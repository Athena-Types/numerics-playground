\documentclass[acmsmall, review]{acmart}

%%
%% \BibTeX command to typeset BibTeX logo in the docs
\AtBeginDocument{%
  \providecommand\BibTeX{{%
    Bib\TeX}}}

%% Rights management information.  This information is sent to you
%% when you complete the rights form.  These commands have SAMPLE
%% values in them; it is your responsibility as an author to replace
%% the commands and values with those provided to you when you
%% complete the rights form.
%\setcopyright{acmlicensed}
%\copyrightyear{2025}
%\acmYear{2025}
%\acmDOI{XXXXXXX.XXXXXXX}
%
%%%
%%% These commands are for a JOURNAL article.
%\acmJournal{JACM}
%\acmVolume{37}
%\acmNumber{4}
%\acmArticle{111}
%\acmMonth{8}

%%
%% Submission ID.
%% Use this when submitting an article to a sponsored event. You'll
%% receive a unique submission ID from the organizers
%% of the event, and this ID should be used as the parameter to this command.
%%\acmSubmissionID{123-A56-BU3}

%%
%% For managing citations, it is recommended to use bibliography
%% files in BibTeX format.
%%
%% You can then either use BibTeX with the ACM-Reference-Format style,
%% or BibLaTeX with the acmnumeric or acmauthoryear sytles, that include
%% support for advanced citation of software artefact from the
%% biblatex-software package, also separately available on CTAN.
%%
%% Look at the sample-*-biblatex.tex files for templates showcasing
%% the biblatex styles.
%%

%%
%% The majority of ACM publications use numbered citations and
%% references.  The command \citestyle{authoryear} switches to the
%% "author year" style.
%%
%% If you are preparing content for an event
%% sponsored by ACM SIGGRAPH, you must use the "author year" style of
%% citations and references.
%% Uncommenting
%% the next command will enable that style.
\citestyle{acmauthoryear}



%% 
% FIGURES
\usepackage{multirow}
\usepackage{xspace}
\usepackage{adjustbox}

%%
% IMAGE
\usepackage{graphicx}
\usepackage{caption}
\usepackage{subcaption}
\usepackage{lscape}

%%
% MATHS
\newtheorem*{remark}{Remark}
\usepackage{amsthm}
\usepackage{amsmath}
\usepackage{bussproofs} 
    \EnableBpAbbreviations
\usepackage{mathpartir} 
\usepackage{mathtools}
\usepackage{stmaryrd}
\newcommand{\R}{\mathbb{R}}
\newcommand{\RR}{\R}
\newcommand{\NNR}{\mathbb{R}^{\geq 0}}
\newcommand{\NN}{\mathbb{N}}
\newcommand{\F}{\mathbb{F}}

%% CONTSANTS
\newcommand{\rnderr}{\ensuremath{\epsilon}}

%% PROGRAMS
\newcommand{\rnd}{\textbf{rnd }}
\newcommand{\ret}{\textbf{ret }}
\newcommand{\letcobind}{\textbf{let-cobind }}
\newcommand{\letassign}{\textbf{let }}
\newcommand{\letdestruct}{\textbf{let-destruct }}
\newcommand{\tin}{\textbf{in }}
\newcommand{\tif}{\textbf{if }}
\newcommand{\tthen}{\textbf{then }}
\newcommand{\letbind}{\textbf{let-bind }}
\newcommand{\inl}{\textbf{inl }}
\newcommand{\inr}{\textbf{inr }}
\newcommand{\op}{\textbf{op}}

%%% CODE
\usepackage{listings}
\usepackage{lstlang}
\usepackage{xcolor}
\usepackage[T1]{fontenc}
\usepackage[scaled]{beramono}
\lstset{mathescape=true,language=fz}

%% TYPES
\newcommand{\unit}{\textbf{unit}}
\newcommand{\num}{\textbf{num}}
\newcommand{\tensor}{\otimes}
\newcommand{\tand}{\ \& \ }
\newcommand{\lin}{\multimap}
\newcommand{\bang}[1]{{!_{#1}}}
\newcommand{\Ra}{\textmd{R}}


\newcommand{\Met}{\mathbf{Met}}
\newcommand{\Set}{\mathbf{Set}}
\newcommand{\denot}[1]{\llbracket {#1} \rrbracket}
\newcommand{\pdenot}[1]{\llparenthesis {#1} \rrparenthesis}
\newcommand{\interpM}[1]{\mathcal{I}({#1})}

%%
%LANG
\newcommand{\Lang}{$\Lambda_\num$\xspace}

%%
% DIAGRAMS
\let\Bbbk\relax % clash with amssymb from acmart.
\usepackage{quiver}

%% 
% CLEVEREF
\usepackage{cleveref}

%%
%% end of the preamble, start of the body of the document source.
\begin{document}

%%
%% The "title" command has an optional parameter,
%% allowing the author to define a "short title" to be used in page headers.
\title{Extensions to Numerical Fuzz}

%%
%% The "author" command and its associated commands are used to define
%% the authors and their affiliations.
%% Of note is the shared affiliation of the first two authors, and the
%% "authornote" and "authornotemark" commands
%% used to denote shared contribution to the research.
\author{Max Fan}
\email{mxf@cs.cornell.edu}
\orcid{0009-0001-3664-6538}
\affiliation{%
  \institution{Cornell University}
  \city{Ithaca}
  \state{New York}
  \country{USA}
}

\author{Justin Hsu}
\email{justin@cs.cornell.edu}
\affiliation{%
  \institution{Cornell University}
  \city{Ithaca}
  \state{New York}
  \country{USA}
}

%%
%% By default, the full list of authors will be used in the page
%% headers. Often, this list is too long, and will overlap
%% other information printed in the page headers. This command allows
%% the author to define a more concise list
%% of authors' names for this purpose.
\renewcommand{\shortauthors}{Fan and Hsu}

%%
%% The abstract is a short summary of the work to be presented in the
%% article.
\begin{abstract}
\end{abstract}

%%
%% The code below is generated by the tool at http://dl.acm.org/ccs.cfm.
%% Please copy and paste the code instead of the example below.
%%
\begin{CCSXML}
<ccs2012>
<concept>
<concept_id>10011007.10011006.10011008.10011009.10011012</concept_id>
<concept_desc>Software and its engineering~Functional languages</concept_desc>
<concept_significance>500</concept_significance>
</concept>
<concept>
<concept_id>10003752.10003790.10011740</concept_id>
<concept_desc>Theory of computation~Type theory</concept_desc>
<concept_significance>500</concept_significance>
</concept>
<concept>
<concept_id>10003752.10003790.10002990</concept_id>
<concept_desc>Theory of computation~Logic and verification</concept_desc>
<concept_significance>500</concept_significance>
</concept>
<concept>
<concept_id>10002950.10003714.10003715.10003725</concept_id>
<concept_desc>Mathematics of computing~Interval arithmetic</concept_desc>
<concept_significance>200</concept_significance>
</concept>
</ccs2012>
\end{CCSXML}

\ccsdesc[500]{Software and its engineering~Functional languages}
\ccsdesc[500]{Theory of computation~Type theory}
\ccsdesc[500]{Theory of computation~Logic and verification}
\ccsdesc[200]{Mathematics of computing~Interval arithmetic}

%%
%% Keywords. The author(s) should pick words that accurately describe
%% the work being presented. Separate the keywords with commas.
\keywords{linear type systems, verification, round-off error, floating point}

%\received{20 February 2007}
%\received[revised]{12 March 2009}
%\received[accepted]{5 June 2009}

%%
%% This command processes the author and affiliation and title
%% information and builds the first part of the formatted document.
\maketitle

\section{Language Syntax}

\begin{figure}[tbp]
  \begin{alignat*}{3}
         &\text{Types } \sigma, \tau &::=~ &\mathbf{unit}
         \mid \num
         \mid \sigma \times \tau 
         \mid \sigma \otimes \tau
         \mid \sigma + \tau 
         \mid \sigma \multimap \tau
         \mid {\bang{s} \sigma}
         \mid {M_u \tau}
         \\
         &\text{Values } v, w \ &::=~ &\langle \rangle
         \mid k \in R
         \mid \langle v,w \rangle 
         \mid (v, w)
         \mid \tin_i \ v
         \mid \lambda x.~e \\
         % & & & \mid [v]
         % \mid\rnd v
         % \mid {\ret v} 
         % \mid \factor v
         % \mid \letbind x = v \ \tin \ f 
         \\
         &\text{Terms } e, f, g &::=~ &x
         \mid v
         \mid \mathbf{op}(e)
         \mid e~f
         \mid {\pi}_i\ e
         \mid \langle e,f \rangle 
         \mid (e, f) \\
         & & & \mid \letpair \ (x,y) = e \ \tin \ f
         \mid \letassign x  = e \ \tin \ f \\
         & & & \mid \tin_i \ e
         \mid 
          \mathbf{case} \ e \ \mathbf{of} \ (\tin_1 \ x.f \ | \ \tin_2 \ x.g) \\
         & & &
         \mid [e]
         \mid \rnd e
         \mid {\ret e} 
         \mid \factor e \\
         & & & 
         \mid {\letbind x = e \ \tin \ f}
         \mid \letcobind x = e \ \tin \ f
         \\
         % Max: find a better name
         &\text{Enviroments } \gamma_0, gamma_1 &::=~ &.
         \mid \gamma, \ x \mapsto v :_s \tau \\
         &\text{Enviroment trees } \sigma &::=~ & \gamma
         \mid \gamma_0; \gamma_1
  \end{alignat*}
  \caption{
    Types, values, and terms. 
    $\mathbf{op} \in \mathcal{O}$.
    $i \in \{1, 2\}$. 
  }
  \label{fig:syntax}
\end{figure}

% Max: For now, I've separated \letbind, \letcobind, \letpair, and
% \letassign to make things more clear.


\section{Dynamic Semantics}

\subsection{Operational Semantics}
\begin{figure}
\begin{center}

\begin{equation*}
\begin{aligned}[c]
	\mathbf{op}(v) &\mapsto op(v)\\
	\pi_i\langle v_1,v_2 \rangle &\mapsto v_i \\
	(\lambda x.e) \ v &\mapsto e[v/x] \\
	%\factor v \ &\mapsto v
\end{aligned}
\quad
\begin{aligned}[c]
	\letassign x = v \ \tin \ e &\mapsto e[v/x] \\
  \letpair (x, y) = (v, w) \ \tin \ e &\mapsto e[v/x][w/y] \\
	\letcobind x = [v] \ \tin \ e &\mapsto e[v/x]
	%\letbind x = \ret v \ \tin \ e &\mapsto e[v/x] \\
\end{aligned}
\end{equation*}
\vskip -1em
\begin{align*}
  \letbind y = (\letbind x = v \ \tin \ f) \ \tin \ g &\mapsto \letbind x = v \ \tin \ \letbind y = f \ \tin \ g \quad x\notin FV(g) 
\end{align*}
\vskip -1.75em
\begin{align*}
	\mathbf{case} \ (\mathbf{in}_i \ v) \ \mathbf{of} \ (\mathbf{in}_1 \ x.e_1 \ | \ \mathbf{in}_2 \ x.e_2 )  &\mapsto e_i[v/x]
  \qquad\qquad(i \in \{1, 2 \})
\end{align*}
\vskip -0.25em

	\AXC{$e \mapsto e'$}
	\UIC{$\letassign x = e \ \tin \ f \mapsto \letassign x = e' \ \tin f$}
	\DisplayProof

\end{center}
    \caption{Evaluation rules for \Lang. Note the side condition for $\letbind$always holds for closed expressions.}
    \label{fig:eval_rules}
\end{figure}

\section{Static Semantics}

\section{Denotational Type Semantics}

\section{Type Soundness}
\begin{lemma}[Termination]
  If $\Gamma \vDash e : \tau$, then $\forall \sigma$ compatible with $\Gamma$, 
  $$\sigma \Vdash e \rightsquigarrow^* \sigma' \Vdash v$$
\end{lemma}
\begin{proof}
  If $\sigma \Vdash e$ is a value, then we are done. Otherwise, if $\sigma
  \Vdash e$ is an expression, we begin by unfolding and applying the definition
  of $\vDash$. By definition, know that $\llbracket \sigma \Vdash e \rrbracket_{\tau} \in
  \llbracket \tau \rrbracket$. We also know that $\bot \not\in \llbracket \tau
  \rrbracket$, a metric space, so $\llbracket \sigma \Vdash e \rrbracket_{\tau}
  \not= \bot$ and is well-defined.

  By inspection of the definition of cases of $\llbracket - \rrbracket_{\tau}$,
  we know that only last case (which deals with all expressions) must apply.
  Therefore, $\sigma \Vdash e \rightsquigarrow^* \sigma' \Vdash v$.
\end{proof}

% \begin{theorem}[Semantic type preservation]
% If a term $e$ is semantically well-typed in context $\Gamma$ with type $\tau$
%   then for all $\sigma \Vdash e \rightsquigarrow \sigma' \Vdash e'$, $\sigma$
%   compatible with $\Gamma$, then there exists a $\Gamma'$ such that $\Gamma'
%   \vDash e' : \tau$.
% \end{theorem}
% \begin{proof}
% If $e$ is a value, we're done. 
% If $e$ is not a value, we proceed by induction over the cases of the rewrite relation
% $\rightsquigarrow$.
% % Our inductive hypothesis is that $e \rightsquigarrow e'$ and $\Gamma \vDash e : \tau$, that is, 
% % $\forall \sigma \text{ compatible with } \Gamma, \llbracket \sigma \Vdash e
% % \rrbracket_{\tau} \in \llbracket \tau \rrbracket$.
% \begin{description}
%   \item[\textsc{Variable lookup.}] If $$
%   \item[\textsc{Ret.}]
%   \item[\textsc{Rnd.}]
% \end{description}
% \end{proof}

% \begin{definition}[Monadic type]\label{def:monadic}
%   A type $\tau$ is monadic if and only if $\exists \tau', q \text{ such that }
%   \tau = M_q \tau'$. It is non-monadic otherwise.
% \end{definition}
% %%% wrong theorem:
% \begin{lemma}[Non-monadic lookup]\label{thm:non-monadic-lookup}
%   For all $\tau$ non-monadic and all $\Gamma, x : \tau'$ compatible with
%   $\sigma$, $\exists \sigma', \sigma = \sigma'[x \mapsto v : \tau']$ is
%   well-defined. Stated equivalently, non-monadic $x$ uniformly maps to $v$ at
%   all enviroments in $\sigma$.
% \end{lemma}
% \begin{proof}
%   TODO
% \end{proof}

\begin{lemma}[Enviroment tree shape]
  The tree height for a 
\end{lemma}

\begin{definition}[Orderings]\label{def:orderings}
  We define the following well-founded partial order over configurations and
  types, which mirrors the definition of machine configuration interpretation
  and will be used for induction in the proceeding proofs.

  First, we define an ordering over types by size of the AST.
  \begin{equation}
    \begin{aligned}[c]
      s(\textbf{unit}) &= 1 \\
      s(\textbf{num}) &= 1 \\
      s(!_s \tau) &= s(\tau) + 1 \\
      s(M_q \tau) &= s(\tau) + 1 \\
      s(\tau_0 \times \tau_1) &= s(\tau_0) + s(\tau_1) + 1 \\
      s(\tau_0 \otimes \tau_1) &= s(\tau_0) + s(\tau_1) + 1 \\
      s(\tau_0 \multimap \tau_1) &= s(\tau_0) + s(\tau_1) + 1 \\
    \end{aligned}
  \end{equation}
  and
  \begin{equation}
    \tau_0 < \tau_0 \iff s(\tau_0) < s(\tau_1)
  \end{equation}

  Secondly, we define an ordering over environment leafs $\gamma$ using the number of bound
  variables:
  \begin{equation}
    \begin{aligned}[c]
      s(.) &= 0 \\
      s(\gamma, x \mapsto v :_s \tau) &= s(\gamma) + 1 \\
    \end{aligned}
  \end{equation}

  and

  \begin{equation}
    \gamma_0 < \gamma_0 \iff s(\gamma_0) < s(\gamma_1)
  \end{equation}

  We can now define an ordering over enviorment trees using the maximum size of
  each enviorment leaf in the tree:
  \begin{equation}
    \begin{aligned}[c]
      s(\gamma; \gamma') &= max(s(\gamma), s(\gamma')) \\
      s(\sigma; \sigma') &= max(s(\sigma), s(\sigma'))
    \end{aligned}
  \end{equation}

  We are now ready to define an ordering over $\textit{config} \times
  \textit{type}$:
  \footnote{This is our termination measure for our machine configuration
  interpretation function.}
  \footnote{Note that the type of the configuration (if the interpretation of
  the configuration belongs to the interpretation of the type) restricts the
  height and shape of environment tree. So we do not need to reason about the
  height of our enviroment trees in our ordering.}

  \begin{equation}
    \sigma \Vdash e : \tau < \sigma' \Vdash e' : \tau' \iff
    s(\tau) < s(\tau') \lor (s(\tau) = s(\tau') \land s(\sigma) < s(\sigma'))
  \end{equation}

\end{definition}

\begin{lemma}[Well-founded partial ordering]
  $\leq$ is a well-founded partial order over $\textit{config} \times
  \textit{type}$.
\end{lemma}
\begin{proof}
  TODO
\end{proof}

\begin{theorem}[Semantic type soundness]
$\Gamma \vdash e : \tau \implies \Gamma \vDash e : \tau$
\end{theorem}
\begin{proof}
From our premise, we wish to show that:
  \begin{enumerate}
    \item $\forall \sigma$ compatible with $\Gamma$, $\llbracket \sigma \Vdash e
      \rrbracket_{\tau} \in \llbracket \tau \rrbracket$.
    \item $\llbracket - \Vdash e
      \rrbracket_{\tau}$ is a \text{1-sensitive} map for all inputs enviroments
      compatible with $\Gamma$
  \end{enumerate}
We proceed by case analysis over whether $e$ is a value or an expression.
  Suppose $e$ is a value. Let us induct over our typing derivation. 
  % In some of the cases, we will need to induct over the size of our enviroment. 
\begin{description}
  \item[\textsc{(ty. rule) Var.}] There is no premise containing a typing
    judgement to this rule. So we have no inductive hypothesis to rely on.
    % We proceed to induct over the type of our variable, $\tau$.
    % \begin{description}
    %   \item{\textbf{unit} and \textbf{num}.}
    %   \item{$\tau_0 \otimes \tau_1$, $\tau_0 + \tau_1$, and $\tau_0 \multimap
    %     \tau_1$.}
    %   \item{$!_s \tau$.}
    %   \item{$M_q \tau$}
    %   \item{$\tau_0 \times \tau_1$.}
    % \end{description}
    Instead, we induct over the the size of our enviorment.
    \begin{description}
      \item[\textit{(env. size) 0.}] We observe that an enviorment size of zero
        implies that $\Gamma$ is empty. So we are trying to show that the
        following:
        $$\Gamma, x : \tau \vDash e : \tau$$
        implies:
        $$\llbracket \sigma \Vdash e \rrbracket_\tau \in \llbracket \tau
        \rrbracket$$
        for compatible $\sigma, \Gamma, x : \tau$. We proceed to case over the type of our
        variable, $\tau$.
        \begin{description}
          \item{(ty. case) \textbf{unit.}} The only configuration compatible with this is
            $x \mapsto \langle \rangle \Vdash x$ and clearly 
            $$\llbracket x \mapsto \langle \rangle \Vdash x
            \rrbracket_{\textbf{unit}} = \llbracket \Vdash \langle \rangle
            \rrbracket_{\textbf{unit}} = * \in \{ * \} = \llbracket \textbf{unit}
            \rrbracket$$
            So, properties (1) and (2) hold by the reasoning above.
          \item{(ty. case) \textbf{num.}}
        \end{description}
      \item[\textit{(env. size) n + 1.}] 
    \end{description}
  \item[\textsc{(ty. rule) Ret.}] 
    From our inductive hypothesis, we have that $\Gamma \vDash e : \tau$ and
    wish to prove that $\Gamma \vDash \mathbf{ret} \ e : \tau$. It suffices to
    show that $(\sigma; \sigma)[\alpha \mapsto v:_1 \tau] \Vdash \alpha$.
  \item[\textsc{(ty. rule) Rnd.}]
\end{description}

  Next, the tree case:
\end{proof}

%%%% IGNORE BELOW %%%%

% todo: Use whatever theorem name people like best here.
Semantics is preserved under operational stepping. If $\sigma \Vdash e : \tau \rightsquigarrow \sigma'
\Vdash e' : \tau$, then $\llbracket \sigma \Vdash e : \tau \rrbracket =
\llbracket \sigma' \Vdash e' : \tau \rrbracket$.

% todo: Use whatever theorem name people like best here.
If the semantics of a program is equivalent to the semantics of a value, it must
reduce to that value. If $\llbracket \sigma \Vdash e : \tau \rrbracket =
\llbracket v : \tau \rrbracket$, then $\sigma \Vdash e : \tau
\rightsquigarrow^{*} \sigma' \Vdash v : \tau$

Syntactically well-typed programs are non-expansive. For $\Gamma \vdash e : \tau
$ and $\llbracket \sigma \rrbracket, \llbracket \sigma' \rrbracket \in
\llbracket \Gamma \rrbracket $ and
$$\sigma \Vdash e : \tau \rightsquigarrow^* v : \tau$$ 
and 
$$\sigma' \Vdash e : \tau \rightsquigarrow^* v' : \tau$$
then
$$
d_{\tau}(v, v') \leq d_{\Gamma}(\sigma, \sigma')
$$

Syntactically ok implies semantically ok. If $\Gamma \vdash e : \tau$, then
$\forall \llbracket \sigma \rrbracket \in \llbracket \Gamma \rrbracket, \exists
v, \llbracket \sigma \Vdash e : \tau \rrbracket = \llbracket v : \tau
\rrbracket$.

Syntactically ok implies operationally ok. If $\Gamma \vdash e : \tau$, then
$\forall \llbracket \sigma \rrbracket \in \llbracket \Gamma \rrbracket, \exists
v, \sigma \Vdash e : \tau \rightsquigarrow^{*} v : \tau$.


\section{Subnormals}

\section{Sharing Error}

\section{Evaluation}


%%
%% The acknowledgments section is defined using the "acks" environment
%% (and NOT an unnumbered section). This ensures the proper
%% identification of the section in the article metadata, and the
%% consistent spelling of the heading.
%\begin{acks}
%  \input{XX-acknowledgments.tex}
%\end{acks}

%%
%% The next two lines define the bibliography style to be used, and
%% the bibliography file.
\bibliographystyle{ACM-Reference-Format}
\bibliography{sample-base}


%%
%% If your work has an appendix, this is the place to put it.
%\appendix
%\input{XX-appendix.tex}

\end{document}
\endinput
%%
%% End of file `sample-acmsmall.tex'.

