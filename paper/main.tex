\documentclass[acmsmall, review]{acmart}

%%
%% \BibTeX command to typeset BibTeX logo in the docs
\AtBeginDocument{%
  \providecommand\BibTeX{{%
    Bib\TeX}}}

%% Rights management information.  This information is sent to you
%% when you complete the rights form.  These commands have SAMPLE
%% values in them; it is your responsibility as an author to replace
%% the commands and values with those provided to you when you
%% complete the rights form.
%\setcopyright{acmlicensed}
%\copyrightyear{2025}
%\acmYear{2025}
%\acmDOI{XXXXXXX.XXXXXXX}
%
%%%
%%% These commands are for a JOURNAL article.
%\acmJournal{JACM}
%\acmVolume{37}
%\acmNumber{4}
%\acmArticle{111}
%\acmMonth{8}

%%
%% Submission ID.
%% Use this when submitting an article to a sponsored event. You'll
%% receive a unique submission ID from the organizers
%% of the event, and this ID should be used as the parameter to this command.
%%\acmSubmissionID{123-A56-BU3}

%%
%% For managing citations, it is recommended to use bibliography
%% files in BibTeX format.
%%
%% You can then either use BibTeX with the ACM-Reference-Format style,
%% or BibLaTeX with the acmnumeric or acmauthoryear sytles, that include
%% support for advanced citation of software artefact from the
%% biblatex-software package, also separately available on CTAN.
%%
%% Look at the sample-*-biblatex.tex files for templates showcasing
%% the biblatex styles.
%%

%%
%% The majority of ACM publications use numbered citations and
%% references.  The command \citestyle{authoryear} switches to the
%% "author year" style.
%%
%% If you are preparing content for an event
%% sponsored by ACM SIGGRAPH, you must use the "author year" style of
%% citations and references.
%% Uncommenting
%% the next command will enable that style.
\citestyle{acmauthoryear}



%% 
% FIGURES
\usepackage{multirow}
\usepackage{xspace}
\usepackage{adjustbox}

%%
% IMAGE
\usepackage{graphicx}
\usepackage{caption}
\usepackage{subcaption}
\usepackage{lscape}

%%
% MATHS
\newtheorem*{remark}{Remark}
\usepackage{amsthm}
\usepackage{amsmath}
\usepackage{bussproofs} 
    \EnableBpAbbreviations
\usepackage{mathpartir} 
\usepackage{mathtools}
\usepackage{stmaryrd}
\newcommand{\R}{\mathbb{R}}
\newcommand{\RR}{\R}
\newcommand{\NNR}{\mathbb{R}^{\geq 0}}
\newcommand{\NN}{\mathbb{N}}
\newcommand{\F}{\mathbb{F}}

%% CONTSANTS
\newcommand{\rnderr}{\ensuremath{\epsilon}}

%% PROGRAMS
\newcommand{\rnd}{\textbf{rnd }}
\newcommand{\ret}{\textbf{ret }}
\newcommand{\letcobind}{\textbf{let-cobind }}
\newcommand{\letassign}{\textbf{let }}
\newcommand{\letpair}{\textbf{let-pair }}
\newcommand{\tin}{\textbf{in}}
\newcommand{\tif}{\textbf{if }}
\newcommand{\tthen}{\textbf{then }}
\newcommand{\letbind}{\textbf{let-bind }}
\newcommand{\inl}{\textbf{inl }}
\newcommand{\inr}{\textbf{inr }}
\newcommand{\factor}{\textbf{factor }}
\newcommand{\op}{\textbf{op}}
\newcommand{\concat}{~\textit{\footnotesize{++}}~}

%%% CODE
\usepackage{listings}
\usepackage{lstlang}
\usepackage{xcolor}
\usepackage[T1]{fontenc}
\usepackage[scaled]{beramono}
\lstset{mathescape=true,language=fz}
\usepackage{xcolor}

%% TYPES
\newcommand{\unit}{\textbf{unit}}
\newcommand{\num}{\textbf{num}}
\newcommand{\tensor}{\otimes}
\newcommand{\tand}{\ \& \ }
\newcommand{\lin}{\multimap}
\newcommand{\bang}[1]{{!_{#1}}}
\newcommand{\Ra}{\textmd{R}}


\newcommand{\Met}{\mathbf{Met}}
\newcommand{\Set}{\mathbf{Set}}
\newcommand{\denot}[1]{\llbracket {#1} \rrbracket}
\newcommand{\pdenot}[1]{\llparenthesis {#1} \rrparenthesis}
\newcommand{\interpM}[1]{\mathcal{I}({#1})}

%%
%LANG
\newcommand{\Lang}{$\Lambda_\num^-$\xspace}

%%
% DIAGRAMS
\let\Bbbk\relax % clash with amssymb from acmart.
\usepackage{quiver}

%% 
% CLEVEREF
\usepackage{cleveref}

%%
%% end of the preamble, start of the body of the document source.
\begin{document}

%%
%% The "title" command has an optional parameter,
%% allowing the author to define a "short title" to be used in page headers.
\title{Extensions to Numerical Fuzz}

%%
%% The "author" command and its associated commands are used to define
%% the authors and their affiliations.
%% Of note is the shared affiliation of the first two authors, and the
%% "authornote" and "authornotemark" commands
%% used to denote shared contribution to the research.
\author{Max Fan}
\email{mxf@cs.cornell.edu}
\orcid{0009-0001-3664-6538}
\affiliation{%
  \institution{Cornell University}
  \city{Ithaca}
  \state{New York}
  \country{USA}
}

\author{Justin Hsu}
\email{justin@cs.cornell.edu}
\affiliation{%
  \institution{Cornell University}
  \city{Ithaca}
  \state{New York}
  \country{USA}
}

%%
%% By default, the full list of authors will be used in the page
%% headers. Often, this list is too long, and will overlap
%% other information printed in the page headers. This command allows
%% the author to define a more concise list
%% of authors' names for this purpose.
\renewcommand{\shortauthors}{Fan and Hsu}

%%
%% The abstract is a short summary of the work to be presented in the
%% article.
\begin{abstract}
\end{abstract}

%%
%% The code below is generated by the tool at http://dl.acm.org/ccs.cfm.
%% Please copy and paste the code instead of the example below.
%%
\begin{CCSXML}
<ccs2012>
<concept>
<concept_id>10011007.10011006.10011008.10011009.10011012</concept_id>
<concept_desc>Software and its engineering~Functional languages</concept_desc>
<concept_significance>500</concept_significance>
</concept>
<concept>
<concept_id>10003752.10003790.10011740</concept_id>
<concept_desc>Theory of computation~Type theory</concept_desc>
<concept_significance>500</concept_significance>
</concept>
<concept>
<concept_id>10003752.10003790.10002990</concept_id>
<concept_desc>Theory of computation~Logic and verification</concept_desc>
<concept_significance>500</concept_significance>
</concept>
<concept>
<concept_id>10002950.10003714.10003715.10003725</concept_id>
<concept_desc>Mathematics of computing~Interval arithmetic</concept_desc>
<concept_significance>200</concept_significance>
</concept>
</ccs2012>
\end{CCSXML}

\ccsdesc[500]{Software and its engineering~Functional languages}
\ccsdesc[500]{Theory of computation~Type theory}
\ccsdesc[500]{Theory of computation~Logic and verification}
\ccsdesc[200]{Mathematics of computing~Interval arithmetic}

%%
%% Keywords. The author(s) should pick words that accurately describe
%% the work being presented. Separate the keywords with commas.
\keywords{linear type systems, verification, round-off error, floating point}

%\received{20 February 2007}
%\received[revised]{12 March 2009}
%\received[accepted]{5 June 2009}

%%
%% This command processes the author and affiliation and title
%% information and builds the first part of the formatted document.
\maketitle

\section{Introduction}

It is natural for ordinary users such as mathematicians and scientists to wish
their computers operate over the reals. But this is impossible. We typically
design languages that compute over the finitary floating-point number
approximation.
As language designers, we wish to assure the ordinary user that we have not
scammed them: that the floating-point programming language semantics we have
provided is close to what they desire. 
To do this, we develop automated numerical analysis techniques that bound the
maximum error introduced by floating-point computation.

% cite Gappa, FPTaylor, Satire etc.

A major challenge of the automated numerical analysis literature is scalability.
In particular, many approaches rely on global optimization, % cite Satire
rewrite saturation, % cite Gappa
or SMT-based methods % find citation
that frequently time-out on large codebases. 
% explain a bit more about why these don't scale

Recent work has applied typed-based analysis approaches to both forwards error
analysis, such as Numerical Fuzz, and backwards numerical error analysis, such
as in Bean. % cite NumFuzz and Bean
In particular, Numerical Fuzz uses a graded effect and co-effect system to
perform a forwards error analysis and is able to outperform many other
state-of-the-art tools on existing benchmarks.
% more about how NumFuzz works

However, Numerical Fuzz does not handle negative numbers and subtraction, which
significantly limits its applicability. There are significant foundational
problems with providing a simultaneously \textit{compositional} and
\textit{scalable} analysis for forwards numerical error in the presence of
subtraction. By \textit{compositional} we mean that the analysis need not be
global. For this we can only rely on a relative notion of error. And by
\textit{scalable} we mean that the analysis remains relatively tight as the
program size balloons.

The problem with providing a compositional and scalable analysis is due to
\textit{catastrophic cancellation}: the phenomenon where two large nearby
numbers with small relative error can be subtracted to produce a small number
with arbitrarily high relative error. Relative notions of error are not
well-behaved in the presence of subtraction and negative numbers. 

% more build-up to the contributions, talk about a posteori bounds

\subsection{Contributions}

Our contributions are as follows:
\begin{itemize}
  \item To side-step the problem of catastrophic cancellation, we develop a
    type-based approach to provide \textit{a posteriori} forwards error bounds
    in the presence of subtraction and negative numbers. 

  \item We discover that it is impossible to share error terms or sensitivities
    in prior type-based work. We address this limitation by introducing a
    previously-untypable \textbf{factor : M_q \tau_0 M_r \tau_1 \multimap
    M_{max(q, r)} (\tau_0 \times \tau_1)} primitive that enables error terms and
    sensitivities to be shared. This allows for tighter and more scalable error
    bounds.

  \item We extend Numerical Fuzz to handle expressions in more places, thereby
    allowing more programs to be typed.
\end{itemize}

\section{Language Syntax}

\begin{figure}[tbp]
  \begin{alignat*}{3}
         &\text{Types } \sigma, \tau &::=~ &\mathbf{unit}
         \mid \num
         \mid \sigma \times \tau 
         \mid \sigma \otimes \tau
         \mid \sigma + \tau 
         \mid \sigma \multimap \tau
         \mid {\bang{s} \sigma}
         \mid {M_u \tau}
         \\
         &\text{Values } v, w \ &::=~ &x
         \mid \langle \rangle
         \mid k \in R
         \mid \langle v,w \rangle 
         \mid  (v, w)
         \mid \inl \ v
         \mid \inr \ v
         \\
         & & & \mid \lambda x.~e 
         \mid [v]
         \mid\rnd v
         \mid {\ret v} 
         \mid \letbind x = \rnd v \ \tin f \\
         &\text{Terms } e, f &::=~ & v
         \mid v~w
         \mid {\pi}_i\ v
         \mid \langle v,e \rangle 
         \mid  (e, f)
         \mid \inl \ e
         \mid \inr \ e
         \mid \letdestruct \ (x,y) = e \ \tin f \\
         & & & \mid \mathbf{case} \ v \ \mathbf{of} \ (\inl x.e \ | \ \inr x.f) 
         \mid \letcobind x = e \ \tin  f
         \mid {\letbind x = e \ \tin f} \\
         & & & \mid \letassign x  = e \ \tin  f 
         \mid \mathbf{op}(v) \quad \mathbf{op} \in \mathcal{O}
  \end{alignat*}
  \caption{Types, values, and terms.}
  \label{fig:syntax}
\end{figure}

% Max: For now, I've separated \letbind, \letcobind, \letdestruct, and
% \letassign to make things more clear.


\section{Static Semantics}
We define an interpretation function $interp(b)$ evaluating bounds to points in
$\textit{num}$. It is only defined over closed types.

\begin{equation}
  \begin{aligned}[c]
    interp(k \in \textit{num}) &\triangleq k \\
    interp(b_0 + b_1) &\triangleq interp(b_0) + interp(b_1) \\
    interp(b_0 \cdot b_1) &\triangleq interp(b_0) \cdot interp(b_1) \\
  \end{aligned}
\end{equation}

\begin{figure}
%% ROW1
\begin{center}
%% var
\AXC{$s \ge 1$}
\RightLabel{(Var)}
\UIC{$\Gamma, x:_s \tau, \Delta \vdash x : \tau$}
\bottomAlignProof
\DisplayProof
\hskip 0.5em
%% fun
\AXC{$\Gamma, x:_1 \tau_0 \vdash e : \tau$}
\RightLabel{($\multimap$ I)}
\UnaryInfC{$\Gamma \vdash \lambda x. e : \tau_0 \multimap \tau $}
\bottomAlignProof
\DisplayProof
\hskip 0.5em
%% app
\AXC{$\Gamma \vdash e : \tau_0 \multimap \tau$}
\AXC{$\Theta \vdash f : \tau_0 $}
\RightLabel{($\multimap$ E)}
\BinaryInfC{$\Gamma + \Theta \vdash ef : \tau $}
\bottomAlignProof
\DisplayProof
\vskip 1em
%%


%% ROW2
\AXC{}
\RightLabel{(Unit)}
\UIC{$\Gamma \vdash \langle \rangle : \mathbf{unit}$}
\bottomAlignProof
\DisplayProof
\hskip 0.5em
%% dep prod intro
\AXC{$\Gamma \vdash e : \tau_0$}
\AXC{$\Gamma \vdash f : \tau_1$}
\RightLabel{($\times$ I)}
\BinaryInfC{$\Gamma \vdash \langle e, f \rangle: \tau_0 \times \tau_1$}
\bottomAlignProof
\DisplayProof
\hskip 0.5em
%% dep prod elim
\AXC{$\Gamma \vdash e : \tau_1 \times \tau_2$}
\RightLabel{($\times$ E)}
\UIC{$\Gamma \vdash {\pi}_i \ e : \tau_i$}
\bottomAlignProof
\DisplayProof
\vskip 1em
%%

%% ind prod intro
\AXC{$\Gamma \vdash e : \tau_0 $}
\AXC{$\Theta \vdash f : \tau_1$}
\RightLabel{($\tensor$ I)}
\BIC{$\Gamma + \Theta \vdash (e, f) : \tau_0 \tensor \tau_1$}
\bottomAlignProof
\DisplayProof
\hskip 0.5em
%% ind prod elim
\AXC{$\Gamma \vdash e : \tau_0 \tensor \tau_1$ }
\AXC{$\Theta,x:_s \tau_0,y:_s\tau_1 \vdash f: \tau $}
\RightLabel{($\tensor$ E)}
\BIC{$s * \Gamma + \Theta \vdash \letpair (x,y) \ = \ e \ \tin \ f : \tau $}
\bottomAlignProof
\DisplayProof
\vskip 1em
%%


%% ROW 4

%% ind sum intro
\AXC{$\Gamma \vdash e : \tau_0$ }
\RightLabel{($+$ $\text{I}_i$)}
\UIC{$\Gamma \vdash \mathbf{in}_i \ e : \tau_0 + \tau_1$}
\bottomAlignProof
\DisplayProof
\hskip 0.5em
% %% ind sum intro
% \AXC{$\Gamma \vdash e : \tau_1$ }
% \RightLabel{($+$ $\text{I}_2$)}
% \UIC{$\Gamma \vdash \mathbf{in}_2 \ e : \tau_0 + \tau_1$}
% \bottomAlignProof
% \DisplayProof
% \hskip 0.5em
% box elim
\AXC{$\Gamma \vdash e : {!_s \tau_0}$}
\AXC{$\Theta, x:_{t*s} \tau_0 \vdash f : \tau$}
\RightLabel{($!$ E)}
\BIC{$t * \Gamma + \Theta \vdash \letcobind x = e \ \tin \ f : \tau$}
\bottomAlignProof
\DisplayProof
\vskip 1em
%%


%% ROW 5

% sum elim
\AXC{$\Gamma \vdash e : \tau_0+\tau_1$}
\AXC{$\Theta, x:_s \tau_0 \vdash f_1 : \tau$ \qquad
$\Theta, x:_s \tau_1 \vdash f_2: \tau$}
\RightLabel{($+$ E)}
\AXC{$s > 0$}
\TIC{$s * \Gamma + \Theta \vdash \mathbf{case} \ e \ \mathbf{of} \ (\mathbf{in}_1 x.f_1 \ | \ \mathbf{in}_2 x.f_2) : \tau$}
\bottomAlignProof
\DisplayProof
\hskip 0.5em
% box intro
\AXC{$\Gamma \vdash e : \tau$ }
\RightLabel{($!$ I)}
\UIC{$s * \Gamma \vdash [e] : {!_s \tau}$}
\bottomAlignProof
\DisplayProof
\vskip 1em

%%% ROW 6

% let 
\AXC{$\Gamma \vdash e :  \tau_0$}
\AXC{$\Theta, x:_{s} \tau \vdash f : \tau$}
\RightLabel{(Let)}
\BIC{$s * \Gamma + \Theta \vdash \letassign x = e \ \tin \ f : \tau$}
\bottomAlignProof
\DisplayProof
\hskip 0.5em

\vskip 1em

%% const
\AXC{$k \in \textit{num}$}
\AXC{$k \in [interp(b_0), interp(b_1)]$}
\RightLabel{(Const)}
\BIC{$\Gamma \vdash k : \num_{(b_0, b_1)}$}
\bottomAlignProof
\DisplayProof
\hskip 0.5em
\vskip 1em

%%% ROW 7

%% subsumption
\AXC{$\Gamma \vdash e :  M_q \tau$}
\AXC{$r \ge q$}
\RightLabel{(Subsumption)}
\BIC{$\Gamma \vdash e :  M_{r} \tau$}
\bottomAlignProof
\DisplayProof
\hskip 0.5em
%% return
\AXC{$\Gamma \vdash e : \tau$}
\RightLabel{(Ret)}
\UIC{$\Gamma \vdash \ret e : M_0 \tau$}
\bottomAlignProof
\DisplayProof
\hskip 0.5em
%% RND
\AXC{$\Gamma \vdash e : \num$}
\RightLabel{(Rnd)}
\UIC{$\Gamma \vdash \rnd \ e : M_u \ \num$}
\bottomAlignProof
\DisplayProof
\vskip 1em


%%% ROW 8


% let-bind
\AXC{$\Gamma \vdash e : M_r \tau_0$}
\AXC{$\Theta, x:_{s} \tau_0 \vdash f : M_{q} \tau$}
\RightLabel{($M_u$ E)}
\BIC{$s * \Gamma + \Theta \vdash \letbind x = e \ \tin \ f : M_{s*r+q} \tau$}
\bottomAlignProof
\DisplayProof

% \hskip 0.5em
\vskip 1em

% funs
\AXC{$\Gamma \vdash e : \tau_0$}
\AXC{$\{ \mathbf{op} :\tau_0 \lin \tau_1 \} \in \Sigma$}
\RightLabel{(Op)}
\BIC{$\Gamma \vdash \mathbf{op}(e) : \tau_1$}
\bottomAlignProof
\DisplayProof

\vskip 1em


%%% ROW 9


% factor
\AXC{$\Gamma \vdash e : (M_q \tau_0) \times (M_r \tau_1)$}
%\AXC{$r + q \leq s$}
%\AXC{$s = max(r,q)$}
\RightLabel{(Factor)}
\UIC{$\Gamma \vdash \factor \ e : M_{max(q,r)} (\tau_0 \times \tau_1)$}
\bottomAlignProof
\DisplayProof
\hskip 0.5em
\AXC{$\Gamma, \epsilon : i \vdash e : \tau$}
\RightLabel{($\forall$-I)}
\UIC{$\Gamma \vdash \Lambda \epsilon . e : \forall \epsilon . \tau$}
\bottomAlignProof
\DisplayProof
\hskip 0.5em
\AXC{$\Gamma \vdash \Lambda \epsilon . e : \forall \epsilon . \tau$}
\RightLabel{($\forall$-E)}
\UIC{$\Gamma \vdash e : \tau[i/\epsilon]$}
\bottomAlignProof
\DisplayProof

\end{center}
    \caption{Typing rules for \Lang, with $s,t,q,r,u \in \NNR \cup \{\infty\}$
    and for $i \in \{ 1, 2 \}$ where $u$ is a fixed constant parameter (see
    Definition~\ref{def:numfuzz-interface} for details on picking an adequate
    constant).}
    \label{fig:typing_rules}
\end{figure}


\section{Dynamic Semantics}

\subsection{Substitution-style Operational Semantics}
The following is defined over untyped terms. In particular, we define the
operational semantics rewrite relation $\mapsto$ to map from (untyped) \Lang
to (untyped) \Lang. In other words, untypable programs can step (but not
necessarily to values).

\begin{figure}
\begin{center}

\begin{equation*}
\begin{aligned}[c]
	\mathbf{op}(v) &\mapsto op(v)\\
	\pi_i\langle v_1,v_2 \rangle &\mapsto v_i \\
	(\lambda x.e) \ v &\mapsto e[v/x] \\
	%\factor v \ &\mapsto v
\end{aligned}
\quad
\begin{aligned}[c]
	\letassign x = v \ \tin \ e &\mapsto e[v/x] \\
  \letpair (x, y) = (v, w) \ \tin \ e &\mapsto e[v/x][w/y] \\
	\letcobind x = v \ \tin \ e &\mapsto e[v/x]
	%\letbind x = \ret v \ \tin \ e &\mapsto e[v/x] \\
\end{aligned}
\end{equation*}
\vskip -1em
\begin{align*}
  \letbind y = (\letbind x = v \ \tin \ f) \ \tin \ g &\mapsto \letbind x = v \ \tin \ \letbind y = f \ \tin \ g \quad x\notin FV(g) 
\end{align*}
\vskip -1.75em
\begin{align*}
	\mathbf{case} \ (\mathbf{in}_i \ v) \ \mathbf{of} \ (\mathbf{in}_1 \ x.e_1 \ | \ \mathbf{in}_2 \ x.e_2 )  &\mapsto e_i[v/x]
  \qquad\qquad(i \in \{1, 2 \})
\end{align*}
\vskip -0.25em

	\AXC{$e \mapsto e'$}
	\UIC{$\letassign x = e \ \tin \ f \mapsto \letassign x = e' \ \tin f$}
	\DisplayProof

  \vskip 0.4em
	\AXC{$e_1 \mapsto e_1'$}
	\AXC{$e_2 \mapsto e_2'$}
  \BIC{$(e_1, e_2) \mapsto (e_1', e_2')$}
	\DisplayProof

  \vskip 0.4em
	\AXC{$e_1 \mapsto e_1'$}
	\AXC{$e_2 \mapsto e_2'$}
  \BIC{$\langle e_1, e_2 \rangle \mapsto \langle e_1', e_2' \rangle$}
	\DisplayProof
\end{center}
    \caption{Substitution-style evaluation rules for \Lang. Note the side condition for $\letbind$always holds for closed expressions.}
    \label{fig:sub_eval_rules}
\end{figure}

\subsection{(Typed) Enviroment-style Operational Semantics}
The following is defined over typed terms. In particular, we define the
operational semantics rewrite relation $\rightsquigarrow$ to map from a typed
term in program multi-enviroments to a typed term in program multi-enviroments. Note
that in this setup, $\letbind x = v \ \tin \ f$, $[x]$, and $x$ are \textit{not}
values.

To be precise, $\rightsquigarrow$ maps an expression $e$ with type $\tau$
running in a enviroment (or multi-enviroment) $\sigma$ mapping variables like
$x_1$ to value $v_1$ with type $\tau'$ and sensitivity budget $s_1$ to a $e'$
with type $\tau'$ and enviroment $\sigma'$. So, $\sigma$ send variables like
$x_1 \to v_1 :_s \tau'$.

A (multi-)enviroment $\sigma$ is compatible with a typing context $\Gamma$ if
$\llbracket \sigma \rrbracket$ is a point within metric space $\llbracket \Gamma
\rrbracket$ (after erasing $\sigma$'s $0$-sensitive variables). Abusing notation
a little, I write this like so: $\llbracket \sigma \rrbracket \in \llbracket
\Gamma \rrbracket$.
% spell out definiion more

% \begin{equation*}
%   \llbracket \sigma \rrbracket \in \llbracket \Gamma \rrbracket 
%   \triangleq \forall
%   \\ 
%   (x \mapsto v :_s \tau) \in \sigma, 0 < s \implies (x :_s \tau) \in 
%   \Gamma
% \end{equation*}

% spell out definiion more
Similarly, $\llbracket \sigma \Vdash e : \tau \rrbracket$ is
interpreted as the point in the metric space $\llbracket \tau \rrbracket$
obtained by running e at $\sigma$.

Note that $\sigma$ is ordered in the case that variables are shadowed. $\sigma[x
\mapsto v :_s \tau]$ denotes lookup from the right (when on the left-hand side
of a rewrite relation) or insertion from the right (when on the right-hand side
of a rewrite relation).

Since we are affine, once we reduce to a value we do not care what happens to
leftover variables in the enviroment. Therefore, the following notation are
equivalent:
$$
\sigma_{idc} \Vdash v = \ \Vdash v = v
$$

To represent computations that may produce multiple values under a
multi-enviroment (e.g. the rounding effect), the following notation is
equivalent:
$$
\sigma; \sigma' \Vdash v = (\sigma \Vdash v); (\sigma' \Vdash v)
$$

% and
% $$
% \sigma \Vdash v_0;...;v_n = \sigma \Vdash v_0; ...; \sigma \Vdash v_n
% $$

% where $\sigma =  \gamma_0;...;\gamma_k$.

$\sigma + \sigma'$ represents pairwise concatenation over multi-enviroments,
defined as follows:

\begin{definition}[Enviroment sum]
  $\gamma_1 + \gamma_2$ = $\gamma_1 :: \gamma_2 $
\end{definition}

\begin{definition}[Enviroment tree sum]
  $. + .$ = $ . $ and $\gamma_0;\gamma_1 + \gamma_2;\gamma_3$ = $(\gamma_0 +
  \gamma_2);(\sigma_1 + \sigma_3)$ and $\_ + \_$ = $\bot$ otherwise.
\end{definition}

\begin{figure}
\begin{center}
\tiny
\begin{equation*}
  \begin{aligned}[c]
    %%%%%%%%%%%%%%%%%%%%%%%%%%%%%%%%%%%%%%%%%%%%%%%%%%%%%%%%%%%%%%%%%%%%%%%%%%%
    % more spicy rules
    % Lookup (sensitivity budget 1)
    \sigma[x \mapsto v :_1 \tau] \Vdash \ x \ : \tau &\rightsquigarrow 
    \sigma \Vdash v : \tau \\
    % Lookup (sensitivity budget greater than 1)
    \sigma[x \mapsto v :_{s} \tau] \Vdash \ x \ : \tau &\rightsquigarrow
    \sigma[x \mapsto v :_{s-1} \tau] \Vdash v : \tau \quad{(\text{with } 1 < s)}
    \\
    %%%%%%%%%%%%%%%%%%%%%%%%%%%%%%%%%%%%%%%%%%%%%%%%%%%%%%%%%%%%%%%%%%%%%%%%%%%
    % Max: there are two ways to write these rounding rules. Curently, the
    % explicit style is favored where rnd and ret can be evaluated at no more
    % than two enviroments simultaneously.
    %%% Ret case:
    % 1. Explicit handling of multiple enviroments:
    % \gamma_0 \Vdash \mathbf{ret} \ v : M_0 \ \tau &\rightsquigarrow
    % \gamma_0 \Vdash v; \gamma_0 \Vdash v : M_0 \ \tau \\
    % \gamma_0; \gamma_1 \Vdash \mathbf{ret} \ v : M_0 \ \tau &\rightsquigarrow
    % \gamma_0 \Vdash v; \gamma_1 \Vdash v : M_0 \ \tau \\
    % 2. Equivalent setup with implicit handling of multiple enviroments:
    % \gamma_0; ...; \gamma_n \Vdash \mathbf{ret} \ v : M_0 \ \tau
    % &\rightsquigarrow \gamma_0 \Vdash v; \gamma_n \Vdash v : M_0 \ \tau \\
    %%% Rounding case:
    % 1. Explicit handling of multiple enviroments:
    % \gamma_0 \Vdash \mathbf{rnd} \ v : M_q \ \mathbf{num} &\rightsquigarrow
    % \gamma_0 \Vdash v; \gamma_0 \Vdash \rho(v) : M_q \ \mathbf{num} \\
    % \gamma_0; \gamma_1 \Vdash \mathbf{rnd} \ v : M_q \ \mathbf{num}
    % &\rightsquigarrow \gamma_0 \Vdash v; \gamma_1 \Vdash \rho(v) : M_q \
    % \mathbf{num} \\
    % 2. Equivalent setup with implicit handling of multiple enviroments:
    % \gamma_0; ...; \gamma_n \Vdash \mathbf{rnd} \ v : M_q \ \mathbf{num} &\rightsquigarrow
    % \Vdash \gamma_0 \Vdash v; \gamma_n \Vdash \rho(v) : M_q \ \mathbf{num} \\
    \sigma \Vdash \mathbf{ret} \ v : M_0 \ \tau &\rightsquigarrow \sigma \Vdash
    \langle v, v \rangle : M_0 \ \tau \\
    \sigma \Vdash \mathbf{rnd} \ v : M_q \ \mathbf{num} &\rightsquigarrow \sigma
    \Vdash \langle v, \rho(v) \rangle : M_q \ \mathbf{num} \\
    %%%%%%%%%%%%%%%%%%%%%%%%%%%%%%%%%%%%%%%%%%%%%%%%%%%%%%%%%%%%%%%%%%%%%%%%%%%
    % lam app
    \sigma \Vdash (\lambda x : \tau' .e) \ v : \tau &\rightsquigarrow \sigma[x
    \mapsto v :_1 \tau'] \Vdash e : \tau \\
    %%%%%%%%%%%%%%%%%%%%%%%%%%%%%%%%%%%%%%%%%%%%%%%%%%%%%%%%%%%%%%%%%%%%%%%%%%%
    % more boring rules
    % op(v) rule
    \sigma \Vdash \mathbf{op}(v) : \tau &\rightsquigarrow \sigma \Vdash op(v) :
    \tau \\
    % proj rule
    \sigma \Vdash \pi_i\langle v_1,v_2 \rangle : \tau &\rightsquigarrow \sigma
    \Vdash v_i : \tau \\ 
  \end{aligned}
\end{equation*}
    %%%%%%%%%%%%%%%%%%%%%%%%%%%%%%%%%%%%%%%%%%%%%%%%%%%%%%%%%%%%%%%%%%%%%%%%%%%
    % Structural rule
    %%%%%%%%%%%%%%%%%%%%%%%%%%%%%%%%%%%%%%%%%%%%%%%%%%%%%%%%%%%%%%%%%%%%%%%%%%%
    \vskip 0.3em
    \AXC{$\sigma_0 \Vdash e_0 \rightsquigarrow \sigma_0' \Vdash e_0'$}
    \AXC{$\sigma_1 \Vdash e_1 \rightsquigarrow \sigma_1' \Vdash e_1'$}
    \BIC{$\sigma_0 \Vdash e_0;\sigma_1 \Vdash e_1 \rightsquigarrow \sigma_0' \Vdash e_0';\sigma_1' \Vdash e_1'$}
    \DisplayProof
    %%%%%%%%%%%%%%%%%%%%%%%%%%%%%%%%%%%%%%%%%%%%%%%%%%%%%%%%%%%%%%%%%%%%%%%%%%%
    % even more spicy rules
    %%%%%%%%%%%%%%%%%%%%%%%%%%%%%%%%%%%%%%%%%%%%%%%%%%%%%%%%%%%%%%%%%%%%%%%%%%%
    % let-bind rule
    \vskip 0.3em
    \AXC{$\frac{1}{s} \cdot \sigma_1 \Vdash e : M_r \tau' \rightsquigarrow^*
    \sigma_1' \Vdash \langle v_0, v_1 \rangle : M_r \tau'$} 
    \UIC{ $\sigma_0 + \sigma_1 \Vdash \textbf{let-bind}_{(s, \tau')} \ x = e \
    \tin \ f  : M_{s*r + q} \ \tau \rightsquigarrow \mu (((\sigma_1' + \sigma_0
    [x \mapsto v_1 :_s \tau']); (\sigma_1' + \sigma_0 [x \mapsto v_0
    :_s \tau'])) \Vdash f : M_{s * r} (M_q \tau)) : M_{s * r + q} \tau$ }
    \DisplayProof
    % Note that to prove that the right hand side of the rewrite is
    % well-defined, we need to show that \sigma_1, \sigma_1', and \sigma_0 all
    % have the same shape / height

    % let-cobind rule
    % \vskip 0.3em
    % \AXC{$\frac{1}{t} \cdot \sigma_0 \Vdash e : \ !_s \tau' \rightsquigarrow^*
    % v^0;v^1;...;v^n: \ !_s \tau'$}
    % \UIC{ $\sigma_0 + \sigma_1 \Vdash \textbf{let-cobind}_{(s, t, \tau')} \ x =
    % e \ \tin \ f : \tau \rightsquigarrow \sigma_1[x \mapsto v^0 :_{t * s}
    % \tau']; \sigma_1^{-1}[x \mapsto v^1 :_{t * s} \tau']; ..; \sigma_1^{-1}[x
    % \mapsto v^n :_{t * s} \tau'] \Vdash f : \tau$}
    % \DisplayProof

    %%%%%%%%%%%%%%%%%%%%%%%%%%%%%%%%%%%%%%%%%%%%%%%%%%%%%%%%%%%%%%%%%%%%%%%%%%%
    % misc stepping rules
    %%%%%%%%%%%%%%%%%%%%%%%%%%%%%%%%%%%%%%%%%%%%%%%%%%%%%%%%%%%%%%%%%%%%%%%%%%%

    % let rule
    % \vskip 0.3em
    % \AXC{$\frac{1}{s} \cdot \sigma_0 \Vdash e : \tau' \rightsquigarrow^*
    % v^0;v^1;...; v^n : \tau'$}
    % \UIC{ $\sigma_0 + \sigma_1 \Vdash \textbf{let}_{(s, \tau')} \ x = e \ \tin \
    % f  : \tau \rightsquigarrow \sigma_1[x \mapsto v^0 :_s \tau'];
    % \sigma_1^{-1}[x \mapsto v^0 :_s \tau'];...; \sigma'[x \mapsto v^n :_s \tau']
    % \Vdash f : \tau$}
    % \DisplayProof

    % let-pair rule
    % \vskip 0.3em
    % \AXC{$\frac{1}{s} \cdot \sigma_0 \Vdash e : \tau' \rightsquigarrow^*
    % (v_0^0, v_1^0);(v_0^1, v_1^1);...; (v_0^n, v_1^n) : \tau_0 \tensor \tau_1$}
    % \UIC{ $\sigma_0 + \sigma_1 \Vdash \textbf{let-pair}_{(s, \tau_0, \tau_1)} \
    % (x, y) = e \ \tin \ f  : \tau \rightsquigarrow \linebreak \sigma_1[x \mapsto
    % v_0^0 :_s \tau_0, y \mapsto v_1^0 :_s \tau_1]; \sigma_1^{-1}[x \mapsto v_0^1
    % :_s \tau_0, y \mapsto v_1^1 :_s \tau_1]; ...; \sigma_1^{-1}[x \mapsto v_0^n
    % :_s \tau_0, y \mapsto v_1^n :_s \tau_1] \Vdash f : \tau$}
    % \DisplayProof

    %%%%%%%%%%%%%%%%%%%%%%%%%%%%%%%%%%%%%%%%%%%%%%%%%%%%%%%%%%%%%%%%%%%%%%%%%%%
    % sharing rule 
    % We need to compute the union because of programs like:
    % \langle let-bind x = rnd v in (fun y => x), 0 \rangle
    % We also need to be using de Bruijn indicies to refer to terms on the
    % enviorment stack. (Otherwise envs might pollute each other.)
    %%%%%%%%%%%%%%%%%%%%%%%%%%%%%%%%%%%%%%%%%%%%%%%%%%%%%%%%%%%%%%%%%%%%%%%%%%%
    \vskip 0.3em
    \AXC{$\sigma \Vdash e_0 : \tau_0 \rightsquigarrow^* \sigma_0 \Vdash v_0:
    \tau_0$}
    \AXC{$\sigma \Vdash e_1 : \tau_1 \rightsquigarrow^* \sigma_1 \Vdash v_1 :
    \tau_1$}
    \BIC{$\sigma \Vdash \langle e_0, e_1 \rangle \ : \tau_0 \times \tau_1
    \rightsquigarrow \sigma_0 + \sigma_1 \Vdash \langle v_0, v_1 \rangle:
    \tau_0 \times \tau_1$}
    \DisplayProof

    \vskip 0.3em
    \AXC{$\sigma \Vdash e_0 : \tau_0 \rightsquigarrow \sigma_0 \Vdash e_0': \tau_0$}
    \AXC{$\sigma' \Vdash e_1 : \tau_1 \rightsquigarrow \sigma_1 \Vdash e_1' : \tau_1$}
    \BIC{$\sigma + \sigma' \Vdash ( e_0, e_1 ) \ : \tau_0 \tensor \tau_1
    \rightsquigarrow \sigma_0 + \sigma_1 \Vdash (e_0', e_1') : \tau_0 \tensor
    \tau_1$}
    \DisplayProof

    % TODO: define evaluation contexts and get rid of this def
    \vskip 0.3em
    \AXC{$\sigma \Vdash e : M_q \tau_1 \times M_r \tau_2 \rightsquigarrow \sigma' \Vdash e' : M_q
    \tau_1 \times M_r \tau_2$}
    \UIC{$\sigma \Vdash \factor e \ : M_s (\tau_1 \times \tau_2)
    \rightsquigarrow \factor e' : M_s (\tau_1 \times \tau_2)$}
    \DisplayProof

\end{center}
    \caption{Enviroment-style evaluation rules for \Lang, ordered by precedence.
    Note that during type checking but prior to running the operational
    semanitcs, the sensitivity information (tracked with metavar $s$) and type
    of bound variables $\tau', \tau_2$, is preserved as annotations in the
    syntax, written $[e]_s$ and $\textbf{let-bind}_{(s, \tau')}$,
    $\textbf{let-cobind}_{(s, t, \tau')}$, and $\lambda x : \tau' . e $.
    Computing the correct split $\sigma + \sigma'$ of an enviroment can be
    performed via type inference.
    % Note that $\textbf{let}^*$ is syntactic sugar
    % for matching all let expressions and their corresponding annotations.
    }
    \label{fig:sub_eval_rules}
\end{figure}

\section{Denotational Semantics}
\begin{definition}[Type interpretation]
  A type $\tau$ is interpreted with $\llbracket - \rrbracket : \textit{type} \to
  \textbf{Met}$ in the same way as in the original NumFuzz system.
  % TODO: put actual definition here.
\end{definition}

\begin{definition}[Typing context interpretation]
  A typing context $\Gamma$ is interpreted with $\llbracket - \rrbracket :
  \textit{context} \to \textbf{Met}$ in the following way:
  \begin{equation}
  \begin{aligned}[c]
    \llbracket . \rrbracket &\triangleq . \\
    \llbracket \Gamma, x :_s \tau \rrbracket &\triangleq \llbracket \Gamma \rrbracket
      \times D_s \llbracket \tau \rrbracket
  \end{aligned}
  \end{equation}
\end{definition}

\begin{definition}[Machine configuration interpretation]
  A machine configuration $\sigma \Vdash e$ is interpreted with $\llbracket -
  \rrbracket_{ - } : \textit{config} \times \textit{type} \hookrightarrow
  \textit{point}$, a partial function, in the following way:
  \begin{equation}
  \begin{aligned}[c]
  %%%%%%%%%%%%%%%%%%%%%%%%%%%%%%%%%%%%%%%%%
  %%%    Base cases (value and leaf)    %%%
  %%%%%%%%%%%%%%%%%%%%%%%%%%%%%%%%%%%%%%%%%
  \llbracket \gamma \Vdash \langle \rangle \rrbracket_{\textbf{unit}}
    &\triangleq * \\
  \llbracket \gamma \Vdash k \rrbracket_{\textbf{num}} &\triangleq k \\
  %%%%%%%%%%%%%%%%%%%%%%%%%%%%%%%%%%%%%%%%%
  %%%                                   %%%
  %%%%%%%%%%%%%%%%%%%%%%%%%%%%%%%%%%%%%%%%%
  \rule{5em}{0.4pt}&\rule{30em}{0.4pt}\\
  \llbracket \gamma [x \mapsto v] \Vdash x \rrbracket_{\tau} &\triangleq
    \llbracket \gamma \Vdash v \rrbracket_{\tau} \\
  \llbracket \gamma \Vdash (v, w) \rrbracket_{\tau_0 \otimes \tau_1} &\triangleq
    (\llbracket \gamma \Vdash v \rrbracket_{\tau_0}, \llbracket \gamma \Vdash v
    \rrbracket_{\tau_1}) \\
  \llbracket \gamma \Vdash \lambda x . e \rrbracket_{\tau_0 \multimap \tau_1}
    &\triangleq \{ (\llbracket \sigma \Vdash v \rrbracket_{\tau_0}, \llbracket
    \gamma + \sigma \Vdash (\lambda x . e) v \rrbracket_{\tau_1}) \> | \>
    \forall \sigma \Vdash v \in \textit{config}, \llbracket \sigma \Vdash v
    \rrbracket_{\tau_0} \not= \bot \} \\
  \llbracket \gamma \Vdash v \rrbracket_{!_s \tau} &\triangleq \llbracket \gamma
    \Vdash v \rrbracket_\tau \\
  % todo: fill in more cases
  \rule{5em}{0.4pt}&\rule{30em}{0.4pt}\\
  %%%%%%%%%%%%%%%%%%%%%%%%%%%%%%%%%%%%%%%%%
  %%%            Tree cases             %%%
  %%%%%%%%%%%%%%%%%%%%%%%%%%%%%%%%%%%%%%%%%
  % todo: check this, maybe encode internal choice somehow?
  \llbracket \sigma; \sigma' \Vdash v \rrbracket_{\tau_0 \times \tau_1}
    &\triangleq (\llbracket \sigma \Vdash v \rrbracket_{\tau_0}, \llbracket
    \sigma' \Vdash v \rrbracket_{\tau_1}) \\
  \llbracket \sigma_0; \sigma_1 \Vdash \mathbf{in_i} \ x \rrbracket_{\tau}
    &\triangleq \llbracket \sigma_i \Vdash x \rrbracket_{\tau} \\
  \llbracket \sigma; \sigma' \Vdash v \rrbracket_{M_q \tau} &\triangleq
    (\llbracket \sigma \Vdash v \rrbracket_{\tau}, \llbracket \sigma' \Vdash v
    \rrbracket_{\tau}) \\
  \rule{5em}{0.4pt}&\rule{30em}{0.4pt}\\
  %%%%%%%%%%%%%%%%%%%%%%%%%%%%%%%%%%%%%%%%%
  %%%            Expr case              %%%
  %%%%%%%%%%%%%%%%%%%%%%%%%%%%%%%%%%%%%%%%%
  \llbracket \sigma \Vdash e \rrbracket_{\tau} &\triangleq \llbracket \sigma'
    \Vdash v\rrbracket_{\tau} \quad{\text{ if } \sigma \Vdash e
    \rightsquigarrow^* \sigma' \Vdash v}
  \end{aligned}
  \end{equation}
  This is a well-founded relation defined first on values and enviroment leafs,
  with $\tau$ decreasing or $\tau$ constant and $\gamma$ decreasing in size. It
  is then defined on enviroment trees and then on expressions that are not
  values, if it can be rewritten to a value.
\end{definition}

\begin{definition}[Environment compatibility]
  % Note: We can rewrite this defintion in a different style (without Coq-style
  % props) if that's easier to understand.
  Enviroment $\sigma$ is compatible with a typing context $\Gamma$ if and only
  if there exists a point $p \in \llbracket \Gamma \rrbracket$ such that:
  $$\sigma \textit{ \underline{com} } \Gamma \textit{ \underline{at} } p
  \textit{ \underline{in} } \llbracket \Gamma \rrbracket $$
  holds (read as $\sigma$ is compatible with $\Gamma$ at point $p$ in metric
  space $\llbracket \Gamma \rrbracket$), which has type:
  $$- \textit{ \underline{com} } - \textit{ \underline{at} } -
  \textit{ \underline{in} } - : \textit{env} \times
  \textit{ctx} \times \textit{point} \times \textbf{Met} \to \text{Prop}$$ 
  The definition is as follows:
  \begin{equation}
    \begin{aligned}[c] 
      % remove :s case
      \sigma \textit{ \underline{com} } \Gamma, x:_s \tau \textit{ \underline{at} } \alpha \textit{ \underline{in} } (d, Y) \times D_s (d_x, X) &\triangleq 
      \sigma \textit{ \underline{com} } \Gamma, x : \tau \textit{ \underline{at} } \alpha \textit{ \underline{in} } (d, Y) \times ((d_x, X)) \\
      % comonad case
      \sigma \textit{ \underline{com} } \Gamma, x: !_s \tau \textit{ \underline{at} } (\alpha, p) \textit{ \underline{in} } (d, Y) \times D_s ((d_x, X)) &\triangleq 
      \sigma \textit{ \underline{com} } \Gamma, x : \tau \textit{ \underline{at} } \alpha \textit{ \underline{in} } (d, Y) \times (d_x, X) \\
      % monad case
      \sigma; \sigma' \textit{ \underline{com} } \Gamma, x: M_r \tau \textit{ \underline{at} } (\alpha, (p, q)) \textit{ \underline{in} } (d, Y) \times T_r ((d_x, X)) &\triangleq 
      \sigma \textit{ \underline{com} } \Gamma, x : \tau \textit{ \underline{at} } (\alpha, p) \textit{ \underline{in} } (d, Y) \times (d_x, X) \\
      & \text{ and } \sigma' \textit{ \underline{com} } \Gamma, x : \tau \text{ \underline{at} } (\alpha, q) \textit{ \underline{in} } (d, Y) \times (d_x, X) \\
      & \text{ and } d_x (p, q) \leq r \\
      & \text{ and } p, q \in X \\
      % unit case
      \sigma[x \mapsto * : \tau] \textit{ \underline{com} } \Gamma, x: \textbf{unit} \textit{ \underline{at} } (\alpha, *) \textit{ \underline{in} } (d, X) \times (d_x, \{ * \}) &\triangleq 
      \sigma \textit{ \underline{com} } \Gamma \textit{ \underline{at} } \alpha \textit{ \underline{in} } (d, X)\\
      % function case
      \sigma[x \mapsto \lambda y . e] \textit{ \underline{com} } \Gamma, x :
      \tau_0 \multimap \tau_1 \textit{ \underline{at} } (\alpha, f) \textit{
        \underline{in} } (d, Y) \times (d_x, X) &\triangleq \sigma \textit{
        \underline{com} } \Gamma \textit{ \underline{at} } \alpha \textit{
        \underline{in} } (d, Y) \\
      & \text{ and } \llbracket \sigma \Vdash \lambda y . e \rrbracket_{\tau_0
      \multimap \tau_1} \subseteq f \\
      & \text{ and } f \in X \text{ and 1-sensitive} \\
      % var case
      \sigma[x \mapsto y] \textit{ \underline{com} } \Gamma, x :
      \tau \textit{ \underline{at} } \alpha \textit{
        \underline{in} } (d, Y) &\triangleq \sigma \textit{
        \underline{com} } \Gamma \textit{ \underline{at} } \alpha \textit{
        \underline{in} } (d, Y) \\
    \end{aligned}
  \end{equation}
\end{definition}


\begin{definition}[Semantic well-typedness]
  An expression $e$ is semantically well-typed in a context $\Gamma$ at type
  $\tau$, written:
  $$\Gamma \vDash e : \tau$$
  if and only if for all $\sigma$ compatible with $\Gamma$, $\llbracket \sigma
  \Vdash e \rrbracket_{\tau} \in \llbracket \tau \rrbracket$.
\end{definition}


\section{Type Soundness}
\begin{lemma}[Termination]\label{thm:termination}
  If $\Gamma \vDash e : \tau$, then $\forall \sigma$ compatible with $\Gamma$, 
  $$\sigma \Vdash e \rightsquigarrow^* \sigma' \Vdash v$$
\end{lemma}
\begin{proof}
  If $\sigma \Vdash e$ is a value, then we are done. Otherwise, if $\sigma
  \Vdash e$ is an expression, we begin by unfolding and applying the definition
  of $\vDash$. By definition, know that $\llbracket \sigma \Vdash e \rrbracket_{\tau} \in
  \llbracket \tau \rrbracket$. We also know that $\bot \not\in \llbracket \tau
  \rrbracket$, a metric space, so $\llbracket \sigma \Vdash e \rrbracket_{\tau}
  \not= \bot$ and is well-defined.

  By inspection of the definition of cases of $\llbracket - \rrbracket_{\tau}$,
  we know that only last case (which deals with all expressions) must apply.
  Therefore, $\sigma \Vdash e \rightsquigarrow^* \sigma' \Vdash v$.
\end{proof}

% \begin{theorem}[Semantic type preservation]
% If a term $e$ is semantically well-typed in context $\Gamma$ with type $\tau$
%   then for all $\sigma \Vdash e \rightsquigarrow \sigma' \Vdash e'$, $\sigma$
%   compatible with $\Gamma$, then there exists a $\Gamma'$ such that $\Gamma'
%   \vDash e' : \tau$.
% \end{theorem}
% \begin{proof}
% If $e$ is a value, we're done. 
% If $e$ is not a value, we proceed by induction over the cases of the rewrite relation
% $\rightsquigarrow$.
% % Our inductive hypothesis is that $e \rightsquigarrow e'$ and $\Gamma \vDash e : \tau$, that is, 
% % $\forall \sigma \text{ compatible with } \Gamma, \llbracket \sigma \Vdash e
% % \rrbracket_{\tau} \in \llbracket \tau \rrbracket$.
% \begin{description}
%   \item[\textsc{Variable lookup.}] If $$
%   \item[\textsc{Ret.}]
%   \item[\textsc{Rnd.}]
% \end{description}
% \end{proof}

% \begin{definition}[Monadic type]\label{def:monadic}
%   A type $\tau$ is monadic if and only if $\exists \tau', q \text{ such that }
%   \tau = M_q \tau'$. It is non-monadic otherwise.
% \end{definition}
% %%% wrong theorem:
% \begin{lemma}[Non-monadic lookup]\label{thm:non-monadic-lookup}
%   For all $\tau$ non-monadic and all $\Gamma, x : \tau'$ compatible with
%   $\sigma$, $\exists \sigma', \sigma = \sigma'[x \mapsto v : \tau']$ is
%   well-defined. Stated equivalently, non-monadic $x$ uniformly maps to $v$ at
%   all enviroments in $\sigma$.
% \end{lemma}
% \begin{proof}
%   TODO
% \end{proof}

\begin{lemma}[Enviroment tree shape]
  The tree height for a 
\end{lemma}

\begin{definition}[Orderings]\label{def:orderings}
  We define the following well-founded partial order over configurations and
  types, which mirrors the definition of machine configuration interpretation
  and will be used for induction in the proceeding proofs.

  First, we define an ordering over types by size of the AST.
  \begin{equation}
    \begin{aligned}[c]
      s(\textbf{unit}) &= 1 \\
      s(\textbf{num}) &= 1 \\
      s(!_s \tau) &= s(\tau) + 1 \\
      s(M_q \tau) &= s(\tau) + 1 \\
      s(\tau_0 \times \tau_1) &= s(\tau_0) + s(\tau_1) + 1 \\
      s(\tau_0 \otimes \tau_1) &= s(\tau_0) + s(\tau_1) + 1 \\
      s(\tau_0 \multimap \tau_1) &= s(\tau_0) + s(\tau_1) + 1 \\
    \end{aligned}
  \end{equation}
  and
  \begin{equation}
    \tau_0 < \tau_0 \iff s(\tau_0) < s(\tau_1)
  \end{equation}

  Secondly, we define an ordering over environment leafs $\gamma$ using the number of bound
  variables:
  \begin{equation}
    \begin{aligned}[c]
      s(.) &= 0 \\
      s(\gamma, x \mapsto v :_s \tau) &= s(\gamma) + 1 \\
    \end{aligned}
  \end{equation}

  and

  \begin{equation}
    \gamma_0 < \gamma_0 \iff s(\gamma_0) < s(\gamma_1)
  \end{equation}

  We can now define an ordering over enviorment trees using the maximum size of
  each enviorment leaf in the tree:
  \begin{equation}
    \begin{aligned}[c]
      s(\gamma; \gamma') &= max(s(\gamma), s(\gamma')) \\
      s(\sigma; \sigma') &= max(s(\sigma), s(\sigma'))
    \end{aligned}
  \end{equation}

  We are now ready to define an ordering over $\textit{config} \times
  \textit{type}$:
  \footnote{This is our termination measure for our machine configuration
  interpretation function.}
  \footnote{Note that the type of the configuration (if the interpretation of
  the configuration belongs to the interpretation of the type) restricts the
  height and shape of environment tree. So we do not need to reason about the
  height of our enviroment trees in our ordering.}

  \begin{equation}
    \sigma \Vdash e : \tau < \sigma' \Vdash e' : \tau' \iff
    s(\tau) < s(\tau') \lor (s(\tau) = s(\tau') \land s(\sigma) < s(\sigma'))
  \end{equation}

\end{definition}

\begin{lemma}[Well-founded partial ordering]
  $\leq$ is a well-founded partial order over $\textit{config} \times
  \textit{type}$.
\end{lemma}
\begin{proof}
  TODO
\end{proof}

\begin{theorem}[Semantic type soundness]
$\Gamma \vdash e : \tau \implies \Gamma \vDash e : \tau$
\end{theorem}
\begin{proof}
From our premise, we wish to show that:
  \begin{enumerate}
    \item $\forall \sigma$ compatible with $\Gamma$, $\llbracket \sigma \Vdash e
      \rrbracket_{\tau} \in \llbracket \tau \rrbracket$ and well-defined.
    \item $\llbracket - \Vdash e \rrbracket_{\tau}$ is a \text{1-sensitive} map
      for all inputs enviroments compatible with $\Gamma$.
  \end{enumerate}
We begin by induction over our typing derivation. 
  % In some of the cases, we will need to induct over the size of our enviroment. 
\begin{description}
  \item[\textsc{(ty. rule) Var.}] There is no premise containing a typing
    judgement to this rule. So we have no inductive hypothesis to rely on.
    (This is a base case.)
    % We proceed to induct over the type of our variable, $\tau$.
    % \begin{description}
    %   \item{\textbf{unit} and \textbf{num}.}
    %   \item{$\tau_0 \otimes \tau_1$, $\tau_0 + \tau_1$, and $\tau_0 \multimap
    %     \tau_1$.}
    %   \item{$!_s \tau$.}
    %   \item{$M_q \tau$}
    %   \item{$\tau_0 \times \tau_1$.}
    % \end{description}
    Instead, we induct over the size of our enviorment.
    \begin{description}
      \item[\textit{(env. size) 0.}] We observe that an enviorment size of zero
        implies that $\Gamma$ is empty. So we are trying to show that the
        following:
        $$x : \tau \vDash x : \tau$$
        implies:
        $$\llbracket \sigma \Vdash x \rrbracket_\tau \in \llbracket \tau
        \rrbracket$$
        for $\sigma$ compatible with typing context $x : \tau$ and is
        \text{1-sensitive}. We proceed to induct over the size of type $\tau$,
        namely $s(\tau)$. 
        \footnote{Note that when we perform inversion over our compatibility
        relation for several of the cases below, the case that $\sigma = x
        \mapsto y$, for $y$ a var, is impossible when the enviroment size is
        zero because $\llbracket \Vdash y \rrbracket = \bot$. This is why we are
        inducting over the enviroment size.}
        \begin{description}
          \item{\underline{(ty. case) \textbf{unit.}}} The only configuration compatible with this is
            $x \mapsto \langle \rangle \Vdash x$ and clearly 
            $$\llbracket x \mapsto \langle \rangle \Vdash x
            \rrbracket_{\textbf{unit}} = \llbracket \Vdash \langle \rangle
            \rrbracket_{\textbf{unit}} = * \in \{ * \} = \llbracket \textbf{unit}
            \rrbracket$$
            So, properties (1) and (2) hold by the reasoning above.
          \item{\underline{(ty. case) $\mathbf{num}$.}} 
            The only configurations compatible
            with this is $x \mapsto k \Vdash x$ for $k \in \mathbb{R}$ and
            clearly $$\llbracket x \mapsto k \Vdash x \rrbracket_{\textbf{num}}
            = \llbracket \Vdash k \rrbracket_{\textbf{num}} = k \in \mathbb{R} =
            \llbracket \textbf{num} \rrbracket$$ implies that $\llbracket -
            \Vdash x \rrbracket_{\mathbf{num}}$ for this case is the
            $\text{1-sensitive}$ identity function over the reals. Therefore,
            properties (1) and (2) both hold.
          \item{\underline{(ty. case) $\times$.}} By inversion on
            our compatibility relation, we know that we must have some $\sigma =
            (x \mapsto v_0; x \mapsto v_1)$ for $\llbracket \Vdash v_0
            \rrbracket_{\tau_0} \in \llbracket \tau_0 \rrbracket$ and
            $\llbracket \Vdash v_1 \rrbracket_{\tau_1} \in \llbracket \tau_1
            \rrbracket$. By definition, $\llbracket x \mapsto v_0; x \mapsto v_1
            \Vdash x \rrbracket_{\tau_0 \times \tau_1} = (\llbracket x \mapsto
            v_0 \Vdash x \rrbracket_{\tau_0}, \llbracket x \mapsto v_1 \Vdash x
            \rrbracket_{\tau_1}) = (\llbracket \Vdash v_0 \rrbracket_{\tau_0},
            \llbracket \Vdash v_1 \rrbracket_{\tau_1})$ which is clearly
            \text{1-sensitive} in the interpretation of our type. So properties
            (1) and (2) hold.
          \item{\underline{(ty. case) $\otimes$.}} By inversion on our
            compatibility relation, we know that we must have some $\sigma = (x
            \mapsto (v_0, v_1))$ for $\llbracket \Vdash v_0 \rrbracket_{\tau_0}
            \in \llbracket \tau_0 \rrbracket$ and $\llbracket \Vdash v_1
            \rrbracket_{\tau_1} \in \llbracket \tau_1 \rrbracket$. By
            definition, $\llbracket x \mapsto (v_0, v_1) \Vdash x
            \rrbracket_{\tau_0 \times \tau_1} = (\llbracket x \mapsto v_0 \Vdash
            x \rrbracket_{\tau_0}, \llbracket x \mapsto v_1 \Vdash x
            \rrbracket_{\tau_1}) = (\llbracket \Vdash v_0 \rrbracket_{\tau_0},
            \llbracket \Vdash v_1 \rrbracket_{\tau_1})$ which is clearly
            \text{1-sensitive} in the interpretation of our type. So properties
            (1) and (2) hold.
          \item{\underline{(ty. case) $+$.}} By inversion on our compatibility
            relation, we know that we have $\sigma = x \mapsto v$ for
            $\llbracket \Vdash v \rrbracket_{\tau_0 + \tau_1} \in \llbracket
            \tau_0 + \tau_1 \rrbracket = \llbracket \tau_0 \rrbracket \uplus
            \llbracket \tau_1 \rrbracket$. 
            There are two cases: 
            (1) $\llbracket \Vdash v \rrbracket_{\tau_0} \in \llbracket \tau_0
            \rrbracket$ and therefore $\llbracket x \mapsto v \Vdash x
            \rrbracket_{\tau_0 + \tau_1} = (0, \llbracket x \mapsto v \Vdash x
            \rrbracket_{\tau_0}) = (0, \llbracket \Vdash v
            \rrbracket_{\tau_0})$; 
            or, (2) $\llbracket \Vdash v \rrbracket_{\tau_1} \in \llbracket
            \tau_1 \rrbracket$ and therefore $\llbracket x \mapsto v \Vdash x
            \rrbracket_{\tau_0 + \tau_1} = (1, \llbracket x \mapsto v \Vdash x
            \rrbracket_{\tau_1}) = (1, \llbracket \Vdash v
            \rrbracket_{\tau_1})$. In both cases property 1 clearly holds.
            Property 2 holds by a similar case analysis.
          \item{\underline{(ty. case) $\tau_0 \multimap \tau_1$.}} 
            This case holds by the conditions on our compatibility relation and
            unfolding the interpretation of a machine configuration.
          \item{\underline{(ty. case) $!_s \tau$.}} 
            We begin by unfolding the definition of our compatibility relation
            and applying our machine configuration interpretation definition:
            $$ 
            \llbracket x \mapsto v :_s \tau \Vdash x \rrbracket_{!_s \tau} = 
            \llbracket x \mapsto v : \tau \Vdash x \rrbracket_{\tau}
            $$
            and then we can apply our inductive hypothesis on $\tau$ to prove
            properties (1) and (2).
          \item{\underline{(ty. case) $M_q \tau$.}}
            By our compatibility relation, we know that: $\sigma = (x \mapsto
            v_0);(x \mapsto v_1)$ for $d_{\tau}(\llbracket \Vdash v_0
            \rrbracket_{\tau}, \llbracket \Vdash v_1 \rrbracket_{\tau}) \leq q$.
            Unfolding our machine configuration interpretation definition, we
            obtain:
            \begin{equation*}
              \begin{aligned}
                \llbracket x \mapsto v_0; x \mapsto v_1 \Vdash x \rrbracket_{M_q
                \tau} &= (\llbracket x \mapsto v_0 \Vdash x \rrbracket_{\tau};
                \llbracket x \mapsto v_1 \Vdash x \rrbracket_{\tau}) \\
                &= (\llbracket \Vdash v_0 \rrbracket_{\tau}, \llbracket \Vdash
                v_1 \rrbracket_{\tau}) \in \llbracket M_q \tau \rrbracket
              \end{aligned}
            \end{equation*}
            which proves property 1. Since distance on the neighborhood monad is
            measured in terms of the first (ideal) component, property 2 also
            holds.
        \end{description}
      \item[\textit{(env. size) n + 1.}] We again induct over our type. 
        Note that in contrast to the base case, when we perform inversion on the
        compatibility relation for the cases below, $\sigma = \sigma'[x \mapsto
        y]$, for $y$ a var, is a possible case. 
        So when we induct over our type, for each type, we have that either:
        \begin{description}
          \item{\underline{\textit{$\sigma = \sigma'[x \mapsto y]$ for $y$ a
            variable.}}} By unfolding the definition of our machine config
            interpretation and inversion over our compatibility relation, we
            know that $\llbracket \sigma \Vdash x \rrbracket_{\tau} = \llbracket
            \sigma' \Vdash y \rrbracket_{\tau} \in \llbracket \tau \rrbracket$
            and is \text{1-sensitive} by our inductive hypothesis (over env.
            size).
          \item{\underline{\textit{Otherwise.}}} Proof for this case mirrors the
            proof for each of the base cases.
        \end{description}
    \end{description}
  \item[\textsc{(ty. rule) Ret.}] 
    From our inductive hypothesis, we have that $\Gamma \vDash e : \tau$ and
    wish to prove that $\Gamma \vDash \mathbf{ret} \ e : M_0 \tau$. Stepping our
    rewrite relation, we have that 
    \begin{equation}
      \begin{aligned}
        \llbracket (\sigma; \sigma)[\alpha \mapsto v:_1 \tau] \Vdash \alpha
        \rrbracket_{M_0 \tau} 
        &= \\
        (\llbracket \sigma[\alpha \mapsto v:_1 \tau] \Vdash \alpha
        \rrbracket_{\tau}, \llbracket \sigma[\alpha
        \mapsto v:_1 \tau] \Vdash \rrbracket_{\tau})
        &\in 
        \llbracket M_0 \tau \rrbracket
      \end{aligned}
    \end{equation}
    by our inductive hypothesis and is clearly \text{1-sensitive} in the metric
    space denoted by its type.
  \item[\textsc{(ty. rule) Rnd.}]
    From our inductive hypothesis, we have that $\Gamma \vDash e : \mathbf{num}$ and
    wish to prove that $\Gamma \vDash \mathbf{rnd} \ e : M_q \mathbf{num}$. Stepping our
    rewrite relation, we can show that:
    \begin{equation}
      \begin{aligned}
        \llbracket (\sigma[\alpha \mapsto v:_1 \mathbf{num}]; \sigma[\alpha \mapsto
        \rho(v) :_1 \mathbf{num}]) \Vdash \alpha \rrbracket_{M_q \mathbf{num}} 
        &= \\
        (\llbracket \sigma[\alpha \mapsto v:_1 \mathbf{num}] \Vdash \alpha
        \rrbracket_{\mathbf{num}}, \llbracket \sigma[\alpha \mapsto \rho(v) :_1
        \mathbf{num}]) \Vdash \alpha \rrbracket_{\mathbf{num}})
        &\in \llbracket M_q \mathbf{num} \rrbracket
      \end{aligned}
    \end{equation}
    by relying on the fact that $d_{\mathbf{num}}(v, \rho(v)) \leq q$. It is
    also clearly \text{1-sensitive} in the metric space denoted by its type as
    distances between monadic values are measured as distances between their
    first (ideal) components.
  \item[\textsc{(ty. rule) $\multimap I$.}] We are given in our inductive
    hypothesis that $\Gamma, x :_1 \vDash e : \tau$ for all types $\tau$. That
    means for all enviroments $\sigma'$ compatible with $\Gamma, x :_1 \tau_0$,
    we have that: $$\llbracket \sigma' \Vdash e \rrbracket_{\tau} \in \tau$$ and
    wish to show that: $\Gamma \Vdash \lambda x . e : \tau_0 \multimap \tau$.
    Unfolding what we wish to show, it suffices to prove that for all $\sigma$
    compatible with $\Gamma$, $\llbracket \sigma \Vdash \lambda x . e
    \rrbracket_{\tau_0 \multimap \tau_1} \in \llbracket \tau_0 \multimap \tau_1
    \rrbracket$ and \text{1-sensitive}.
    Let $\sigma$ be an arbitrary enviroment compatible with $\Gamma$. We proceed
    by inducting over the size of type $\tau_0$, namely $s(\tau_0)$.
    \begin{description}
      \item{\underline{(ty. case) $\textbf{unit}$.}} TODO.
      \item{\underline{(ty. case) $\textbf{num}$.}} TODO.
      \item{\underline{(ty. case) $\tau_0 \times \tau_1$.}} TODO. 
      \item{\underline{(ty. case) $\tau_0 \otimes \tau_1$.}} TODO.
      \item{\underline{(ty. case) $\tau_0 + \tau_1$.}} TODO.
      \item{\underline{(ty. case) $\tau_0 \multimap \tau_1$.}} TODO. 
      \item{\underline{(ty. case) $!_s \tau$.}} TODO.
      \item{\underline{(ty. case) $M_u \tau$.}} TODO.
    \end{description}
    TODO
  \item[\textsc{(ty. rule) Unit.}] Holds trivially.
  \item[\textsc{(ty. rule) Const.}] Holds trivially.
  \item[\textsc{(ty. rule) Subsumption.}] Holds trivially.
\end{description}
\end{proof}

\subsection{Modeling fidelity of real-world numerical programs}
Our operational and denotational semantics model both our ideal and approximate
numerical computations simultaneously. This enables an explicit encoding of the
computational content of $\textbf{factor}$ (which can be viewed as rearranging
various mixed ideal/approximate intermediate computations in a type-sound
fashion) and multiplication in our monad (which can be viewed as "forgetting"
mixed intermediate computations through the type-sound application of triangle
inequality).

However, a semantics that unifies both ideal and approximate computations makes
it less obvious that our type soundness theorem is actually useful in bounding
round-off error: the difference between what the user wanted (a program over the
reals with no round-off error) and what the user got (a program over the
floats).

In this section, we allay such concerns by annotating the operational semantics,
now colored with a \textcolor{red}{red} rewrite relation
\textcolor{red}{$\rightsquigarrow$}. The annotated operational semantics allow
us to cleanly track and separate the ideal and approximate computations during
stepping:
\begin{enumerate}
  \item For portions of the configuration that correspond to our \textit{ideal}
    computation, without round-off error, we \underline{underline} the
    computation. For example, our rounding stepping rule would look like (with
    only our ideal annotations):
    \begin{equation*}
      \sigma \Vdash \underline{\mathbf{rnd} \ v} : M_q \ \mathbf{num} \ \textcolor{red}{\rightsquigarrow} \
      \underline{\sigma [\alpha \mapsto v :_1 \mathbf{num}]}; \sigma [\alpha \mapsto \rho(v) :_1
      \mathbf{num}] \Vdash \underline{\alpha} : M_q \ \mathbf{num}
    \end{equation*}
  \item For portions of the configuration that correspond to our
    \textit{approximate} computation, with round-off error, we
    $\overline{\text{overline}}$ the computation. For example, our rounding
    stepping rule would look like (with only our approximate annotations):
    \begin{equation*}
      \sigma \Vdash \overline{\mathbf{rnd} \ v} : M_q \ \mathbf{num} \ \textcolor{red}{\rightsquigarrow} \ \sigma
      [\alpha \mapsto v :_1 \mathbf{num}]; \overline{\sigma [\alpha \mapsto \rho(v) :_1
      \mathbf{num}]} \Vdash \overline{\alpha} : M_q \ \mathbf{num}
    \end{equation*}
\end{enumerate}
For the fully annotated rounding stepping rule, see
Equation~\ref{eq:rnd-annotated}.
A user may choose to selectively consider the ideal semantics by \textit{only}
caring about the underlined portions of the operational semantics rewrite rules,
or, a user may selectively implement the approximate floating-point semantics by
\textit{only} computing the overlined portions. Our unified operational
semantics allows for both perspectives to co-exist.
Importantly, each perspective is self-contained: to consider the ideal
semantics, the user can completely ignore the approximate semantics (rounding)
and vice-versa.
With annotations, we prove our error soundness theorem
(Theorem~\ref{thm:error-soundness}), which states that all semantically
well-typed $e : M_q \ \mathbf{num}$ reduce to a pair of values, with the first
component side fully ideal (underlined; what the user wanted) and the second component fully
approximate (overlined; what the user got and can compute in-hardware), with the distance
between computations --- our round-off error --- bounded by $q$.

\subsubsection{Annotated operational semantics}

Note that different nodes in the enviroment tree may have mixed annotations,
corresponding to when the ideal and approximate computations are mixed. For
example, the fully annotated rounding stepping rule looks like:
\begin{equation}\label{eq:rnd-annotated}
  \sigma \Vdash \underline{\overline{\mathbf{rnd} \ v}} : M_q \ \mathbf{num} \ \textcolor{red}{\rightsquigarrow} \
  \underline{\sigma [\alpha \mapsto v :_1 \mathbf{num}]}; \overline{\sigma [\alpha \mapsto \rho(v) :_1
  \mathbf{num}]} \Vdash \underline{\overline{\alpha}} : M_q \ \mathbf{num}
\end{equation}
where $\sigma$ itself contain under and overlined annotations that are
propagated through. In some cases, expressions will be simultaneously
$\underline{\overline{\text{under-over-lined}}}$. 

\subsubsection{Annotated operational semantics}

\subsubsection{Relations over configurations and environment trees}
% TODO: fix off-by-one issue with grammars
To help prove error soundness, we define a unary relation over well-annotated
configurations and prove that it is preserved under stepping. The following
relation enforces two properties that are useful to maintain under stepping:
\begin{definition}[Well-annotated configuration]
  A configuration is well-annotated if and only if the following two properties
  hold:
  \begin{enumerate}
    \item On the left-hand side of $\textcolor{red}{\rightsquigarrow}$, annotated
      enviroment trees belong to the following grammar:
      \begin{alignat*}{3}
            &\sigma, \sigma_0, \sigma_1 &::=~ & \gamma
            \ \mid \ \underline{\sigma_0}; \overline{\sigma_1}
            & \ \mid \ \overline{\underline{\sigma_0}}; \overline{\underline{\sigma_1}}
      \end{alignat*}
    \item On the right-hand side of $\textcolor{red}{\rightsquigarrow}$, 
      expressions are always accessible from both the ideal and approximate
      perspectives. That is, $\sigma \Vdash \underline{\overline{e}}$.
  \end{enumerate}
\end{definition}

\begin{definition}[Fully ideal configuration]
  A configuration is fully ideal if and only if it has the form
  $\sigma_0 \Vdash \underline{e}$ where $\sigma$ belongs to the following grammar:
  \footnote{Note that e may also be overlined.}
      \begin{alignat*}{3}
        &\sigma_0, \sigma_1 &::=~ & \underline{\gamma}
            \ \mid \ \underline{\overline{\gamma}} \ \mid \ \sigma_0; \sigma_1
      \end{alignat*}

\end{definition}

\begin{definition}[Fully approximate configuration]
  Similarly, a configuration is fully approximate if and only if it has the form
  $\sigma_0 \Vdash \overline{e}$ where $\sigma$ belongs to the following
  grammar:
  \footnote{Note that e may also be underlined.}
      \begin{alignat*}{3}
        &\sigma_0, \sigma_1 &::=~ & \overline{\gamma}
            \ \mid \ \underline{\overline{\gamma}} \ \mid \ \sigma_0; \sigma_1
      \end{alignat*}
\end{definition}

\subsubsection{Error soundness}
\begin{lemma}[Well-annotation relation preservation]\label{thm:expressions-preserve-annotations} 
  For all $\sigma, \sigma', e, e'$, if $\sigma \Vdash e \rightsquigarrow \sigma'
  \Vdash e'$, then if $\sigma \Vdash \underline{\overline{e}}$ is well-annotated
  and $\sigma \Vdash \underline{\overline{e}} \
  \textcolor{red}{\rightsquigarrow} \ \sigma' \Vdash \underline{\overline{e'}}$,
  then $\sigma' \Vdash \underline{\overline{e'}}$ is well-annotated.
\end{lemma}
\begin{proof}
  By inspection of the annotated rewrite ($\textcolor{red}{\rightsquigarrow}$) relation.
\end{proof}

With annotations, we can now state the following theorem:
\begin{theorem}[Error soundness]\label{thm:error-soundness}
  For all semantically well-typed expressions $e : M_q \ \mathbf{num}$ in the
  empty context, $. \Vdash \underline{\overline{e}} \
  \textcolor{red}{\rightsquigarrow^*} \ \underline{\gamma}; \overline{\gamma'}
  \Vdash \underline{\overline{v}}$ where:
  $$
  \llbracket . \Vdash e \rrbracket_{M_q \ \mathbf{num}} 
  = \llbracket \underline{\gamma}; \overline{\gamma'} \Vdash
  \underline{\overline{v}} \rrbracket_{M_q \ \mathbf{num}} 
  = (\llbracket \underline{\gamma \Vdash v} \rrbracket_{\mathbf{num}},
  \llbracket \overline{\gamma' \Vdash v} \rrbracket_{\mathbf{num}}) 
  = (k_0, k_1)
  $$
  for $d_{\mathbf{num}}(k_0, k_1) \leq q$.
\end{theorem}
\begin{proof}
  Since $. \vDash e : M_q \ \mathbf{num}$, we know that $\llbracket . \Vdash e
  \rrbracket_{M_q \ \mathbf{num}} \in \llbracket M_q \ \mathbf{num}
  \rrbracket$. 
  By Lemma~\ref{thm:termination} (termination) and preservation of
  well-annotated configurations, we have that 
  $. \Vdash \underline{\overline{e}} \rightsquigarrow^{*} \sigma \Vdash
  \underline{\overline{v}}$ for some $\sigma$.
  It remains to be shown that $\sigma = \gamma; \gamma'$ where $\llbracket \underline{\gamma \Vdash v}
  \rrbracket_{\mathbf{num}} = k_0$ and $\llbracket \overline{\gamma' \Vdash v}
  \rrbracket_{\mathbf{num}} = k_1$ and $d_{\mathbf{num}}(k_0, k_1) \leq q$

  \begin{equation}
    \begin{aligned}
      \llbracket . \Vdash \underline{\overline{e}} \rrbracket_{M_q \ \mathbf{num}} 
        &=
      \llbracket \underline{\gamma}; \overline{\gamma'} \Vdash \underline{\overline{v}} \rrbracket_{M_q \ \mathbf{num}} \\
      &=
      (\llbracket \underline{\gamma} \Vdash \underline{\overline{v}} \rrbracket_{\mathbf{num}}, \llbracket \overline{\gamma'} \Vdash \underline{\overline{v}} \rrbracket_{\mathbf{num}}) \\
    \end{aligned}
  \end{equation}
\end{proof}

%%%% IGNORE BELOW %%%%

% % todo: Use whatever theorem name people like best here.
% Semantics is preserved under operational stepping. If $\sigma \Vdash e : \tau \rightsquigarrow \sigma'
% \Vdash e' : \tau$, then $\llbracket \sigma \Vdash e : \tau \rrbracket =
% \llbracket \sigma' \Vdash e' : \tau \rrbracket$.
%
% % todo: Use whatever theorem name people like best here.
% If the semantics of a program is equivalent to the semantics of a value, it must
% reduce to that value. If $\llbracket \sigma \Vdash e : \tau \rrbracket =
% \llbracket v : \tau \rrbracket$, then $\sigma \Vdash e : \tau
% \rightsquigarrow^{*} \sigma' \Vdash v : \tau$
%
% Syntactically well-typed programs are non-expansive. For $\Gamma \vdash e : \tau
% $ and $\llbracket \sigma \rrbracket, \llbracket \sigma' \rrbracket \in
% \llbracket \Gamma \rrbracket $ and
% $$\sigma \Vdash e : \tau \rightsquigarrow^* v : \tau$$ 
% and 
% $$\sigma' \Vdash e : \tau \rightsquigarrow^* v' : \tau$$
% then
% $$
% d_{\tau}(v, v') \leq d_{\Gamma}(\sigma, \sigma')
% $$
%
% Syntactically ok implies semantically ok. If $\Gamma \vdash e : \tau$, then
% $\forall \llbracket \sigma \rrbracket \in \llbracket \Gamma \rrbracket, \exists
% v, \llbracket \sigma \Vdash e : \tau \rrbracket = \llbracket v : \tau
% \rrbracket$.
%
% Syntactically ok implies operationally ok. If $\Gamma \vdash e : \tau$, then
% $\forall \llbracket \sigma \rrbracket \in \llbracket \Gamma \rrbracket, \exists
% v, \sigma \Vdash e : \tau \rightsquigarrow^{*} v : \tau$.


\section{Implementation, language instantiation, and type inference}

Conceptually, we can think of our type inference and checking algorithm in two
stages. Users write programs in \Lang. We then perform the type sensitivity and
inference algorithm developed in \cite{NumFuzz}, extended to our core language
without bound polymorphism. Finally, we automatically infer polymorphic bound
annotations to generate a valid typing derivation of the program in \bnd{\Lang}.

\begin{equation}
  \begin{aligned}[c]
    add &: \bnd{\forall \epsilon_0, \epsilon_1 \ .} \
    \textbf{num}_{\bnd{\epsilon_0}} \times \textbf{num}_{\bnd{\epsilon_1}}
    \multimap \textbf{num}_{\bnd{\epsilon_0 + \epsilon_1}} \\
    sub &: \bnd{\forall \epsilon_0, \epsilon_1 \ .} \
    \textbf{num}_{\bnd{\epsilon_0}} \times \textbf{num}_{\bnd{\epsilon_1}}
    \multimap \textbf{num}_{\bnd{\epsilon_0 - \epsilon_1}} \\
    mult &: \bnd{\forall \epsilon_0, \epsilon_1 \ .} \
    \textbf{num}_{\bnd{\epsilon_0}} \times \textbf{num}_{\bnd{\epsilon_1}}
    \multimap \textbf{num}_{\bnd{\epsilon_0 \cdot \epsilon_1}} \\
  \end{aligned}
\end{equation}

\subsection{Type inference}
\begin{example}[Unannotated program]
  \begin{equation}
    \lambda x \ . \ \mathbf{let} \ s \ = \ add \ x \ \tin \ (\mathbf{rnd} \ s)
    : \textbf{num} \times \textbf{num} \multimap \textbf{num}
  \end{equation}
\end{example}
\begin{example}[Annotated program]
  \begin{equation}
    \bnd{\Lambda \epsilon_0, \epsilon_1 \ .} \lambda x : num_{\bnd{\epsilon_0}}
    \times num_{\bnd{\epsilon_1}} \ . \ 
    \mathbf{let} \ s \ = \ add \ \bnd{\{\epsilon_0\}} \ \{\bnd{\epsilon_1\}} \ x
    \ \tin \ (\mathbf{rnd} \ s)
    : 
    \bnd{\forall \epsilon_0, \epsilon_1 \ .} \ 
    \textbf{num}_{\bnd{\epsilon_0}} \times \textbf{num}_{\bnd{\epsilon_1}}
    \multimap
    \textbf{num}_{\bnd{\epsilon_0 + \epsilon_1}}
  \end{equation}
\end{example}


%%
%% The acknowledgments section is defined using the "acks" environment
%% (and NOT an unnumbered section). This ensures the proper
%% identification of the section in the article metadata, and the
%% consistent spelling of the heading.
%\begin{acks}
%  \input{XX-acknowledgments.tex}
%\end{acks}

%%
%% The next two lines define the bibliography style to be used, and
%% the bibliography file.
\bibliographystyle{ACM-Reference-Format}
\bibliography{citations}


%%
%% If your work has an appendix, this is the place to put it.
%\appendix
%\input{XX-appendix.tex}

\end{document}
\endinput
%%
%% End of file `sample-acmsmall.tex'.

