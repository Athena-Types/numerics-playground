\documentclass[acmsmall, review]{acmart}

%%
%% \BibTeX command to typeset BibTeX logo in the docs
\AtBeginDocument{%
  \providecommand\BibTeX{{%
    Bib\TeX}}}

%% Rights management information.  This information is sent to you
%% when you complete the rights form.  These commands have SAMPLE
%% values in them; it is your responsibility as an author to replace
%% the commands and values with those provided to you when you
%% complete the rights form.
%\setcopyright{acmlicensed}
%\copyrightyear{2025}
%\acmYear{2025}
%\acmDOI{XXXXXXX.XXXXXXX}
%
%%%
%%% These commands are for a JOURNAL article.
%\acmJournal{JACM}
%\acmVolume{37}
%\acmNumber{4}
%\acmArticle{111}
%\acmMonth{8}

%%
%% Submission ID.
%% Use this when submitting an article to a sponsored event. You'll
%% receive a unique submission ID from the organizers
%% of the event, and this ID should be used as the parameter to this command.
%%\acmSubmissionID{123-A56-BU3}

%%
%% For managing citations, it is recommended to use bibliography
%% files in BibTeX format.
%%
%% You can then either use BibTeX with the ACM-Reference-Format style,
%% or BibLaTeX with the acmnumeric or acmauthoryear sytles, that include
%% support for advanced citation of software artefact from the
%% biblatex-software package, also separately available on CTAN.
%%
%% Look at the sample-*-biblatex.tex files for templates showcasing
%% the biblatex styles.
%%

%%
%% The majority of ACM publications use numbered citations and
%% references.  The command \citestyle{authoryear} switches to the
%% "author year" style.
%%
%% If you are preparing content for an event
%% sponsored by ACM SIGGRAPH, you must use the "author year" style of
%% citations and references.
%% Uncommenting
%% the next command will enable that style.
\citestyle{acmauthoryear}



%% 
% FIGURES
\usepackage{multirow}
\usepackage{xspace}
\usepackage{adjustbox}

%%
% IMAGE
\usepackage{graphicx}
\usepackage{caption}
\usepackage{subcaption}
\usepackage{lscape}

%%
% MATHS
\newtheorem*{remark}{Remark}
\usepackage{amsthm}
\usepackage{amsmath}
\usepackage{bussproofs} 
    \EnableBpAbbreviations
\usepackage{mathpartir} 
\usepackage{mathtools}
\usepackage{stmaryrd}
\newcommand{\R}{\mathbb{R}}
\newcommand{\RR}{\R}
\newcommand{\NNR}{\mathbb{R}^{\geq 0}}
\newcommand{\NN}{\mathbb{N}}
\newcommand{\F}{\mathbb{F}}

%% CONTSANTS
\newcommand{\rnderr}{\ensuremath{\epsilon}}

%% PROGRAMS
\newcommand{\rnd}{\textbf{rnd }}
\newcommand{\ret}{\textbf{ret }}
\newcommand{\letcobind}{\textbf{let-cobind }}
\newcommand{\letassign}{\textbf{let }}
\newcommand{\letdestruct}{\textbf{let-destruct }}
\newcommand{\tin}{\textbf{in}}
\newcommand{\tif}{\textbf{if }}
\newcommand{\tthen}{\textbf{then }}
\newcommand{\letbind}{\textbf{let-bind }}
\newcommand{\inl}{\textbf{inl }}
\newcommand{\inr}{\textbf{inr }}
\newcommand{\op}{\textbf{op}}

%%% CODE
\usepackage{listings}
\usepackage{lstlang}
\usepackage{xcolor}
\usepackage[T1]{fontenc}
\usepackage[scaled]{beramono}
\lstset{mathescape=true,language=fz}

%% TYPES
\newcommand{\unit}{\textbf{unit}}
\newcommand{\num}{\textbf{num}}
\newcommand{\tensor}{\otimes}
\newcommand{\tand}{\ \& \ }
\newcommand{\lin}{\multimap}
\newcommand{\bang}[1]{{!_{#1}}}
\newcommand{\Ra}{\textmd{R}}


\newcommand{\Met}{\mathbf{Met}}
\newcommand{\Set}{\mathbf{Set}}
\newcommand{\denot}[1]{\llbracket {#1} \rrbracket}
\newcommand{\pdenot}[1]{\llparenthesis {#1} \rrparenthesis}
\newcommand{\interpM}[1]{\mathcal{I}({#1})}

%%
%LANG
\newcommand{\Lang}{$\Lambda_\num$\xspace}

%%
% DIAGRAMS
\let\Bbbk\relax % clash with amssymb from acmart.
\usepackage{quiver}

%% 
% CLEVEREF
\usepackage{cleveref}

%%
%% end of the preamble, start of the body of the document source.
\begin{document}

%%
%% The "title" command has an optional parameter,
%% allowing the author to define a "short title" to be used in page headers.
\title{Extensions to Numerical Fuzz}

%%
%% The "author" command and its associated commands are used to define
%% the authors and their affiliations.
%% Of note is the shared affiliation of the first two authors, and the
%% "authornote" and "authornotemark" commands
%% used to denote shared contribution to the research.
\author{Max Fan}
\email{mxf@cs.cornell.edu}
\orcid{0009-0001-3664-6538}
\affiliation{%
  \institution{Cornell University}
  \city{Ithaca}
  \state{New York}
  \country{USA}
}

\author{Justin Hsu}
\email{justin@cs.cornell.edu}
\affiliation{%
  \institution{Cornell University}
  \city{Ithaca}
  \state{New York}
  \country{USA}
}

%%
%% By default, the full list of authors will be used in the page
%% headers. Often, this list is too long, and will overlap
%% other information printed in the page headers. This command allows
%% the author to define a more concise list
%% of authors' names for this purpose.
\renewcommand{\shortauthors}{Fan and Hsu}

%%
%% The abstract is a short summary of the work to be presented in the
%% article.
\begin{abstract}
\end{abstract}

%%
%% The code below is generated by the tool at http://dl.acm.org/ccs.cfm.
%% Please copy and paste the code instead of the example below.
%%
\begin{CCSXML}
<ccs2012>
<concept>
<concept_id>10011007.10011006.10011008.10011009.10011012</concept_id>
<concept_desc>Software and its engineering~Functional languages</concept_desc>
<concept_significance>500</concept_significance>
</concept>
<concept>
<concept_id>10003752.10003790.10011740</concept_id>
<concept_desc>Theory of computation~Type theory</concept_desc>
<concept_significance>500</concept_significance>
</concept>
<concept>
<concept_id>10003752.10003790.10002990</concept_id>
<concept_desc>Theory of computation~Logic and verification</concept_desc>
<concept_significance>500</concept_significance>
</concept>
<concept>
<concept_id>10002950.10003714.10003715.10003725</concept_id>
<concept_desc>Mathematics of computing~Interval arithmetic</concept_desc>
<concept_significance>200</concept_significance>
</concept>
</ccs2012>
\end{CCSXML}

\ccsdesc[500]{Software and its engineering~Functional languages}
\ccsdesc[500]{Theory of computation~Type theory}
\ccsdesc[500]{Theory of computation~Logic and verification}
\ccsdesc[200]{Mathematics of computing~Interval arithmetic}

%%
%% Keywords. The author(s) should pick words that accurately describe
%% the work being presented. Separate the keywords with commas.
\keywords{linear type systems, verification, round-off error, floating point}

%\received{20 February 2007}
%\received[revised]{12 March 2009}
%\received[accepted]{5 June 2009}

%%
%% This command processes the author and affiliation and title
%% information and builds the first part of the formatted document.
\maketitle

\section{Language Syntax}

\begin{figure}[tbp]
  \begin{alignat*}{3}
         &\text{Types } \sigma, \tau &::=~ &\mathbf{unit}
         \mid \num
         \mid \sigma \times \tau 
         \mid \sigma \otimes \tau
         \mid \sigma + \tau 
         \mid \sigma \multimap \tau
         \mid {\bang{s} \sigma}
         \mid {M_u \tau}
         \\
         &\text{Values } v, w \ &::=~ &\langle \rangle
         \mid k \in R
         \mid \langle v,w \rangle 
         \mid (v, w)
         \mid \tin_i \ v
         \mid \lambda x.~e \\
         % & & & \mid [v]
         % \mid\rnd v
         % \mid {\ret v} 
         % \mid \factor v
         % \mid \letbind x = v \ \tin \ f 
         \\
         &\text{Terms } e, f, g &::=~ &x
         \mid v
         \mid \mathbf{op}(e)
         \mid e~f
         \mid {\pi}_i\ e
         \mid \langle e,f \rangle 
         \mid (e, f) \\
         & & & \mid \letpair \ (x,y) = e \ \tin \ f
         \mid \letassign x  = e \ \tin \ f \\
         & & & \mid \tin_i \ e
         \mid 
          \mathbf{case} \ e \ \mathbf{of} \ (\tin_1 \ x.f \ | \ \tin_2 \ x.g) \\
         & & &
         \mid [e]
         \mid \rnd e
         \mid {\ret e} 
         \mid \factor e \\
         & & & 
         \mid {\letbind x = e \ \tin \ f}
         \mid \letcobind x = e \ \tin \ f
         \\
         % Max: find a better name
         &\text{Enviroments } \gamma_0, gamma_1 &::=~ &.
         \mid \gamma, \ x \mapsto v :_s \tau \\
         &\text{Enviroment trees } \sigma &::=~ & \gamma
         \mid \gamma_0; \gamma_1
  \end{alignat*}
  \caption{
    Types, values, and terms. 
    $\mathbf{op} \in \mathcal{O}$.
    $i \in \{1, 2\}$. 
  }
  \label{fig:syntax}
\end{figure}

% Max: For now, I've separated \letbind, \letcobind, \letpair, and
% \letassign to make things more clear.


\section{Dynamic Semantics}

\subsection{Operational Semantics}
\begin{figure}
\begin{center}

\begin{equation*}
\begin{aligned}[c]
	\mathbf{op}(v) &\mapsto op(v)\\
	\pi_i\langle v_1,v_2 \rangle &\mapsto v_i \\
	(\lambda x.e) \ v &\mapsto e[v/x] \\
	%\factor v \ &\mapsto v
\end{aligned}
\quad
\begin{aligned}[c]
	\letassign x = v \ \tin \ e &\mapsto e[v/x] \\
  \letpair (x, y) = (v, w) \ \tin \ e &\mapsto e[v/x][w/y] \\
	\letcobind x = [v] \ \tin \ e &\mapsto e[v/x]
	%\letbind x = \ret v \ \tin \ e &\mapsto e[v/x] \\
\end{aligned}
\end{equation*}
\vskip -1em
\begin{align*}
  \letbind y = (\letbind x = v \ \tin \ f) \ \tin \ g &\mapsto \letbind x = v \ \tin \ \letbind y = f \ \tin \ g \quad x\notin FV(g) 
\end{align*}
\vskip -1.75em
\begin{align*}
	\mathbf{case} \ (\mathbf{in}_i \ v) \ \mathbf{of} \ (\mathbf{in}_1 \ x.e_1 \ | \ \mathbf{in}_2 \ x.e_2 )  &\mapsto e_i[v/x]
  \qquad\qquad(i \in \{1, 2 \})
\end{align*}
\vskip -0.25em

	\AXC{$e \mapsto e'$}
	\UIC{$\letassign x = e \ \tin \ f \mapsto \letassign x = e' \ \tin f$}
	\DisplayProof

\end{center}
    \caption{Evaluation rules for \Lang. Note the side condition for $\letbind$always holds for closed expressions.}
    \label{fig:eval_rules}
\end{figure}

\section{Abstract Interpretation}

\subsection{Abstract Domain}

\section{Static Semantics}

\begin{figure}
%% ROW1
\begin{center}
%% var
\AXC{$s \ge 1$}
\RightLabel{(Var)}
\UIC{$\Gamma, x:_s \sigma, \Delta \vdash x : \sigma$}
\bottomAlignProof
\DisplayProof
\hskip 0.5em
%% fun
\AXC{$\Gamma, x:_1 \sigma \vdash e : \tau$}
\RightLabel{($\multimap$ I)}
\UnaryInfC{$\Gamma \vdash \lambda x. e : \sigma \multimap \tau $}
\bottomAlignProof
\DisplayProof
\hskip 0.5em
%% app
\AXC{$\Gamma \vdash e : \sigma \multimap \tau$}
\AXC{$\Theta \vdash f : \sigma $}
\RightLabel{($\multimap$ E)}
\BinaryInfC{$\Gamma + \Theta \vdash ef : \tau $}
\bottomAlignProof
\DisplayProof
\vskip 1em
%%


%% ROW2
\AXC{}
\RightLabel{(Unit)}
\UIC{$\Gamma \vdash \langle \rangle : \mathbf{unit}$}
\bottomAlignProof
\DisplayProof
\hskip 0.5em
%% dep prod intro
\AXC{$\Gamma \vdash e : \sigma$}
\AXC{$\Gamma \vdash f : \tau$}
\RightLabel{($\times$ I)}
\BinaryInfC{$\Gamma \vdash \langle e, f \rangle: \sigma \times \tau $}
\bottomAlignProof
\DisplayProof
\hskip 0.5em
%% dep prod elim
\AXC{$\Gamma \vdash e : \tau_1 \times \tau_2$}
\RightLabel{($\times$ E)}
\UIC{$\Gamma \vdash {\pi}_i \ e : \tau_i$}
\bottomAlignProof
\DisplayProof
\vskip 1em
%%

%% ind prod intro
\AXC{$\Gamma \vdash e : \sigma $}
\AXC{$\Theta \vdash f : \tau$}
\RightLabel{($\tensor$ I)}
\BIC{$\Gamma + \Theta \vdash (e, f) : \sigma \tensor \tau$}
\bottomAlignProof
\DisplayProof
\hskip 0.5em
%% ind prod elim
\AXC{$\Gamma \vdash e : \sigma \tensor \tau$ }
\AXC{$\Theta,x:_s \sigma,y:_s\tau \vdash f: \rho $}
\RightLabel{($\tensor$ E)}
\BIC{$s * \Gamma + \Theta \vdash \letpair (x,y) \ = \ e \ \tin \ f : \rho $}
\bottomAlignProof
\DisplayProof
\vskip 1em
%%


%% ROW 4

%% ind sum intro
\AXC{$\Gamma \vdash e : \sigma$ }
\RightLabel{($+$ $\text{I}_1$)}
\UIC{$\Gamma \vdash \mathbf{in}_1 \ e : \sigma + \tau$}
\bottomAlignProof
\DisplayProof
\hskip 0.5em
%% ind sum intro
\AXC{$\Gamma \vdash e : \tau$ }
\RightLabel{($+$ $\text{I}_2$)}
\UIC{$\Gamma \vdash \mathbf{in}_2 \ e : \sigma + \tau$}
\bottomAlignProof
\DisplayProof
\hskip 0.5em
% box elim
\AXC{$\Gamma \vdash e : {!_s \sigma}$}
\AXC{$\Theta, x:_{t*s} \sigma \vdash f : \tau$}
\RightLabel{($!$ E)}
\BIC{$t * \Gamma + \Theta \vdash \letcobind x = e \ \tin \ f : \tau$}
\bottomAlignProof
\DisplayProof
\vskip 1em
%%


%% ROW 5

% sum elim
\AXC{$\Gamma \vdash e : \sigma+\tau$}
\AXC{$\Theta, x:_s \sigma \vdash f : \rho$ \qquad
$\Theta, x:_s \tau \vdash g: \rho$}
\RightLabel{($+$ E)}
\AXC{$s > 0$}
\TIC{$s * \Gamma + \Theta \vdash \mathbf{case} \ e \ \mathbf{of} \ (\mathbf{in}_1 x.f \ | \ \mathbf{in}_2 x.g) : \rho$}
\bottomAlignProof
\DisplayProof
\hskip 0.5em
% box intro
\AXC{$\Gamma \vdash e : \sigma$ }
\RightLabel{($!$ I)}
\UIC{$s * \Gamma \vdash [e] : {!_s \sigma}$}
\bottomAlignProof
\DisplayProof
\vskip 1em

%%% ROW 6

% let 
\AXC{$\Gamma \vdash e :  \tau$}
\AXC{$\Theta, x:_{s} \tau \vdash f : \sigma$}
\RightLabel{(Let)}
\BIC{$s * \Gamma + \Theta \vdash \letassign x = e \ \tin \ f : \sigma$}
\bottomAlignProof
\DisplayProof
\hskip 0.5em
%% const
\AXC{$k \in R$}
\RightLabel{(Const)}
\UIC{$\Gamma \vdash k : \num$}
\bottomAlignProof
\DisplayProof
\hskip 0.5em
\vskip 1em

%%% ROW 7

%% subsumption
\AXC{$\Gamma \vdash e :  M_q \tau$}
\AXC{$r \ge q$}
\RightLabel{(Subsumption)}
\BIC{$\Gamma \vdash e :  M_{r} \tau$}
\bottomAlignProof
\DisplayProof
\hskip 0.5em
%% return
\AXC{$\Gamma \vdash e : \tau$}
\RightLabel{(Ret)}
\UIC{$\Gamma \vdash \ret e : M_0 \tau$}
\bottomAlignProof
\DisplayProof
\hskip 0.5em
%% RND
\AXC{$\Gamma \vdash e : \num$}
\RightLabel{(Rnd)}
\UIC{$\Gamma \vdash \rnd \ e : M_q \ \num$}
\bottomAlignProof
\DisplayProof
\vskip 1em


%%% ROW 8


% let-bind
\AXC{$\Gamma \vdash e : M_r \sigma$}
\AXC{$\Theta, x:_{s} \sigma \vdash e : M_{q} \tau$}
\RightLabel{($M_u$ E)}
\BIC{$s * \Gamma + \Theta \vdash \letbind x = e \ \tin \ f : M_{s*r+q} \tau$}
\bottomAlignProof
\DisplayProof
\hskip 0.5em
% funs
\AXC{$\Gamma \vdash e : \sigma$}
\AXC{$\{ \mathbf{op} :\sigma \lin \num \} \in {\Sigma}$}
\RightLabel{(Op)}
\BIC{$\Gamma \vdash \mathbf{op}(e) : \num$}
\bottomAlignProof
\DisplayProof

\vskip 1em


%%% ROW 8


% factor
\AXC{$\Gamma \vdash e : (M_q \tau) \times (M_r \sigma)$}
%\AXC{$r + q \leq s$}
%\AXC{$s = max(r,q)$}
\RightLabel{(Factor)}
\UIC{$\Gamma \vdash \factor \ e : M_{max(q,r)} (\tau \times \sigma)$}
\bottomAlignProof
\DisplayProof

\end{center}
    \caption{Typing rules for \Lang, with $s,t,q,r \in \NNR \cup \{\infty\}$. }
    \label{fig:typing_rules}
\end{figure}


\section{Denotational Type Semantics}

\section{Type Soundness}

\section{Subnormals}

\section{Sharing Error}

\section{Evaluation}


%%
%% The acknowledgments section is defined using the "acks" environment
%% (and NOT an unnumbered section). This ensures the proper
%% identification of the section in the article metadata, and the
%% consistent spelling of the heading.
%\begin{acks}
%  \input{XX-acknowledgments.tex}
%\end{acks}

%%
%% The next two lines define the bibliography style to be used, and
%% the bibliography file.
\bibliographystyle{ACM-Reference-Format}
\bibliography{sample-base}


%%
%% If your work has an appendix, this is the place to put it.
%\appendix
%\input{XX-appendix.tex}

\end{document}
\endinput
%%
%% End of file `sample-acmsmall.tex'.

