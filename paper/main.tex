\documentclass[acmsmall, review]{acmart}

%%
%% \BibTeX command to typeset BibTeX logo in the docs
\AtBeginDocument{%
  \providecommand\BibTeX{{%
    Bib\TeX}}}

%% Rights management information.  This information is sent to you
%% when you complete the rights form.  These commands have SAMPLE
%% values in them; it is your responsibility as an author to replace
%% the commands and values with those provided to you when you
%% complete the rights form.
%\setcopyright{acmlicensed}
%\copyrightyear{2025}
%\acmYear{2025}
%\acmDOI{XXXXXXX.XXXXXXX}
%
%%%
%%% These commands are for a JOURNAL article.
%\acmJournal{JACM}
%\acmVolume{37}
%\acmNumber{4}
%\acmArticle{111}
%\acmMonth{8}

%%
%% Submission ID.
%% Use this when submitting an article to a sponsored event. You'll
%% receive a unique submission ID from the organizers
%% of the event, and this ID should be used as the parameter to this command.
%%\acmSubmissionID{123-A56-BU3}

%%
%% For managing citations, it is recommended to use bibliography
%% files in BibTeX format.
%%
%% You can then either use BibTeX with the ACM-Reference-Format style,
%% or BibLaTeX with the acmnumeric or acmauthoryear sytles, that include
%% support for advanced citation of software artefact from the
%% biblatex-software package, also separately available on CTAN.
%%
%% Look at the sample-*-biblatex.tex files for templates showcasing
%% the biblatex styles.
%%

%%
%% The majority of ACM publications use numbered citations and
%% references.  The command \citestyle{authoryear} switches to the
%% "author year" style.
%%
%% If you are preparing content for an event
%% sponsored by ACM SIGGRAPH, you must use the "author year" style of
%% citations and references.
%% Uncommenting
%% the next command will enable that style.
\citestyle{acmauthoryear}



%% 
% FIGURES
\usepackage{multirow}
\usepackage{xspace}
\usepackage{adjustbox}

%%
% IMAGE
\usepackage{graphicx}
\usepackage{caption}
\usepackage{subcaption}
\usepackage{lscape}

%%
% MATHS
\newtheorem*{remark}{Remark}
\usepackage{amsthm}
\usepackage{amsmath}
\usepackage{bussproofs} 
    \EnableBpAbbreviations
\usepackage{mathpartir} 
\usepackage{mathtools}
\usepackage{stmaryrd}
\newcommand{\R}{\mathbb{R}}
\newcommand{\RR}{\R}
\newcommand{\NNR}{\mathbb{R}^{\geq 0}}
\newcommand{\NN}{\mathbb{N}}
\newcommand{\F}{\mathbb{F}}

%% CONTSANTS
\newcommand{\rnderr}{\ensuremath{\epsilon}}

%% PROGRAMS
\newcommand{\rnd}{\textbf{rnd }}
\newcommand{\ret}{\textbf{ret }}
\newcommand{\letcobind}{\textbf{let-cobind }}
\newcommand{\letassign}{\textbf{let }}
\newcommand{\letpair}{\textbf{let-pair }}
\newcommand{\tin}{\textbf{in}}
\newcommand{\tif}{\textbf{if }}
\newcommand{\tthen}{\textbf{then }}
\newcommand{\letbind}{\textbf{let-bind }}
\newcommand{\inl}{\textbf{inl }}
\newcommand{\inr}{\textbf{inr }}
\newcommand{\factor}{\textbf{factor }}
\newcommand{\op}{\textbf{op}}
\newcommand{\concat}{~\textit{\footnotesize{++}}~}

%%% CODE
\usepackage{listings}
\usepackage{lstlang}
\usepackage{xcolor}
\usepackage[T1]{fontenc}
\usepackage[scaled]{beramono}
\lstset{mathescape=true,language=fz}
\usepackage{xcolor}

%% TYPES
\newcommand{\unit}{\textbf{unit}}
\newcommand{\num}{\textbf{num}}
\newcommand{\tensor}{\otimes}
\newcommand{\tand}{\ \& \ }
\newcommand{\lin}{\multimap}
\newcommand{\bang}[1]{{!_{#1}}}
\newcommand{\Ra}{\textmd{R}}


\newcommand{\Met}{\mathbf{Met}}
\newcommand{\Set}{\mathbf{Set}}
\newcommand{\denot}[1]{\llbracket {#1} \rrbracket}
\newcommand{\pdenot}[1]{\llparenthesis {#1} \rrparenthesis}
\newcommand{\interpM}[1]{\mathcal{I}({#1})}

%%
%LANG
\newcommand{\Lang}{$\Lambda_\num^-$\xspace}

%%
% DIAGRAMS
\let\Bbbk\relax % clash with amssymb from acmart.
\usepackage{quiver}

%% 
% CLEVEREF
\usepackage{cleveref}

%%
%% end of the preamble, start of the body of the document source.
\begin{document}

%%
%% The "title" command has an optional parameter,
%% allowing the author to define a "short title" to be used in page headers.
\title{Extensions to Numerical Fuzz}

%%
%% The "author" command and its associated commands are used to define
%% the authors and their affiliations.
%% Of note is the shared affiliation of the first two authors, and the
%% "authornote" and "authornotemark" commands
%% used to denote shared contribution to the research.
\author{Max Fan}
\email{mxf@cs.cornell.edu}
\orcid{0009-0001-3664-6538}
\affiliation{%
  \institution{Cornell University}
  \city{Ithaca}
  \state{New York}
  \country{USA}
}

\author{Justin Hsu}
\email{justin@cs.cornell.edu}
\affiliation{%
  \institution{Cornell University}
  \city{Ithaca}
  \state{New York}
  \country{USA}
}

%%
%% By default, the full list of authors will be used in the page
%% headers. Often, this list is too long, and will overlap
%% other information printed in the page headers. This command allows
%% the author to define a more concise list
%% of authors' names for this purpose.
\renewcommand{\shortauthors}{Fan and Hsu}

%%
%% The abstract is a short summary of the work to be presented in the
%% article.
\begin{abstract}
\end{abstract}

%%
%% The code below is generated by the tool at http://dl.acm.org/ccs.cfm.
%% Please copy and paste the code instead of the example below.
%%
\begin{CCSXML}
<ccs2012>
<concept>
<concept_id>10011007.10011006.10011008.10011009.10011012</concept_id>
<concept_desc>Software and its engineering~Functional languages</concept_desc>
<concept_significance>500</concept_significance>
</concept>
<concept>
<concept_id>10003752.10003790.10011740</concept_id>
<concept_desc>Theory of computation~Type theory</concept_desc>
<concept_significance>500</concept_significance>
</concept>
<concept>
<concept_id>10003752.10003790.10002990</concept_id>
<concept_desc>Theory of computation~Logic and verification</concept_desc>
<concept_significance>500</concept_significance>
</concept>
<concept>
<concept_id>10002950.10003714.10003715.10003725</concept_id>
<concept_desc>Mathematics of computing~Interval arithmetic</concept_desc>
<concept_significance>200</concept_significance>
</concept>
</ccs2012>
\end{CCSXML}

\ccsdesc[500]{Software and its engineering~Functional languages}
\ccsdesc[500]{Theory of computation~Type theory}
\ccsdesc[500]{Theory of computation~Logic and verification}
\ccsdesc[200]{Mathematics of computing~Interval arithmetic}

%%
%% Keywords. The author(s) should pick words that accurately describe
%% the work being presented. Separate the keywords with commas.
\keywords{linear type systems, verification, round-off error, floating point}

%\received{20 February 2007}
%\received[revised]{12 March 2009}
%\received[accepted]{5 June 2009}

%%
%% This command processes the author and affiliation and title
%% information and builds the first part of the formatted document.
\maketitle

\section{Introduction}
It is natural for ordinary programmers such as mathematicians and scientists to
wish their computers operate over the reals. 
But this is impossible and programming languages typically compute over the
finitary floating-point number approximation.
As language designers, we wish to convince the ordinary programmer, who are
largely not experts in numerical analysis, that we have not scammed them: that
the floating-point programming language semantics we have provided is close to
what they desire. 
Towards this goal, we develop automated numerical analysis techniques that bound
the maximum error introduced by floating-point computation.

A major challenge in the automated numerical analysis literature is scalability.
Many existing approaches rely on global optimization \cite{fptaylor} \cite{satire},
rewrite saturation \cite{gappa}, or SMT-based methods \cite{rosa}. 
However, as programs scale, these analysis approaches frequently time-out.
Recent work has applied typed-based analysis approaches to both forwards and
backwards error analysis, such as Numerical Fuzz \cite{numfuzz} and Bean \cite{bean}.
The advantage of type-based methods is that they are inherently compositional
and scalable: all of the information necessary to perform an error analysis on a
function is contained within the function type --- no global optimization,
rewrite saturation, or bit-blasting is necessary.

However, prior type-based work for forwards error analysis is not capable of
handling negative numbers and subtraction. The problem is essentially:
compositional type-based error bounds seem to require relative notions of error;
however, relative notions of error are not well-behaved in the presence of
subtraction and negative numbers. This phenomenon is known as
\textit{catastrophic cancellation}: two large nearby numbers with small relative
error can be subtracted to produce a small number with arbitrarily high relative
error. 

In this paper, we are interested in addressing the above limitations by
providing a type-based forwards error analysis method that can handle broader
classes of programs and provide tigher error bounds. 
Concretely, our contributions are as follows:
\begin{itemize}
  \item We develop the first type-based approach to forwards error analysis that
    enables both \textit{a priori} and \textit{a posterori} compositional error
    bounds in the presence of subtraction and negative numbers. We prove that
    our approach, which can handle strictly more programs, leads to forwards
    error bounds no looser than prior type-based work for forwards analysis.

  \item We extend Numerical Fuzz by introducing a previously-untypable primitive
    that enables the error and sensitivities in rounded terms to be shared,
    allowing for tighter error bounds. We also extend Numerical Fuzz to enable
    expressions in more places. We prove the soundness of these extensions to
    Numerical Fuzz.

  \item We implement our type-based approach for forwards error analysis. We
    evaluate our implementation against FPTaylor and Gappa on a suite of
    benchmarks translated into our core language and demonstrate that we obtain
    competitive error bounds with often faster performance.
    % todo: put in hard numbers
\end{itemize}

% Numerical Fuzz in particular uses a graded effect and co-effect type system to
% track both the function sensitivity and round-off error of each function.
% By simultaneously tracking both pieces of information, Numerical Fuzz is able to
% compositionally reason about error.
% However, Numerical Fuzz does not handle negative numbers and subtraction, which
% significantly limits its applicability. 

% There are significant foundational
% problems with providing a simultaneously \textit{compositional} and
% \textit{scalable} analysis for forwards numerical error in the presence of
% subtraction. By \textit{compositional} we mean that the analysis need not be
% global. For this we can only rely on a relative notion of error. And by
% \textit{scalable} we mean that the analysis remains relatively tight as the
% program size balloons.

% more build-up to the contributions, talk about a posteori bounds

\section{Language Syntax}

\begin{figure}[tbp]
  \begin{alignat*}{3}
         &\text{Types } \sigma, \tau &::=~ &\mathbf{unit}
         \mid \num
         \mid \sigma \times \tau 
         \mid \sigma \otimes \tau
         \mid \sigma + \tau 
         \mid \sigma \multimap \tau
         \mid {\bang{s} \sigma}
         \mid {M_u \tau}
         \\
         &\text{Values } v, w \ &::=~ &\langle \rangle
         \mid k \in R
         \mid \langle v,w \rangle 
         \mid (v, w)
         \mid \tin_i \ v
         \mid \lambda x.~e \\
         % & & & \mid [v]
         % \mid\rnd v
         % \mid {\ret v} 
         % \mid \factor v
         % \mid \letbind x = v \ \tin \ f 
         \\
         &\text{Terms } e, f, g &::=~ &x
         \mid v
         \mid \mathbf{op}(e)
         \mid e~f
         \mid {\pi}_i\ e
         \mid \langle e,f \rangle 
         \mid (e, f) \\
         & & & \mid \letpair \ (x,y) = e \ \tin \ f
         \mid \letassign x  = e \ \tin \ f \\
         & & & \mid \tin_i \ e
         \mid 
          \mathbf{case} \ e \ \mathbf{of} \ (\tin_1 \ x.f \ | \ \tin_2 \ x.g) \\
         & & &
         \mid [e]
         \mid \rnd e
         \mid {\ret e} 
         \mid \factor e \\
         & & & 
         \mid {\letbind x = e \ \tin \ f}
         \mid \letcobind x = e \ \tin \ f
         \\
         % Max: find a better name
         &\text{Enviroments } \gamma_0, gamma_1 &::=~ &.
         \mid \gamma, \ x \mapsto v :_s \tau \\
         &\text{Enviroment trees } \sigma &::=~ & \gamma
         \mid \gamma_0; \gamma_1
  \end{alignat*}
  \caption{
    Types, values, and terms. 
    $\mathbf{op} \in \mathcal{O}$.
    $i \in \{1, 2\}$. 
  }
  \label{fig:syntax}
\end{figure}

% Max: For now, I've separated \letbind, \letcobind, \letpair, and
% \letassign to make things more clear.


\section{Static Semantics}
For the remainder of the paper, we only care about closed types.
% We define an interpretation function $interp(b)$ evaluating bounds to points in
% $\textit{num}$. 
%
% \begin{equation}
%   \begin{aligned}[c]
%     interp(k \in \textit{num}) &\triangleq k \\
%     interp(b_0 + b_1) &\triangleq interp(b_0) + interp(b_1) \\
%     interp(b_0 \cdot b_1) &\triangleq interp(b_0) \cdot interp(b_1) \\
%   \end{aligned}
% \end{equation}

\begin{figure}
%% ROW1
\begin{center}
%% var
\AXC{$s \ge 1$}
\RightLabel{(Var)}
\UIC{$\Delta \ | \ \Gamma, x:_s \tau, \Theta \vdash x : \tau$}
\bottomAlignProof
\DisplayProof
\hskip 0.5em
%% fun
\AXC{$\Delta \ | \ \Gamma, x:_1 \tau_0 \vdash e : \tau$}
\RightLabel{($\multimap$ I)}
\UnaryInfC{$\Delta \ | \ \Gamma \vdash \lambda x. e : \tau_0 \multimap \tau $}
\bottomAlignProof
\DisplayProof
\vskip 1em

%% app
\AXC{$\Delta_0 \ | \ \Gamma \vdash e : \tau_0 \multimap \tau$}
\AXC{$\Delta_1 \ | \ \Theta \vdash f : \tau_0 $}
\RightLabel{($\multimap$ E)}
\BinaryInfC{$\Delta_0 + \Delta_1 \ | \ \Gamma + \Theta \vdash ef : \tau $}
\bottomAlignProof
\DisplayProof
\vskip 1em
%%


%% ROW2
\AXC{}
\RightLabel{(Unit)}
\UIC{$\Delta \ | \ \Gamma \vdash \langle \rangle : \mathbf{unit}$}
\bottomAlignProof
\DisplayProof
\hskip 0.5em
%% dep prod intro
\AXC{$\Delta \ | \ \Gamma \vdash e : \tau_0$}
\AXC{$\Delta \ | \ \Gamma \vdash f : \tau_1$}
\RightLabel{($\times$ I)}
\BinaryInfC{$\Delta \ | \ \Gamma \vdash \langle e, f \rangle: \tau_0 \times \tau_1$}
\bottomAlignProof
\DisplayProof
\hskip 0.5em
%% dep prod elim
\AXC{$\Delta \ | \ \Gamma \vdash e : \tau_1 \times \tau_2$}
\RightLabel{($\times$ E)}
\UIC{$\Delta \ | \ \Gamma \vdash {\pi}_i \ e : \tau_i$}
\bottomAlignProof
\DisplayProof
\vskip 1em
%%

%% ind prod intro
\AXC{$\Delta_0 \ | \ \Gamma \vdash e : \tau_0 $}
\AXC{$\Delta_1 \ | \ \Theta \vdash f : \tau_1$}
\RightLabel{($\tensor$ I)}
\BIC{$\Delta_0 + \Delta_1 \ | \ \Gamma + \Theta \vdash (e, f) : \tau_0 \tensor \tau_1$}
\bottomAlignProof
\DisplayProof
\vskip 1em

%% ind prod elim
\AXC{$\Delta_0 \ | \ \Gamma \vdash e : \tau_0 \tensor \tau_1$ }
\AXC{$\Delta_1 \ | \ \Theta,x:_s \tau_0,y:_s\tau_1 \vdash f: \tau $}
\RightLabel{($\tensor$ E)}
\BIC{$\Delta_0 + \Delta_1 \ | \ s * \Gamma + \Theta \vdash \letpair (x,y) \ = \ e \ \tin \ f : \tau $}
\bottomAlignProof
\DisplayProof
\hskip 0.5em
%% ind sum intro
\AXC{$\Delta \ | \ \Gamma \vdash e : \tau_0$ }
\RightLabel{($+$ $\text{I}_i$)}
\UIC{$\Delta \ | \ \Gamma \vdash \mathbf{in}_i \ e : \tau_0 + \tau_1$}
\bottomAlignProof
\DisplayProof
\vskip 1em
% %% ind sum intro
% \AXC{$\Gamma \vdash e : \tau_1$ }
% \RightLabel{($+$ $\text{I}_2$)}
% \UIC{$\Gamma \vdash \mathbf{in}_2 \ e : \tau_0 + \tau_1$}
% \bottomAlignProof
% \DisplayProof
% \hskip 0.5em
% box elim
\AXC{$\Delta_0 \ | \ \Gamma \vdash e : {!_s \tau_0}$}
\AXC{$\Delta_1 \ | \ \Theta, x:_{t*s} \tau_0 \vdash f : \tau$}
\RightLabel{($!$ E)}
\BIC{$\Delta_0 + \Delta_1 \ | \ t * \Gamma + \Theta \vdash \letcobind x = e \ \tin \ f : \tau$}
\bottomAlignProof
\DisplayProof
\vskip 1em
%%


%% ROW 5

% sum elim
\AXC{$\Delta_0 \ | \ \Gamma \vdash e : \tau_0+\tau_1$}
\AXC{$\Delta_1 \ | \ \Theta, x:_s \tau_0 \vdash f_1 : \tau$ \qquad
$\Delta_1 \ | \ \Theta, x:_s \tau_1 \vdash f_2: \tau$}
\RightLabel{($+$ E)}
\AXC{$s > 0$}
\TIC{$\Delta_0 + \Delta_1 \ | \ s * \Gamma + \Theta \vdash \mathbf{case} \ e \ \mathbf{of} \ (\mathbf{in}_1 x.f_1 \ | \ \mathbf{in}_2 x.f_2) : \tau$}
\bottomAlignProof
\DisplayProof
\hskip 0.5em
% box intro
\AXC{$\Delta \ | \ \Gamma \vdash e : \tau$ }
\RightLabel{($!$ I)}
\UIC{$\Delta \ | \ s * \Gamma \vdash [e] : {!_s \tau}$}
\bottomAlignProof
\DisplayProof
\vskip 1em

%%% ROW 6

% let 
\AXC{$\Delta_0 \ | \ \Gamma \vdash e :  \tau_0$}
\AXC{$\Delta_1 \ | \ \Theta, x:_{s} \tau \vdash f : \tau$}
\RightLabel{(Let)}
\BIC{$\Delta_0 + \Delta_1 \ | \ s * \Gamma + \Theta \vdash \letassign x = e \ \tin \ f : \tau$}
\bottomAlignProof
\DisplayProof
\hskip 0.5em

\vskip 1em

%% const
\AXC{$k \in \textit{num}$}
\AXC{$k \in interp(\Gamma, i)$}
\RightLabel{(Const)}
\BIC{$\Delta \ | \ \Gamma \vdash k : \num_{b}$}
\bottomAlignProof
\DisplayProof
\hskip 0.5em
\vskip 1em

%%% ROW 7

%% subsumption
\AXC{$\Delta \ | \ \Gamma \vdash e :  M_q \tau$}
\AXC{$r \ge q$}
\RightLabel{(Subsumption)}
\BIC{$\Delta \ | \ \Gamma \vdash e :  M_{r} \tau$}
\bottomAlignProof
\DisplayProof
\hskip 0.5em
%% return
\AXC{$\Delta \ | \ \Gamma \vdash e : \tau$}
\RightLabel{(Ret)}
\UIC{$\Delta \ | \ \Gamma \vdash \ret e : M_0 \tau$}
\bottomAlignProof
\DisplayProof
\hskip 0.5em
%% RND
\AXC{$\Delta \ | \ \Gamma \vdash e : \num$}
\RightLabel{(Rnd)}
\UIC{$\Delta \ | \ \Gamma \vdash \rnd \ e : M_u \ \num$}
\bottomAlignProof
\DisplayProof
\vskip 1em


%%% ROW 8


% let-bind
\AXC{$\Delta_0 \ | \ \Gamma \vdash e : M_r \tau_0$}
\AXC{$\Delta_1 \ | \ \Theta, x:_{s} \tau_0 \vdash f : M_{q} \tau$}
\RightLabel{($M_u$ E)}
\BIC{$\Delta_0 + \Delta_1 \ | \ s * \Gamma + \Theta \vdash \letbind x = e \ \tin \ f : M_{s*r+q} \tau$}
\bottomAlignProof
\DisplayProof

% \hskip 0.5em
\vskip 1em

% funs
\AXC{$\Delta \ | \ \Gamma \vdash e : \tau_0$}
\AXC{$\{ \mathbf{op} :\tau_0 \lin \tau_1 \} \in \Sigma$}
\RightLabel{(Op)}
\BIC{$\Delta \ | \ \Gamma \vdash \mathbf{op}(e) : \tau_1$}
\bottomAlignProof
\DisplayProof

\vskip 1em


%%% ROW 9


% factor
\AXC{$\Delta \ | \ \Gamma \vdash e : (M_q \tau_0) \times (M_r \tau_1)$}
%\AXC{$r + q \leq s$}
%\AXC{$s = max(r,q)$}
\RightLabel{(Factor)}
\UIC{$\Delta \ | \ \Gamma \vdash \factor \ e : M_{max(q,r)} (\tau_0 \times \tau_1)$}
\bottomAlignProof
\DisplayProof

\vskip 1em

\AXC{$\Delta \ | \ \Gamma \vdash e : \tau$}
% \AXC{$\Delta \vdash i : \textbf{interval}$}
\RightLabel{($\forall$-I)}
\UIC{$\Delta \ | \ \Gamma \vdash \Lambda \epsilon . e : \forall \epsilon . \tau$}
\bottomAlignProof
\DisplayProof
\hskip 0.5em
\AXC{$\Delta \ | \ \Gamma \vdash \Lambda \epsilon . e : \forall \epsilon . \tau$}
\AXC{$\Delta \vdash i : \textbf{interval}$}
\RightLabel{($\forall$-E)}
\BIC{$\Delta \ | \ \Gamma \vdash e : \tau[i/\epsilon]$}
\bottomAlignProof
\DisplayProof

\end{center}
    \caption{Typing rules for \Lang, with $s,t,q,r,u \in \NNR \cup \{\infty\}$
    and for $i \in \{ 1, 2 \}$ where $u$ is a fixed constant parameter (see
    Definition~\ref{def:numfuzz-interface} for details on picking an adequate
    constant).}
    \label{fig:typing_rules}
\end{figure}


\section{Dynamic Semantics}

\subsection{Substitution-style Operational Semantics}
The following is defined over untyped terms. In particular, we define the
operational semantics rewrite relation $\mapsto$ to map from (untyped) \Lang
to (untyped) \Lang. In other words, untypable programs can step (but not
necessarily to values).

\begin{figure}
\begin{center}

\begin{equation*}
\begin{aligned}[c]
	\mathbf{op}(v) &\mapsto op(v)\\
	\pi_i\langle v_1,v_2 \rangle &\mapsto v_i \\
	(\lambda x.e) \ v &\mapsto e[v/x] \\
	%\factor v \ &\mapsto v
\end{aligned}
\quad
\begin{aligned}[c]
	\letassign x = v \ \tin \ e &\mapsto e[v/x] \\
  \letpair (x, y) = (v, w) \ \tin \ e &\mapsto e[v/x][w/y] \\
	\letcobind x = v \ \tin \ e &\mapsto e[v/x]
	%\letbind x = \ret v \ \tin \ e &\mapsto e[v/x] \\
\end{aligned}
\end{equation*}
\vskip -1em
\begin{align*}
  \letbind y = (\letbind x = v \ \tin \ f) \ \tin \ g &\mapsto \letbind x = v \ \tin \ \letbind y = f \ \tin \ g \quad x\notin FV(g) 
\end{align*}
\vskip -1.75em
\begin{align*}
	\mathbf{case} \ (\mathbf{in}_i \ v) \ \mathbf{of} \ (\mathbf{in}_1 \ x.e_1 \ | \ \mathbf{in}_2 \ x.e_2 )  &\mapsto e_i[v/x]
  \qquad\qquad(i \in \{1, 2 \})
\end{align*}
\vskip -0.25em

	\AXC{$e \mapsto e'$}
	\UIC{$\letassign x = e \ \tin \ f \mapsto \letassign x = e' \ \tin f$}
	\DisplayProof

\end{center}
    \caption{Substitution-style evaluation rules for \Lang. Note the side condition for $\letbind$always holds for closed expressions.}
    \label{fig:sub_eval_rules}
\end{figure}

\subsection{(Typed) Enviroment-style Operational Semantics}
The following is defined over typed terms. In particular, we define the
operational semantics rewrite relation $\rightsquigarrow$ to map from a typed
term in an program enviroment to a typed term in an program enviroment. Note
that in this setup, $\letbind x = v \ \tin \ f$ and $[x]$ are \textit{not}
values.

To be precise, $\rightsquigarrow$ maps an expression $e$ with type $\tau$
running in an enviroment $\sigma$ mapping variables like $x_1$ to value $v_1$
with type $\tau_1$ and sensitivity budget $s_1$ to a $e'$ with type $\tau'$ and
enviroment $\sigma'$.
So, $\sigma$ send variables like $x_1 \to v_1 :_s \tau_1$.

An enviroment $\sigma$ is compatible with a typing context $\Gamma$ if
$\llbracket \sigma \rrbracket$ is a point within metric space $\llbracket \Gamma
\rrbracket$ (after erasing $\sigma$'s $0$-sensitive variables). Abusing notation
a little, I write this like so: $\llbracket \sigma \rrbracket \in \llbracket
\Gamma \rrbracket$.

\begin{equation*}
  \llbracket \sigma \rrbracket \in \llbracket \Gamma \rrbracket 
  \triangleq \forall
  \\ 
  (x \mapsto v :_s \tau) \in \sigma, 0 < s \implies (x :_s \tau) \in \llbracket
  \Gamma \rrbracket
\end{equation*}

Similarly, $\llbracket \sigma \Vdash e : \tau \rrbracket$ is
interpreted as the point in the metric space $\llbracket \tau \rrbracket$
obtained by running e at $\sigma$.

Useful adequacy-flavored theorem to prove. If $\sigma \Vdash e : \tau \rightsquigarrow \sigma'
\Vdash e' : \tau'$, then $\llbracket \sigma \Vdash e : \tau \rrbracket =
\llbracket \sigma' \Vdash e' : \tau' \rrbracket$.

Another adequacy-flavored theorem (in reverse). If $\llbracket \sigma \Vdash e :
\tau \rrbracket = \llbracket \sigma_{empty} \Vdash v : \tau' \rrbracket$, then $\sigma
\Vdash e : \tau \rightsquigarrow^{*} \sigma_{empty} \Vdash v : \tau'$

A syntatic type-soundness flavored theorem. If $\Gamma \vdash e : \tau$, then
$\forall \llbracket \sigma \rrbracket \in \llbracket \Gamma \rrbracket, \exists
\sigma' \ v \ \tau', \sigma \Vdash e : \tau \rightsquigarrow^{*} \sigma' \Vdash v :
\tau'$ and $\llbracket \tau \rrbracket = \llbracket \tau' \rrbracket$.

Another flavored soundness theorem. For $\llbracket \Gamma \vdash e : \tau
\rrbracket$ and $\sigma, \sigma' \in \Gamma$ and
$$\sigma \Vdash e : \tau \rightsquigarrow^* \sigma_{empty} \Vdash v : \tau'$$ 
and 
$$\sigma' \Vdash e : \tau \rightsquigarrow^* \sigma_{empty'} \Vdash v' : \tau''$$
%the distance between $\llbracket \sigma \rrbracket$ and $\llbracket \sigma'
%\rrbracket$ in metric space $\llbracket \Gamma \rrbracket$ is greater than or
%equal to the distance between $\llbracket \sigma \Vdash e : \tau \rrbracket$ and
%$\llbracket \sigma' \Vdash e : \tau \rrbracket$. In other words,
then
$$
d_{\llbracket \tau \rrbracket}(v, v') \leq d_{\llbracket \Gamma \rrbracket}(\sigma, \sigma')
$$
where $\sigma_{empty}, \sigma_{empty'}$ means
$$
\forall (x \mapsto v :_s \tau) \in \sigma_{empty}, \sigma_{empty'}, s = 0
$$

Note that $\sigma$ is ordered in the case that variables are shadowed. $\sigma[x
\mapsto v :_s \tau]$ denotes lookup (when on the left-hand side of a rewrite
relation) or insertion from the right (when on the right-hand side of a rewrite
relation).

\begin{figure}
\begin{center}
\begin{equation*}
  \begin{aligned}[c]
    %%%%%%%%%%%%%%%%%%%%%%%%%%%%%%%%%%%%%%%%%%%%%%%%%%%%%%%%%%%%%%%%%%%%%%%%%%%
    % more spicy rules
    % Lookup (sensitivity budget 1)
    \sigma[x \mapsto v :_1 \tau] \Vdash \ x \ : \tau &\rightsquigarrow 
    \sigma \Vdash v : \tau \\
    % Lookup (sensitivity budget greater than 1)
    \sigma[x \mapsto v :_{s} \tau] \Vdash \ x \ : \tau &\rightsquigarrow
    \sigma[x \mapsto v :_{s-1} \tau] \Vdash v : \tau \quad{(\text{with } 1 < s)}
    \\
    % let-bind rule
    \sigma \Vdash \textbf{let-bind}_{(s, \tau_1)} \ x = v \ : M_{q} \tau \ \tin \ f &\rightsquigarrow \sigma[x
    \mapsto v :_s \tau_1]
    \Vdash f : M_q \tau \\
    % let-cobind rule
    \sigma \Vdash \textbf{let-cobind}_{(s, t, \tau_1)} \ x = v \ : \tau \ \tin \ f &\rightsquigarrow (\sigma[x
    \mapsto v :_s \tau_1]) * t
    \Vdash f : \tau \\
    % [e] rule
    \sigma \Vdash [e]_s \ : \ \tau &\rightsquigarrow \sigma * s \Vdash e \ : \ \tau \\
    % lam app
    \sigma \Vdash (\lambda x : \tau_1 .e) \ v : \tau &\rightsquigarrow \sigma[x
    \mapsto v :_1 \tau_1] \Vdash e : \tau \\
    %%%%%%%%%%%%%%%%%%%%%%%%%%%%%%%%%%%%%%%%%%%%%%%%%%%%%%%%%%%%%%%%%%%%%%%%%%%
    % more boring rules
    % op(v) rule
    \sigma \Vdash \mathbf{op}(v) : \tau &\rightsquigarrow \sigma \Vdash op(v) :
    \tau \\
    % proj rule
    \sigma \Vdash \pi_i\langle v_1,v_2 \rangle : \tau &\rightsquigarrow \sigma
    \Vdash v_i : \tau \\ 
    %%%%%%%%%%%%%%%%%%%%%%%%%%%%%%%%%%%%%%%%%%%%%%%%%%%%%%%%%%%%%%%%%%%%%%%%%%%
    % let stepping rule
    %%%%%%%%%%%%%%%%%%%%%%%%%%%%%%%%%%%%%%%%%%%%%%%%%%%%%%%%%%%%%%%%%%%%%%%%%%%
  \end{aligned}
\end{equation*}

  \AXC{$\sigma \Vdash e : \tau_1 \rightsquigarrow \sigma' \Vdash e' : \tau_2$}
  \UIC{$\sigma \Vdash \textbf{let}^* \ x = e \ : \tau_1 \ \tin \ \tau_2
  \rightsquigarrow \sigma' \Vdash \textbf{let}^* \ x = e' \ : \tau_1 \
  \tin \ \tau_2$}
	\DisplayProof

\end{center}
    \caption{Enviroment-style evaluation rules for \Lang. Note that during type
    checking but prior to running the operational semanitcs, the sensitivity
    information (tracked with metavar $s$) and type of bound variables $\tau_1,
    \tau_2$, is preserved as annotations in the syntax, written $[e]_s$ and
    $\textbf{let-bind}_{(s, \tau_1)}$, $\textbf{let-cobind}_{(s, t, \tau_1)}$,
    and $\lambda x : \tau_1 . e $. Note that $\textbf{let}^*$ is syntactic sugar
    for matching all let expressions and their corresponding annotations.}
    \label{fig:sub_eval_rules}
\end{figure}

\section{Denotational Semantics}

\begin{definition}[Type interpretation]
  A type $\tau$ is interpreted with $\llbracket - \rrbracket : \textit{type} \to
  \textbf{Met}$ in the same way as in the original NumFuzz system.
  % TODO: put actual definition here.
\end{definition}


\begin{definition}[Typing context interpretation]
  A typing context $\Gamma$ is interpreted with $\llbracket - \rrbracket :
  \textit{context} \to \textbf{Met}$ in the following way:
  \begin{equation}
  \begin{aligned}[c]
    \llbracket . \rrbracket &\triangleq . \\
    \llbracket \Gamma, x :_s \tau \rrbracket &\triangleq \llbracket \Gamma \rrbracket
      \times D_s \llbracket \tau \rrbracket
  \end{aligned}
  \end{equation}
\end{definition}

Let $CV(\tau)$ be the closed values of type $\tau$ and $CE(\tau)$ be the closed
expressions of type $\tau$. Then we can define a unary logical relation over types which
capture the core information needed to prove our error soundness theorem.

\begin{definition}[Logical relation]
  \begin{equation}
  \begin{aligned}[c]
    % same from Ariel's paper
    \mathcal{R_{\tau}} &\triangleq 
      \{ e \ | \ e \in CE(\tau) \text{ and } \exists v
        \in CV(\tau) \text{ s.t. } e \mapsto^{*} v \text{ and } v \in \mathcal{VR_{\tau}} 
      \} \\
    \mathcal{VR_{\mathbf{unit}}} &\triangleq \{ \langle \rangle \} \\
    \mathcal{VR_{\mathbf{num}}} &\triangleq \mathbb{R} \\
    \mathcal{VR_{\mathbf{\tau_0 \times \tau_1}}} &\triangleq 
      \{ \langle v, w \rangle \ | 
        \ v \in \mathcal{R}_{\tau_0} \text{ and } w \in \mathcal{R}_{\tau_1}
      \} \\
    \mathcal{VR_{\mathbf{\tau_0 \otimes \tau_1}}} &\triangleq 
      \{ ( v, w ) \ | 
        \ v \in \mathcal{R}_{\tau_0} \text{ and } w \in \mathcal{R}_{\tau_1}
      \} \\
    \mathcal{VR_{\mathbf{\tau_0 + \tau_1}}} &\triangleq 
      \{ \mathbf{inl}~v \ | \ v \in \mathcal{R}_{\tau_0} \} 
      \cup
      \{ \mathbf{inr}~v \ | \ v \in \mathcal{R}_{\tau_1} \} \\
    \mathcal{VR_{\mathbf{\tau_0 \multimap \tau_1}}} &\triangleq 
      \{ \lambda x . e \ | \ \forall w_0, w_1 \in \mathcal{VR}_{\tau_0}, \\ & \quad \quad \ (\lambda x.e)~w_0, (\lambda x . e)~w_1 \in
      \mathcal{R}_{\tau_1} \text{ and } \mathcal{SD}_{\tau_1}((\lambda x . e)~w_0, (\lambda x . e)~w_1) \leq
      \mathcal{SD}_{\tau_0}(w_0, w_1) \} \\
    \mathcal{VR_{\mathbf{!_s \tau}}} &\triangleq 
      \{ [~v~] \ | \ v \in \mathcal{R}_{\tau} \} \\
    % spicy hot new stuff
    \mathcal{VR_{\mathbf{M_q \tau}}} &\triangleq 
      \{ (v, w) \ | \ v, w \in \mathcal{R}_{\tau} \text{ and } \mathcal{SDV}_{\tau}(v, w)
      \leq q \} \\
  \end{aligned}
  \end{equation}
\end{definition}

Our definition for distance (in our dentoational semantics), $d_\tau$ and the
distance between syntactic values $\mathcal{SD}_\tau$ (for $\mathcal{S}$yntactic
$\mathcal{D}$istance) and $\mathcal{SDV}_\tau$ (for $\mathcal{S}$yntactic
$\mathcal{D}$istance for $\mathcal{V}$alues) are closely related. Some care is
needed to ensure that it is well-founded. We define our distance over syntactic
values as follows:

\begin{definition}[Distance between closed syntactic terms]
  \begin{equation}
  \begin{aligned}[c]
    \mathcal{SD_{\tau}}(e_0, e_1) &\triangleq \mathcal{SDV}_{\tau}(v_0, v_1)
    &\text{ if } e_0 \mapsto^{*} v_0 \text{ and } e_1 \mapsto^{*} v_1 \\
    \mathcal{SD_{\tau}}(e_0, e_1) &\triangleq \infty &\text{ otherwise } \\
    \mathcal{SDV_{\mathbf{unit}}}(v, w) &\triangleq 0 &\text{ for } v, w = \langle \rangle \\
    \mathcal{SDV_{\mathbf{num}}}(k_0, k_1) &\triangleq 
      d_\mathbb{R} &\text{ for } k_0, k_1 \in \mathbb{R} \\
    \mathcal{SDV}_{\tau_0 \times \tau_1}((v_0, v_1), (w_0, w_1)) 
      &\triangleq max(\mathcal{SDV}_{\tau_0}(v_0, w_0),~\mathcal{SDV}_{\tau_0}(v_1, w_1))
    \\
    \mathcal{SDV}_{\tau_0 \otimes \tau_1}((v_0, v_1), (w_0, w_1)) 
      &\triangleq \mathcal{SDV}_{\tau_0}(v_0, w_0) + \mathcal{SDV}_{\tau_0}(v_1, w_1))
    \\
    \mathcal{SDV}_{\tau_0 + \tau_1}(\in_i v, \in_i w) 
      &\triangleq \mathcal{SDV}_{\tau_i}(v, w)
    \\
    \mathcal{SDV}_{\tau_0 \multimap \tau_1}(v_0, v_1) 
      &\triangleq \text{sup}_{w \in \mathcal{VR}_{\tau_0}} \mathcal{SD}_{\tau_1}(v_0~w,~v_1~w)
    \\
    % max: double check
    \mathcal{SDV}_{!_s \tau}(v, w) 
      &\triangleq s \cdot \mathcal{SDV}_{\tau}(v, w)
    \\
    \mathcal{SDV}_{M_q~\tau}((v_0, v_1), (w_0, w_1)) 
      &\triangleq \mathcal{SDV}_{\tau}(v_0, w_0)
    \\
    \mathcal{SDV_{\tau}}(v, w) &\triangleq \infty &\text{ otherwise } \\
  \end{aligned}
  \end{equation}
\end{definition}


Numerical Fuzz is a modular family of programming langauges paramterized by the
appropriate $\rho$, constant parameter $u$, and the appropriate set of numeric
computations $\Sigma$.  
We can now state a few assumptions about how the operational and static
semantics of $\mathbf{rnd}$ and $\Sigma$ (for $\mathbf{op}$) relate.
It is the proof obligation for any language designer instantiating the language
to demonstrate that these properties hold in order for our paramterized
soundness theorems to follow.
\begin{definition}[Interface for instantiating Numerical Fuzz.]
  \label{def:numfuzz-interface}
  The interface for Numerical Fuzz consists of $\rho$, $u$, and $\Sigma$ such
  that the following properties hold: 
\begin{description}
  \item[a) Property of $\rho$ and constant parameter $u$.] We assume that the
    $\forall e \in \mathcal{R}_{num}, \mathcal{SD}_{\mathbf{num}}(\rho(e), e)
    \leq q$ where $q$ is the grade in the $\mathbf{rnd}$ typing rule.
  \item[b) Property of $\mathbf{op}$.] We also assume that for every operation
    $\mathbf{op} : \tau_0 \multimap \tau_1 \ \in \ \Sigma$ we have a
    corresponding function $op$ mapping syntactic values in
    $\mathcal{VR}_{\tau_0}$ to $\mathcal{VR}_{\tau_1}$ and where
    the following two properties hold:
    \begin{description}
      \item[\underline{1) Type denotation.}] For any $e \in
        \mathcal{R}_{\tau_0}$ that $\mathbf{op}(e) \in \mathcal{R}_{\tau_1}$.
      \item[\underline{2) Sensitivity.}] For any $e_1, e_2 \in \mathcal{R}_{\tau_0}$ that
        $\mathcal{SD}_{\tau_1}(\mathbf{op}(e_1), \mathbf{op}(e_2)) \leq
        \mathcal{SD}_{\tau_0}(e_1, e_2)$.
    \end{description}
\end{description}
\end{definition}

\begin{lemma}[$\mathcal{SD}$ is a metric]
  $\mathcal{SD}$ forms a metric over our syntactic terms; in particular it
  satisfies:
  \begin{enumerate}
    \item Distance from any point to itself is zero: $\forall x,~\mathcal{SD}(x,
      x) = 0$.
    \item Positivity: $\forall x, y,~\mathcal{SD}(x, y) \geq 0$.
    \item Symmetry: $\forall x, y,~\mathcal{SD}(x, y) = \mathcal{SD}(y, x)$.
    \item Triangle inequality: $\forall x, y, z,~\mathcal{SD}(x, z) \leq
      \mathcal{SD}(x, y) + \mathcal{SD}(y, z)$.
  \end{enumerate}
\end{lemma}
\begin{proof}
  The properties holds for the base cases of $\mathcal{SDV}$ and follow for the
  remaining cases by our inductive hypothesis. Since our operational semntics is
  deterministic, the properties follow for $\mathcal{SD}$.
\end{proof}

\begin{lemma}[$\mathcal{SD}$ is preserved under stepping]
  If $e_0 \mapsto e'_0$, then for any $e_1$, $\mathcal{SD}(e_0, e_1) =
  \mathcal{SD}(e'_0, e_1)$.
\end{lemma}
\begin{proof}
  Holds by inspection of the definition of $\mathcal{SD}$.
\end{proof}
Note that by metric symmetry, $\mathcal{SD}$ is preserved under stepping on both
sides.

% probs needs some theorem relating SD to d, what do property do we want to
% enforce here? (probs need to factor out a lemma or something when doing the
% logical relations proof)


\section{Type Soundness}
\begin{lemma}[Termination]
  If $\Gamma \vDash e : \tau$, then $\forall \sigma$ compatible with $\Gamma$, 
  $$\sigma \Vdash e \rightsquigarrow^* \sigma' \Vdash v$$
\end{lemma}
\begin{proof}
  If $\sigma \Vdash e$ is a value, then we are done. Otherwise, if $\sigma
  \Vdash e$ is an expression, we begin by unfolding and applying the definition
  of $\vDash$. By definition, know that $\llbracket \sigma \Vdash e \rrbracket_{\tau} \in
  \llbracket \tau \rrbracket$. We also know that $\bot \not\in \llbracket \tau
  \rrbracket$, a metric space, so $\llbracket \sigma \Vdash e \rrbracket_{\tau}
  \not= \bot$ and is well-defined.

  By inspection of the definition of cases of $\llbracket - \rrbracket_{\tau}$,
  we know that only last case (which deals with all expressions) must apply.
  Therefore, $\sigma \Vdash e \rightsquigarrow^* \sigma' \Vdash v$.
\end{proof}

% \begin{theorem}[Semantic type preservation]
% If a term $e$ is semantically well-typed in context $\Gamma$ with type $\tau$
%   then for all $\sigma \Vdash e \rightsquigarrow \sigma' \Vdash e'$, $\sigma$
%   compatible with $\Gamma$, then there exists a $\Gamma'$ such that $\Gamma'
%   \vDash e' : \tau$.
% \end{theorem}
% \begin{proof}
% If $e$ is a value, we're done. 
% If $e$ is not a value, we proceed by induction over the cases of the rewrite relation
% $\rightsquigarrow$.
% % Our inductive hypothesis is that $e \rightsquigarrow e'$ and $\Gamma \vDash e : \tau$, that is, 
% % $\forall \sigma \text{ compatible with } \Gamma, \llbracket \sigma \Vdash e
% % \rrbracket_{\tau} \in \llbracket \tau \rrbracket$.
% \begin{description}
%   \item[\textsc{Variable lookup.}] If $$
%   \item[\textsc{Ret.}]
%   \item[\textsc{Rnd.}]
% \end{description}
% \end{proof}

% \begin{definition}[Monadic type]\label{def:monadic}
%   A type $\tau$ is monadic if and only if $\exists \tau', q \text{ such that }
%   \tau = M_q \tau'$. It is non-monadic otherwise.
% \end{definition}
% %%% wrong theorem:
% \begin{lemma}[Non-monadic lookup]\label{thm:non-monadic-lookup}
%   For all $\tau$ non-monadic and all $\Gamma, x : \tau'$ compatible with
%   $\sigma$, $\exists \sigma', \sigma = \sigma'[x \mapsto v : \tau']$ is
%   well-defined. Stated equivalently, non-monadic $x$ uniformly maps to $v$ at
%   all enviroments in $\sigma$.
% \end{lemma}
% \begin{proof}
%   TODO
% \end{proof}

\begin{lemma}[Enviroment tree shape]
  The tree height for a 
\end{lemma}

\begin{definition}[Orderings]\label{def:orderings}
  We define the following well-founded partial order over configurations and
  types, which mirrors the definition of machine configuration interpretation
  and will be used for induction in the proceeding proofs.

  First, we define an ordering over types by size of the AST.
  \begin{equation}
    \begin{aligned}[c]
      s(\textbf{unit}) &= 1 \\
      s(\textbf{num}) &= 1 \\
      s(!_s \tau) &= s(\tau) + 1 \\
      s(M_q \tau) &= s(\tau) + 1 \\
      s(\tau_0 \times \tau_1) &= s(\tau_0) + s(\tau_1) + 1 \\
      s(\tau_0 \otimes \tau_1) &= s(\tau_0) + s(\tau_1) + 1 \\
      s(\tau_0 \multimap \tau_1) &= s(\tau_0) + s(\tau_1) + 1 \\
    \end{aligned}
  \end{equation}
  and
  \begin{equation}
    \tau_0 < \tau_0 \iff s(\tau_0) < s(\tau_1)
  \end{equation}

  Secondly, we define an ordering over environment leafs $\gamma$ using the number of bound
  variables:
  \begin{equation}
    \begin{aligned}[c]
      s(.) &= 0 \\
      s(\gamma, x \mapsto v :_s \tau) &= s(\gamma) + 1 \\
    \end{aligned}
  \end{equation}

  and

  \begin{equation}
    \gamma_0 < \gamma_0 \iff s(\gamma_0) < s(\gamma_1)
  \end{equation}

  We can now define an ordering over enviorment trees using the maximum size of
  each enviorment leaf in the tree:
  \begin{equation}
    \begin{aligned}[c]
      s(\gamma; \gamma') &= max(s(\gamma), s(\gamma')) \\
      s(\sigma; \sigma') &= max(s(\sigma), s(\sigma'))
    \end{aligned}
  \end{equation}

  We are now ready to define an ordering over $\textit{config} \times
  \textit{type}$:
  \footnote{This is our termination measure for our machine configuration
  interpretation function.}
  \footnote{Note that the type of the configuration (if the interpretation of
  the configuration belongs to the interpretation of the type) restricts the
  height and shape of environment tree. So we do not need to reason about the
  height of our enviroment trees in our ordering.}

  \begin{equation}
    \sigma \Vdash e : \tau < \sigma' \Vdash e' : \tau' \iff
    s(\tau) < s(\tau') \lor (s(\tau) = s(\tau') \land s(\sigma) < s(\sigma'))
  \end{equation}

\end{definition}

\begin{lemma}[Well-founded partial ordering]
  $\leq$ is a well-founded partial order over $\textit{config} \times
  \textit{type}$.
\end{lemma}
\begin{proof}
  TODO
\end{proof}

\begin{theorem}[Semantic type soundness]
$\Gamma \vdash e : \tau \implies \Gamma \vDash e : \tau$
\end{theorem}
\begin{proof}
From our premise, we wish to show that:
  \begin{enumerate}
    \item $\forall \sigma$ compatible with $\Gamma$, $\llbracket \sigma \Vdash e
      \rrbracket_{\tau} \in \llbracket \tau \rrbracket$.
    \item $\llbracket - \Vdash e
      \rrbracket_{\tau}$ is a \text{1-sensitive} map for all inputs enviroments
      compatible with $\Gamma$
  \end{enumerate}
We proceed by case analysis over whether $e$ is a value or an expression.
  Suppose $e$ is a value. Let us induct over our typing derivation. 
  % In some of the cases, we will need to induct over the size of our enviroment. 
\begin{description}
  \item[\textsc{(ty. rule) Var.}] There is no premise containing a typing
    judgement to this rule. So we have no inductive hypothesis to rely on.
    % We proceed to induct over the type of our variable, $\tau$.
    % \begin{description}
    %   \item{\textbf{unit} and \textbf{num}.}
    %   \item{$\tau_0 \otimes \tau_1$, $\tau_0 + \tau_1$, and $\tau_0 \multimap
    %     \tau_1$.}
    %   \item{$!_s \tau$.}
    %   \item{$M_q \tau$}
    %   \item{$\tau_0 \times \tau_1$.}
    % \end{description}
    Instead, we induct over the the size of our enviorment.
    \begin{description}
      \item[\textit{(env. size) 0.}] We observe that an enviorment size of zero
        implies that $\Gamma$ is empty. So we are trying to show that the
        following:
        $$\Gamma, x : \tau \vDash e : \tau$$
        implies:
        $$\llbracket \sigma \Vdash e \rrbracket_\tau \in \llbracket \tau
        \rrbracket$$
        for compatible $\sigma, \Gamma, x : \tau$. We proceed to case over the type of our
        variable, $\tau$.
        \begin{description}
          \item{(ty. case) \textbf{unit.}} The only configuration compatible with this is
            $x \mapsto \langle \rangle \Vdash x$ and clearly 
            $$\llbracket x \mapsto \langle \rangle \Vdash x
            \rrbracket_{\textbf{unit}} = \llbracket \Vdash \langle \rangle
            \rrbracket_{\textbf{unit}} = * \in \{ * \} = \llbracket \textbf{unit}
            \rrbracket$$
            So, properties (1) and (2) hold by the reasoning above.
          \item{(ty. case) \textbf{num.}}
        \end{description}
      \item[\textit{(env. size) n + 1.}] 
    \end{description}
  \item[\textsc{(ty. rule) Ret.}] 
    From our inductive hypothesis, we have that $\Gamma \vDash e : \tau$ and
    wish to prove that $\Gamma \vDash \mathbf{ret} \ e : \tau$. It suffices to
    show that $(\sigma; \sigma)[\alpha \mapsto v:_1 \tau] \Vdash \alpha$.
  \item[\textsc{(ty. rule) Rnd.}]
\end{description}

  Next, the tree case:
\end{proof}

%%%% IGNORE BELOW %%%%

% todo: Use whatever theorem name people like best here.
Semantics is preserved under operational stepping. If $\sigma \Vdash e : \tau \rightsquigarrow \sigma'
\Vdash e' : \tau$, then $\llbracket \sigma \Vdash e : \tau \rrbracket =
\llbracket \sigma' \Vdash e' : \tau \rrbracket$.

% todo: Use whatever theorem name people like best here.
If the semantics of a program is equivalent to the semantics of a value, it must
reduce to that value. If $\llbracket \sigma \Vdash e : \tau \rrbracket =
\llbracket v : \tau \rrbracket$, then $\sigma \Vdash e : \tau
\rightsquigarrow^{*} \sigma' \Vdash v : \tau$

Syntactically well-typed programs are non-expansive. For $\Gamma \vdash e : \tau
$ and $\llbracket \sigma \rrbracket, \llbracket \sigma' \rrbracket \in
\llbracket \Gamma \rrbracket $ and
$$\sigma \Vdash e : \tau \rightsquigarrow^* v : \tau$$ 
and 
$$\sigma' \Vdash e : \tau \rightsquigarrow^* v' : \tau$$
then
$$
d_{\tau}(v, v') \leq d_{\Gamma}(\sigma, \sigma')
$$

Syntactically ok implies semantically ok. If $\Gamma \vdash e : \tau$, then
$\forall \llbracket \sigma \rrbracket \in \llbracket \Gamma \rrbracket, \exists
v, \llbracket \sigma \Vdash e : \tau \rrbracket = \llbracket v : \tau
\rrbracket$.

Syntactically ok implies operationally ok. If $\Gamma \vdash e : \tau$, then
$\forall \llbracket \sigma \rrbracket \in \llbracket \Gamma \rrbracket, \exists
v, \sigma \Vdash e : \tau \rightsquigarrow^{*} v : \tau$.


\section{Abstract Interpretation}

\subsection{Abstract Domain}

\section{Subnormals}

\section{Sharing Error}

\section{Evaluation}


%%
%% The acknowledgments section is defined using the "acks" environment
%% (and NOT an unnumbered section). This ensures the proper
%% identification of the section in the article metadata, and the
%% consistent spelling of the heading.
%\begin{acks}
%  \input{XX-acknowledgments.tex}
%\end{acks}

%%
%% The next two lines define the bibliography style to be used, and
%% the bibliography file.
\bibliographystyle{ACM-Reference-Format}
\bibliography{sample-base}


%%
%% If your work has an appendix, this is the place to put it.
%\appendix
%\input{XX-appendix.tex}

\end{document}
\endinput
%%
%% End of file `sample-acmsmall.tex'.

