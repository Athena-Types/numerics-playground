% \begin{definition}[Type interpretation]
%   A type $\tau$ is interpreted with $\llbracket - \rrbracket : \textit{type} \to
%   \textbf{Met}$ in the same way as in the original NumFuzz system.
%   % TODO: put actual definition here.
% \end{definition}
%
%
% \begin{definition}[Typing context interpretation]
%   A typing context $\Gamma$ is interpreted with $\llbracket - \rrbracket :
%   \textit{context} \to \textbf{Met}$ in the following way:
%   \begin{equation}
%   \begin{aligned}[c]
%     \llbracket . \rrbracket &\triangleq . \\
%     \llbracket \Gamma, x :_s \tau \rrbracket &\triangleq \llbracket \Gamma \rrbracket
%       \times D_s \llbracket \tau \rrbracket
%   \end{aligned}
%   \end{equation}
% \end{definition}

In our language, we only reason about closed types and terms. Let $CV(\tau)$ be
the closed values of type $\tau$ and $CE(\tau)$ be the closed expressions of
type $\tau$. Then we can define a unary logical relation over types which
capture the core information needed to prove our error soundness theorem:

\begin{definition}[Logical relation]
  \begin{equation}
  \begin{aligned}[c]
    \mathcal{R_{\tau}} &\triangleq 
      \{ e \ | \ e \in CE(\tau) \text{ and } \exists v
        \in CV(\tau) \text{ s.t. } e \mapsto^{*} v \text{ and } v \in \mathcal{VR_{\tau}} 
      \} \\
    \mathcal{VR_{\mathbf{unit}}} &\triangleq \{ \langle \rangle \} \\
    \mathcal{VR}_{\mathbf{num}_{b}} &\triangleq 
      \mathcal{VR}_{\mathbf{num}_{(k_0, k_1)}} 
      \hspace{22em} \text{if } b \mapsto^* (k_0, k_1) \\
    \mathcal{VR}_{\mathbf{num}_{(k_0, k_1)}} &\triangleq 
      \{ k \ | \ k \in \mathit{num} \text{ and } k_0 \leq k \leq k_1 \} \\
    \mathcal{VR_{\mathbf{\tau_0 \times \tau_1}}} &\triangleq 
      \{ \langle v, w \rangle \ | 
        \ v \in \mathcal{R}_{\tau_0} \text{ and } w \in \mathcal{R}_{\tau_1}
      \} \\
    \mathcal{VR_{\mathbf{\tau_0 \otimes \tau_1}}} &\triangleq 
      \{ ( v, w ) \ | 
        \ v \in \mathcal{R}_{\tau_0} \text{ and } w \in \mathcal{R}_{\tau_1}
      \} \\
    \mathcal{VR_{\mathbf{\tau_0 + \tau_1}}} &\triangleq 
      \{ \mathbf{inl}~v \ | \ v \in \mathcal{R}_{\tau_0} \} 
      \cup
      \{ \mathbf{inr}~v \ | \ v \in \mathcal{R}_{\tau_1} \} \\
    \mathcal{VR_{\mathbf{\tau_0 \multimap \tau_1}}} &\triangleq 
      \{ \lambda x . e \ | \ \forall w_0, w_1 \in \mathcal{VR}_{\tau_0}, \\ & \quad \quad \ (\lambda x.e)~w_0, (\lambda x . e)~w_1 \in
      \mathcal{R}_{\tau_1} \text{ and } \mathcal{SD}_{\tau_1}((\lambda x . e)~w_0, (\lambda x . e)~w_1) \leq
      \mathcal{SD}_{\tau_0}(w_0, w_1) \} \\
    \mathcal{VR_{\mathbf{!_s \tau}}} &\triangleq 
      \{ [~v~] \ | \ v \in \mathcal{R}_{\tau} \} \\
    % spicy hot new stuff
    \mathcal{VR_{\mathbf{M_q \tau}}} &\triangleq 
      \{ (v, w) \ | \ v, w \in \mathcal{R}_{\tau} \text{ and } \mathcal{SDV}_{\tau}(v, w)
      \leq q \} \\
    \mathcal{VR}_{\textbf{bnd}} &\triangleq \{ (k_0, k_1) \ | \ k_0 \leq k_1 \ \forall k_0, k_1 \in
    \textit{ num } \} \\
    \mathcal{VR_{\forall \epsilon. \tau}} &\triangleq 
      \{ ((k_0, k_1), v) \ | \ \forall (k_0, k_1) \in
      \mathcal{VR}_{\textbf{bnd}} \text{ s.t. } v ~ \{(k_0, k_1)\} \in
      \mathcal{R}_{\tau[(k_0, k_1) / \epsilon]}\} \\
  \end{aligned}
  \end{equation}
\end{definition}

Our definition for distance (in our dentoational semantics), $d_\tau$ and the
distance between syntactic values $\mathcal{SD}_\tau$ (for $\mathcal{S}$yntactic
$\mathcal{D}$istance) and $\mathcal{SDV}_\tau$ (for $\mathcal{S}$yntactic
$\mathcal{D}$istance for $\mathcal{V}$alues) are closely related. Some care is
needed to ensure that it is well-founded. We define our distance over syntactic
values as follows:

\begin{definition}[Distance between closed syntactic terms]
  \begin{equation}
  \begin{aligned}[c]
    \mathcal{SD_{\tau}}(e_0, e_1) &\triangleq \mathcal{SDV}_{\tau}(v_0, v_1)
    &\text{ if } e_0 \mapsto^{*} v_0 \text{ and } e_1 \mapsto^{*} v_1 \\
    \mathcal{SD_{\tau}}(e_0, e_1) &\triangleq \infty &\text{ otherwise } \\
    \mathcal{SDV_{\mathbf{unit}}}(v, w) &\triangleq 0 &\text{ for } v, w =
      \langle \rangle \\
    \mathcal{SDV}_{\mathbf{num_{b}}}(k_0, k_1) &\triangleq 
      d_\mathbb{R} &\text{ for } k_0, k_1 \in \mathbb{R} \\
    \mathcal{SDV}_{\tau_0 \times \tau_1}((v_0, v_1), (w_0, w_1)) 
      &\triangleq max(\mathcal{SDV}_{\tau_0}(v_0, w_0),~\mathcal{SDV}_{\tau_0}(v_1, w_1))
    \\
    \mathcal{SDV}_{\tau_0 \otimes \tau_1}((v_0, v_1), (w_0, w_1)) 
      &\triangleq \mathcal{SDV}_{\tau_0}(v_0, w_0) + \mathcal{SDV}_{\tau_0}(v_1, w_1))
    \\
    \mathcal{SDV}_{\tau_0 + \tau_1}(\tin_i~v, \tin_i~w) 
      &\triangleq \mathcal{SDV}_{\tau_i}(v, w)
    \\
    \mathcal{SDV}_{\tau_0 \multimap \tau_1}(v_0, v_1) 
      &\triangleq \text{sup}_{w \in \mathcal{VR}_{\tau_0}} \mathcal{SD}_{\tau_1}(v_0~w,~v_1~w)
    \\
    % max: double check
    \mathcal{SDV}_{!_s \tau}([v], [w]) 
      &\triangleq s \cdot \mathcal{SDV}_{\tau}(v, w)
    \\
    \mathcal{SDV}_{M_q~\tau}((v_0, v_1), (w_0, w_1)) 
      &\triangleq \mathcal{SDV}_{\tau}(v_0, w_0)
    \\
    \mathcal{SDV}_{\textbf{bnd}}((k_0, k_1), (k_0, k_1)) 
      &\triangleq 0 
    \\
    \mathcal{SDV}_{\textbf{bnd}}((k_0, k_1), (k'_0, k'_1)) 
      &\triangleq \infty &\text{ otherwise }
    \\
    \mathcal{SDV}_{\forall \epsilon . \tau}(v_0, v_1) 
      &\triangleq \text{sup}_{w \in \mathcal{VR}_{\textbf{bnd}}} 
      \mathcal{SD}_{\tau_1}(v_0~\{w\},~v_1~\{w\})
    \\
    \mathcal{SDV_{\tau}}(v, w) &\triangleq \infty &\text{ otherwise } \\
  \end{aligned}
  \end{equation}
\end{definition}

\begin{lemma}[$\mathcal{SD}$ is a metric]
  $\mathcal{SD}$ forms a metric over our syntactic terms; in particular it
  satisfies:
  \begin{enumerate}
    \item Distance from any point to itself is zero: $\forall x,~\mathcal{SD}(x,
      x) = 0$.
    \item Positivity: $\forall x, y,~\mathcal{SD}(x, y) \geq 0$.
    \item Symmetry: $\forall x, y,~\mathcal{SD}(x, y) = \mathcal{SD}(y, x)$.
    \item Triangle inequality: $\forall x, y, z,~\mathcal{SD}(x, z) \leq
      \mathcal{SD}(x, y) + \mathcal{SD}(y, z)$.
  \end{enumerate}
\end{lemma}
\begin{proof}
  The properties holds for the base cases of $\mathcal{SDV}$ and follow for the
  remaining cases by our inductive hypothesis. Since our operational semntics is
  deterministic, the properties follow for $\mathcal{SD}$.
\end{proof}

\begin{lemma}[$\mathcal{SD}$ is preserved under stepping]
  If $e_0 \mapsto e'_0$, then for any $e_1$, $\mathcal{SD}(e_0, e_1) =
  \mathcal{SD}(e'_0, e_1)$.
\end{lemma}
\begin{proof}
  Holds by inspection of the definition of $\mathcal{SD}$.
\end{proof}
Note that by metric symmetry, $\mathcal{SD}$ is preserved under stepping on both
sides.

% probs needs some theorem relating SD to d, what do property do we want to
% enforce here? (probs need to factor out a lemma or something when doing the
% logical relations proof)

