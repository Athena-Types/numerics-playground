\section{Denotational Semantics}

\newpage

\begin{definition}[Type interpretation]
  A type $\tau$ is interpreted with $\llbracket - \rrbracket : \textit{type} \to
  \textbf{Met}$ in the same way as in the original NumFuzz system.
  % TODO: put actual definition here.
\end{definition}


\begin{definition}[Typing context interpretation]
  A typing context $\Gamma$ is interpreted with $\llbracket - \rrbracket :
  \textit{context} \to \textbf{Met}$ in the following way:
  \begin{equation}
  \begin{aligned}[c]
    \llbracket . \rrbracket &\triangleq . \\
    \llbracket \Gamma, x :_s \tau \rrbracket &\triangleq \llbracket \Gamma \rrbracket
      \times D_s \llbracket \tau \rrbracket
  \end{aligned}
  \end{equation}
\end{definition}

\newpage


\begin{definition}[Machine configuration interpretation]
  A machine configuration $\sigma \Vdash e$ is interpreted with $\llbracket -
  \rrbracket_{ - } : \textit{config} \times \textit{type} \hookrightarrow
  \textit{point}$, a partial function, in the following way:
  \begin{equation}
  \begin{aligned}[c]
  %%%%%%%%%%%%%%%%%%%%%%%%%%%%%%%%%%%%%%%%%
  %%%    Base cases (value and leaf)    %%%
  %%%%%%%%%%%%%%%%%%%%%%%%%%%%%%%%%%%%%%%%%
  \llbracket \gamma \Vdash \langle \rangle \rrbracket_{\textbf{unit}}
    &\triangleq * \\
  \llbracket \gamma \Vdash k \rrbracket_{\textbf{num}} &\triangleq k \\
  %%%%%%%%%%%%%%%%%%%%%%%%%%%%%%%%%%%%%%%%%
  %%%                                   %%%
  %%%%%%%%%%%%%%%%%%%%%%%%%%%%%%%%%%%%%%%%%
  % \rule{5em}{0.4pt}&\rule{30em}{0.4pt}\\
  % \llbracket \gamma [x \mapsto v] \Vdash x \rrbracket_{\tau} &\triangleq
  %   \llbracket \gamma \Vdash v \rrbracket_{\tau} \\
  %%%%%%%%%%%%%%%%%%%%%%%%%%%%%%%%%%%%%%%%%
  %%%          lifted rules             %%%
  %%%%%%%%%%%%%%%%%%%%%%%%%%%%%%%%%%%%%%%%%
  % \llbracket \gamma \Vdash (v, w) \rrbracket_{\tau_0 \otimes \tau_1} &\triangleq
  %   (\llbracket \gamma \Vdash v \rrbracket_{\tau_0}, \llbracket \gamma \Vdash w
  %   \rrbracket_{\tau_1}) \\
  % \llbracket \gamma \Vdash \lambda x . e \rrbracket_{\tau_0 \multimap \tau_1}
  %   &\triangleq \{ (\llbracket \sigma \Vdash v \rrbracket_{\tau_0}, \llbracket
  %   \gamma + \sigma \Vdash (\lambda x . e) v \rrbracket_{\tau_1}) \> | \>
  %   \forall \sigma \Vdash v \in \textit{config}, \llbracket \sigma \Vdash v
  %   \rrbracket_{\tau_0} \not= \bot \} \\
  % \llbracket \gamma \Vdash v \rrbracket_{!_s \tau} &\triangleq \llbracket \gamma
  %   \Vdash v \rrbracket_\tau \\
  % \llbracket \gamma \Vdash v \rrbracket_{\tau_0 + \tau_1} &\triangleq
  %   \begin{cases}
  %     (\llbracket \gamma \Vdash v \rrbracket_{\tau_0}, 0) \quad &\text{if well-defined} \\
  %     (0, \llbracket \gamma \Vdash v \rrbracket_{\tau_1}) \quad &\text{if well-defined}
  %   \end{cases} \\
  \rule{5em}{0.4pt}&\rule{30em}{0.4pt}\\
  \llbracket \sigma [x \mapsto v] \Vdash x \rrbracket_{\tau} &\triangleq
    \llbracket \sigma \Vdash v \rrbracket_{\tau} \quad \text{if well-defined} \\
  %%%%%%%%%%%%%%%%%%%%%%%%%%%%%%%%%%%%%%%%%
  \llbracket \sigma \Vdash (v, w) \rrbracket_{\tau_0 \otimes \tau_1} &\triangleq
    (\llbracket \sigma \Vdash v \rrbracket_{\tau_0}, \llbracket \sigma \Vdash w
    \rrbracket_{\tau_1}) \\
  \llbracket \sigma \Vdash \mathbf{op} \ v \rrbracket_{\textbf{num}}
    &\triangleq \textit{op} (\llbracket \sigma \Vdash v \rrbracket_{\tau_0})
    \quad \text{for \textbf{op} : $\tau_0 \multimap \textbf{num} \in \Sigma$} \\
  \llbracket \sigma_0; \sigma_1 \Vdash \mathbf{in_i} \ x \rrbracket_{\tau}
    &\triangleq \llbracket \sigma_i \Vdash x \rrbracket_{\tau} \\
  \llbracket \sigma_0 \Vdash v \rrbracket_{\tau_0 \multimap \tau_1}
    &\triangleq \{ (\llbracket \sigma_1 \Vdash w \rrbracket_{\tau_0}, \llbracket
    \sigma_0 @ \sigma_1 \Vdash v \> w \rrbracket_{\tau_1}) \> | \>
    \forall \sigma_1 \Vdash w \in \textit{config}, \llbracket \sigma_1 \Vdash w
    \rrbracket_{\tau_0} \not= \bot \} \\
    \llbracket s \cdot \sigma \Vdash v \rrbracket_{!_s \tau} &\triangleq \llbracket
    \sigma \Vdash v \rrbracket_\tau \\
  \llbracket \sigma \Vdash v \rrbracket_{\tau_0 + \tau_1} &\triangleq
    \begin{cases}
      (0, \llbracket \sigma \Vdash v \rrbracket_{\tau_0}) \quad &\text{if well-defined} \\
      (1, \llbracket \sigma \Vdash v \rrbracket_{\tau_1}) \quad &\text{if well-defined}
    \end{cases} \\
  %%%%%%%%%%%%%%%%%%%%%%%%%%%%%%%%%%%%%%%%%
  %%%            Tree cases             %%%
  %%%%%%%%%%%%%%%%%%%%%%%%%%%%%%%%%%%%%%%%%
  % \rule{5em}{0.4pt}&\rule{30em}{0.4pt}\\
  \llbracket \sigma; \sigma' \Vdash v \rrbracket_{\tau_0 \times \tau_1}
    &\triangleq (\llbracket \sigma \Vdash v \rrbracket_{\tau_0}, \llbracket
    \sigma' \Vdash v \rrbracket_{\tau_1}) \\
  \llbracket \sigma; \sigma' \Vdash v \rrbracket_{M_q \tau} &\triangleq
    (\llbracket \sigma \Vdash v \rrbracket_{\tau}, \llbracket \sigma' \Vdash v
    \rrbracket_{\tau}) \\
  %%%%%%%%%%%%%%%%%%%%%%%%%%%%%%%%%%%%%%%%%
  %%%            Expr case              %%%
  %%%%%%%%%%%%%%%%%%%%%%%%%%%%%%%%%%%%%%%%%
  \rule{5em}{0.4pt}&\rule{30em}{0.4pt}\\
  \llbracket \sigma \Vdash e \rrbracket_{\tau} &\triangleq \llbracket \sigma'
    \Vdash v\rrbracket_{\tau} \quad{\text{ if } \sigma \Vdash e
    \rightsquigarrow^* \sigma' \Vdash v}
  \end{aligned}
  \end{equation}
  This definition is well-founded. It is defined first over values. It is then
  defined over types. At any point, either the type is getting smaller or the
  type is the same size and the enviroment size is getting smaller. To be
  precise, the termination measure for the interpetation over values is defined
  in Definition~\ref{def:orderings}. Finally, the intepretation is defined over
  expressions that are not values, if it can be rewritten to a value.
\end{definition}

\begin{definition}[Environment compatibility]
  We have the following relation with type:
  $$- \textit{ \underline{com} } - \textit{ \underline{at} } - : \textit{env}
  \times \textit{ctx} \times \textit{point} \to \text{Prop}$$ 
  defined as follows:
  \begin{equation}
    \begin{normalsize}
    \begin{aligned}[c] 
      % unit case
      \sigma[x \mapsto * : \textbf{unit}] \textit{ \underline{com} } \Gamma, x:
      \textbf{unit} \textit{ \underline{at} } (\star, p) &\triangleq \sigma
      \textit{ \underline{com} } \Gamma \textit{ \underline{at} } p \\
      % num case
      \sigma[x \mapsto k : \textbf{num}] \textit{ \underline{com} } \Gamma, x:
      \textbf{num} \textit{ \underline{at} } (k, p) &\triangleq \sigma \textit{
        \underline{com} } \Gamma \textit{ \underline{at} } p \\
      &\text{where } k \in \mathbb{R} \\
      % var case
      \sigma[x \mapsto y] \textit{ \underline{com} } \Gamma, x : \tau \textit{
        \underline{at} } (p, q) & \triangleq \sigma \textit{ \underline{com} }
      \Gamma \textit{ \underline{at} } q \\
      & \text{ and } \llbracket \sigma \Vdash y \rrbracket_{\tau} = p\in
      \llbracket \tau \rrbracket \\
      % op case
      \sigma[x \mapsto \mathbf{op}(v)] \textit{ \underline{com} } \Gamma, x:
      \tau_1 \textit{ \underline{at} } (p, q) &\triangleq \sigma \textit{
        \underline{com} } \Gamma \textit{ \underline{at} } q \\ 
      & \text{ and } \llbracket \sigma \Vdash v \rrbracket_{\tau_0} = p \in
      \llbracket \tau_0 \rrbracket \\
      & \text{ where } op : \tau_0 \multimap \tau_1 \in \Sigma \\
      % function case
      \sigma[x \mapsto v] \textit{ \underline{com} } \Gamma, x : \tau_0
      \multimap \tau_1 \textit{ \underline{at} } (p, q) &\triangleq \sigma
      \textit{ \underline{com} } \Gamma \textit{ \underline{at} } q \\ 
      & \text{ and } \llbracket \sigma \Vdash v \rrbracket_{\tau_0 \multimap
      \tau_1} = p \in \llbracket \tau_0 \multimap \tau_1 \rrbracket
      % & \text{ and } \llbracket \sigma \Vdash \lambda y . e \rrbracket_{\tau_0
      %   \multimap \tau_1} =_{extenstional} f \\ 
      % & \text{ and } f \in \llbracket \tau_0 \multimap \tau_1 \rrbracket
        \footnote{This enforces that f is \text{1-sensitive}.} \\
      % remove :s case
      \sigma[x \mapsto v :_s \tau] \textit{ \underline{com} } \Gamma, x:_s \tau
      \textit{ \underline{at} } (p, q) &\triangleq
      \sigma \textit{ \underline{com} } \Gamma, x : \tau \textit{ \underline{at}
      } (p, q) \\
      & \text{ and } \llbracket \sigma \Vdash v \rrbracket_{\tau} = p \in
      \llbracket \tau \rrbracket \\
      % comonad case
      \sigma \textit{ \underline{com} } \Gamma, x: !_s \tau \textit{ \underline{at} } (p, q) &\triangleq
      \sigma \textit{ \underline{com} } \Gamma, x :_s \tau \textit{ \underline{at} } (p, q) \\
      % monad case
      \sigma; \sigma' \textit{ \underline{com} } \Gamma, x: M_r \tau \textit{
        \underline{at} } ((a, b), p) &\triangleq 
      \sigma \textit{ \underline{com} } \Gamma, x : \tau \textit{ \underline{at}
      } (a, p) \\
      & \text{ and } \sigma' \textit{ \underline{com} } \Gamma, x : \tau
      \textit{ \underline{at} } (b, p) \\
      & \text{ and } (\llbracket \sigma \Vdash x \rrbracket_{\tau}, \llbracket
      \sigma' \Vdash x \rrbracket_{\tau}) = (a, b) \in \llbracket M_r \tau \rrbracket \\
      % linear with case
      \sigma[x \mapsto v_0];\sigma[x \mapsto v_1] \textit{ \underline{com} }
      \Gamma, x : \tau_0 \times \tau_1 \textit{ \underline{at} } ((a, b), p) &\triangleq 
      \sigma \textit{ \underline{com} } \Gamma \textit{ \underline{at} } (a, p) \\ 
      & \text{ and } \sigma \textit{ \underline{com} } \Gamma \textit{ \underline{at} } (b, p) \\
      & \text{ and } \llbracket \sigma \Vdash v_0 \rrbracket_{\tau_0} = a \in
          \llbracket \tau_0 \rrbracket
        \text{ and } \llbracket \sigma \Vdash v_1 \rrbracket_{\tau_1} = b \in
          \llbracket \tau_1 \rrbracket \\
      % tensor case
      \sigma[x \mapsto (v_0, v_1)] \textit{ \underline{com} } \Gamma, x : \tau_0
      \otimes \tau_1 \textit{ \underline{at} } ((a, b), p) &\triangleq
      \sigma[x \mapsto v_0] \textit{ \underline{com} } \Gamma  \textit{
        \underline{at} } (a, p) \\
      & \text{ and } \sigma[x \mapsto v_1] \textit{ \underline{com} } \Gamma
        \textit{ \underline{at} } (b, p) \\
      & \text{ and } \llbracket \sigma \Vdash v_0 \rrbracket_{\tau_0} = a \in
          \llbracket \tau_0 \rrbracket
        \text{ and } \llbracket \sigma \Vdash v_1 \rrbracket_{\tau_1} = b \in
          \llbracket \tau_1 \rrbracket \\
      % sum case
      \sigma[x \mapsto v] \textit{ \underline{com} } \Gamma, x : \tau_0 + \tau_1 \textit{ \underline{at} } (p, q)
      &\triangleq 
      (\sigma[x \mapsto v] \textit{ \underline{com} } \Gamma, x : \tau_0
      \textit{ \underline{at} } (p, q) \\
      &\> \> \> \> \> \text{ and } \llbracket \sigma \Vdash v \rrbracket_{\tau_0} = p
      \in \llbracket \tau_0 \rrbracket) \\
      & \text{ or } (\sigma[x \mapsto v] \textit{ \underline{com} } \Gamma, x :
      \tau_1 \textit{ \underline{at} } (p, q) \\
      &\> \> \> \> \> \text{ and } \llbracket \sigma \Vdash v \rrbracket_{\tau_1} = p
      \in \llbracket \tau_1 \rrbracket) \\
    \end{aligned}
    \end{normalsize}
  \end{equation}
  % Note: We can rewrite this defintion in a different style (without Coq-style
  % props) if that's easier to understand.
  Then, an enviroment $\sigma$ is compatible with a typing context $\Gamma$ at
  point $p$ if the above relation holds. Intuitively, this means that there
  exists a point $p \in \llbracket \Gamma \rrbracket$ that $\sigma$ represents.
\end{definition}

\begin{lemma}[Unique point intepretation]
  For any enviroment $\sigma$ and context $\Gamma$, there is at most one point
  $p \in \llbracket \Gamma \rrbracket$ such that $\sigma \textit{
    \underline{com} } \Gamma \textit { \underline{at} } p$.
\end{lemma}
\begin{proof}
  TODO.
\end{proof}

\begin{definition}[Semantic well-typedness]
  An expression $e$ is semantically well-typed in a context $\Gamma$ at type
  $\tau$, written:
  $$\Gamma \vDash e : \tau$$
  if and only if for all $\sigma$ compatible with $\Gamma$, $\llbracket \sigma
  \Vdash e \rrbracket_{\tau} \in \llbracket \tau \rrbracket$ and $\llbracket -
  \Vdash e \rrbracket_{\tau}$ is a 1-sensitive map from the points $p$ at which
  $\sigma$ and $\Gamma$ are compatible with; sending $p \in \llbracket \Gamma
  \rrbracket$ to $\llbracket \tau \rrbracket$.
\end{definition}

Finally, we define $\Sigma_{\tau_0 \multimap \tau_1}$ to be the lifted set of
1-sensitive functions mapping $\llbracket \tau_0 \rrbracket$ to $\llbracket
\tau_1 \rrbracket$.

\begin{definition}[Operations in NumFuzz]
  \begin{equation}
    \Sigma_{\tau_0 \multimap \tau_1} \triangleq \{ f \ | \ f \in \llbracket \tau_0
    \multimap \tau_1 \rrbracket \}
  \end{equation}
\end{definition}
