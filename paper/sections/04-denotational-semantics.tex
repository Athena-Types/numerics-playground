% \begin{definition}[Type interpretation]
%   A type $\tau$ is interpreted with $\llbracket - \rrbracket : \textit{type} \to
%   \textbf{Met}$ in the same way as in the original NumFuzz system.
%   % TODO: put actual definition here.
% \end{definition}
%
%
% \begin{definition}[Typing context interpretation]
%   A typing context $\Gamma$ is interpreted with $\llbracket - \rrbracket :
%   \textit{context} \to \textbf{Met}$ in the following way:
%   \begin{equation}
%   \begin{aligned}[c]
%     \llbracket . \rrbracket &\triangleq . \\
%     \llbracket \Gamma, x :_s \tau \rrbracket &\triangleq \llbracket \Gamma \rrbracket
%       \times D_s \llbracket \tau \rrbracket
%   \end{aligned}
%   \end{equation}
% \end{definition}


% In the following section, 
% The following notation is to mirror the look of the Fuzz metric preservation
% theorem statement but contains differences in the setup. For example, our
% logical relation is neither coinductive nor step-indexed. It is also unary,
% using mutual (well-founded) recursion between the definition of a syntactic term
% falling within the logical relation and the definition of the syntactic distance
% between terms.

% probs needs some theorem relating SD to d, what do property do we want to
% enforce here? (probs need to factor out a lemma or something when doing the
% logical relations proof)

