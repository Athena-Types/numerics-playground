\section{Type inference}

\subsection{Subtyping}
We extend the subtyping relation from \cite{NumFuzz} to work with interval
bounds in the following way:

$$
\begin{aligned}[c]
\textbf{num}_{(k_0, k_1)} \sqsubseteq \textbf{num}_{(k'_0, k'_1)} \triangleq 
k'_0 \leq k_0 \leq k_1 \leq k'_1 \\
\textbf{num}_{i} \sqsubseteq \textbf{num}_{i'} \triangleq 
\forall \sigma : \Delta, \
\textbf{num}_{i}~[\sigma] \sqsubseteq \textbf{num}_{i'}~[\sigma]
\end{aligned}
$$

\begin{theorem}[Admissibility of subtyping]
  The admissibility of subtyping follows from the proof in \cite{NumFuzz} and the
new interval widening rule in our type system, corresponding to the
$\textbf{num}$ subtyping case.
\end{theorem}

\subsection{Algorithmic Type Inference}

Conceptually, we can think of our type inference and checking algorithm in two
stages. Users write programs in \Lang. We then perform the type sensitivity and
inference algorithm developed in \cite{NumFuzz}, extended to our core language
without bound polymorphism. Finally, we automatically infer polymorphic bound
annotations to generate a valid typing derivation of the program in \bnd{\Lang}.

\begin{equation}
  \begin{aligned}[c]
    add &: \bnd{\forall \epsilon_0, \epsilon_1 \ .} \
    \textbf{num}_{\bnd{\epsilon_0}} \times \textbf{num}_{\bnd{\epsilon_1}}
    \multimap \textbf{num}_{\bnd{\epsilon_0 + \epsilon_1}} \\
    sub &: \bnd{\forall \epsilon_0, \epsilon_1 \ .} \
    \textbf{num}_{\bnd{\epsilon_0}} \times \textbf{num}_{\bnd{\epsilon_1}}
    \multimap \textbf{num}_{\bnd{\epsilon_0 - \epsilon_1}} \\
    mult &: \bnd{\forall \epsilon_0, \epsilon_1 \ .} \
    \textbf{num}_{\bnd{\epsilon_0}} \times \textbf{num}_{\bnd{\epsilon_1}}
    \multimap \textbf{num}_{\bnd{\epsilon_0 \cdot \epsilon_1}} \\
  \end{aligned}
\end{equation}

\begin{example}[Unannotated program]
  \begin{equation}
    \lambda x \ . \ \mathbf{let} \ s \ = \ add \ x \ \tin \ (\mathbf{rnd} \ s)
    : \textbf{num} \times \textbf{num} \multimap \textbf{num}
  \end{equation}
\end{example}
\begin{example}[Annotated program]
  \begin{equation}
    \bnd{\Lambda \epsilon_0, \epsilon_1 \ .} \lambda x : num_{\bnd{\epsilon_0}}
    \times num_{\bnd{\epsilon_1}} \ . \ 
    \mathbf{let} \ s \ = \ add \ \bnd{\{\epsilon_0\}} \ \{\bnd{\epsilon_1\}} \ x
    \ \tin \ (\mathbf{rnd} \ s)
    : 
    \bnd{\forall \epsilon_0, \epsilon_1 \ .} \ 
    \textbf{num}_{\bnd{\epsilon_0}} \times \textbf{num}_{\bnd{\epsilon_1}}
    \multimap
    \textbf{num}_{\bnd{\epsilon_0 + \epsilon_1}}
  \end{equation}
\end{example}

\subsection{Actual setup}

We have context skeletons $\Gamma^{\bullet}$, type contexts $\Gamma$ without any
bounds, and fully annotated contexts $\bnd{\Delta \ | \ \Gamma}$ with bound
polymorphism. 
We also have erased programs $e^{\bullet}$ with polymorphic abstraction,
instantiation, etc. removed.

% The goal of type inference, then, is to take a program $e^{\bullet}$ in
% context skeleton $\Gamma^{\bullet}$ and produce a program, possibly using bound
% polymorphism, $e$, such that .

\begin{definition}[Type inference and checking problem]
Given an erased context skeleton $\Delta \ | \ \Gamma^{\bullet}$ and program
$e^{\bullet}$ in \Lang with erased type $\tau^{\bullet}$, produce a derivation
of $\Delta \ | \ \Gamma \vdash e : \tau$ for $e$ in \bnd{\Lang} if such a
derivation exists. We write this like: $$\Delta \ | \ \Gamma^{\bullet};
e^{\bullet} \implies \Delta \ | \ \Gamma'; e'; \tau$$
\end{definition}

\begin{figure}
%% ROW1
\begin{center}
\begin{equation}
\begin{aligned}[c]
\end{aligned}
\end{equation}

\AXC{}
\UIC{$I(\textbf{unit}^{\bullet}, \Gamma, \Delta) \to \textbf{unit}; .$}
\bottomAlignProof
\DisplayProof
\hskip 0.5em
% \AXC{$\epsilon \not \in FTV(\Gamma) \cup DOM(\Delta)$}
\AXC{$\epsilon \text{ globally fresh}$}
\UIC{$ I(\textbf{num}^{\bullet}, \Gamma, \Delta) \to \textbf{num}_{\epsilon};
\epsilon $}
\bottomAlignProof
\DisplayProof
\vskip 1em

\AXC{$I(\tau_0^{\bullet}, \Gamma, \Delta) \to \tau_0; \alpha$}
\AXC{$I(\tau_1^{\bullet}, \Gamma, \Delta) \to \tau_1; \beta$}
\BIC{$I((\tau_0 \times \tau_1)^{\bullet}, \Gamma, \Delta) \to \tau_0 \times
\tau_1; \alpha, \beta$}
\bottomAlignProof
\DisplayProof
\hskip 0.5em
\AXC{$I(\tau_0^{\bullet}, \Gamma, \Delta) \to \tau_0; \alpha$}
\AXC{$I(\tau_1^{\bullet}, \Gamma, \Delta) \to \tau_1; \beta$}
\BIC{$I((\tau_0 \otimes \tau_1)^{\bullet}, \Gamma, \Delta) \to \tau_0 \otimes
\tau_1; \alpha, \beta$}
\bottomAlignProof
\DisplayProof
\vskip 1em

\AXC{$I(\tau_0^{\bullet}, \Gamma, \Delta) \to \tau_0; \alpha$}
\AXC{$I(\tau_1^{\bullet}, \Gamma, \Delta) \to \tau_1; \beta$}
\BIC{$I((\tau_0 + \tau_1)^{\bullet}, \Gamma, \Delta) \to \tau_0 +
\tau_1; \alpha, \beta$}
\bottomAlignProof
\DisplayProof
\hskip 0.5em
\AXC{$I(\tau_0^{\bullet}, \Gamma, \Delta) \to \tau_0; \alpha$}
\AXC{$I(\tau_1^{\bullet}, \Gamma, \Delta) \to \tau_1; \beta$}
\BIC{$I((\tau_0 \multimap \tau_1)^{\bullet}, \Gamma, \Delta) \to \tau_0
\multimap \tau_1; \alpha, \beta$}
\bottomAlignProof
\DisplayProof
\vskip 1em

\AXC{$I(\tau^{\bullet}, \Gamma, \Delta) \to \tau; \beta$}
\UIC{$I(!_s (\tau^{\bullet}), \Gamma, \Delta) \to !_s \tau ; \alpha, \beta$}
\bottomAlignProof
\DisplayProof
\hskip 0.5em
\AXC{$I(\tau^{\bullet}, \Gamma, \Delta) \to \tau; \beta$}
\UIC{$I(M_u (\tau^{\bullet}), \Gamma, \Delta) \to M_u \tau ; \alpha, \beta$}
\bottomAlignProof
\DisplayProof
\vskip 1em

\vskip 3em

\AXC{}
\UIC{$E(\textbf{unit}, \Gamma, \Delta) \to .$}
\bottomAlignProof
\DisplayProof
\hskip 0.5em
\AXC{$\epsilon \text{ globally fresh}$}
\UIC{$ E(\textbf{num}_i, \Gamma, \Delta) \to i$}
\bottomAlignProof
\DisplayProof
\vskip 1em

\AXC{$E(\tau_0, \Gamma, \Delta) \to \alpha$}
\AXC{$E(\tau_1, \Gamma, \Delta) \to \beta$}
\BIC{$E((\tau_0 \times \tau_1), \Gamma, \Delta) \to \alpha, \beta$}
\bottomAlignProof
\DisplayProof
\hskip 0.5em
\AXC{$E(\tau_0, \Gamma, \Delta) \to \alpha$}
\AXC{$E(\tau_1, \Gamma, \Delta) \to \beta$}
\BIC{$E((\tau_0 \otimes \tau_1), \Gamma, \Delta) \to \alpha, \beta$}
\bottomAlignProof
\DisplayProof
\vskip 1em

\AXC{$E(\tau_0, \Gamma, \Delta) \to \alpha$}
\AXC{$E(\tau_1, \Gamma, \Delta) \to \beta$}
\BIC{$E((\tau_0 + \tau_1), \Gamma, \Delta) \to \alpha, \beta$}
\bottomAlignProof
\DisplayProof
\hskip 0.5em
\AXC{$E(\tau_0, \Gamma, \Delta) \to \alpha$}
\AXC{$E(\tau_1, \Gamma, \Delta) \to \beta$}
\BIC{$E((\tau_0 \multimap \tau_1), \Gamma, \Delta) \to \alpha, \beta$}
\bottomAlignProof
\DisplayProof
\vskip 1em

\AXC{$E(\tau, \Gamma, \Delta) \to \beta$}
\UIC{$E(!_s (\tau), \Gamma, \Delta) \to \alpha, \beta$}
\bottomAlignProof
\DisplayProof
\hskip 0.5em
\AXC{$E(\tau, \Gamma, \Delta) \to \beta$}
\UIC{$E(M_u (\tau), \Gamma, \Delta) \to \alpha, \beta$}
\bottomAlignProof
\DisplayProof
\end{center}
    \caption{Ancillary functions for defining algorithmic type inference.}
    \label{fig:helper_type_inference}
\end{figure}

\newpage

\begin{figure}
\begin{center}
%% unit
\AXC{}
\RightLabel{(Unit)}
\UIC{$\Delta \ | \ \Gamma^{\bullet}; \langle \rangle \implies \Delta \ | \
\Gamma^{0}; \langle \rangle; \textbf{unit}$}
\bottomAlignProof
\DisplayProof
\vskip 1em

%% var
\AXC{}
\RightLabel{(Var)}
\UIC{$\Delta \ | \ \Gamma^{\bullet}, x : \sigma; \langle \rangle \implies \Delta \ | \
\Gamma^{0}, x :_1 \sigma; x; \sigma$}
\bottomAlignProof
\DisplayProof
\vskip 1em

%% bang intro
\AXC{$\Delta \ | \ \Gamma^{\bullet}; [e]_{\annotate{s}} \implies \Delta' \ | \
\Gamma; [e]_{\annotate{s}}; \tau$}
\RightLabel{($!$ I)}
\UIC{$\Delta \ | \ \Gamma^{\bullet}; [e]_{\annotate{s}} \implies \Delta' \ | \
s * \Gamma; [e]_{\annotate{s}}; !_s~\tau$}
\bottomAlignProof
\DisplayProof
\vskip 1em

%% fun
\AXC{$I(\tau_0^{\bullet}, \Gamma^{\bullet}, \Delta) \to \tau_0; \epsilon_0, \ldots,
\epsilon_n$}
\AXC{$\Delta, \epsilon_0 : \textbf{bnd} , \ldots, \epsilon_n : \textbf{bnd}
\ | \ \Gamma^{\bullet}, x : \tau_0; e^{\bullet} \implies \Delta' \ | \ \Gamma, x
:_s \tau_0; e; \tau$}
% \AXC{$\epsilon_0, \ldots, \epsilon_n \not\in FTV(\Gamma) \cup DOM(\Delta)$}
\RightLabel{($\multimap$ I)}
\BIC{$\Delta \ | \ \Gamma^{\bullet}; \lambda (x : \annotate{\tau_0^{\bullet}}).
e^{\bullet} \implies \Delta' \ | \ \Gamma; \Lambda \epsilon_0, \ldots,
\epsilon_n (\lambda (x : \tau_0) . e); \tau_0 \multimap
\tau$}
\bottomAlignProof
\DisplayProof
\vskip 1em

%% imp elim
\AXC{$\Delta \ | \ \Gamma^{\bullet}; f^{\bullet} \implies \Delta'_0 \ | \ \Gamma_0; f; \tau_0 \multimap \tau$}
\AXC{$\Delta \ | \ \Gamma^{\bullet}; e^{\bullet} \implies \Delta'_1 \ | \ \Gamma_1; e; \tau'_0$}
\AXC{$\tau'_0 \sqsubseteq \tau_0$}
\RightLabel{($\multimap$ E)}
\TIC{$\Delta \ | \ \Gamma^{\bullet}; f^{\bullet}~e^{\bullet} \implies \Delta_0 +
\Delta_1 \ | \ \Gamma_0 + \Gamma_1 ; f~e; \tau$}
\bottomAlignProof
\DisplayProof
\vskip 1em

%% imp elim, forall
\AXC{$\begin{aligned}[t]
E(\tau_0, \Gamma^{\bullet}, \Delta) &\to i_0, \ldots, i_n \\
\tau'_0 &\sqsubseteq \tau_0[\epsilon_0 / i_0, \ldots, \epsilon_n / i_n] \\
\end{aligned}
$}
\AXC{$
\begin{aligned}[t]
\Delta \ | \ \Gamma^{\bullet}; e^{\bullet} &\implies \Delta'_0 \ | \ \Gamma_0; f;
\forall \epsilon_0, \ldots, \epsilon_n \ . \ \tau_0 \multimap \tau \\
\Delta \ | \ \Gamma^{\bullet}; f^{\bullet} &\implies \Delta'_1 \ | \ \Gamma_1;
e; \tau'_0 \\
\end{aligned}
$}
\AXC{$
\begin{aligned}[t]
\tau &\mapsto^{*} \tau' \\
\tau' &\not\mapsto 
\end{aligned}
$}
\RightLabel{($\forall \multimap$ E)}
\TIC{$\Delta \ | \ \Gamma^{\bullet}; f^{\bullet}~e^{\bullet} \implies \Delta'_0
+ \Delta'_1 \ | \ \Gamma_0 + \Gamma_1 ; (f~\{ i_0 \}~\ldots~\{ i_n \})~e; \tau'$}
\bottomAlignProof
\DisplayProof
\vskip 1em

\RightLabel{(\textbf{op})}
\AXC{$\{\textbf{op} : \tau\} \in \Sigma$}
\UIC{$\Delta \ | \ \Gamma^{\bullet}; \textbf{op} 
\implies \Delta \ | \ \Gamma^{0}; \textbf{op}; \tau$}
\bottomAlignProof
\DisplayProof
\vskip 1em

\RightLabel{(\textbf{factor})}
\AXC{$\Delta \ | \ \Gamma^{\bullet}; e^{\bullet} 
\implies \Delta' \ | \ \Gamma; e; (M_q \tau_0 \times M_r \tau_1)$}
\UIC{$\Delta \ | \ \Gamma^{\bullet}; \textbf{factor}~e^{\bullet} \implies \Delta
\ | \ \Gamma; \textbf{factor}~e; M_{max(q, r)} (\tau_0 \times \tau_1)$}
\bottomAlignProof
\DisplayProof
\vskip 1em

\vskip 3em

%% times intro 
\AXC{$\Delta \ | \ \Gamma^{\bullet}; e^{\bullet} \implies \Delta'_0 \ | \ \Gamma_0; e; \tau_0$}
\AXC{$\Delta \ | \ \Gamma^{\bullet}; f^{\bullet} \implies \Delta'_1 \ | \ \Gamma_1; f; \tau_1$}
% \AXC{$
% \begin{aligned}[t]
% \Delta'_0 \sqsubseteq \Delta' \\
% \Delta'_1 \sqsubseteq \Delta' \\
% \end{aligned}
% $}
% \AXC{$
% \begin{aligned}[t]
% \Gamma'_0 \sqsubseteq \Gamma' \\
% \Gamma'_1 \sqsubseteq \Gamma' \\
% \end{aligned}
% $}
\RightLabel{($\times$ I)}
\BIC{$\Delta \ | \ \Gamma^{\bullet}; \langle f^{\bullet}, e^{\bullet} \rangle
\implies \Delta'_0 + \Delta'_1 \ | \ max(\Gamma'_0, \Gamma'_1); \langle f, e
\rangle; \tau_0 \times \tau_1$}
\bottomAlignProof
\DisplayProof
\vskip 1em
% todo: go over with Justin this

%% otimes intro 
\AXC{$\Delta \ | \ \Gamma^{\bullet}; e^{\bullet} \implies \Delta'_0 \ | \ \Gamma_0; e; \tau_0$}
\AXC{$\Delta \ | \ \Gamma^{\bullet}; f^{\bullet} \implies \Delta'_1 \ | \ \Gamma_1; f; \tau_1$}
\RightLabel{($\otimes$ I)}
\BIC{$\Delta \ | \ \Gamma^{\bullet}; \langle f^{\bullet}, e^{\bullet} \rangle
\implies \Delta'_0 + \Delta'_1 \ | \ \Gamma'_0 + \Gamma'_1; (f, e); \tau_0 \otimes \tau_1$}
\bottomAlignProof
\DisplayProof
\vskip 1em

\RightLabel{(\textbf{let-bind})}
\AXC{$\Delta \ | \ \Gamma^{\bullet}; e^{\bullet} \implies \Delta_0 \ | \ \Gamma; e; M_r~\tau_0$}
\AXC{$\Delta \ | \ \Gamma^{\bullet}, x; f^{\bullet} \implies \Delta_1 \ | \ \Theta, x
:_s \tau_0; M_q~\tau$}
\BIC{$\Delta \ | \ \Gamma^{\bullet}; \textbf{let-bind} \ x \ = \ e^{\bullet} \
\textbf{in} \ f^{\bullet} \implies \Delta_0 + \Delta_1 \ | \ s * \Gamma +
\Theta; \textbf{let-bind} \ x \ = \ e \ \textbf{in} \ f; M_{s*r + q}~\tau$}
\bottomAlignProof
\DisplayProof
\vskip 1em

\end{center}
    \caption{Algorithmic rules. \annotate{Red text} are user-required
    annotations. To help define the $\implies$ relation, we define a $\to$
    relation that annotates (removes the bullet) for a given type.}
    \label{fig:type_inference}
\end{figure}


\section{Implementation and language instantiation}
