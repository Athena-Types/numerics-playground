\section{Encoding} \label{sec:encoding}
In this section we encode Numerical Fuzz to provide \textit{a priori} and
\textit{a posterori} error bounds in the presence of subtraction and negative
numbers.
In Section~\ref{sec:instantiation}, we instantiate Numerical Fuzz using the modular interface defined in Definition~\ref{def:numfuzz-interface} using a $\textit{paired representation}$ introduced below.
Then, in Section~\ref{sec:application} we demonstrate how to read out the $\textit{paired-representation}$ as concrete bounds for standard, unpaired, floating-point arithmetic.

\subsection{Instantiation} \label{sec:instantiation}
Our forwards error analysis relies on bounding Lipschitz function sensitivity.
However, subtraction, and by extension addition with negative numbers, has
unbounded Lipschitz sensitivity over the reals. This poses a challenge to a
compositional type-based analysis of programs involving subtraction and negative
numbers.
To avoid this problem, we encode our semantics through a \textit{paired
representation} where we associate for each real $r \in \mathbb{R}$ a pair $(r,
a, b) \in \mathbb{P} = \mathbb{R} \times \mathbb{R}^+ \times \mathbb{R}^+ = \mathit{num}$ such that $r = a -
b$. 
We measure distance in $\mathbb{P}$ via the following function: $d_\mathbb{P}((r, a, b), (r, a', b'))
\triangleq max(d_\mathbb{R}(a,a'), d_\mathbb{R}(b,b'))$.

\begin{lemma}[$\textit{num}$ forms a metric space.]
\end{lemma}
\begin{proof}
Symmetry and reflexivity hold trivially. Triangle inequality holds from triangle
inequality on $d_{\mathbb{R}}$.
\end{proof}

% To avoid this problem, we encode our semantics through a \textit{paired
% representation} where we associate for each real $r \in \mathbb{R}$ a pair $(a,
% b) \in \mathbb{R}^2$ such that $r = a - b$. 
Our operations and interval bounds operate over the paired representation $\mathbb{P}$.
In this manner, subtraction and negative numbers can be encoded using bounded-sensitivity operations over the non-negative reals, avoiding the problem of infinite sensitivity.
Each of our operations $\Sigma = \{ \textbf{add}, \ \textbf{add}, \ \textbf{mul}
\}$ and $\Sigma_\textbf{num} = \{ +, -, \cdot \}$ are defined as follows:

\begin{equation}
\begin{aligned}[c]
  &\textbf{add} : \forall \epsilon_1, \epsilon_2, \mathbf{num}_{\epsilon_1} \times
  \mathbf{num}_{\epsilon_2} \multimap \mathbf{num}_{\epsilon_1 + \epsilon_2} \\
  &\textbf{add} \ \bnd{\{\epsilon_1\}} \ \bnd{\{\epsilon_2\}} \ (\langle (r, a,
  b), (r', a', b') \rangle) \mapsto (r + r', a + a', b + b') \\
  \\
  &\textbf{sub} : \forall \epsilon_1, \epsilon_2, \mathbf{num}_{\epsilon_1} \times
  \mathbf{num}_{\epsilon_2} \multimap \mathbf{num}_{\epsilon_1 - \epsilon_2} \\
  &\textbf{sub} \ \bnd{\{\epsilon_1\}} \ \bnd{\{\epsilon_2\}} \ (\langle (r, a, b),
  (r, a', b') \rangle ) \mapsto (r - r', a + b', b + a') \\
  \\
  &\textbf{mul} : \forall \epsilon_1, \epsilon_2, \mathbf{num}_{\epsilon_1} \times
  \mathbf{num}_{\epsilon_2} \multimap \mathbf{num}_{\epsilon_1 \cdot \epsilon_2} \\
  &\textbf{mul} \ \bnd{\{\epsilon_1\}} \ \bnd{\{\epsilon_2\}} \ (((r, a, b), (r, a',
  b'))) \mapsto (r * r', a * b + b * b', b * a' + a * b') \\
\end{aligned}
\end{equation}

Given a rounding function $\rho_\mathbb{R}$ over the reals, we can also
associate a paired $\rho_{\mathbb{P}}((r, a, b)) = (\rho_{\mathbb{R}}(r), a
\cdot \frac{\rho_{\mathbb{R}}(r)}{r}, b \cdot \frac{\rho_{\mathbb{R}}(r)}{r})$. 
It is sound to use the same round-off error bound $u$ associated with $\rho_\mathbb{R}$ for $\rho_\mathbb{P}$:

\begin{lemma}[Property of rounding and constant parameter $u$]
We wish to show that given a $u$ and $\rho_{\mathbb{R}}$ such that 
    $\forall e \in d_\mathbb{R} (\rho(e), e)
    \leq u$, then
    $\forall e \in \mathcal{R}_{num}, \mathcal{SD}_{\mathbf{num}}(\rho_{\mathbb{P}}(e), e)
    \leq u$. In other words, our translated rounding function from the unpaired representation respects the error bounds on the unpaired representation.
\end{lemma}
\begin{proof}
$(\rho_{\mathbb{P}}((r, a, b)), (r, a, b)) = 
(\rho_{\mathbb{R}}(r), a
\cdot \frac{\rho_{\mathbb{R}}(r)}{r}, b \cdot \frac{\rho_{\mathbb{R}}(r)}{r})$ So, 
$
d_{\mathbb{P}}((\rho_{\mathbb{R}}(r), a
\cdot \frac{\rho_{\mathbb{R}}(r)}{r}, b \cdot \frac{\rho_{\mathbb{R}}(r)}{r}), (r, a, b)) = max(\rho_{\mathbb{R}}(r), \rho_{\mathbb{R}}(r)) = \rho_{\mathbb{R}} (r)
$.
\end{proof}

Moreover, our operations satisfy metric preservation and our interval operations
are closed over intervals ($\textit{num}^2$).
% todo: these are closed over something other than num?

\begin{lemma}[Operations satisfy metric preservation]
Each $\textbf{op} : \tau \in \Sigma$ satisfies metric preservation.
\end{lemma}
\begin{proof}
Holds by unfolding.
\end{proof}


\begin{lemma}[Interval operations closure.]
Each interval operation is closed over our intervals $\textit{num}^2$.
\end{lemma}
\begin{proof}
Holds by inspection.
\end{proof}

By stapling together the proceeding lemmas, the interface ($\mathbb{P}, d_{\mathbb{P}}, \rho_{\mathbb{P}}, u, \Sigma, \Sigma_{\mathbf{num}}$) satisfy the interface for instantiating Numerical Fuzz specified in Definition~\ref{def:numfuzz-interface}.

\subsection{Application} \label{sec:application}
For a given $. \vdash e : M_q \textbf{num}_{(k_0, k_1)}$, we wish to bound the
error between $e$ computed ideally over the reals versus the floating-point approximation. We now prove our paired error theorems (Theorem~\ref{thm:paired-a-posteriori}, Theorem~\ref{thm:paired-a-priori}), which for a triple $(r, a, b)$ relates error on the paired components $a, b$ with error on $r$.

\begin{theorem}[Paired error theorem; a posteriori] \label{thm:paired-a-posteriori}
For $. \vdash e : M_q~\mathbf{num}_{(k_0, k_1)}$, $e \mapsto ((r, a, b), (r', a', b'))$ where $d_{\mathbb{R}}(r, r')$ is less than or equal to all of the following bounds: 
\begin{enumerate}
  \item $max(|ln(e^u + \frac{a}{r'}(1-e^{2u}))|,|ln(e^{-u} + \frac{a}{r'}(1-e^{-2u}))|)$
  \item $max(|ln(e^{-u} + \frac{b}{r'}(e^{-2u}- 1))|,|ln(e^u + \frac{b}{r'}(e^{2u}-1))|)$
\end{enumerate}
\end{theorem} 
\begin{proof}
  By our logical relation and metric preservation theorem (Theorem
  \ref{thm:metric-preservation}), we know that $e \mapsto ((r, a, b), (r', a', b'))$ such that $d_{\mathbb{P}}((r, a, b), (r', a', b')) \leq q$.
  Therefore, $d_{\mathbb{R}}(a, a'), d_{\mathbb{R}}(b, b') \leq q$.
  Unfolding the definition of $d_\mathbb{R}$, which uses relative error, we wish
  to bound $|ln(\frac{r}{r'})| = |ln(\frac{a - b}{a' - b'})|$.
  We proceed to prove the first bound:

  There are two cases, $r \leq r'$ or $r > r'$. In the first case, we wish to
  minimize $a-b$ and maximize $a' - b'$. So, 
  \begin{equation} \label{eq:a-priori-bnd-a-pos}
  \begin{aligned}[c]
    |ln(\frac{r}{r'})| \leq 
    |ln(\frac{a - b'e^u}{a' - b'})| =
    |ln(\frac{a - b' e^u + a'e^u - a'e^u}{a' - b'})| \\ =
    |ln(e^u + \frac{a - a'e^u}{a' - b'})| \leq
    |ln(e^u + \frac{a - ae^{2u}}{a' - b'})| =
    |ln(e^u + \frac{a}{r'}(1-e^{2u}))|
  \end{aligned}
  \end{equation}
  Similarly, for the other case, we have:
  \begin{equation} \label{eq:a-priori-bnd-a-neg}
  \begin{aligned}[c]
    |ln(\frac{r}{r'})| \leq 
    |ln(\frac{a - b'e^{-u}}{a' - b'})| =
    |ln(\frac{a - b'e^{-u} + a'e^{-u} - a'e^{-u}}{a' - b'})| \\ =
    |ln(e^{-u} + \frac{a - a'e^{-u}}{a' - b'})| \leq
    |ln(e^{-u} + \frac{a - ae^{-2u}}{a' - b'})| =
    |ln(e^{-u} + \frac{a}{r'}(1-e^{-2u}))|
  \end{aligned}[c]
  \end{equation}
  A sound bound for both cases is the maximum of Equation~\ref{eq:a-priori-bnd-a-pos} and Equation~\ref{eq:a-priori-bnd-a-neg}, which is exactly our first bound (1).
  The proof strategy for the second bound (2) mirrors that of the first bound.
\end{proof}

\begin{theorem}[Paired error theorem; a priori] \label{thm:paired-a-priori}
  For $. \vdash e : M_q~\mathbf{num}_{(k_0, k_1)}$, $e \mapsto ((r, a, b), (r', a', b'))$ where $d_{\mathbb{R}}(r, r')$ is less than or equal to all of the following bounds: 
\begin{enumerate}
  \item $max(|ln(e^{u} + \frac{a}{a-b}(e^{-u}- e^{u}))|, |ln(e^{-u} + \frac{a}{a-b}(e^{u}- e^{-u}))|)$
  \item $max(|ln(e^u + \frac{b}{a-b}(e^u - b e^{-u}))|,|ln(e^u + \frac{b}{a-b}(e^u - e^{-u}))|)$
\end{enumerate} 
\end{theorem} 
\begin{proof}
  Our error metric is symmetric so for this theorem we wish to prove a bound on
  $d_{\mathbb{R}}(r', r)$. Similarly to Theorem~\ref{thm:paired-a-posteriori},
  we prove each bound and, for each bound, split into two cases: $r \leq r'$ or
  $r > r'$. In the first case, we have:
  \begin{equation} \label{eq:a-priori-bnd-a-pos}
  \begin{aligned}[c]
    |ln(\frac{r'}{r})| \leq 
    |ln(\frac{a' - be^u}{a - b})| =
    |ln(\frac{a' - be^u + ae^u - ae^u}{a - b})| \\ =
    |ln(e^u + \frac{a' - ae^u}{a - b})| \leq
    |ln(e^u + \frac{ae^{-u} - ae^u}{a - b})| =
    |ln(e^u + \frac{a}{a - b}(e^{-u} - e^u))|
  \end{aligned}
  \end{equation}
  In the second case, we have:
  \begin{equation} \label{eq:a-priori-bnd-a-neg}
  \begin{aligned}[c]
    |ln(\frac{r'}{r})| \leq 
    |ln(\frac{a' - be^{-u}}{a - b})| =
    |ln(\frac{a' - be^{-u} + ae^{-u} - ae^{-u}}{a - b})| \\ =
    |ln(e^{-u} + \frac{a' - ae^{-u}}{a - b})| \leq
    |ln(e^{-u} + \frac{ae^{u} - ae^{-u}}{a - b})| =
    |ln(e^{-u} + \frac{a}{a - b}(e^{u} - e^{-u}))|
  \end{aligned}
  \end{equation}
  A sound bound for both cases is the maximum of Equation~\ref{eq:a-priori-bnd-a-pos} and Equation~\ref{eq:a-priori-bnd-a-neg}, which is exactly our first bound (1).
  The proof strategy for the second bound (2) mirrors that of the first bound.
\end{proof}

% TODO: detail how to plug in?

\subsection{Tightness} \label{sec:tightness}
A natural question to ask is whether the paired representation introduced in the
proceeding section leads to looser error bounds on programs that could have been
typed using the standard, unpaired representation. In the following section, we
demonstrate that for the programs obtained using the unpaired representation,
our bounds obtained through the paired representation are no looser.

To demonstrate this, we instantiate our language with the following interface
$(\mathbb{P}^{+}, d, \rho_{\mathbb{P}}, u, \Sigma = \{ \mathbf{add},
\mathbf{mul} \}, \Sigma_{\mathbf{num}})$, where $\mathbb{P}^+ = \{(r, r, 0) \ | \ \forall r \in \mathbb{R}^{+} \}$ and $d((r, a, b), (r', a', b')) = d_{\mathbb{R}}(r, r')$.
Let us write programs derivable in this type system using the $\vdash_{\mathbb{R}}$ judgement.
We observe that programs step the same in both instantiations of the language
and also that programs derivable in $\vdash_{\mathbb{R}}$ have the same type
with $\vdash$.
\begin{lemma}[Derivablity]
If $\Delta \ | \ \Gamma \vdash_{\mathbb{R}} e : \tau$,
then
$\Delta \ | \ \Gamma \vdash e : \tau$ 
\end{lemma}
\begin{proof}
  $\vdash_{\mathbb{R}}$ has strictly less operations and the associated rounding
  functions $\rho_{\mathbb{P}}$ and $u$ bounds are the same.
\end{proof}

We also observe that for a $. \vdash_{\mathbb{R}} e :  M_q \mathbf{num}_i$, $q$ suffices to bound the
maximum permitted round-off error.

\begin{theorem}[Unpaired error theorem]
For a program $. \vdash_{\mathbb{R}} e :  M_q \mathbf{num}_i$, we have that $ e \mapsto^{*} ((r, a,
b), (r', a', b'))$ where $d_{\mathbb{R}}(r, r') \leq q$.
\end{theorem}
\begin{proof}
  Holds by our logical relation and error metric.
\end{proof}

We further observe that for a program with non-negative numbers and only-growing
functions (e.g. $\mathbf{add}, \mathbf{mul}$), our type inference algo 

TODO: state type inference algo theorem. A bit out of order tho.

We are now ready to state our tightness theorem which guarantees that the paired
representation produced error bounds no looser than the unpaired representation.

\begin{theorem}[Tightness]
For a $. \vdash_{\mathbb{R}} e :  M_q \mathbf{num}_i$, we know that $. \vdash e :  M_q \mathbf{num}_i$ and $e \mapsto^{*} ((r, a, b), (r', a', b'))$ where the following bounds are equivalent:
\begin{itemize}
  \item $q$
  % \item $max(|ln(e^u + \frac{a}{r'}(1-e^{2u}))|,|ln(e^{-u} + \frac{a}{r'}(1-e^{-2u}))|)$
  \item $max(|ln(e^{-u} + \frac{b}{r'}(e^{-2u}- 1))|,|ln(e^u + \frac{b}{r'}(e^{2u}-1))|)$ (a posteriori)
  % \item $max(|ln(e^{u} + \frac{a}{a-b}(e^{-u}- e^{u}))|, |ln(e^{-u} + \frac{a}{a-b}(e^{u}- e^{-u}))|)$
  \item $max(|ln(e^u + \frac{b}{a-b}(e^u - b e^{-u}))|,|ln(e^u + \frac{b}{a-b}(e^u - e^{-u}))|)$ (a priori)
\end{itemize}
\end{theorem}
\begin{proof}
Observe that $b$ is zero by construction of the unpaired instantiation of our
language. TODO.
\end{proof}

% TODO: clarify how to start / use interval analysis, stitch together theorems
