\section{Language} \label{sec:lang}
We are interested in developing a modular family of languages for analyzing
numerical round-off error. Modularity enables differing floating-point bit
precisions (e.g. 32-bit floats, 64-bit floats) and floating-point operations to
be soundly instantiated.

% In this section, we detail the construction of our modular family of languages.
% We extend the Numerical Fuzz family of languages to support two key features:
% \begin{enumerate}
%   \item Error and sensitivity sharing between terms, through the addition of a
%   \textbf{factor} primitive (Section~\ref{sec:error-sharing}). This enables more
%   programs to be typed with tighter error bounds.
%   % todo: add section ref
%   \item Interval analysis, through the incorporation of \textit{bound
%   polymorphism} (Section~\ref{sec:bound-poly}). In
%   (Section~\ref{sec:tightness}), we will prove that this will result in error
%   bounds no looser than prior type-based approaches for forwards analysis.
% \end{enumerate}
% Finally, we
% provide a formal specification for soundly instantiating a particular
% language and prove soundness via a logical relations argument.
%
% \subsection{Sharing Error} \label{sec:error-sharing}
%
% \subsection{Type-based Interval Analysis and Bound Polymorphism} \label{sec:bound-poly}

\subsection{Syntax}
We present the full syntax for types, terms, and evaluation contexts in
Figure~\ref{fig:syntax}. Our language is an extension of the Numerical Fuzz
\cite{NumFuzz} language, a call-by-value affine lambda calculus.

\subsubsection*{Types}
Our type system has monad types $M_u \tau$ for tracking the round-off error in
$\tau$ and bounding it by a real, non-negative grade $u$, scaled metric types
$!_s \tau$, and linear function types $\tau_0 \multimap \tau$ for 1-sensitive
functions. We also also incorporate an interval-style analysis by annotating
$\textbf{num}_{\bnd{i}}$ types with a subscript bound with the grammar $\bnd{i}$
shown in Figure~\ref{fig:syntax}. For the remainder of this paper we use the
color $\bnd{\textit{blue}}$ to denote bound polymorphism.

Many programs we wish to type call the same function many times. To ensure that
our interval analysis is scalable and that functions types can be reused, we
support \textit{bound polymorphism}, which allows us to specialize our function
to have different concrete bounds for each function call site. We write types
$\tau$ polymorphic in interval variable $\epsilon$ as $\forall \epsilon. \tau$.

\subsubsection*{Terms}
Our term syntax has explicit terms to represent polymorphic abstraction
$\bnd{(\Lambda \epsilon. e)}$ and instantiation $\bnd{(e \{i\})}$. We also have
explicit terms to represent scaling ($[e]$), rounding ($\textbf{rnd}~e$).
For sequencing and combining computation, we have $\textbf{let}$,
$\textbf{let-pair}$, $\textbf{let-bind}$, and $\textbf{let-cobind}$ for
sequencing assignment, tuple unpacking, monadic, and comonadic computations
respectively.

In linear and affine logic, we have the alternative conjunction $\tau_0 \times
\tau_1$, (sometimes written $\tau_0~\&~\tau_1$). We can view the introduction
rule as allowing $\tau_0$ and $\tau_1$ to share resources in $\Gamma$, their
construction. Correspondingly, the elimination rules can be viewed as forcing
consumers to internally choose between allocating the resources in $\Gamma$
towards either constructing $\tau_1$ or $\tau_2$.

However, the ability to share resources interacts poorly with our sequenced
$\textbf{let-bind}$ and call-by-value evaluation strategy, which pessimistically
prohibits the sharing of resources between the bound argument and the body. As a
toy example, if we have two monadic types $M_q \tau_0 \times M_q \tau_1$ that
shared resources in their construction, we have no way of constructing $M_q
(\tau_0 \times \tau_1)$ with a shared error grade. A similar problem exists for
sharing context sensitivities. To enabling the sharing of round-off and
sensitivity information, we introduce a new primitive new primitive
($\textbf{factor} : (M_q \tau_0 \times M_r \tau_1) \multimap M_{max(q, r)}
(\tau_0 \times \tau_1)$) which allows for error and sensitivity information to
be shared between monadic terms.


\subsubsection*{Evaluation contexts}
We extend the Numerical Fuzz language to allow expressions in more places. To
separate the structural plumbing of the operational semantics from the more
interesting portions of the operational semantics, we define evaluation
contexts.

\section{Language Syntax}

\begin{figure}[tbp]
  \begin{alignat*}{3}
         &\text{Types } \sigma, \tau &::=~ &\mathbf{unit}
         \mid \num
         \mid \sigma \times \tau 
         \mid \sigma \otimes \tau
         \mid \sigma + \tau 
         \mid \sigma \multimap \tau
         \mid {\bang{s} \sigma}
         \mid {M_u \tau}
         \\
         &\text{Values } v, w \ &::=~ &\langle \rangle
         \mid k \in R
         \mid \langle v,w \rangle 
         \mid (v, w)
         \mid \tin_i \ v
         \mid \lambda x.~e \\
         % & & & \mid [v]
         % \mid\rnd v
         % \mid {\ret v} 
         % \mid \factor v
         % \mid \letbind x = v \ \tin \ f 
         \\
         &\text{Terms } e, f, g &::=~ &x
         \mid v
         \mid \mathbf{op}(e)
         \mid e~f
         \mid {\pi}_i\ e
         \mid \langle e,f \rangle 
         \mid (e, f) \\
         & & & \mid \letpair \ (x,y) = e \ \tin \ f
         \mid \letassign x  = e \ \tin \ f \\
         & & & \mid \tin_i \ e
         \mid 
          \mathbf{case} \ e \ \mathbf{of} \ (\tin_1 \ x.f \ | \ \tin_2 \ x.g) \\
         & & &
         \mid [e]
         \mid \rnd e
         \mid {\ret e} 
         \mid \factor e \\
         & & & 
         \mid {\letbind x = e \ \tin \ f}
         \mid \letcobind x = e \ \tin \ f
         \\
         % Max: find a better name
         &\text{Enviroments } \gamma_0, gamma_1 &::=~ &.
         \mid \gamma, \ x \mapsto v :_s \tau \\
         &\text{Enviroment trees } \sigma &::=~ & \gamma
         \mid \gamma_0; \gamma_1
  \end{alignat*}
  \caption{
    Types, values, and terms. 
    $\mathbf{op} \in \mathcal{O}$.
    $i \in \{1, 2\}$. 
  }
  \label{fig:syntax}
\end{figure}

% Max: For now, I've separated \letbind, \letcobind, \letpair, and
% \letassign to make things more clear.



\subsection{Static Semantics}
Our static semantics extends Numerical Fuzz by adding expressions in more places
(e.g. the application rule), a $\textbf{factor}$ primitive, and bound
polymorphism. We outline the two key additions below.

\subsubsection*{Factoring and Distributing Error}
Suppose, by way of example, that we wish to compute the sum 
$w~\tilde{+}~x~\tilde{+}~y~\tilde{+}~z$, where $\tilde{+}$ represents addition
with round-off error.
It is well known that round-off error grows linearly in the height of a
summation tree. So, a user might naturally want to sequence the summation as
a perfect binary tree: $(w~\tilde{+}~x)~\tilde{+}~(y~\tilde{+}~z)$.
However Numerical Fuzz forces all monadic computation to be sequenced, forcing
the error analysis 
to always compute the worst-case round-off error associated with the
pathological degenerate tree: $((w~\tilde{+}~x)~\tilde{+}~y)~\tilde{+}~z$.

Our extension enables the user to structure the computation tree arbitrarily,
providing tighter error bounds when possible. It is inspired by the resource
interpretation of linear logic: we can either give our error budget $M_q$ to
$\tau_0$ or $\tau_1$ or equivalently give the same error budget $M_q$ to
$\tau_0$ \textit{with} $\tau_1$. To show that $M_q ~ (\tau_0 \times \tau_1)$
is equivalent to
$(M_q ~ \tau_0) \times (M_q \tau_1)$, we need to construct two 1-sensitive
functions in the forwards and backwards direction.

The program $\textbf{distribute} : M_q (\tau_0 \times \tau_1) \multimap (M_q \tau_0)
\times (M_q \tau_1)$ is derivable in Numerical Fuzz, so no special primitive is
needed:
\begin{equation*} \label{eq:distribute}
\begin{aligned}[c]
\textbf{distribute} &: M_q (\tau_1 \times \tau_2) \multimap (M_q \tau_1 \times M_q \tau_2) \\
 & \triangleq \lambda ~ y. ~ 
   \langle
     \textbf{let-bind} \ x ~ = ~ y \ \textbf{in} \ \textbf{ret}(\pi_1 ~ x),
     \textbf{let-bind} \ x ~ = ~ y \ \textbf{in} \ \textbf{ret}(\pi_2 ~ x),
   \rangle
\end{aligned}
\end{equation*}

By contrast, inverse program $\textbf{factor}: (M_q \tau_0) \times (M_q \tau_1)
\multimap M_q (\tau_0 \times \tau_1)$ is neither derivable nor admissible (but
is sound). $\textbf{factor}$ is a particularly useful primitive to have for
sequencing computations. For example, consider the following side-by-side
comparison in Figure~\ref{fig:factor-side-by-side} with interval bounds erased
for presentational purposes.

\begin{figure}[ht]
\centering

\begin{subfigure}[t]{0.48\textwidth}
\begin{lstlisting}
// (w + x) + (y + z) : M[3*u] num 
let-bind a = addfp <w, x> in
let-bind b = addfp <y, z> in
let-bind c = addfp <a, b> in
addfp c 
\end{lstlisting}
\caption{Without factor}
\end{subfigure}
\hfill
\begin{subfigure}[t]{0.48\textwidth}
\begin{lstlisting}
// (w + x) + (y + z) : M[2*u] num 
let-bind a = 
  factor <addfp <w, x>, addfp <y, z>> in
  addfp a
\end{lstlisting}
\caption{With factor}
\end{subfigure}

\caption{Side-by-side comparison with and without the \textbf{factor} primitive.}
\label{fig:factor-side-by-side}
\end{figure}
Note that $u$ represents the unit round-off associated with $addfp :
\textbf{num} \times \textbf{num} \multimap M_u \textbf{num}$. In the summation
example, the program with factor has an error bound of two ULPs whereas the
program without factor has an error bound of three ULPs. Clearly,
$\textbf{factor}$ helps users design programs that achieve tighter error bounds.

\subsubsection*{Interval analysis and bound polymorphism}
We wish to extend the type system to be able to perform an interval analysis. A
naive approach would be to annotate each $\textbf{num}_{\bnd{i}}$ with a
subscript interval $\bnd{i}$. However, if we have a function, such as the
identity function over numbers $id : \textbf{num}_{\bnd{i}} \multimap
\textbf{num}_{\bnd{i}}$ called at two different call sites with intervals $j$
and $k$, the inferred bound for $i$ would be $j \cup k$.

To solve this problem, we add bounds and \textit{bound polymorphism} for
functions, allowing the functions to be typed like so: $id : \bnd{\forall
\epsilon.} ~ \textbf{num}_{\bnd{\epsilon}} \multimap
\textbf{num}_{\bnd{\epsilon}}$. We can introduce and eliminate polymorphism via
the $\forall$ introduction and elimination rules provided in
Figure~\ref{fig:typing_rules}.

\section{Static Semantics}
For the remainder of the paper, we only care about closed types.
% We define an interpretation function $interp(b)$ evaluating bounds to points in
% $\textit{num}$. 
%
% \begin{equation}
%   \begin{aligned}[c]
%     interp(k \in \textit{num}) &\triangleq k \\
%     interp(b_0 + b_1) &\triangleq interp(b_0) + interp(b_1) \\
%     interp(b_0 \cdot b_1) &\triangleq interp(b_0) \cdot interp(b_1) \\
%   \end{aligned}
% \end{equation}

\begin{figure}
%% ROW1
\begin{center}
%% var
\AXC{$s \ge 1$}
\RightLabel{(Var)}
\UIC{$\Delta \ | \ \Gamma, x:_s \tau, \Theta \vdash x : \tau$}
\bottomAlignProof
\DisplayProof
\hskip 0.5em
%% fun
\AXC{$\Delta \ | \ \Gamma, x:_1 \tau_0 \vdash e : \tau$}
\RightLabel{($\multimap$ I)}
\UnaryInfC{$\Delta \ | \ \Gamma \vdash \lambda x. e : \tau_0 \multimap \tau $}
\bottomAlignProof
\DisplayProof
\vskip 1em

%% app
\AXC{$\Delta_0 \ | \ \Gamma \vdash e : \tau_0 \multimap \tau$}
\AXC{$\Delta_1 \ | \ \Theta \vdash f : \tau_0 $}
\RightLabel{($\multimap$ E)}
\BinaryInfC{$\Delta_0 + \Delta_1 \ | \ \Gamma + \Theta \vdash ef : \tau $}
\bottomAlignProof
\DisplayProof
\vskip 1em
%%


%% ROW2
\AXC{}
\RightLabel{(Unit)}
\UIC{$\Delta \ | \ \Gamma \vdash \langle \rangle : \mathbf{unit}$}
\bottomAlignProof
\DisplayProof
\hskip 0.5em
%% dep prod intro
\AXC{$\Delta \ | \ \Gamma \vdash e : \tau_0$}
\AXC{$\Delta \ | \ \Gamma \vdash f : \tau_1$}
\RightLabel{($\times$ I)}
\BinaryInfC{$\Delta \ | \ \Gamma \vdash \langle e, f \rangle: \tau_0 \times \tau_1$}
\bottomAlignProof
\DisplayProof
\hskip 0.5em
%% dep prod elim
\AXC{$\Delta \ | \ \Gamma \vdash e : \tau_1 \times \tau_2$}
\RightLabel{($\times$ E)}
\UIC{$\Delta \ | \ \Gamma \vdash {\pi}_i \ e : \tau_i$}
\bottomAlignProof
\DisplayProof
\vskip 1em
%%

%% ind prod intro
\AXC{$\Delta_0 \ | \ \Gamma \vdash e : \tau_0 $}
\AXC{$\Delta_1 \ | \ \Theta \vdash f : \tau_1$}
\RightLabel{($\tensor$ I)}
\BIC{$\Delta_0 + \Delta_1 \ | \ \Gamma + \Theta \vdash (e, f) : \tau_0 \tensor \tau_1$}
\bottomAlignProof
\DisplayProof
\vskip 1em

%% ind prod elim
\AXC{$\Delta_0 \ | \ \Gamma \vdash e : \tau_0 \tensor \tau_1$ }
\AXC{$\Delta_1 \ | \ \Theta,x:_s \tau_0,y:_s\tau_1 \vdash f: \tau $}
\RightLabel{($\tensor$ E)}
\BIC{$\Delta_0 + \Delta_1 \ | \ s * \Gamma + \Theta \vdash \letpair (x,y) \ = \ e \ \tin \ f : \tau $}
\bottomAlignProof
\DisplayProof
\hskip 0.5em
%% ind sum intro
\AXC{$\Delta \ | \ \Gamma \vdash e : \tau_0$ }
\RightLabel{($+$ $\text{I}_i$)}
\UIC{$\Delta \ | \ \Gamma \vdash \mathbf{in}_i \ e : \tau_0 + \tau_1$}
\bottomAlignProof
\DisplayProof
\vskip 1em
% %% ind sum intro
% \AXC{$\Gamma \vdash e : \tau_1$ }
% \RightLabel{($+$ $\text{I}_2$)}
% \UIC{$\Gamma \vdash \mathbf{in}_2 \ e : \tau_0 + \tau_1$}
% \bottomAlignProof
% \DisplayProof
% \hskip 0.5em
% box elim
\AXC{$\Delta_0 \ | \ \Gamma \vdash e : {!_s \tau_0}$}
\AXC{$\Delta_1 \ | \ \Theta, x:_{t*s} \tau_0 \vdash f : \tau$}
\RightLabel{($!$ E)}
\BIC{$\Delta_0 + \Delta_1 \ | \ t * \Gamma + \Theta \vdash \letcobind x = e \ \tin \ f : \tau$}
\bottomAlignProof
\DisplayProof
\vskip 1em
%%


%% ROW 5

% sum elim
\AXC{$\Delta_0 \ | \ \Gamma \vdash e : \tau_0+\tau_1$}
\AXC{$\Delta_1 \ | \ \Theta, x:_s \tau_0 \vdash f_1 : \tau$ \qquad
$\Delta_1 \ | \ \Theta, x:_s \tau_1 \vdash f_2: \tau$}
\RightLabel{($+$ E)}
\AXC{$s > 0$}
\TIC{$\Delta_0 + \Delta_1 \ | \ s * \Gamma + \Theta \vdash \mathbf{case} \ e \ \mathbf{of} \ (\mathbf{in}_1 x.f_1 \ | \ \mathbf{in}_2 x.f_2) : \tau$}
\bottomAlignProof
\DisplayProof
\hskip 0.5em
% box intro
\AXC{$\Delta \ | \ \Gamma \vdash e : \tau$ }
\RightLabel{($!$ I)}
\UIC{$\Delta \ | \ s * \Gamma \vdash [e] : {!_s \tau}$}
\bottomAlignProof
\DisplayProof
\vskip 1em

%%% ROW 6

% let 
\AXC{$\Delta_0 \ | \ \Gamma \vdash e :  \tau_0$}
\AXC{$\Delta_1 \ | \ \Theta, x:_{s} \tau \vdash f : \tau$}
\RightLabel{(Let)}
\BIC{$\Delta_0 + \Delta_1 \ | \ s * \Gamma + \Theta \vdash \letassign x = e \ \tin \ f : \tau$}
\bottomAlignProof
\DisplayProof
\hskip 0.5em

\vskip 1em

%% const
\AXC{$k \in \textit{num}$}
\AXC{$k \in interp(\Gamma, i)$}
\RightLabel{(Const)}
\BIC{$\Delta \ | \ \Gamma \vdash k : \num_{b}$}
\bottomAlignProof
\DisplayProof
\hskip 0.5em
\vskip 1em

%%% ROW 7

%% subsumption
\AXC{$\Delta \ | \ \Gamma \vdash e :  M_q \tau$}
\AXC{$r \ge q$}
\RightLabel{(Subsumption)}
\BIC{$\Delta \ | \ \Gamma \vdash e :  M_{r} \tau$}
\bottomAlignProof
\DisplayProof
\hskip 0.5em
%% return
\AXC{$\Delta \ | \ \Gamma \vdash e : \tau$}
\RightLabel{(Ret)}
\UIC{$\Delta \ | \ \Gamma \vdash \ret e : M_0 \tau$}
\bottomAlignProof
\DisplayProof
\hskip 0.5em
%% RND
\AXC{$\Delta \ | \ \Gamma \vdash e : \num$}
\RightLabel{(Rnd)}
\UIC{$\Delta \ | \ \Gamma \vdash \rnd \ e : M_u \ \num$}
\bottomAlignProof
\DisplayProof
\vskip 1em


%%% ROW 8


% let-bind
\AXC{$\Delta_0 \ | \ \Gamma \vdash e : M_r \tau_0$}
\AXC{$\Delta_1 \ | \ \Theta, x:_{s} \tau_0 \vdash f : M_{q} \tau$}
\RightLabel{($M_u$ E)}
\BIC{$\Delta_0 + \Delta_1 \ | \ s * \Gamma + \Theta \vdash \letbind x = e \ \tin \ f : M_{s*r+q} \tau$}
\bottomAlignProof
\DisplayProof

% \hskip 0.5em
\vskip 1em

% funs
\AXC{$\Delta \ | \ \Gamma \vdash e : \tau_0$}
\AXC{$\{ \mathbf{op} :\tau_0 \lin \tau_1 \} \in \Sigma$}
\RightLabel{(Op)}
\BIC{$\Delta \ | \ \Gamma \vdash \mathbf{op}(e) : \tau_1$}
\bottomAlignProof
\DisplayProof

\vskip 1em


%%% ROW 9


% factor
\AXC{$\Delta \ | \ \Gamma \vdash e : (M_q \tau_0) \times (M_r \tau_1)$}
%\AXC{$r + q \leq s$}
%\AXC{$s = max(r,q)$}
\RightLabel{(Factor)}
\UIC{$\Delta \ | \ \Gamma \vdash \factor \ e : M_{max(q,r)} (\tau_0 \times \tau_1)$}
\bottomAlignProof
\DisplayProof

\vskip 1em

\AXC{$\Delta \ | \ \Gamma \vdash e : \tau$}
% \AXC{$\Delta \vdash i : \textbf{interval}$}
\RightLabel{($\forall$-I)}
\UIC{$\Delta \ | \ \Gamma \vdash \Lambda \epsilon . e : \forall \epsilon . \tau$}
\bottomAlignProof
\DisplayProof
\hskip 0.5em
\AXC{$\Delta \ | \ \Gamma \vdash \Lambda \epsilon . e : \forall \epsilon . \tau$}
\AXC{$\Delta \vdash i : \textbf{interval}$}
\RightLabel{($\forall$-E)}
\BIC{$\Delta \ | \ \Gamma \vdash e : \tau[i/\epsilon]$}
\bottomAlignProof
\DisplayProof

\end{center}
    \caption{Typing rules for \Lang, with $s,t,q,r,u \in \NNR \cup \{\infty\}$
    and for $i \in \{ 1, 2 \}$ where $u$ is a fixed constant parameter (see
    Definition~\ref{def:numfuzz-interface} for details on picking an adequate
    constant).}
    \label{fig:typing_rules}
\end{figure}



\subsection{Dynamic Semantics}
We only define our language over closed terms. In Figure~\ref{fig:typing_rules},
we define the operational semantics rewrite relation $\mapsto$ to map from
closed terms in \Lang to closed terms in \Lang.

\section{Dynamic Semantics}

\subsection{Substitution-style Operational Semantics}
The following is defined over untyped terms. In particular, we define the
operational semantics rewrite relation $\mapsto$ to map from (untyped) \Lang
to (untyped) \Lang. In other words, untypable programs can step (but not
necessarily to values).

\begin{figure}
\begin{center}

\begin{equation*}
\begin{aligned}[c]
	\mathbf{op}(v) &\mapsto op(v)\\
	\pi_i\langle v_1,v_2 \rangle &\mapsto v_i \\
	(\lambda x.e) \ v &\mapsto e[v/x] \\
	%\factor v \ &\mapsto v
\end{aligned}
\quad
\begin{aligned}[c]
	\letassign x = v \ \tin \ e &\mapsto e[v/x] \\
  \letpair (x, y) = (v, w) \ \tin \ e &\mapsto e[v/x][w/y] \\
	\letcobind x = v \ \tin \ e &\mapsto e[v/x]
	%\letbind x = \ret v \ \tin \ e &\mapsto e[v/x] \\
\end{aligned}
\end{equation*}
\vskip -1em
\begin{align*}
  \letbind y = (\letbind x = v \ \tin \ f) \ \tin \ g &\mapsto \letbind x = v \ \tin \ \letbind y = f \ \tin \ g \quad x\notin FV(g) 
\end{align*}
\vskip -1.75em
\begin{align*}
	\mathbf{case} \ (\mathbf{in}_i \ v) \ \mathbf{of} \ (\mathbf{in}_1 \ x.e_1 \ | \ \mathbf{in}_2 \ x.e_2 )  &\mapsto e_i[v/x]
  \qquad\qquad(i \in \{1, 2 \})
\end{align*}
\vskip -0.25em

	\AXC{$e \mapsto e'$}
	\UIC{$\letassign x = e \ \tin \ f \mapsto \letassign x = e' \ \tin f$}
	\DisplayProof

\end{center}
    \caption{Substitution-style evaluation rules for \Lang. Note the side condition for $\letbind$always holds for closed expressions.}
    \label{fig:sub_eval_rules}
\end{figure}

\subsection{(Typed) Enviroment-style Operational Semantics}
The following is defined over typed terms. In particular, we define the
operational semantics rewrite relation $\rightsquigarrow$ to map from a typed
term in an program enviroment to a typed term in an program enviroment. Note
that in this setup, $\letbind x = v \ \tin \ f$ and $[x]$ are \textit{not}
values.

To be precise, $\rightsquigarrow$ maps an expression $e$ with type $\tau$
running in an enviroment $\sigma$ mapping variables like $x_1$ to value $v_1$
with type $\tau_1$ and sensitivity budget $s_1$ to a $e'$ with type $\tau'$ and
enviroment $\sigma'$.
So, $\sigma$ send variables like $x_1 \to v_1 :_s \tau_1$.

An enviroment $\sigma$ is compatible with a typing context $\Gamma$ if
$\llbracket \sigma \rrbracket$ is a point within metric space $\llbracket \Gamma
\rrbracket$ (after erasing $\sigma$'s $0$-sensitive variables). Abusing notation
a little, I write this like so: $\llbracket \sigma \rrbracket \in \llbracket
\Gamma \rrbracket$.

\begin{equation*}
  \llbracket \sigma \rrbracket \in \llbracket \Gamma \rrbracket 
  \triangleq \forall
  \\ 
  (x \mapsto v :_s \tau) \in \sigma, 0 < s \implies (x :_s \tau) \in \llbracket
  \Gamma \rrbracket
\end{equation*}

Similarly, $\llbracket \sigma \Vdash e : \tau \rrbracket$ is
interpreted as the point in the metric space $\llbracket \tau \rrbracket$
obtained by running e at $\sigma$.

Useful adequacy-flavored theorem to prove. If $\sigma \Vdash e : \tau \rightsquigarrow \sigma'
\Vdash e' : \tau'$, then $\llbracket \sigma \Vdash e : \tau \rrbracket =
\llbracket \sigma' \Vdash e' : \tau' \rrbracket$.

Another adequacy-flavored theorem (in reverse). If $\llbracket \sigma \Vdash e :
\tau \rrbracket = \llbracket \sigma_{empty} \Vdash v : \tau' \rrbracket$, then $\sigma
\Vdash e : \tau \rightsquigarrow^{*} \sigma_{empty} \Vdash v : \tau'$

A syntatic type-soundness flavored theorem. If $\Gamma \vdash e : \tau$, then
$\forall \llbracket \sigma \rrbracket \in \llbracket \Gamma \rrbracket, \exists
\sigma' \ v \ \tau', \sigma \Vdash e : \tau \rightsquigarrow^{*} \sigma' \Vdash v :
\tau'$ and $\llbracket \tau \rrbracket = \llbracket \tau' \rrbracket$.

Another flavored soundness theorem. For $\llbracket \Gamma \vdash e : \tau
\rrbracket$ and $\sigma, \sigma' \in \Gamma$ and
$$\sigma \Vdash e : \tau \rightsquigarrow^* \sigma_{empty} \Vdash v : \tau'$$ 
and 
$$\sigma' \Vdash e : \tau \rightsquigarrow^* \sigma_{empty'} \Vdash v' : \tau''$$
%the distance between $\llbracket \sigma \rrbracket$ and $\llbracket \sigma'
%\rrbracket$ in metric space $\llbracket \Gamma \rrbracket$ is greater than or
%equal to the distance between $\llbracket \sigma \Vdash e : \tau \rrbracket$ and
%$\llbracket \sigma' \Vdash e : \tau \rrbracket$. In other words,
then
$$
d_{\llbracket \tau \rrbracket}(v, v') \leq d_{\llbracket \Gamma \rrbracket}(\sigma, \sigma')
$$
where $\sigma_{empty}, \sigma_{empty'}$ means
$$
\forall (x \mapsto v :_s \tau) \in \sigma_{empty}, \sigma_{empty'}, s = 0
$$

Note that $\sigma$ is ordered in the case that variables are shadowed. $\sigma[x
\mapsto v :_s \tau]$ denotes lookup (when on the left-hand side of a rewrite
relation) or insertion from the right (when on the right-hand side of a rewrite
relation).

\begin{figure}
\begin{center}
\begin{equation*}
  \begin{aligned}[c]
    %%%%%%%%%%%%%%%%%%%%%%%%%%%%%%%%%%%%%%%%%%%%%%%%%%%%%%%%%%%%%%%%%%%%%%%%%%%
    % more spicy rules
    % Lookup (sensitivity budget 1)
    \sigma[x \mapsto v :_1 \tau] \Vdash \ x \ : \tau &\rightsquigarrow 
    \sigma \Vdash v : \tau \\
    % Lookup (sensitivity budget greater than 1)
    \sigma[x \mapsto v :_{s} \tau] \Vdash \ x \ : \tau &\rightsquigarrow
    \sigma[x \mapsto v :_{s-1} \tau] \Vdash v : \tau \quad{(\text{with } 1 < s)}
    \\
    % let-bind rule
    \sigma \Vdash \textbf{let-bind}_{(s, \tau_1)} \ x = v \ : M_{q} \tau \ \tin \ f &\rightsquigarrow \sigma[x
    \mapsto v :_s \tau_1]
    \Vdash f : M_q \tau \\
    % let-cobind rule
    \sigma \Vdash \textbf{let-cobind}_{(s, t, \tau_1)} \ x = v \ : \tau \ \tin \ f &\rightsquigarrow (\sigma[x
    \mapsto v :_s \tau_1]) * t
    \Vdash f : \tau \\
    % [e] rule
    \sigma \Vdash [e]_s \ : \ \tau &\rightsquigarrow \sigma * s \Vdash e \ : \ \tau \\
    % lam app
    \sigma \Vdash (\lambda x : \tau_1 .e) \ v : \tau &\rightsquigarrow \sigma[x
    \mapsto v :_1 \tau_1] \Vdash e : \tau \\
    %%%%%%%%%%%%%%%%%%%%%%%%%%%%%%%%%%%%%%%%%%%%%%%%%%%%%%%%%%%%%%%%%%%%%%%%%%%
    % more boring rules
    % op(v) rule
    \sigma \Vdash \mathbf{op}(v) : \tau &\rightsquigarrow \sigma \Vdash op(v) :
    \tau \\
    % proj rule
    \sigma \Vdash \pi_i\langle v_1,v_2 \rangle : \tau &\rightsquigarrow \sigma
    \Vdash v_i : \tau \\ 
    %%%%%%%%%%%%%%%%%%%%%%%%%%%%%%%%%%%%%%%%%%%%%%%%%%%%%%%%%%%%%%%%%%%%%%%%%%%
    % let stepping rule
    %%%%%%%%%%%%%%%%%%%%%%%%%%%%%%%%%%%%%%%%%%%%%%%%%%%%%%%%%%%%%%%%%%%%%%%%%%%
  \end{aligned}
\end{equation*}

  \AXC{$\sigma \Vdash e : \tau_1 \rightsquigarrow \sigma' \Vdash e' : \tau_2$}
  \UIC{$\sigma \Vdash \textbf{let}^* \ x = e \ : \tau_1 \ \tin \ \tau_2
  \rightsquigarrow \sigma' \Vdash \textbf{let}^* \ x = e' \ : \tau_1 \
  \tin \ \tau_2$}
	\DisplayProof

\end{center}
    \caption{Enviroment-style evaluation rules for \Lang. Note that during type
    checking but prior to running the operational semanitcs, the sensitivity
    information (tracked with metavar $s$) and type of bound variables $\tau_1,
    \tau_2$, is preserved as annotations in the syntax, written $[e]_s$ and
    $\textbf{let-bind}_{(s, \tau_1)}$, $\textbf{let-cobind}_{(s, t, \tau_1)}$,
    and $\lambda x : \tau_1 . e $. Note that $\textbf{let}^*$ is syntactic sugar
    for matching all let expressions and their corresponding annotations.}
    \label{fig:sub_eval_rules}
\end{figure}


\subsection{Logical Relation}
\section{Denotational Semantics}

\begin{definition}[Type interpretation]
  A type $\tau$ is interpreted with $\llbracket - \rrbracket : \textit{type} \to
  \textbf{Met}$ in the same way as in the original NumFuzz system.
  % TODO: put actual definition here.
\end{definition}


\begin{definition}[Typing context interpretation]
  A typing context $\Gamma$ is interpreted with $\llbracket - \rrbracket :
  \textit{context} \to \textbf{Met}$ in the following way:
  \begin{equation}
  \begin{aligned}[c]
    \llbracket . \rrbracket &\triangleq . \\
    \llbracket \Gamma, x :_s \tau \rrbracket &\triangleq \llbracket \Gamma \rrbracket
      \times D_s \llbracket \tau \rrbracket
  \end{aligned}
  \end{equation}
\end{definition}

Let $CV(\tau)$ be the closed values of type $\tau$ and $CE(\tau)$ be the closed
expressions of type $\tau$. Then we can define a unary logical relation over types which
capture the core information needed to prove our error soundness theorem.

\begin{definition}[Logical relation]
  \begin{equation}
  \begin{aligned}[c]
    % same from Ariel's paper
    \mathcal{R_{\tau}} &\triangleq 
      \{ e \ | \ e \in CE(\tau) \text{ and } \exists v
        \in CV(\tau) \text{ s.t. } e \mapsto^{*} v \text{ and } v \in \mathcal{VR_{\tau}} 
      \} \\
    \mathcal{VR_{\mathbf{unit}}} &\triangleq \{ \langle \rangle \} \\
    \mathcal{VR_{\mathbf{num}}} &\triangleq \mathbb{R} \\
    \mathcal{VR_{\mathbf{\tau_0 \times \tau_1}}} &\triangleq 
      \{ \langle v, w \rangle \ | 
        \ v \in \mathcal{R}_{\tau_0} \text{ and } w \in \mathcal{R}_{\tau_1}
      \} \\
    \mathcal{VR_{\mathbf{\tau_0 \otimes \tau_1}}} &\triangleq 
      \{ ( v, w ) \ | 
        \ v \in \mathcal{R}_{\tau_0} \text{ and } w \in \mathcal{R}_{\tau_1}
      \} \\
    \mathcal{VR_{\mathbf{\tau_0 + \tau_1}}} &\triangleq 
      \{ \mathbf{inl}~v \ | \ v \in \mathcal{R}_{\tau_0} \} 
      \cup
      \{ \mathbf{inr}~v \ | \ v \in \mathcal{R}_{\tau_1} \} \\
    \mathcal{VR_{\mathbf{\tau_0 \multimap \tau_1}}} &\triangleq 
      \{ \lambda x . e \ | \ \forall w_0, w_1 \in \mathcal{VR}_{\tau_0}, \\ & \quad \quad \ (\lambda x.e)~w_0, (\lambda x . e)~w_1 \in
      \mathcal{R}_{\tau_1} \text{ and } \mathcal{SD}_{\tau_1}((\lambda x . e)~w_0, (\lambda x . e)~w_1) \leq
      \mathcal{SD}_{\tau_0}(w_0, w_1) \} \\
    \mathcal{VR_{\mathbf{!_s \tau}}} &\triangleq 
      \{ [~v~] \ | \ v \in \mathcal{R}_{\tau} \} \\
    % spicy hot new stuff
    \mathcal{VR_{\mathbf{M_q \tau}}} &\triangleq 
      \{ (v, w) \ | \ v, w \in \mathcal{R}_{\tau} \text{ and } \mathcal{SDV}_{\tau}(v, w)
      \leq q \} \\
  \end{aligned}
  \end{equation}
\end{definition}

Our definition for distance (in our dentoational semantics), $d_\tau$ and the
distance between syntactic values $\mathcal{SD}_\tau$ (for $\mathcal{S}$yntactic
$\mathcal{D}$istance) and $\mathcal{SDV}_\tau$ (for $\mathcal{S}$yntactic
$\mathcal{D}$istance for $\mathcal{V}$alues) are closely related. Some care is
needed to ensure that it is well-founded. We define our distance over syntactic
values as follows:

\begin{definition}[Distance between closed syntactic terms]
  \begin{equation}
  \begin{aligned}[c]
    \mathcal{SD_{\tau}}(e_0, e_1) &\triangleq \mathcal{SDV}_{\tau}(v_0, v_1)
    &\text{ if } e_0 \mapsto^{*} v_0 \text{ and } e_1 \mapsto^{*} v_1 \\
    \mathcal{SD_{\tau}}(e_0, e_1) &\triangleq \infty &\text{ otherwise } \\
    \mathcal{SDV_{\mathbf{unit}}}(v, w) &\triangleq 0 &\text{ for } v, w = \langle \rangle \\
    \mathcal{SDV_{\mathbf{num}}}(k_0, k_1) &\triangleq 
      d_\mathbb{R} &\text{ for } k_0, k_1 \in \mathbb{R} \\
    \mathcal{SDV}_{\tau_0 \times \tau_1}((v_0, v_1), (w_0, w_1)) 
      &\triangleq max(\mathcal{SDV}_{\tau_0}(v_0, w_0),~\mathcal{SDV}_{\tau_0}(v_1, w_1))
    \\
    \mathcal{SDV}_{\tau_0 \otimes \tau_1}((v_0, v_1), (w_0, w_1)) 
      &\triangleq \mathcal{SDV}_{\tau_0}(v_0, w_0) + \mathcal{SDV}_{\tau_0}(v_1, w_1))
    \\
    \mathcal{SDV}_{\tau_0 + \tau_1}(\in_i v, \in_i w) 
      &\triangleq \mathcal{SDV}_{\tau_i}(v, w)
    \\
    \mathcal{SDV}_{\tau_0 \multimap \tau_1}(v_0, v_1) 
      &\triangleq \text{sup}_{w \in \mathcal{VR}_{\tau_0}} \mathcal{SD}_{\tau_1}(v_0~w,~v_1~w)
    \\
    % max: double check
    \mathcal{SDV}_{!_s \tau}(v, w) 
      &\triangleq s \cdot \mathcal{SDV}_{\tau}(v, w)
    \\
    \mathcal{SDV}_{M_q~\tau}((v_0, v_1), (w_0, w_1)) 
      &\triangleq \mathcal{SDV}_{\tau}(v_0, w_0)
    \\
    \mathcal{SDV_{\tau}}(v, w) &\triangleq \infty &\text{ otherwise } \\
  \end{aligned}
  \end{equation}
\end{definition}


Numerical Fuzz is a modular family of programming langauges paramterized by the
appropriate $\rho$, constant parameter $u$, and the appropriate set of numeric
computations $\Sigma$.  
We can now state a few assumptions about how the operational and static
semantics of $\mathbf{rnd}$ and $\Sigma$ (for $\mathbf{op}$) relate.
It is the proof obligation for any language designer instantiating the language
to demonstrate that these properties hold in order for our paramterized
soundness theorems to follow.
\begin{definition}[Interface for instantiating Numerical Fuzz.]
  \label{def:numfuzz-interface}
  The interface for Numerical Fuzz consists of $\rho$, $u$, and $\Sigma$ such
  that the following properties hold: 
\begin{description}
  \item[a) Property of $\rho$ and constant parameter $u$.] We assume that the
    $\forall e \in \mathcal{R}_{num}, \mathcal{SD}_{\mathbf{num}}(\rho(e), e)
    \leq q$ where $q$ is the grade in the $\mathbf{rnd}$ typing rule.
  \item[b) Property of $\mathbf{op}$.] We also assume that for every operation
    $\mathbf{op} : \tau_0 \multimap \tau_1 \ \in \ \Sigma$ we have a
    corresponding function $op$ mapping syntactic values in
    $\mathcal{VR}_{\tau_0}$ to $\mathcal{VR}_{\tau_1}$ and where
    the following two properties hold:
    \begin{description}
      \item[\underline{1) Type denotation.}] For any $e \in
        \mathcal{R}_{\tau_0}$ that $\mathbf{op}(e) \in \mathcal{R}_{\tau_1}$.
      \item[\underline{2) Sensitivity.}] For any $e_1, e_2 \in \mathcal{R}_{\tau_0}$ that
        $\mathcal{SD}_{\tau_1}(\mathbf{op}(e_1), \mathbf{op}(e_2)) \leq
        \mathcal{SD}_{\tau_0}(e_1, e_2)$.
    \end{description}
\end{description}
\end{definition}

\begin{lemma}[$\mathcal{SD}$ is a metric]
  $\mathcal{SD}$ forms a metric over our syntactic terms; in particular it
  satisfies:
  \begin{enumerate}
    \item Distance from any point to itself is zero: $\forall x,~\mathcal{SD}(x,
      x) = 0$.
    \item Positivity: $\forall x, y,~\mathcal{SD}(x, y) \geq 0$.
    \item Symmetry: $\forall x, y,~\mathcal{SD}(x, y) = \mathcal{SD}(y, x)$.
    \item Triangle inequality: $\forall x, y, z,~\mathcal{SD}(x, z) \leq
      \mathcal{SD}(x, y) + \mathcal{SD}(y, z)$.
  \end{enumerate}
\end{lemma}
\begin{proof}
  The properties holds for the base cases of $\mathcal{SDV}$ and follow for the
  remaining cases by our inductive hypothesis. Since our operational semntics is
  deterministic, the properties follow for $\mathcal{SD}$.
\end{proof}

\begin{lemma}[$\mathcal{SD}$ is preserved under stepping]
  If $e_0 \mapsto e'_0$, then for any $e_1$, $\mathcal{SD}(e_0, e_1) =
  \mathcal{SD}(e'_0, e_1)$.
\end{lemma}
\begin{proof}
  Holds by inspection of the definition of $\mathcal{SD}$.
\end{proof}
Note that by metric symmetry, $\mathcal{SD}$ is preserved under stepping on both
sides.

% probs needs some theorem relating SD to d, what do property do we want to
% enforce here? (probs need to factor out a lemma or something when doing the
% logical relations proof)



\subsection{Modular Interface}
The following notation is to mirror the look of the Fuzz metric preservation
theorem statement but contains differences in the setup. For example, our
logical relation is neither coinductive nor step-indexed. It is also unary,
using mutual (well-founded) recursion between the definition of a syntactic term
falling within the logical relation and the definition of the syntactic distance
between terms.

For any two closed bounds, we can write $i \sim_0 i$ or $i \sim_{\infty} i$. 
For any two closed expressions $e_0, e_1$ falling in same type relation $e_0,
e_1 \in R_\tau$ (where $\tau$ is a closed type), we can write $e_0 \sim_r e_1$
where $\mathcal{SD}_{\tau}(e_0, e_1) \leq r$. 
Generally, we can write vectors of expressions (e.g. for subsitutions) $\sigma =
\sigma_{\Delta} \ | \ \sigma_{\Gamma}, \sigma' = \sigma'_{\Delta} \ | \
\sigma'_{\Gamma}$
for a given typing context $\Delta \ | \ \Gamma$ such that: $\sigma \sim_{\gamma} \sigma'
: \Delta \ | \ \Gamma$ for a \textit{distance vector}
$\gamma = r_0, r_1, \ldots$
where
$\sigma = (k^l_0, k^r_0),~(k^l_1, k^r_1),~\ldots \ | \ e_0,~e_1,~\ldots$ 
and 
$\sigma' = (k'^l_0, k'^r_0),~(k'^l_1, k'^r_1),~\ldots \ | \ e_0,~e_1,~\ldots$ 
such that:
$$
(k^l_0, k^r_0) \sim_{r_0} (k'^l_0, k'^r_0) : \mathbf{bnd},~(k^l_1, k^r_1) \sim_{r_1} (k'^l_1, k'^r_1) : \mathbf{bnd},~\ldots \
| \ e_0 \sim_{r_m} e'_0 :
\tau_0~[\sigma_{\Delta}],~e_0 \sim_{r_{m+1}} e'_0 :
\tau_1~[\sigma_{\Delta}],~\ldots
$$
We also say that a substitution vector 
$[(k^l_0, k^r_0) / \epsilon_0, (k^l_1, k^r_1) / \epsilon_1, \ldots \ | \ e_1/x_1, e_2/x_2, \ldots]$ 
is \textit{compatible} with a typing context 
$i_0 : \mathbf{bnd}, i_1 : \mathbf{bnd}, \ldots \ | \ x_1 : \tau_1, x_2 : \tau_2, \ldots$
if each term 
$(k^l_0, k^r_0) \in \mathcal{R}_{\mathbf{bnd}}, 
(k^l_1, k^r_1) \in \mathcal{R}_{\mathbf{bnd}}, 
\ldots \ | 
\ e_1 \in \mathcal{R}_{\tau_1~[\sigma_{\Delta}]}, 
e_2 \in \mathcal{R}_{\tau_2~[\sigma_{\Delta}]}, \ldots$ where all types are closed.

Our definition for the dot product of a distance vector is the same as Fuzz. We
also write, for a distance vector $\gamma$ and variable $x$, $\gamma(x)$ for the
lookup of the distance of variable $x$ in $\gamma$. If the variable $x$ is not
in the domain, $\gamma(x) = 0$ by default. From here on in the paper, we'll
treat and represent our distance vector $\gamma$ as a lookup function and assume
that there is an implicit fixed ordering on the variables.

Numerical Fuzz is a modular family of programming langauges paramterized by the
appropriate $\rho$, constant parameter $u$, and the appropriate set of numeric
computations $\Sigma$.  
We can now state a few assumptions about how the operational and static
semantics of $\mathbf{rnd}$ and $\Sigma$ (for $\mathbf{op}$) relate.
It is the proof obligation for any language designer instantiating the language
to demonstrate that these properties hold in order for our paramterized
soundness theorems to follow.
\begin{definition}[Interface for instantiating Numerical Fuzz.]
  \label{def:numfuzz-interface}
  The interface for Numerical Fuzz consists of $\rho$, $u$, $\Sigma$, and
  $\Sigma_{\mathbf{bnd}}$ such that the following properties hold: 
\begin{description}
  \item[a) Property of $\rho$ and constant parameter $u$.] We assume that the
    $\forall e \in \mathcal{R}_{num}, \mathcal{SD}_{\mathbf{num}}(\rho(e), e)
    \leq q$ where $q$ is the grade in the $\mathbf{rnd}$ typing rule.
  \item[b) Property of $\mathbf{op}$.] 
    We can view $\textit{op}$ as a (possibly constant) metalevel function.
    We also assume that for every operation
    $\mathbf{op} : \tau \in \ \Sigma$ we have a
    corresponding function $op$ mapping syntactic values in
    $\mathcal{VR}_{\tau}$ where metric preservation holds: For all $\sigma
    \sim_{\gamma} \sigma' : \Gamma$, $\textit{op}~\sigma \sim_{\gamma \cdot
    \Gamma} \textit{op} : \tau$.
    Or, expanded:
    \begin{description}
      \item[\underline{1) Type denotation.}] $\mathit{op} \in
        \mathcal{R}_{\tau}$. The metalevel operation will step to something in the
        value relation.
      \item[\underline{2) Context sensitivity.}]
        This property states that for every $\textbf{op}$, the
        corresponding $\textit{op}$ preserves the metric. For any two
        substitutions $\sigma, \sigma'$ such that:
        $\sigma \sim_{\gamma} \sigma' : \Gamma$
        we have that
        $\mathcal{SD}_{\tau}(\textit{op}~\sigma, \textit{op}~\sigma') \leq \gamma \cdot \Gamma$.
    \end{description} \item[c) Property of $\mathbf{iop}$.] We further need to
      assume that for every operation $\mathbf{iop}_c : \Sigma_\textbf{num}$, we
      have that for any sequence of bounds $b_0, b_1, \ldots : \textbf{bnd}$ the
      corresponding metalevel function will actually map to a concrete bound
      where $\textit{iop}_c(b_0, b_1, \ldots) : \textbf{bnd} = (k_0, k_1)$
      holds.
\end{description}
\end{definition}

\subsection{Soundness}
\section{Type Soundness}
\begin{lemma}[Termination]
  If $\Gamma \vDash e : \tau$, then $\forall \sigma$ compatible with $\Gamma$, 
  $$\sigma \Vdash e \rightsquigarrow^* \sigma' \Vdash v$$
\end{lemma}
\begin{proof}
  If $\sigma \Vdash e$ is a value, then we are done. Otherwise, if $\sigma
  \Vdash e$ is an expression, we begin by unfolding and applying the definition
  of $\vDash$. By definition, know that $\llbracket \sigma \Vdash e \rrbracket_{\tau} \in
  \llbracket \tau \rrbracket$. We also know that $\bot \not\in \llbracket \tau
  \rrbracket$, a metric space, so $\llbracket \sigma \Vdash e \rrbracket_{\tau}
  \not= \bot$ and is well-defined.

  By inspection of the definition of cases of $\llbracket - \rrbracket_{\tau}$,
  we know that only last case (which deals with all expressions) must apply.
  Therefore, $\sigma \Vdash e \rightsquigarrow^* \sigma' \Vdash v$.
\end{proof}

% \begin{theorem}[Semantic type preservation]
% If a term $e$ is semantically well-typed in context $\Gamma$ with type $\tau$
%   then for all $\sigma \Vdash e \rightsquigarrow \sigma' \Vdash e'$, $\sigma$
%   compatible with $\Gamma$, then there exists a $\Gamma'$ such that $\Gamma'
%   \vDash e' : \tau$.
% \end{theorem}
% \begin{proof}
% If $e$ is a value, we're done. 
% If $e$ is not a value, we proceed by induction over the cases of the rewrite relation
% $\rightsquigarrow$.
% % Our inductive hypothesis is that $e \rightsquigarrow e'$ and $\Gamma \vDash e : \tau$, that is, 
% % $\forall \sigma \text{ compatible with } \Gamma, \llbracket \sigma \Vdash e
% % \rrbracket_{\tau} \in \llbracket \tau \rrbracket$.
% \begin{description}
%   \item[\textsc{Variable lookup.}] If $$
%   \item[\textsc{Ret.}]
%   \item[\textsc{Rnd.}]
% \end{description}
% \end{proof}

% \begin{definition}[Monadic type]\label{def:monadic}
%   A type $\tau$ is monadic if and only if $\exists \tau', q \text{ such that }
%   \tau = M_q \tau'$. It is non-monadic otherwise.
% \end{definition}
% %%% wrong theorem:
% \begin{lemma}[Non-monadic lookup]\label{thm:non-monadic-lookup}
%   For all $\tau$ non-monadic and all $\Gamma, x : \tau'$ compatible with
%   $\sigma$, $\exists \sigma', \sigma = \sigma'[x \mapsto v : \tau']$ is
%   well-defined. Stated equivalently, non-monadic $x$ uniformly maps to $v$ at
%   all enviroments in $\sigma$.
% \end{lemma}
% \begin{proof}
%   TODO
% \end{proof}

\begin{lemma}[Enviroment tree shape]
  The tree height for a 
\end{lemma}

\begin{definition}[Orderings]\label{def:orderings}
  We define the following well-founded partial order over configurations and
  types, which mirrors the definition of machine configuration interpretation
  and will be used for induction in the proceeding proofs.

  First, we define an ordering over types by size of the AST.
  \begin{equation}
    \begin{aligned}[c]
      s(\textbf{unit}) &= 1 \\
      s(\textbf{num}) &= 1 \\
      s(!_s \tau) &= s(\tau) + 1 \\
      s(M_q \tau) &= s(\tau) + 1 \\
      s(\tau_0 \times \tau_1) &= s(\tau_0) + s(\tau_1) + 1 \\
      s(\tau_0 \otimes \tau_1) &= s(\tau_0) + s(\tau_1) + 1 \\
      s(\tau_0 \multimap \tau_1) &= s(\tau_0) + s(\tau_1) + 1 \\
    \end{aligned}
  \end{equation}
  and
  \begin{equation}
    \tau_0 < \tau_0 \iff s(\tau_0) < s(\tau_1)
  \end{equation}

  Secondly, we define an ordering over environment leafs $\gamma$ using the number of bound
  variables:
  \begin{equation}
    \begin{aligned}[c]
      s(.) &= 0 \\
      s(\gamma, x \mapsto v :_s \tau) &= s(\gamma) + 1 \\
    \end{aligned}
  \end{equation}

  and

  \begin{equation}
    \gamma_0 < \gamma_0 \iff s(\gamma_0) < s(\gamma_1)
  \end{equation}

  We can now define an ordering over enviorment trees using the maximum size of
  each enviorment leaf in the tree:
  \begin{equation}
    \begin{aligned}[c]
      s(\gamma; \gamma') &= max(s(\gamma), s(\gamma')) \\
      s(\sigma; \sigma') &= max(s(\sigma), s(\sigma'))
    \end{aligned}
  \end{equation}

  We are now ready to define an ordering over $\textit{config} \times
  \textit{type}$:
  \footnote{This is our termination measure for our machine configuration
  interpretation function.}
  \footnote{Note that the type of the configuration (if the interpretation of
  the configuration belongs to the interpretation of the type) restricts the
  height and shape of environment tree. So we do not need to reason about the
  height of our enviroment trees in our ordering.}

  \begin{equation}
    \sigma \Vdash e : \tau < \sigma' \Vdash e' : \tau' \iff
    s(\tau) < s(\tau') \lor (s(\tau) = s(\tau') \land s(\sigma) < s(\sigma'))
  \end{equation}

\end{definition}

\begin{lemma}[Well-founded partial ordering]
  $\leq$ is a well-founded partial order over $\textit{config} \times
  \textit{type}$.
\end{lemma}
\begin{proof}
  TODO
\end{proof}

\begin{theorem}[Semantic type soundness]
$\Gamma \vdash e : \tau \implies \Gamma \vDash e : \tau$
\end{theorem}
\begin{proof}
From our premise, we wish to show that:
  \begin{enumerate}
    \item $\forall \sigma$ compatible with $\Gamma$, $\llbracket \sigma \Vdash e
      \rrbracket_{\tau} \in \llbracket \tau \rrbracket$.
    \item $\llbracket - \Vdash e
      \rrbracket_{\tau}$ is a \text{1-sensitive} map for all inputs enviroments
      compatible with $\Gamma$
  \end{enumerate}
We proceed by case analysis over whether $e$ is a value or an expression.
  Suppose $e$ is a value. Let us induct over our typing derivation. 
  % In some of the cases, we will need to induct over the size of our enviroment. 
\begin{description}
  \item[\textsc{(ty. rule) Var.}] There is no premise containing a typing
    judgement to this rule. So we have no inductive hypothesis to rely on.
    % We proceed to induct over the type of our variable, $\tau$.
    % \begin{description}
    %   \item{\textbf{unit} and \textbf{num}.}
    %   \item{$\tau_0 \otimes \tau_1$, $\tau_0 + \tau_1$, and $\tau_0 \multimap
    %     \tau_1$.}
    %   \item{$!_s \tau$.}
    %   \item{$M_q \tau$}
    %   \item{$\tau_0 \times \tau_1$.}
    % \end{description}
    Instead, we induct over the the size of our enviorment.
    \begin{description}
      \item[\textit{(env. size) 0.}] We observe that an enviorment size of zero
        implies that $\Gamma$ is empty. So we are trying to show that the
        following:
        $$\Gamma, x : \tau \vDash e : \tau$$
        implies:
        $$\llbracket \sigma \Vdash e \rrbracket_\tau \in \llbracket \tau
        \rrbracket$$
        for compatible $\sigma, \Gamma, x : \tau$. We proceed to case over the type of our
        variable, $\tau$.
        \begin{description}
          \item{(ty. case) \textbf{unit.}} The only configuration compatible with this is
            $x \mapsto \langle \rangle \Vdash x$ and clearly 
            $$\llbracket x \mapsto \langle \rangle \Vdash x
            \rrbracket_{\textbf{unit}} = \llbracket \Vdash \langle \rangle
            \rrbracket_{\textbf{unit}} = * \in \{ * \} = \llbracket \textbf{unit}
            \rrbracket$$
            So, properties (1) and (2) hold by the reasoning above.
          \item{(ty. case) \textbf{num.}}
        \end{description}
      \item[\textit{(env. size) n + 1.}] 
    \end{description}
  \item[\textsc{(ty. rule) Ret.}] 
    From our inductive hypothesis, we have that $\Gamma \vDash e : \tau$ and
    wish to prove that $\Gamma \vDash \mathbf{ret} \ e : \tau$. It suffices to
    show that $(\sigma; \sigma)[\alpha \mapsto v:_1 \tau] \Vdash \alpha$.
  \item[\textsc{(ty. rule) Rnd.}]
\end{description}

  Next, the tree case:
\end{proof}

%%%% IGNORE BELOW %%%%

% todo: Use whatever theorem name people like best here.
Semantics is preserved under operational stepping. If $\sigma \Vdash e : \tau \rightsquigarrow \sigma'
\Vdash e' : \tau$, then $\llbracket \sigma \Vdash e : \tau \rrbracket =
\llbracket \sigma' \Vdash e' : \tau \rrbracket$.

% todo: Use whatever theorem name people like best here.
If the semantics of a program is equivalent to the semantics of a value, it must
reduce to that value. If $\llbracket \sigma \Vdash e : \tau \rrbracket =
\llbracket v : \tau \rrbracket$, then $\sigma \Vdash e : \tau
\rightsquigarrow^{*} \sigma' \Vdash v : \tau$

Syntactically well-typed programs are non-expansive. For $\Gamma \vdash e : \tau
$ and $\llbracket \sigma \rrbracket, \llbracket \sigma' \rrbracket \in
\llbracket \Gamma \rrbracket $ and
$$\sigma \Vdash e : \tau \rightsquigarrow^* v : \tau$$ 
and 
$$\sigma' \Vdash e : \tau \rightsquigarrow^* v' : \tau$$
then
$$
d_{\tau}(v, v') \leq d_{\Gamma}(\sigma, \sigma')
$$

Syntactically ok implies semantically ok. If $\Gamma \vdash e : \tau$, then
$\forall \llbracket \sigma \rrbracket \in \llbracket \Gamma \rrbracket, \exists
v, \llbracket \sigma \Vdash e : \tau \rrbracket = \llbracket v : \tau
\rrbracket$.

Syntactically ok implies operationally ok. If $\Gamma \vdash e : \tau$, then
$\forall \llbracket \sigma \rrbracket \in \llbracket \Gamma \rrbracket, \exists
v, \sigma \Vdash e : \tau \rightsquigarrow^{*} v : \tau$.



\subsection{Tightness} \label{sec:tightness}



