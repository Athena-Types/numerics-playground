\section{Conclusion}
Our work is also the first type-based approach that is able to reason about
round-off error in the presence of subtraction and negative numbers. 
Since Negative Fuzz uses a type-based approach to numerical analysis, our
approach also offers several qualitative advantages over competing tools. For
example, programs that call a library function would not need to type-check
the underlying library code and could instead rely on the function type as an
interface specification, saving type checking time. 
As another example of the potential benefit of avoiding a global optimization
problem, type checking could also be performed in parallel or incrementally over
the dependency graph of the program.

% Ideas:
% - type checking allows for an incremental approach (we can "save" the inferred
%   sensitivites and bounds)
