\section{Introduction}

It is natural for ordinary users such as mathematicians and scientists to wish
their computers operate over the reals. But this is impossible. We typically
design languages that compute over the finitary floating-point number
approximation.
As language designers, we wish to assure the ordinary user that we have not
scammed them: that the floating-point programming language semantics we have
provided is close to what they desire. 
To do this, we develop automated numerical analysis techniques that bound the
maximum error introduced by floating-point computation.

% cite Gappa, FPTaylor, Satire etc.

A major challenge of the automated numerical analysis literature is scalability.
In particular, many approaches rely on global optimization, % cite Satire
rewrite saturation, % cite Gappa
or SMT-based methods % find citation
that frequently time-out on large codebases. 
% explain a bit more about why these don't scale

Recent work has applied typed-based analysis approaches to both forwards error
analysis, such as Numerical Fuzz, and backwards numerical error analysis, such
as in Bean. % cite NumFuzz and Bean
In particular, Numerical Fuzz uses a graded effect and co-effect system to
perform a forwards error analysis and is able to outperform many other
state-of-the-art tools on existing benchmarks.
% more about how NumFuzz works

However, Numerical Fuzz is not capable of handling negative numbers and
subtraction, which significantly limits its applicability.
There are significant foundational problems with providing a simultaneously
\textit{compositional} and \textit{scalable} analyis for forwards numerical
error in the presence of subtraction. By \textit{compositional} we mean that
the analysis need not be global. For this we can only rely on a relative notion
of error. And by \textit{scalable} we mean that the analysis remains relatively
tight as the program size balloons.

The problem with providing a compositional and scalable analysis is due to
\textit{catastrophic cancellation}: the phenomenon where two large nearby
numbers with small relative error can be subtracted to produce a small number
with arbitrarily high relative error. 


% contributions
