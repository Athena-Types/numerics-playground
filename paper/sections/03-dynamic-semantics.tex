\section{Dynamic Semantics}

\iffalse
\subsection{Substitution-style Operational Semantics}
The following is defined over untyped terms. In particular, we define the
operational semantics rewrite relation $\mapsto$ to map from (untyped) \Lang
to (untyped) \Lang. In other words, untypable programs can step (but not
necessarily to values).

\begin{figure}
\begin{center}

\begin{equation*}
\begin{aligned}[c]
	\mathbf{op}(v) &\mapsto op(v)\\
	\pi_i\langle v_1,v_2 \rangle &\mapsto v_i \\
	(\lambda x.e) \ v &\mapsto e[v/x] \\
	%\factor v \ &\mapsto v
\end{aligned}
\quad
\begin{aligned}[c]
	\letassign x = v \ \tin \ e &\mapsto e[v/x] \\
  \letpair (x, y) = (v, w) \ \tin \ e &\mapsto e[v/x][w/y] \\
	\letcobind x = v \ \tin \ e &\mapsto e[v/x]
	%\letbind x = \ret v \ \tin \ e &\mapsto e[v/x] \\
\end{aligned}
\end{equation*}
\vskip -1em
\begin{align*}
  \letbind y = (\letbind x = v \ \tin \ f) \ \tin \ g &\mapsto \letbind x = v \ \tin \ \letbind y = f \ \tin \ g \quad x\notin FV(g) 
\end{align*}
\vskip -1.75em
\begin{align*}
	\mathbf{case} \ (\mathbf{in}_i \ v) \ \mathbf{of} \ (\mathbf{in}_1 \ x.e_1 \ | \ \mathbf{in}_2 \ x.e_2 )  &\mapsto e_i[v/x]
  \qquad\qquad(i \in \{1, 2 \})
\end{align*}
\vskip -0.25em

	\AXC{$e \mapsto e'$}
	\UIC{$\letassign x = e \ \tin \ f \mapsto \letassign x = e' \ \tin f$}
	\DisplayProof

  \vskip 0.4em
	\AXC{$e_1 \mapsto e_1'$}
	\AXC{$e_2 \mapsto e_2'$}
  \BIC{$(e_1, e_2) \mapsto (e_1', e_2')$}
	\DisplayProof

  \vskip 0.4em
	\AXC{$e_1 \mapsto e_1'$}
	\AXC{$e_2 \mapsto e_2'$}
  \BIC{$\langle e_1, e_2 \rangle \mapsto \langle e_1', e_2' \rangle$}
	\DisplayProof
\end{center}
    \caption{Substitution-style evaluation rules for \Lang. Note the side condition for $\letbind$always holds for closed expressions.}
    \label{fig:sub_eval_rules}
\end{figure}
\fi

\subsection{Enviroment-style Operational Semantics}
The following is defined over typed terms. In particular, we define the
operational semantics rewrite relation $\rightsquigarrow$ to map from a typed
term in program enviroment trees to a typed term in program enviroment trees. 

$$
\textit{config} \subseteq \textit{env} \times \textit{expr}
$$

$$
\rightsquigarrow \subseteq \textit{config} \times \textit{config}
$$

To be precise, $\rightsquigarrow$ maps an expression $e$ with type $\tau$
running in a collection of enviroment(s) (leaf or tree) $\sigma$ mapping
variables like $x_1$ to value $v_1$ with type $\tau'$ and sensitivity budget
$s_1$ to a $e'$ with type $\tau'$ and enviroment $\sigma'$. So, $\sigma$ send
variables like $x_1 \to v_1 :_s \tau'$.

$\sigma + \sigma'$ represent pointwise concatenation over multi-enviroments,
defined as follows:

\begin{definition}[Enviroment sum, pairwise]
  \begin{equation}
    \gamma_1 + \gamma_2 = \gamma_1 :: \gamma_2 
  \end{equation}
\end{definition}

\begin{definition}[Enviroment tree sum, pairwise]
  \begin{equation}
    \begin{aligned}[c]
      . + . &=  . \\ 
      \gamma_0;\gamma_1 + \gamma_2;\gamma_3 &= (\gamma_0 + \gamma_2);(\sigma_1 + \sigma_3) \\ 
      \_ + \_ &= \bot \quad{\text{(otherwise)}}
    \end{aligned}
  \end{equation}
\end{definition}

$\sigma @ \sigma'$ represent adding $\sigma'$ at each of the leafs of $\sigma$,
defined as follows:
\begin{definition}[Enviroment sum, at leaf]
  \begin{equation}
    \begin{aligned}[c]
      \sigma_1; \sigma_2 @ \sigma &= (\sigma_1 @ \sigma);(\sigma_2 @ \sigma) \\
      \gamma @ (\sigma_1; \sigma_2) &= (\gamma @ \sigma_1);(\gamma @ \sigma_2)
    \end{aligned}
  \end{equation}
\end{definition}

%% TODO: type up definition

\begin{figure}
\begin{center}
\tiny
\begin{equation*}
  \begin{aligned}[c]
    %%%%%%%%%%%%%%%%%%%%%%%%%%%%%%%%%%%%%%%%%%%%%%%%%%%%%%%%%%%%%%%%%%%%%%%%%%%
    % more spicy rules
    % Lookup (sensitivity budget 1)
    \sigma[x \mapsto v :_1 \tau_0 \multimap \tau_1] \Vdash \ x \ w \ : \tau_1 &\rightsquigarrow
    \sigma \Vdash v \ w : \tau_1 \\
    % Lookup (sensitivity budget greater than 1)
    \sigma[x \mapsto v :_{s} \tau_0 \multimap \tau_1] \Vdash \ x \ w \ : \tau_1
    &\rightsquigarrow \sigma[x \mapsto v :_{s-1} \tau_0 \multimap \tau_1] \Vdash
    v \ w : \tau_1 \quad{(\text{with } 1 < s)} \\
    %%%%%%%%%%%%%%%%%%%%%%%%%%%%%%%%%%%%%%%%%%%%%%%%%%%%%%%%%%%%%%%%%%%%%%%%%%%
    % Max: there are two ways to write these rounding rules. Curently, the
    % explicit style is favored where rnd and ret can be evaluated at no more
    % than two enviroments simultaneously.
    %%% Ret case:
    % 1. Explicit handling of multiple enviroments:
    % \gamma_0 \Vdash \mathbf{ret} \ v : M_0 \ \tau &\rightsquigarrow
    % \gamma_0 \Vdash v; \gamma_0 \Vdash v : M_0 \ \tau \\
    % \gamma_0; \gamma_1 \Vdash \mathbf{ret} \ v : M_0 \ \tau &\rightsquigarrow
    % \gamma_0 \Vdash v; \gamma_1 \Vdash v : M_0 \ \tau \\
    % 2. Equivalent setup with implicit handling of multiple enviroments:
    % \gamma_0; ...; \gamma_n \Vdash \mathbf{ret} \ v : M_0 \ \tau
    % &\rightsquigarrow \gamma_0 \Vdash v; \gamma_n \Vdash v : M_0 \ \tau \\
    %%% Rounding case:
    % 1. Explicit handling of multiple enviroments:
    % \gamma_0 \Vdash \mathbf{rnd} \ v : M_q \ \mathbf{num} &\rightsquigarrow
    % \gamma_0 \Vdash v; \gamma_0 \Vdash \rho(v) : M_q \ \mathbf{num} \\
    % \gamma_0; \gamma_1 \Vdash \mathbf{rnd} \ v : M_q \ \mathbf{num}
    % &\rightsquigarrow \gamma_0 \Vdash v; \gamma_1 \Vdash \rho(v) : M_q \
    % \mathbf{num} \\
    % 2. Equivalent setup with implicit handling of multiple enviroments:
    % \gamma_0; ...; \gamma_n \Vdash \mathbf{rnd} \ v : M_q \ \mathbf{num} &\rightsquigarrow
    % \Vdash \gamma_0 \Vdash v; \gamma_n \Vdash \rho(v) : M_q \ \mathbf{num} \\
    \sigma \Vdash \mathbf{ret} \ v : M_0 \ \tau &\rightsquigarrow (\sigma;
    \sigma) [\alpha \mapsto v :_1 \tau] \Vdash \alpha : M_0 \ \tau \\
    \sigma \Vdash \mathbf{rnd} \ v : M_q \ \mathbf{num} &\rightsquigarrow \sigma
    [\alpha \mapsto v :_1 \mathbf{num}]; \sigma [\alpha \mapsto \rho(v) :_1
    \mathbf{num}] \Vdash \alpha : M_q \ \mathbf{num} \\
    %%%%%%%%%%%%%%%%%%%%%%%%%%%%%%%%%%%%%%%%%%%%%%%%%%%%%%%%%%%%%%%%%%%%%%%%%%%
    % lam app
    \sigma \Vdash (\lambda x : \tau' .e) \ v : \tau &\rightsquigarrow \sigma[x
    \mapsto v :_1 \tau'] \Vdash e : \tau \\
    %%%%%%%%%%%%%%%%%%%%%%%%%%%%%%%%%%%%%%%%%%%%%%%%%%%%%%%%%%%%%%%%%%%%%%%%%%%
    % more boring rules
    % op(v) rule
    \sigma \Vdash \mathbf{op}(v) : \textbf{num} &\rightsquigarrow \sigma \Vdash op(v) :
    \textbf{num} \\
    % linear with rule
    \sigma \Vdash \langle v_0,v_1 \rangle : \tau_0 \times \tau_1
    &\rightsquigarrow \sigma[\alpha \mapsto v_0 :_1 t_0]; \sigma[\alpha \mapsto
    v_1 :_1 t_1] \Vdash \alpha : \tau_0 \times \tau_1\\
    % proj rule
    \sigma_0; \sigma_1 \Vdash \pi_i \ x : \tau &\rightsquigarrow \sigma_i \Vdash x
    : \tau \\ 
    % ini rules
    \sigma \Vdash \tin_i v &\rightsquigarrow \sigma[\alpha \mapsto v]
    \Vdash \ \tin_i \ \alpha \quad \text{(where $v$ is not a var)} \\
    % case .. of .. rule
    \sigma \Vdash \mathbf{case} \ \tin_i v \ \mathbf{of} \ (\tin_1 \ x.f_1 \ | \
    \tin_2 \ x.f_2 ) : \tau &\rightsquigarrow \sigma[x \mapsto v :_s \tau]
    \Vdash f_i : \tau \\
    \sigma [x \mapsto \tin_i \ y :_s \tau'] \Vdash \mathbf{case} \ x \ \mathbf{of} \ (\tin_1 \
    z.f_1 \ | \ \tin_2 \ z.f_2 ) : \tau &\rightsquigarrow \sigma[x \mapsto \ \tin_i \ y
    :_{s-1} \tau][z \mapsto y :_1 \tau'] \Vdash f_i : \tau \\ 
  \end{aligned}
\end{equation*}
    %%%%%%%%%%%%%%%%%%%%%%%%%%%%%%%%%%%%%%%%%%%%%%%%%%%%%%%%%%%%%%%%%%%%%%%%%%%
    % Structural rule
    %%%%%%%%%%%%%%%%%%%%%%%%%%%%%%%%%%%%%%%%%%%%%%%%%%%%%%%%%%%%%%%%%%%%%%%%%%%
    % \vskip 0.3em
    % \AXC{$\sigma_0 \Vdash e_0 \rightsquigarrow \sigma_0' \Vdash e_0'$}
    % \AXC{$\sigma_1 \Vdash e_1 \rightsquigarrow \sigma_1' \Vdash e_1'$}
    % \BIC{$\sigma_0 \Vdash e_0;\sigma_1 \Vdash e_1 \rightsquigarrow \sigma_0' \Vdash e_0';\sigma_1' \Vdash e_1'$}
    % \DisplayProof
    %%%%%%%%%%%%%%%%%%%%%%%%%%%%%%%%%%%%%%%%%%%%%%%%%%%%%%%%%%%%%%%%%%%%%%%%%%%
    % even more spicy rules
    %%%%%%%%%%%%%%%%%%%%%%%%%%%%%%%%%%%%%%%%%%%%%%%%%%%%%%%%%%%%%%%%%%%%%%%%%%%
    % let-bind rule
    \vskip 0.3em
    \AXC{$\sigma_1 \Vdash e : M_r \tau' \rightsquigarrow^*
    \sigma_1';\sigma_1'' \Vdash v_1 : M_r \tau'$}
    \AXC{$\sigma_1' @ \sigma_0 [x \mapsto v_1 :_s \tau'] \Vdash f \rightsquigarrow^*
    \sigma_2;\sigma_3 \Vdash v_2 : M_r \tau'$} 
    \AXC{$\sigma_1'' @ \sigma_0 [x \mapsto v_1 :_s \tau'] \Vdash f \rightsquigarrow^*
    \sigma_4;\sigma_5 \Vdash v_2 : M_r \tau'$} 
    \TIC{ $\sigma_0 + s \cdot \sigma_1 \Vdash \textbf{let-bind}_{(s, \tau')} \ x
    = e \ \tin \ f  : M_{s*r + q} \ \tau \rightsquigarrow \sigma_2; \sigma_5
    \Vdash v_2 : M_{s*r + q} \ \tau$}
    \DisplayProof
    % Note that to prove that the right hand side of the rewrite is
    % well-defined, we need to show that \sigma_1, \sigma_1', and \sigma_0 all
    % have the same shape / height 
    % We also need to have v_2 be syntatically the same in both branches.
    \vskip 0.3em
    \AXC{$\sigma_1 \Vdash e : !_s \tau' \rightsquigarrow^*
    \sigma_1' \Vdash v_1 : !_s \tau'$}
    \AXC{$\sigma_1' @ \sigma_0 [x \mapsto v_1 :_{s * t} \tau'] \Vdash f \rightsquigarrow^*
    \sigma_2 \Vdash v_2 : M_r \tau'$} 
    \BIC{ $\sigma_0 + s \cdot \sigma_1 \Vdash \textbf{let-cobind}_{(s, \tau')} \
    x = e \ \tin \ f  : \tau \rightsquigarrow \sigma_2 \Vdash v_2 : \tau $}
    \DisplayProof
    % let-pair rule
    \vskip 0.3em
    \AXC{$\sigma_1 \Vdash e : \tau_0 \rightsquigarrow^* \sigma_1' \Vdash (v_0,
    v_1) : \tau_0$}
    \AXC{$\sigma_1' @ \sigma_0 [x \mapsto v_0 : \tau_0][y \mapsto v_1 : \tau_1]
    \Vdash f \rightsquigarrow^* \sigma_2 \Vdash v_2 : M_r \tau'$} 
    \BIC{ $\sigma_0 + t \cdot \sigma_1 \Vdash \textbf{let-pair}_{(s, \tau_0,
    \tau_1)} \ (x, y) = e \ \tin \ f  : \tau \rightsquigarrow \sigma_2 \Vdash
    v_2 : \tau $}
    \DisplayProof
    %%%%%%%%%%%%%%%%%%%%%%%%%%%%%%%%%%%%%%%%%%%%%%%%%%%%%%%%%%%%%%%%%%%%%%%%%%%
    % misc stepping rules
    %%%%%%%%%%%%%%%%%%%%%%%%%%%%%%%%%%%%%%%%%%%%%%%%%%%%%%%%%%%%%%%%%%%%%%%%%%%
    % let rule
    % \vskip 0.3em
    % \AXC{$\frac{1}{s} \cdot \sigma_0 \Vdash e : \tau' \rightsquigarrow^*
    % v^0;v^1;...; v^n : \tau'$}
    % \UIC{ $\sigma_0 + \sigma_1 \Vdash \textbf{let}_{(s, \tau')} \ x = e \ \tin \
    % f  : \tau \rightsquigarrow \sigma_1[x \mapsto v^0 :_s \tau'];
    % \sigma_1^{-1}[x \mapsto v^0 :_s \tau'];...; \sigma'[x \mapsto v^n :_s \tau']
    % \Vdash f : \tau$}
    % \DisplayProof

    % let-pair rule
    % \vskip 0.3em
    % \AXC{$\frac{1}{s} \cdot \sigma_0 \Vdash e : \tau' \rightsquigarrow^*
    % (v_0^0, v_1^0);(v_0^1, v_1^1);...; (v_0^n, v_1^n) : \tau_0 \tensor \tau_1$}
    % \UIC{ $\sigma_0 + \sigma_1 \Vdash \textbf{let-pair}_{(s, \tau_0, \tau_1)} \
    % (x, y) = e \ \tin \ f  : \tau \rightsquigarrow \linebreak \sigma_1[x \mapsto
    % v_0^0 :_s \tau_0, y \mapsto v_1^0 :_s \tau_1]; \sigma_1^{-1}[x \mapsto v_0^1
    % :_s \tau_0, y \mapsto v_1^1 :_s \tau_1]; ...; \sigma_1^{-1}[x \mapsto v_0^n
    % :_s \tau_0, y \mapsto v_1^n :_s \tau_1] \Vdash f : \tau$}
    % \DisplayProof

    %%%%%%%%%%%%%%%%%%%%%%%%%%%%%%%%%%%%%%%%%%%%%%%%%%%%%%%%%%%%%%%%%%%%%%%%%%%
    % sharing rule 
    % We need to compute the union because of programs like:
    % \langle let-bind x = rnd v in (fun y => x), 0 \rangle
    % We also need to be using de Bruijn indicies to refer to terms on the
    % enviorment stack. (Otherwise envs might pollute each other.)
    %%%%%%%%%%%%%%%%%%%%%%%%%%%%%%%%%%%%%%%%%%%%%%%%%%%%%%%%%%%%%%%%%%%%%%%%%%%
    \vskip 0.3em
    \AXC{$\sigma \Vdash e_0 : \tau_0 \rightsquigarrow^* \sigma_0 \Vdash v_0:
    \tau_0$}
    \AXC{$\sigma \Vdash e_1 : \tau_1 \rightsquigarrow^* \sigma_1 \Vdash v_1 :
    \tau_1$}
    \BIC{$\sigma \Vdash \langle e_0, e_1 \rangle \ : \tau_0 \times \tau_1
    \rightsquigarrow \sigma_0 + \sigma_1 \Vdash \langle v_0, v_1 \rangle:
    \tau_0 \times \tau_1$}
    \DisplayProof

    \vskip 0.3em
    \AXC{$\sigma \Vdash e_0 : \tau_0 \rightsquigarrow^* \sigma_0 \Vdash v_0: \tau_0$}
    \AXC{$\sigma' \Vdash e_1 : \tau_1 \rightsquigarrow^* \sigma_1 \Vdash v_1 : \tau_1$}
    \BIC{$\sigma + \sigma' \Vdash ( e_0, e_1 ) \ : \tau_0 \tensor \tau_1
    \rightsquigarrow \sigma_0 + \sigma_1 \Vdash (v_0, v_1) : \tau_0 \tensor
    \tau_1$}
    \DisplayProof

    \vskip 0.3em
    \AXC{$\sigma \Vdash e : M_q \tau_1 \times M_r \tau_2 \rightsquigarrow
    (\sigma_0;\sigma_1);(\sigma_2; \sigma_3) \Vdash v : M_q \tau_1 \times M_r
    \tau_2$}
    \UIC{$\sigma \Vdash \factor e \ : M_{max(q, r)} (\tau_1 \times \tau_2)
    \rightsquigarrow (\sigma_0; \sigma_2);(\sigma_1; \sigma_3) \Vdash v : M_{max(q, r)}
    (\tau_1 \times \tau_2)$}
    \DisplayProof
    % TODO: define evaluation contexts and get rid of this def
    % \vskip 0.3em
    % \AXC{$\sigma \Vdash e : M_q \tau_1 \times M_r \tau_2 \rightsquigarrow \sigma' \Vdash e' : M_q
    % \tau_1 \times M_r \tau_2$}
    % \UIC{$\sigma \Vdash \factor e \ : M_s (\tau_1 \times \tau_2)
    % \rightsquigarrow \factor e' : M_s (\tau_1 \times \tau_2)$}
    % \DisplayProof

\end{center}
    \caption{Enviroment-style evaluation rules for \Lang, ordered by precedence.
    Note that $\alpha$ is a metavar. It denotes a fresh variable when appearing
    on the right hand side of the rewrite relation.
    Note also that during type checking but prior to running the operational
    semanitcs, the sensitivity information (tracked with metavar $s$) and type
    of bound variables $\tau', \tau_2$, is preserved as annotations in the
    syntax, written $[e]_s$ and $\textbf{let-bind}_{(s, \tau')}$,
    $\textbf{let-cobind}_{(s, t, \tau')}$, and $\lambda x : \tau' . e $.
    Computing the correct split $\sigma + \sigma'$ of an enviroment can be
    performed via type inference.
    Also note that the grade $q$ in $M_q$ for the monadic stepping rule upper
    bounds the difference $d_{\mathbf{num}}(k, \rho(k)) \leq q$ for all $k \in
    \llbracket \mathbf{num} \rrbracket$.
    % Note that $\textbf{let}^*$ is syntactic sugar
    % for matching all let expressions and their corresponding annotations.
    }
    \label{fig:sub_eval_rules}
\end{figure}
