\section{Dynamic Semantics}

\subsection{Substitution-style Operational Semantics}
The following is defined over untyped terms. In particular, we define the
operational semantics rewrite relation $\mapsto$ to map from (untyped) \Lang
to (untyped) \Lang. In other words, untypable programs can step (but not
necessarily to values).

\begin{figure}
\begin{center}

\begin{equation*}
\begin{aligned}[c]
	\mathbf{op}(v) &\mapsto op(v)\\
	\pi_i\langle v_1,v_2 \rangle &\mapsto v_i \\
	(\lambda x.e) \ v &\mapsto e[v/x] \\
	%\factor v \ &\mapsto v
\end{aligned}
\quad
\begin{aligned}[c]
	\letassign x = v \ \tin \ e &\mapsto e[v/x] \\
  \letpair (x, y) = (v, w) \ \tin \ e &\mapsto e[v/x][w/y] \\
	\letcobind x = [v] \ \tin \ e &\mapsto e[v/x]
	%\letbind x = \ret v \ \tin \ e &\mapsto e[v/x] \\
\end{aligned}
\end{equation*}
\vskip -1em
\begin{align*}
  \letbind y = (\letbind x = v \ \tin \ f) \ \tin \ g &\mapsto \letbind x = v \ \tin \ \letbind y = f \ \tin \ g \quad x\notin FV(g) 
\end{align*}
\vskip -1.75em
\begin{align*}
	\mathbf{case} \ (\mathbf{in}_i \ v) \ \mathbf{of} \ (\mathbf{in}_1 \ x.e_1 \ | \ \mathbf{in}_2 \ x.e_2 )  &\mapsto e_i[v/x]
  \qquad\qquad(i \in \{1, 2 \})
\end{align*}
\vskip -0.25em

	\AXC{$e \mapsto e'$}
	\UIC{$\letassign x = e \ \tin \ f \mapsto \letassign x = e' \ \tin f$}
	\DisplayProof

\end{center}
    \caption{Substitution-style evaluation rules for \Lang. Note the side condition for $\letbind$always holds for closed expressions.}
    \label{fig:sub_eval_rules}
\end{figure}

\subsection{(Typed) Enviroment-style Operational Semantics}
The following is defined over typed terms. In particular, we define the
operational semantics rewrite relation $\rightsquigarrow$ to map from a typed
term in an program enviroment to a typed term in an program enviroment.

To be precise, $\rightsquigarrow$ maps an expression $e$ with type $\tau$
running in an enviroment $\sigma$ mapping variables like $x_1$ to value $v_1$
with type $\tau_1$ and sensitivity budget $s_1$ to a $e'$ with type $\tau'$ and
enviroment $\sigma'$.

For an enviroment $\sigma$ compatible with typing context $\Gamma$, $\llbracket
\sigma \rrbracket$ is interpreted as a point within metric space $\llbracket
\Gamma \rrbracket$. Similarly, $\llbracket \sigma \Vdash e : \tau \rrbracket$ is
interpreted as the point in the metric space $\llbracket \tau \rrbracket$
obtained by running e at $\sigma$.

Useful theorem to prove. If $\sigma \Vdash e : \tau \rightsquigarrow \sigma'
\Vdash e' : \tau'$, then $\llbracket \sigma \Vdash e : \tau \rrbracket =
\llbracket \sigma' \Vdash e' : \tau' \rrbracket$.

Potentially wrong / bad theorem. If $\llbracket \sigma \Vdash e : \tau
\rrbracket = \llbracket \sigma' \Vdash v : \tau' \rrbracket$, then $\sigma
\Vdash e : \tau \rightsquigarrow^{*} \sigma' \Vdash v : \tau'$

Another theorem. For $\llbracket \Gamma \vdash e : \tau \rrbracket$ and a
$\sigma, \sigma'$ compatible with $\Gamma$, the distance between $\llbracket
\sigma \rrbracket$ and $\llbracket \sigma' \rrbracket$ in metric space
$\llbracket \Gamma \rrbracket$ is less than or equal to the distance between
$\llbracket \sigma \Vdash e : \tau \rrbracket$ and $\llbracket \sigma' \Vdash e : \tau \rrbracket$.

% Max: this \Vdash notation is intentionally highly suggestive.
% Max: not sure what to call this theorem, probably, there is something like
% this in prior work.
