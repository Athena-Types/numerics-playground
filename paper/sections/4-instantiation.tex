\section{Instantiation} \label{sec:encoding}
Numerical Fuzz's forwards error analysis relies on bounding Lipschitz function
sensitivity. However, subtraction, and by extension addition with negative
numbers, has unbounded Lipschitz sensitivity over the reals. This poses a
challenge to a compositional type-based analysis of programs involving
subtraction and negative numbers.

To avoid this problem, we translate our semantics to a \textit{paired
representation} where we associate for each real $r \in \mathbb{R}$ a lattice of
triples $(r, a, b) \in \mathbb{R} \times \mathbb{R}^+ \times \mathbb{R}^+
\triangleq \mathbb{P} = \mathit{num}$ under the $\leq$ relation over the first
component such that $r = a - b$. 
Accordingly, we equip our set $\mathbb{P}$ with a distance function
$d_{\textit{num}}((r, a, b)(r', a', b'))$ defined as the maxmimum distance over
the paried components (Definition~\ref{def:distance-p}).
Instead of reasoning over the real component, we instantiate our graded type
system to track distances and error over the always-growing paired components.
We then instantiate our automatic typed-based bounds analysis and apply our main
error theorem to translate our error bound back over the unpaired component.

The remainder of this section is organized as follows. In
Section~\ref{sec:instantiation}, we instantiate Negative Fuzz's modular
interface (Definition~\ref{def:numfuzz-interface}) with our $\textit{paired
representation}$. In particular, we define and type our supported operations
including subtraction under the paired representation. We describe our bounds
analysis setup and provide intuition for our choice of domain. In
Section~\ref{sec:application}, we use the results of our automatic type-based
bounds analysis to translate the monadic grade over the paired components into
useful bounds over the underlying unpaired real. We also present our main error
theorem over the unpaired real. We defer intrducing our approach to automatic
type inference to Section~\ref{sec:inference}.


% In the remainder of this section, we demonstrate how to use Negative Fuzz to
% provide error bounds in the presence of subtraction and negative numbers. 
% Then, in Section~\ref{sec:application} we use the bounds analysis to translate
% the grade on the paired representation into useful error bounds for the
% standard, unpaired, floating-point arithmetic.
% TODO: signpost more detailed

% outline:
% - explanation, signpost
% - instantiating modular interfacde (direct, concise, and technical)
% - reading out things (technical but mixed with some explanation)

\subsection{Instantiating the modular interface} \label{sec:instantiation}
To soundly use the paired representation, we must instantiate the modular
interface ($\textit{num}$, $d_\textbf{num}$, $\rho$, $u$, $\Sigma$,
$(\bnd{\mathbb{B}}, \leq)$, $\Sigma_{\bnd{bnd}}$) defined in
Definition~\ref{def:numfuzz-interface} and show that it satisfies the properties
specified. In the subsequent subsections, we instantiate the interface with the
appropriate rounding functions and round-off parameters
(Section~\ref{sec:rho-and-u}), supported arithmetic operations
(Section~\ref{sec:paired-ops}), bounds domain and bounds operations
(Section~\ref{sec:bnd-domain-and-op}), and finally prove the that our
instantation adheres to the spec demanded by Negative Fuzz
(Sections~\ref{sec:prop-a}-\ref{sec:prop-e}).

\subsubsection{Defining the rounding function and round-off parameters for the
paired representation.} \label{sec:rho-and-u}
We set $\textit{num} = \mathbb{P}$ equipped with $\leq$ defined above.
We also set $\rho = \rho_{\mathbb{P}}$ (Definition~\ref{def:rounding-p}) to
``simulate" the round-off error on the unpaired component.
Our choice of $u$ depends on the format and rounding mode used. 
% TODO: change thm statement to reflect this, add label and ref here
It is sound to use same corresponding choice of $u$ representing the
unit-roundoff constant as Numerical Fuzz (Lemma~\ref{thm:rounding-param}); we
reproduce the corresponding choices from Numerical Fuzz broken out by
floating-point format and rounding mode used in Table~\ref{tab:formats} and
Table~\ref{tab:rnd_modes} respectively. We define our operations $\Sigma$, bound
domain $(\bnd{\mathbb{B}}, \leq)$, and bound operations $\Sigma_{\bnd{bnd}}$
below.

\begin{table}
\caption{Parameters for floating-point number sets in the IEEE 754-2008
standard. For each, $emin = 1 - emax$.}
\label{tab:formats}
\begin{tabular}{ c c c c }
 \hline
\textbf{Parameter}& \textbf{binary32} & \textbf{binary64} &\textbf{binary128} \\
 \hline
 $p$   & 24    & 53 & 113     \\
$emax$ &   $+127$  & $+1023$ & $+16383$ \\
 \hline
\end{tabular}
\end{table}

\begin{table}
\caption{Common Rounding Functions (modes).}
\label{tab:rnd_modes}
\begin{tabular}{ c c c c }
 \hline
\textbf{Rounding mode} & \textbf{Behavior} & \textbf{Notation} & \textbf{Unit Roundoff ($u$)} \\
\hline
 Round towards $+\infty$   & $\min\{y \in \F \mid y \ge x \}$  & $\rho_{RU}(x)$ & $\beta^{1-p}$   \\
 Round towards $-\infty$   & $\max\{y \in \F \mid y \le x \}$  &   $\rho_{RD}(x)$ & $\beta^{1-p}$ \\
 Round towards $0$   &   $\rho_{RU}(x)$ if $x < 0 $, otherwise  $\rho_{RD}(x)$  & $\rho_{RZ}(x)$
	& $\beta^{1-p} $ \\
 Round towards nearest \footnote{For round towards nearest, there 
	are several possible tie-breaking choices.}   & $\{y \in \F \mid \forall z \in \F,  |x - y| \le |x-z| \} 
	 $   & $\rho_{RN(x})$ & $\frac{1}{2}\beta^{1-p}  $    \\
 \hline
\end{tabular}
\end{table}

\subsubsection{Defining our operations over the paired representation}
\label{sec:paired-ops}
We define $\Sigma = \{ \textbf{add}, \ \textbf{sub}, \ \textbf{mul} \}$, which
compute are are typed over our paired representationn in the following way:
\begin{equation}
\begin{aligned}[c]
  &\textit{add} : \forall \epsilon_1, \epsilon_2, \mathbf{num}_{\epsilon_1} \times
  \mathbf{num}_{\epsilon_2} \multimap \mathbf{num}_{\epsilon_1 + \epsilon_2} \\
  &\textit{add} \ \bnd{\{\epsilon_1\}} \ \bnd{\{\epsilon_2\}} \ (\langle (r, a,
  b), (r', a', b') \rangle) \mapsto (r + r', a + a', b + b') \\
  \\
  &\textit{sub} : \forall \epsilon_1, \epsilon_2, \mathbf{num}_{\epsilon_1} \times
  \mathbf{num}_{\epsilon_2} \multimap \mathbf{num}_{\epsilon_1 - \epsilon_2} \\
  &\textit{sub} \ \bnd{\{\epsilon_1\}} \ \bnd{\{\epsilon_2\}} \ (\langle (r, a, b),
  (r, a', b') \rangle ) \mapsto (r - r', a + b', b + a') \\
  \\
  &\textit{mul} : \forall \epsilon_1, \epsilon_2, \mathbf{num}_{\epsilon_1} \times
  \mathbf{num}_{\epsilon_2} \multimap \mathbf{num}_{\epsilon_1 \cdot \epsilon_2} \\
  &\textit{mul} \ \bnd{\{\epsilon_1\}} \ \bnd{\{\epsilon_2\}} \ (((r, a, b), (r, a',
  b'))) \mapsto (r * r', a * b + b * b', b * a' + a * b') \\
\end{aligned}
\end{equation}


% TODO: now we describe / define bounds and bound ops. make sure that the
% lettering of each of the props are ok
\subsubsection{Defining our bounds domain and bound operations}
\label{sec:bnd-domain-and-op}
Our bounds analysis is inspired by interval analysis. For each component in our
triple, we track the lower bound and upper bound. For example, for a pair $(1,
2, 3)$ we might track that $r^{\downarrow}$ (pronounced ``r lower bound") and
$r^{\uparrow}$ (pronounced ``r upper bound") and $a^{\downarrow}, a^{\uparrow},
b^{\downarrow}, b^{\uparrow}$ each bound their respective components. 
To improve the precision of our bounds analysis, we observe that frequently only
one of the paired components is greater than zero. In fact, an inspection of our
supported operations shows that $a > 0 \land b > 0$ can only be true when
either: (1) the program contains subtraction, or (2) the program contains
negative numbers. Further, we observe that when the ($a > 0 \land b > 0$)
condition holds, the bound drastically tightens.
% TODO: brag about Horner being much tighter (do we foreshadow or just drop some
% random eval claim here?)

% TODO: intuition on the bit, why we have it and what it tracks. emp on common
% case
Accordingly, for a concrete tuple 
$(r, a, b) \in \mathbb{P}$
we set our bound domain $\bnd{\mathbb{B}}$ to track seven things:
$((r^{\downarrow}, r^{\uparrow}), (a^{\downarrow}, a^{\uparrow}),
(b^{\downarrow}, b^{\uparrow}), d)$
under the following constraints:
\begin{enumerate}
  \item Lower and upper bounds on $r$, with the following constraints $r^{\uparrow} \leq r \leq r^{\downarrow}$
  \item Lower and upper bounds on $a$, with the following constraints $a^{\uparrow} \leq a \leq a^{\downarrow}$
  \item Lower and upper bounds on $b$, with the following constraints $b^{\uparrow} \leq b \leq b^{\downarrow}$
  \item A boolean $d$ tracking whether the condition $a > 0 \land b > 0$ is true.
\end{enumerate}
To define $\leq$ over our bounds domain, we first define a mapping of bounds to
sets:
\begin{definition}[Mapping of bounds to sets.]
Our mapping from bounds to sets $\textbf{toSet} : \bnd{\mathbb{B}} \to
\mathbf{Set}$ is defined as follows:
\begin{equation}
  \begin{aligned}[c]
   \textbf{toSet} ((r^{\downarrow}, r^{\uparrow}), (a^{\downarrow}, a^{\uparrow}), (b^{\downarrow}, b^{\uparrow}), d) 
    = \\ \{ (r, a, b) \ | \
    (r^{\downarrow} \leq r \leq r^{\uparrow}) \land
      (a^{\downarrow} \leq a \leq a^{\uparrow}) 
      \land (b^{\downarrow} \leq b \leq b^{\uparrow})
      \land ((a > 0 \land b > 0) = d) \\
    \forall (r, a, b) \in \mathbb{R} \times \mathbb{R}^+ \times \mathbb{R}^+\}
  \end{aligned}
\end{equation}
\end{definition}

Then, our $\leq$ relation over $\bnd{\mathbb{B}}$ is simply the $\subseteq$
relation over the \textbf{toSet} interpretation of each bound:
\begin{definition}[$\leq$ over $\bnd{\mathbb{B}}$.]
  $b_0 \leq b_1 \iff \textbf{toSet} b_0 \subseteq \textbf{toSet} b_1$
\end{definition}

Finally, we define our bound operations $\Sigma_{\bnd{bnd}} = \{+, -, \cdot \}$
mapping $\bnd{\mathbb{B}} \times \bnd{\mathbb{B}} \to \bnd{\mathbb{B}}$.
Each bound operation corresponds to the primitive operation in the surface
syntax of the language. The correctness of the definition of each bound
operation is ensured by the interface requirement that that each primitive
operation's corresponding type obeys metric preservation.
\begin{equation}
  \begin{aligned}[c]
    ((r^{\downarrow}, r^{\uparrow}), (a^{\downarrow}, a^{\uparrow}), (b^{\downarrow}, b^{\uparrow}), d) + 
    ((r'^{\downarrow}, r'^{\uparrow}), (a'^{\downarrow}, a'^{\uparrow}), (b'^{\downarrow}, b'^{\uparrow}), d')
    &= \\
    ((r^{\downarrow}_{add}, 
    r^{\uparrow}_{add}), 
      (a^{\downarrow} + a'^{\downarrow}, 
      a^{\uparrow} + a'^{\uparrow}),
      (b^{\downarrow} + b'^{\downarrow},
      b^{\uparrow} + b'^{\uparrow}),
      (d \land d' \land 
      ((0 \leq r^{\downarrow} \land 0 \leq r'^{\downarrow}) 
        \lor
    (r^{\uparrow} \leq 0 \land r'^{\uparrow} \leq 0)))) & \\
    ((r^{\downarrow}, r^{\uparrow}), (a^{\downarrow}, a^{\uparrow}), (b^{\downarrow}, b^{\uparrow}), d) -
    ((r'^{\downarrow}, r'^{\uparrow}), (a'^{\downarrow}, a'^{\uparrow}), (b'^{\downarrow}, b'^{\uparrow}), d')
    &= \\
    ((r^{\downarrow}_{sub}, r^{\uparrow}_{sub}),
    (a^{\downarrow} + b'^{\downarrow}, a^{\uparrow} + b'^{\uparrow}),
    (b^{\downarrow} + a'^{\downarrow}, b^{\uparrow} + a'^{\uparrow}),
      (d \land d' \land 
      ((0 \leq r^{\downarrow} \land r'^{\uparrow} \leq 0) 
        \lor
    (r^{\uparrow} \leq 0 \land 0 \leq r'^{\downarrow})))) &\\
    ((r^{\downarrow}, a^{\downarrow}, b^{\downarrow}),(r^{\uparrow}, a^{\uparrow}, b^{\uparrow})) \cdot
    ((r'^{\downarrow}, a'^{\downarrow}, b'^{\downarrow}),(r'^{\uparrow}, a'^{\uparrow}, b'^{\uparrow})) 
    &= \\
    ((r^{\downarrow}_{mul}, r^{\uparrow}_{mul}),
    (
    \text{min}(a^{\downarrow} a'^{\downarrow}, b^{\downarrow} b'^{\downarrow})
    , 
    \text{min}(a^{\uparrow} a'^{\uparrow}, b^{\uparrow} b'^{\uparrow})
    ), 
    (
    \text{max}(a^{\downarrow} b'^{\downarrow}, b^{\downarrow} a'^{\downarrow})
    , 
    \text{max}(a^{\uparrow} b'^{\uparrow}, b^{\uparrow} a'^{\uparrow}), 
    d \land d')) \\ 
    \text{if $d \land d'$}\\ 
    ((r^{\downarrow}, a^{\downarrow}, b^{\downarrow}),(r^{\uparrow}, a^{\uparrow}, b^{\uparrow})) \cdot
    ((r'^{\downarrow}, a'^{\downarrow}, b'^{\downarrow}),(r'^{\uparrow}, a'^{\uparrow}, b'^{\uparrow})) 
    &= \\
    ((r^{\downarrow}_{mul}, r^{\uparrow}_{mul}),
    (a^{\downarrow} a'^{\downarrow} + b^{\downarrow} b'^{\downarrow}, 
    a^{\uparrow} a'^{\uparrow} + b^{\uparrow} b'^{\uparrow}), 
    (a^{\downarrow} b'^{\downarrow} + b^{\downarrow} a'^{\downarrow}, 
    a^{\uparrow} b'^{\uparrow} + b^{\uparrow} a'^{\uparrow}), d \land d') \\
    \text{otherwise}\\ 
  \end{aligned}
\end{equation}
where 
$$r^{\downarrow}_{add} = min(\{x + y \ | \ \forall x \in [r^{\downarrow}, r^{\uparrow}] \ \forall y \in [r'^{\downarrow}, r'^{\uparrow}]\})$$
$$r^{\uparrow}_{add} = max(\{x + y \ | \ \forall x \in [r^{\downarrow}, r^{\uparrow}] \ \forall y \in [r'^{\downarrow}, r'^{\uparrow}]\})$$
$$r^{\downarrow}_{sub} = min(\{x - y \ | \ \forall x \in [r^{\downarrow}, r^{\uparrow}] \ \forall y \in [r'^{\downarrow}, r'^{\uparrow}]\})$$
$$r^{\uparrow}_{sub} = max(\{x - y \ | \ \forall x \in [r^{\downarrow}, r^{\uparrow}] \ \forall y \in [r'^{\downarrow}, r'^{\uparrow}]\})$$
$$r^{\downarrow}_{mul} = min(\{x \cdot y \ | \ \forall x \in [r^{\downarrow}, r^{\uparrow}] \ \forall y \in [r'^{\downarrow}, r'^{\uparrow}]\})$$
$$r^{\uparrow}_{mul} = max(\{x \cdot y \ | \ \forall x \in [r^{\downarrow},
r^{\uparrow}] \ \forall y \in [r'^{\downarrow}, r'^{\uparrow}]\})$$
% TODO: do we need more intuition here? maybe connect to abstract transformers?
% TODO: add intuition on logic for the bit, maybe break up and define each op
% one-by-one? maybe also tease horner pathological example?

We now prove that our instantation adheres to the properties demanded by the
interface spec of Negative Fuzz. By stapling together the following lemmas, the
interface ($\mathbb{P}, d_{\mathbb{P}}, \rho_{\mathbb{P}}, u, \Sigma,
(\mathbb{B}, \leq), \Sigma_{\mathbf{num}}$) satisfies the spec for instantiating
Negative Fuzz specified in Definition~\ref{def:numfuzz-interface}.

\subsubsection{Property A: The \textit{num} and distance function
$d_\textbf{num}$ form a metric space.} \label{sec:prop-a}
\begin{lemma}[Metric space]
  $\mathbb{P}$ forms a metric space under $d_\mathbb{P}$.
\end{lemma}
\begin{proof}
Symmetry and reflexivity hold trivially. Triangle inequality holds by unfolding
and appliation of triangle inequality with respect to $d_{\mathbb{R}}$.
\end{proof}

Our operations and bounds operate over the paired representation
$\mathbb{P}$. In this manner, subtraction and negative numbers can be encoded
using bounded-sensitivity operations over the non-negative reals, avoiding the
problem of infinite sensitivity.


\subsubsection{Property B: Property of $u$ and rounding function $\rho$.}

% TODO: keep the below
To show a correspondance with the constants for $u$ given in Numerical Fuzz and
Negative Fuzz, we show that it is sound to use the same round-off error bound
$u$ associated with $\rho_\mathbb{R}$ for $\rho_\mathbb{P}$.
In other words, we wish to show that the translated rounding function can share
the same round-off constant as the grade used in Numerical Fuzz over an unpaired
representation.

Given a rounding function $\rho_\mathbb{R}$ over the real with up to $u$
round-off error, recall that we define a ``simualted" rounding function
$\rho_{\mathbb{P}}$ over the paired representation
(Definition~\ref{def:rounding-p})
which satisfies the constraint that $r = a - b$ at all times. We need to show
that our round-off constants $u$ still work for our new paired representation
and paired rounding function $\rho_{\mathbb{P}}$.
\begin{lemma}[Property of rounding and constant parameter $u$]
We wish to show that given a unit round-off bound $u$ and $\rho_{\mathbb{R}}$ such that 
$\forall r \in \mathbb{R}, d_\mathbb{R} (\rho(r), r) \leq u$, then
    $\forall e \in \mathcal{R}_{num}, \mathcal{SD}_{\mathbf{num}}(\rho_{\mathbb{P}}(e), e)
    \leq u$. 
\end{lemma}
\begin{proof}
$(\rho_{\mathbb{P}}((r, a, b)), (r, a, b)) = 
(\rho_{\mathbb{R}}(r), a
\cdot \frac{\rho_{\mathbb{R}}(r)}{r}, b \cdot \frac{\rho_{\mathbb{R}}(r)}{r})$. 
So, 
$
d_{\mathbb{P}}((\rho_{\mathbb{R}}(r), a
\cdot \frac{\rho_{\mathbb{R}}(r)}{r}, b \cdot \frac{\rho_{\mathbb{R}}(r)}{r}), (r, a, b)) = max(\rho_{\mathbb{R}}(r), \rho_{\mathbb{R}}(r)) = \rho_{\mathbb{R}} (r)
$.
\end{proof}

\subsubsection{Property C: Property of $\mathbf{op} \in \Sigma$: metric preservation.}
Our operations satisfy metric preservation.
% TODO: elaborate some more?

\begin{lemma}[Operations satisfy metric preservation]
Each $\textbf{op} \in \Sigma$ satisfies metric preservation.
\end{lemma}
\begin{proof}
Holds by a case analysis for each operation in $\Sigma$ and unfolding the
definition of our metric function.
\end{proof}

\subsubsection{Property D: Partial order of bounds.}
\begin{lemma}[Partial order of $\bnd{\mathbb{B}}$.]
  $(\mathbb{B}, \leq)$ form a partial order.
\end{lemma}
\begin{proof}
  Follows from the fact that $\subseteq$ forms a partial order over
  $\mathbb{Set}$.
\end{proof}

\subsubsection{Property E: Property of $\mathbf{bop} \in \Sigma_{\bnd{bnd}}$:
closure over bounds ($\bnd{\mathbb{B}}$).} \label{sec:prop-e}

\begin{lemma}[Closure over bounds]
For every operation $\mathbf{bop}(b_0, \ldots, b_n) : \Sigma_\textbf{num}$, we
have that if $c_0, \ldots, c_n \in \mathbb{B}$ are $\textit{bop}(c_0, \ldots,
c_n) = c' : \textbf{bnd}$ holds.
\end{lemma}
\begin{proof}
  Follows by inspection of the definition of our three supported operations.
\end{proof}

\subsection{Translating our error bounds from the paired to the unpaired
representation} \label{sec:application} 
% TODO: completed up to here
For a given $. \vdash e : M_q \textbf{num}_{\bnd{b}}$, we wish to bound the
error between $e$ computed ideally over the reals versus the floating-point
approximation. 
To do this, we prove our paired error theorems
(Theorem~\ref{thm:paired-a-priori-rel},
Theorem~\ref{thm:paired-a-posteriori-rel}), which for a triple $(r, a, b)$
relates error on the paired components $a, b$ with error on $r$. By plugging in
the theorems proved below, we are able obtain useful bounds over the unpaired
representation for a given typed program.
We have two main error theorems in this paper, detailed below: a relative error
theorem (Theorem~\ref{thm:paired-a-priori-rel}) and an absolute error theorem
(Theorem~\ref{thm:paired-a-priori-abs}).

Note that for the below theorem we have the condition $0 < r^{\downarrow}$. This
can be generalized to $r^{\downarrow}$ and $r^{\uparrow}$ having the same sign
by a mirrored proof. When the lower and upper bounds on $r$ straddle zero, we
cannot produce a relative error bound.
\begin{theorem}[Paired error theorem; a priori relative] \label{thm:paired-a-priori-rel}
  For any numbers $((r, a, b), (r', a', b')) \in \mathbb{P}^2$ and where we have
  $r^{\downarrow}, r^{\uparrow}, a^{\downarrow}, a^{\uparrow}, b^{\downarrow}, b^{\uparrow} \in \mathbb{R}^{+}$
  the following bounds
  \begin{enumerate}
    \item $d_{\mathbb{R}}(a, a'), d_{\mathbb{R}}(b, b') \leq q$
    \item $r \in [r^{\downarrow}, r^{\uparrow}]$ and $0 < r^{\downarrow}$
    \item $a \in [a^{\downarrow}, a^{\uparrow}]$
    \item $b \in [b^{\downarrow}, b^{\uparrow}]$
  \end{enumerate}
  we have that $d_{\mathbb{R}}(r, r')$ is less than or equal to all of the
  following bounds: 
\begin{enumerate}
  \item[(a)] $max(|ln(e^{-q} + \frac{a^{\uparrow}}{r^{\downarrow}}(e^{q} - e^{-q}))|, |ln(e^{q} + \frac{a^{\downarrow}}{r^{\uparrow}}(e^{-q} - e^{q}))|)$
  \item[(b)] $max(|ln(e^q + \frac{b^{\uparrow}}{r^{\downarrow}}(e^q - e^{-q}))|,|ln(e^{-q} + \frac{b^{\downarrow}}{r^{\uparrow}}(e^{-q} - e^{q}))|)$
\end{enumerate} 
\end{theorem}
\begin{proof}
  Our error metric is symmetric so for this theorem we wish to prove a bound on
  $d_{\mathbb{R}}(r', r)$. We prove each bound and, for each bound, split into
  two cases: $r \leq r'$ or $r > r'$. In the first case, we have:
  \begin{equation}
  \begin{aligned}[c]
  1 \leq 
  \frac{r'}{r} \leq
  \frac{a' - be^{-q}}{a - b} =
  \frac{a' - be^{-q} + ae^{-q} - ae^{-q}}{a - b} \\ =
  e^{-q} + \frac{a' - ae^{-q}}{a - b} \leq
  e^{-q} + \frac{ae^{q} - ae^{-q}}{a - b} =
  e^{-q} + \frac{a}{a - b}(e^{q} - e^{-q}) \\ \leq
  e^{-q} + \frac{a^{\uparrow}}{r^{\downarrow}}(e^{q} - e^{-q}) 
  \end{aligned}
  \end{equation}
  Therefore, 
  \begin{equation} \label{eq:a-priori-bnd-a-pos}
  \begin{aligned}[c]
  0 \leq ln(\frac{r'}{r}) \leq 
  ln(e^{-q} + \frac{a^{\uparrow}}{r^{\downarrow}}(e^q - e^{-q}))
  \end{aligned}
  \end{equation}
  In the second case, we have:
  \begin{equation}
  \begin{aligned}[c]
    1 \geq
    \frac{r'}{r} \geq
    \frac{a' - be^{q}}{a - b} = 
    \frac{a' - be^{q} + ae^{q} - ae^{q}}{a - b} \\ =
    e^{q} + \frac{a' - ae^{q}}{a - b} \geq
    e^{q} + \frac{ae^{-q} - ae^{q}}{a - b} = 
    e^{q} + \frac{a}{a-b}(e^{-q} - e^{q}) \\ \geq
    e^{q} + \frac{a^{\downarrow}}{r^{\uparrow}}(e^{-q} - e^{q})
  \end{aligned}
  \end{equation}
  Therefore, 
  \begin{equation} \label{eq:a-priori-bnd-a-neg}
  \begin{aligned}[c]
    0 \geq ln(\frac{r'}{r}) \geq
    ln(e^{q} + \frac{a^{\downarrow}}{r^{\uparrow}}(e^{-q} - e^{q}))
  \end{aligned}
  \end{equation}

  A sound bound for both cases is the maximum of the absolute value of
  Equation~\ref{eq:a-priori-bnd-a-pos} and Equation~\ref{eq:a-priori-bnd-a-neg},
  which is exactly our first bound (a). The proof strategy for the second bound
  (b) mirrors that of the first bound.
\end{proof}

\begin{corollary}
  For $. \vdash e : M_q~\mathbf{num}_{((r^{\downarrow}, r^{\uparrow}),
  (a^{\downarrow}, a^{\uparrow}), (b^{\downarrow}, b^{\uparrow}), d)}$, $e
  \mapsto ((r, a, b), (r', a', b'))$ where $d_{\mathbb{R}}(r, r')$ is less than
  or equal to all of the following bounds: 
\begin{enumerate}
  \item[(a)] $max(|ln(e^{-q} + \frac{a^{\uparrow}}{r^{\downarrow}}(e^{q} - e^{-q}))|, |ln(e^{q} + \frac{a^{\downarrow}}{r^{\uparrow}}(e^{-q} - e^{q}))|)$
  \item[(b)] $max(|ln(e^q + \frac{b^{\uparrow}}{r^{\downarrow}}(e^q - e^{-q}))|,|ln(e^{-q} + \frac{b^{\downarrow}}{r^{\uparrow}}(e^{-q} - e^{q}))|)$
\end{enumerate} 
\end{corollary} 
\begin{proof}
  By our logical relation and metric preservation theorem (Theorem
  \ref{thm:metric-preservation}), we know that $e \mapsto ((r, a, b), (r', a',
  b'))$ such that $d_{\mathbb{P}}((r, a, b), (r', a', b')) \leq q$. Therefore,
  $d_{\mathbb{R}}(a, a'), d_{\mathbb{R}}(b, b') \leq q$. Applying the previous
  theorem proves the corollary.
\end{proof}

\begin{theorem}[Paired error theorem; a priori absolute] \label{thm:paired-a-priori-abs}
  For any numbers $((r, a, b), (r', a', b')) \in \mathbb{P}^2$ and where we have
  the following bounds for $r^{\downarrow}, r^{\uparrow} \in \mathbb{R}$, 
  $a^{\downarrow}, a^{\uparrow}, b^{\downarrow}, b^{\uparrow} \in \mathbb{R}^{+}$:
  \begin{enumerate}
    \item $d_{\mathbb{R}}(a, a'), d_{\mathbb{R}}(b, b') \leq q$
    \item $r \in [r^{\downarrow}, r^{\uparrow}]$
    \item $a \in [a^{\downarrow}, a^{\uparrow}]$
    \item $b \in [b^{\downarrow}, b^{\uparrow}]$
  \end{enumerate}
  we have that $|r - r'|$ is the maximum of the following:
\begin{enumerate}
  \item[(a)] $a^{\uparrow}(1-e^{-q}) - b^{\downarrow}(1 - e^q)$
  \item[(b)] $a^{\downarrow}(1-e^{-q}) - b^{\uparrow}(1 - e^q)$
  \item[(c)] $a^{\uparrow}(e^{q}-1) - b^{\downarrow}(e^{-q} - 1)$
  \item[(d)] $a^{\downarrow}(e^{q}-1) - b^{\uparrow}(e^{-q} - 1)$
\end{enumerate} 
\end{theorem} 
\begin{proof}
  We case into whether $r \geq r'$ (or not) and $r \geq 0$ (or not). This
  produces four cases:
  \begin{description}
    \item [Case $r \geq r'$ and $r \geq 0$.] We have that:
      \begin{equation}
        \begin{aligned}
          (a - b) - (a' - b') &\leq (a - b) - (ae^{-q} - be^{q}) \\
                              &= a(1-e^{-q}) - b(1 - e^q)
                              &= a^{\uparrow}(1-e^{-q}) - b^{\downarrow}(1 - e^q)
        \end{aligned}
      \end{equation}
    \item [Case $r \geq r'$ and $r < 0$.]
      \begin{equation}
        \begin{aligned}
          (a - b) - (a' - b') &\leq (a - b) - (ae^{-q} - be^{q}) \\
                              &= a(1-e^{-q}) - b(1 - e^q)
                              &= a^{\downarrow}(1-e^{-q}) - b^{\uparrow}(1 - e^q)
        \end{aligned}
      \end{equation}
    \item [Case $r < r'$ and $r \geq 0$.]
      \begin{equation}
        \begin{aligned}
          (a' - b') - (a - b) &\leq ((ae^q - be^{-q}) - (a - b)) \\
                              &= a(e^{q}-1) - b(e^{-q} - 1)
                              &= a^{\uparrow}(e^{q}-1) - b^{\downarrow}(e^{-q} - 1)
        \end{aligned}
      \end{equation}
    \item [Case $r < r'$ and $r < 0$.]
      \begin{equation}
        \begin{aligned}
          (a' - b') - (a - b) &\leq ((ae^q - be^{-q}) - (a - b)) \\
                              &= a(e^{q}-1) - b(e^{-q} - 1)
                              &= a(e^{q}-1) - b(e^{-q} - 1)
        \end{aligned}
      \end{equation}
  \end{description}
  By taking the maximum of all possible cases, we can obtain a sound a priori
  absolute error bound.
\end{proof}

\begin{corollary}
  For $. \vdash e : M_q~\mathbf{num}_{((r^{\downarrow}, r^{\uparrow}),
  (a^{\downarrow}, a^{\uparrow}), (b^{\downarrow}, b^{\uparrow}), d)}$, 
  we know that $e \mapsto^{*} ((r, a, b), (r', a', b'))$ where $|r - r'|$ is the
  maximum of the following:
\begin{enumerate}
  \item $a^{\uparrow}(1-e^{-q}) - b^{\downarrow}(1 - e^q)$
  \item $a^{\downarrow}(1-e^{-q}) - b^{\uparrow}(1 - e^q)$
  \item $a^{\uparrow}(e^{q}-1) - b^{\downarrow}(e^{-q} - 1)$
  \item $a^{\downarrow}(e^{q}-1) - b^{\uparrow}(e^{-q} - 1)$
\end{enumerate} 
\end{corollary} 
\begin{proof}
  By our logical relation and metric preservation theorem (Theorem
  \ref{thm:metric-preservation}), we know that $e \mapsto ((r, a, b), (r', a',
  b'))$ such that $d_{\mathbb{P}}((r, a, b), (r', a', b')) \leq q$. Therefore,
  $d_{\mathbb{R}}(a, a'), d_{\mathbb{R}}(b, b') \leq q$. Applying the previous
  theorem proves the corollary.
\end{proof}

% TODO: is there an a posterori absolute error theorem we can come up with?

