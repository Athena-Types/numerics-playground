\section{Type Soundness}
Following Fuzz, we prove the following syntanctic properties relating our type
systema and operational semantics.

\begin{lemma}[Weakening]\label{thm:weakening}
  If $\Gamma \vdash e : \tau$, then $\Gamma + \Sigma \vdash e : \tau$.
\end{lemma}
\begin{proof}
  We induct over our typing devivation. 
  For each case, we can ignore arbitrary variables in our context (the type system is
  affine). 
  In particular, for the \textbf{!E}, \textbf{$\otimes$ E}, \textbf{+ E},
  \textbf{$M_u$~E}, and \textbf{$\otimes$ I} rules, the enviroment must be split
  into a $\Gamma$ (which can in some cases be scaled) and $\Delta$ (which does
  not ever get scaled); for each of these cases we choose to push unused
  variables into $\Delta$. 
\end{proof}

Following Fuzz, we prove a $r$-sensitive substitution lemma.
\begin{lemma}[$r$-sensitive subsitution]\label{thm:substitution}
  Let $\Gamma \vdash e : \tau$ and $\Theta, x :_r \tau \vdash e' : \tau'$, then 
  $r \cdot \Gamma + \Theta \vdash e'[e/x] : \tau'$.
\end{lemma}
\begin{proof}
  Follows by induction over the height of our typing derivation and applying the
  inductive hypothesis to each applicable premise.
\end{proof}

The following notation mirrors the look of the Fuzz metric perservation theorem
statement but contains differences in the construction. In particular, logical our
relation is neither coinductive nor step-indexed. It is also unary, using mutual
(well-founded) recursion between the definition of a syntactic term falling
within the logical relation and the definition of the syntactic distance between
terms.

For any two closed expressions $e_0, e_1$ falling in same type relation $e_0,
e_1 \in R_\tau$, we can write $e_0 \sim_r e_1$ where $\mathcal{SD}_{\tau}(e_0,
e_1) \leq r$. Similarly, we can write lists of expressions (e.g. for
subsitutions) $\sigma_0, \sigma_1$ for a given typing context $\Gamma$ such
that: $\sigma_0 \sim_{\gamma} \sigma_1 : \Gamma$ for a \textit{distance vector}
$\gamma$
where
$\sigma_0 = e^0_0,~e^0_1,~\ldots e^0_n$ 
and $\sigma_1 = e^1_0,~e^1_1,~\ldots e^1_n$ 
and $\gamma = r_0, r_1, \ldots r_n$ such that:
$$e^0_0 \sim_{r_0} e^1_0 :
\tau_0,~e^0_0 \sim_{r_0} e^1_0 : \tau_0,~\ldots e^0_n \sim_{r_1} e^1_n :
\tau_n$$

Our definition for the dot product of a distance vector is the same as Fuzz.
% todo: put it in here

\begin{theorem}[Metric preservation]
  For any $\Gamma \vdash e : \tau$ and substitutions $\sigma, \sigma'$ such that
  $\sigma \sim_{\gamma} \sigma' : \Gamma$, then 
  $e~\sigma \sim_{\gamma \cdot \Gamma} e~\sigma'$.
\end{theorem}
\begin{proof}
  We induct over our typing derivation. The base cases (Var, Unit, Const) follow
  trivially. The subsumption case also follows trivially. We detail the
  remaining cases here:
  \begin{description}
    \item[Case $\multimap$ I.] 
      We wish to show that for any $\Gamma \vdash \lambda x . e : \tau$ and
      subsitutions $\sigma \sim_{\gamma} \sigma' : \Gamma$ that $\lambda x .
      e~\sigma \sim_{\gamma \cdot \Gamma} \lambda x . e~\sigma'$. Unfolding, it
      suffices to show that both:
      \begin{enumerate}
        \item $\lambda x . e~\sigma, \lambda x . e~\sigma'$ are in
          $\mathcal{R}_{\tau_0 \multimap \tau}$
        \item $\mathcal{SD}_{\tau_0 \multimap \tau}(\lambda x . e~\sigma,
          \lambda x . e~\sigma') \leq \gamma \cdot \Gamma$
      \end{enumerate}
    
      Observe that we have by our inductive hypothesis that for any $\Delta, x:
      \tau' \vdash e : \tau$ and subsitutions $\delta[v_0/x] \sim_{\gamma'}
      \delta'[v_1/x] : \Gamma,~x : \tau'$ and that $e~\delta[v_0/x]
      \sim_{\gamma' \cdot \Gamma} e~\delta'[v_1/x]$. Let us now carry on with
      the proof:

      \begin{description}
        \item[\underline{Property 1.}] We need to show that $\sigma$ and
          $\sigma'$ are both in the relation $\mathcal{R}_{\tau_0 \multimap
          \tau}$. The cases are symmetric so we only show one case. Unfolding
          the definition of $\mathcal{R}_{\tau_0 \multimap \tau}$, let $w_0, w_1
          \in \mathcal{VR}_{\tau_0}$. 
          $(\lambda x . e) w_0 \mapsto e[w_0/x]$
          and
          $(\lambda x . e) w_1 \mapsto e[w_1/x]$
          and so it suffices to show that
          $e~\sigma[w_0/x], e~\sigma[w_1/x]$ 
          are closed expressions falling in $R_{\tau_0}$.
          This is true by our inductive hypothesis, instantiating with $\sigma =
          \delta = \delta'$, $v_0 = w_0$, and $v_1 = w_1$.
        \item[\underline{Property 2.}] Unfolding the definition of
          $\mathcal{SD}_{\tau_0 \multimap \tau}$, it suffices to show that:
          $$
          \mathcal{SD}_{\tau}
          ((\lambda x . e~w)~\sigma, (\lambda x . e~w)~\sigma') 
          \leq \gamma \cdot \Gamma
          $$
          Stepping, it suffices to show that
          $$
          \mathcal{SD}_{\tau}
          (e~\sigma[w/x], e~\sigma'[w/x]) 
          \leq \gamma \cdot \Gamma
          $$
          which holds by application of our inductive hypothesis when $\delta =
          \sigma$, $\delta' = \sigma'$, $v_0 = v_1 = w$, and $\gamma' = \gamma
          :: 0$.
      \end{description}
      So, by our inductive hypothesis and unfolding the definitions of
      $\mathcal{R}$ and $\mathcal{SD}$, we have properties (1) and (2)
      respectively. 
  \end{description}
\end{proof}

% \begin{lemma}[Metric preservation]
% \end{lemma}
%
% \begin{lemma}[Type error soundness]
% \end{lemma}
