\section{Type Soundness}
\begin{lemma}[Termination]
  If $\Gamma \vDash e : \tau$, then $\forall \sigma$ compatible with $\Gamma$, 
  $$\sigma \Vdash e \rightsquigarrow^* \sigma' \Vdash v$$
\end{lemma}
\begin{proof}
  If $\sigma \Vdash e$ is a value, then we are done. Otherwise, if $\sigma
  \Vdash e$ is an expression, we begin by unfolding and applying the definition
  of $\vDash$. By definition, know that $\llbracket \sigma \Vdash e \rrbracket_{\tau} \in
  \llbracket \tau \rrbracket$. We also know that $\bot \not\in \llbracket \tau
  \rrbracket$, a metric space, so $\llbracket \sigma \Vdash e \rrbracket_{\tau}
  \not= \bot$ and is well-defined.

  By inspection of the definition of cases of $\llbracket - \rrbracket_{\tau}$,
  we know that only last case (which deals with all expressions) must apply.
  Therefore, $\sigma \Vdash e \rightsquigarrow^* \sigma' \Vdash v$.
\end{proof}

% \begin{theorem}[Semantic type preservation]
% If a term $e$ is semantically well-typed in context $\Gamma$ with type $\tau$
%   then for all $\sigma \Vdash e \rightsquigarrow \sigma' \Vdash e'$, $\sigma$
%   compatible with $\Gamma$, then there exists a $\Gamma'$ such that $\Gamma'
%   \vDash e' : \tau$.
% \end{theorem}
% \begin{proof}
% If $e$ is a value, we're done. 
% If $e$ is not a value, we proceed by induction over the cases of the rewrite relation
% $\rightsquigarrow$.
% % Our inductive hypothesis is that $e \rightsquigarrow e'$ and $\Gamma \vDash e : \tau$, that is, 
% % $\forall \sigma \text{ compatible with } \Gamma, \llbracket \sigma \Vdash e
% % \rrbracket_{\tau} \in \llbracket \tau \rrbracket$.
% \begin{description}
%   \item[\textsc{Variable lookup.}] If $$
%   \item[\textsc{Ret.}]
%   \item[\textsc{Rnd.}]
% \end{description}
% \end{proof}

% \begin{definition}[Monadic type]\label{def:monadic}
%   A type $\tau$ is monadic if and only if $\exists \tau', q \text{ such that }
%   \tau = M_q \tau'$. It is non-monadic otherwise.
% \end{definition}
% %%% wrong theorem:
% \begin{lemma}[Non-monadic lookup]\label{thm:non-monadic-lookup}
%   For all $\tau$ non-monadic and all $\Gamma, x : \tau'$ compatible with
%   $\sigma$, $\exists \sigma', \sigma = \sigma'[x \mapsto v : \tau']$ is
%   well-defined. Stated equivalently, non-monadic $x$ uniformly maps to $v$ at
%   all enviroments in $\sigma$.
% \end{lemma}
% \begin{proof}
%   TODO
% \end{proof}

\begin{lemma}[Enviroment tree shape]
  The tree height for a 
\end{lemma}

\begin{definition}[Orderings]\label{def:orderings}
  We define the following well-founded partial order over configurations and
  types, which mirrors the definition of machine configuration interpretation
  and will be used for induction in the proceeding proofs.

  First, we define an ordering over types by size of the AST.
  \begin{equation}
    \begin{aligned}[c]
      s(\textbf{unit}) &= 1 \\
      s(\textbf{num}) &= 1 \\
      s(!_s \tau) &= s(\tau) + 1 \\
      s(M_q \tau) &= s(\tau) + 1 \\
      s(\tau_0 \times \tau_1) &= s(\tau_0) + s(\tau_1) + 1 \\
      s(\tau_0 \otimes \tau_1) &= s(\tau_0) + s(\tau_1) + 1 \\
      s(\tau_0 \multimap \tau_1) &= s(\tau_0) + s(\tau_1) + 1 \\
    \end{aligned}
  \end{equation}
  and
  \begin{equation}
    \tau_0 < \tau_0 \iff s(\tau_0) < s(\tau_1)
  \end{equation}

  Secondly, we define an ordering over environment leafs $\gamma$ using the number of bound
  variables:
  \begin{equation}
    \begin{aligned}[c]
      s(.) &= 0 \\
      s(\gamma, x \mapsto v :_s \tau) &= s(\gamma) + 1 \\
    \end{aligned}
  \end{equation}

  and

  \begin{equation}
    \gamma_0 < \gamma_0 \iff s(\gamma_0) < s(\gamma_1)
  \end{equation}

  We can now define an ordering over enviorment trees using the maximum size of
  each enviorment leaf in the tree:
  \begin{equation}
    \begin{aligned}[c]
      s(\gamma; \gamma') &= max(s(\gamma), s(\gamma')) \\
      s(\sigma; \sigma') &= max(s(\sigma), s(\sigma'))
    \end{aligned}
  \end{equation}

  We are now ready to define an ordering over $\textit{config} \times
  \textit{type}$:
  \footnote{This is our termination measure for our machine configuration
  interpretation function.}
  \footnote{Note that the type of the configuration (if the interpretation of
  the configuration belongs to the interpretation of the type) restricts the
  height and shape of environment tree. So we do not need to reason about the
  height of our enviroment trees in our ordering.}

  \begin{equation}
    \sigma \Vdash e : \tau < \sigma' \Vdash e' : \tau' \iff
    s(\tau) < s(\tau') \lor (s(\tau) = s(\tau') \land s(\sigma) < s(\sigma'))
  \end{equation}

\end{definition}

\begin{lemma}[Well-founded partial ordering]
  $\leq$ is a well-founded partial order over $\textit{config} \times
  \textit{type}$.
\end{lemma}
\begin{proof}
  TODO
\end{proof}

\begin{theorem}[Semantic type soundness for values]
$\Gamma \vdash v : \tau \implies \Gamma \vDash v : \tau$
\end{theorem}
\begin{proof}
From our premise, we wish to show that:
  \begin{enumerate}
    \item $\forall \sigma$ compatible with $\Gamma$, $\llbracket \sigma \Vdash v
      \rrbracket_{\tau} \in \llbracket \tau \rrbracket$ and well-defined.
    \item $\llbracket - \Vdash v \rrbracket_{\tau}$ is a \text{1-sensitive} map
      for all inputs enviroments compatible with $\Gamma$.
  \end{enumerate}
We begin by induction over our typing derivation. 
  % In some of the cases, we will need to induct over the size of our enviroment. 
\begin{description}
  \item[\textsc{(ty. rule) Var.}] There is no premise containing a typing
    judgement to this rule. So we have no inductive hypothesis to rely on.
    (This is a base case.)
    % We proceed to induct over the type of our variable, $\tau$.
    % \begin{description}
    %   \item{\textbf{unit} and \textbf{num}.}
    %   \item{$\tau_0 \otimes \tau_1$, $\tau_0 + \tau_1$, and $\tau_0 \multimap
    %     \tau_1$.}
    %   \item{$!_s \tau$.}
    %   \item{$M_q \tau$}
    %   \item{$\tau_0 \times \tau_1$.}
    % \end{description}
    Instead, we induct over the the size of our enviorment.
    \begin{description}
      \item[\textit{(env. size) 0.}] We observe that an enviorment size of zero
        implies that $\Gamma$ is empty. So we are trying to show that the
        following:
        $$x : \tau \vDash x : \tau$$
        implies:
        $$\llbracket \sigma \Vdash x \rrbracket_\tau \in \llbracket \tau
        \rrbracket$$
        for $\sigma$ compatible with typing context $x : \tau$ and is
        \text{1-sensitive}. We proceed to induct over the size of type $\tau$,
        namely $s(\tau)$:
        \begin{description}
          \item{\underline{(ty. case) \textbf{unit.}}} The only configuration compatible with this is
            $x \mapsto \langle \rangle \Vdash x$ and clearly 
            $$\llbracket x \mapsto \langle \rangle \Vdash x
            \rrbracket_{\textbf{unit}} = \llbracket \Vdash \langle \rangle
            \rrbracket_{\textbf{unit}} = * \in \{ * \} = \llbracket \textbf{unit}
            \rrbracket$$
            So, properties (1) and (2) hold by the reasoning above.
          \item{\underline{(ty. case) $\mathbf{num}$.}} 
            The only configurations compatible
            with this is $x \mapsto k \Vdash x$ for $k \in \mathbb{R}$ and
            clearly $$\llbracket x \mapsto k \Vdash x \rrbracket_{\textbf{num}}
            = \llbracket \Vdash k \rrbracket_{\textbf{num}} = k \in \mathbb{R} =
            \llbracket \textbf{num} \rrbracket$$ implies that $\llbracket -
            \Vdash x \rrbracket_{\mathbf{num}}$ for this case is the
            $\text{1-sensitive}$ identity function over the reals. Therefore,
            properties (1) and (2) both hold.
          \item{\underline{(ty. cases) $\times$ and $\otimes$.}} By inversion on
            our compatibility relation, we know that we must have some $\sigma =
            (x \mapsto v_0; x \mapsto v_1)$ for $\llbracket \Vdash v_0
            \rrbracket_{\tau_0} \in \llbracket \tau_0 \rrbracket$ and
            $\llbracket \Vdash v_1 \rrbracket_{\tau_1} \in \llbracket \tau_1
            \rrbracket$. By definition, $\llbracket x \mapsto v_0; x \mapsto v_1
            \Vdash x \rrbracket_{\tau_0 \times \tau_1} = (\llbracket x \mapsto
            v_0 \Vdash x \rrbracket_{\tau_0}, \llbracket x \mapsto v_1 \Vdash x
            \rrbracket_{\tau_1}) = (\llbracket \Vdash v_0 \rrbracket_{\tau_0},
            \llbracket \Vdash v_1 \rrbracket_{\tau_1})$ which is clearly
            \text{1-sensitive} in the interpretation of our type. So properties
            (1) and (2) hold.
          \item{\underline{(ty. case) $+$.}} By inversion on our compatibility
            relation, we know that we have $\sigma = x \mapsto v$ for
            $\llbracket \Vdash v \rrbracket_{\tau_0 + \tau_1} \in \llbracket
            \tau_0 + \tau_1 \rrbracket = \llbracket \tau_0 \rrbracket \uplus
            \llbracket \tau_1 \rrbracket$. 
            There are two cases: 
            (1) $\llbracket \Vdash v \rrbracket_{\tau_0} \in \llbracket \tau_0
            \rrbracket$ and therefore $\llbracket x \mapsto v \Vdash x
            \rrbracket_{\tau_0 + \tau_1} = (0, \llbracket x \mapsto v \Vdash x
            \rrbracket_{\tau_0}) = (0, \llbracket \Vdash v
            \rrbracket_{\tau_0})$; 
            or, (2) $\llbracket \Vdash v \rrbracket_{\tau_1} \in \llbracket
            \tau_1 \rrbracket$ and therefore $\llbracket x \mapsto v \Vdash x
            \rrbracket_{\tau_0 + \tau_1} = (1, \llbracket x \mapsto v \Vdash x
            \rrbracket_{\tau_1}) = (1, \llbracket \Vdash v
            \rrbracket_{\tau_1})$. In both cases property 1 clearly holds.
            Property 2 holds by a similar case analysis.
          \item{\underline{(ty. case) $\tau_0 \multimap \tau_1$.}} 
            This case holds by the conditions on our compatibility relation and
            unfolding the interpretation of a machine configuration.
          \item{\underline{(ty. case) $!_s \tau$.}} 
            We begin by unfolding the definition of our compatibility relation
            and applying our machine configuration interpretation definition:
            $$ 
            \llbracket x \mapsto v :_s \tau \Vdash x \rrbracket_{!_s \tau} = 
            \llbracket x \mapsto v : \tau \Vdash x \rrbracket_{\tau}
            $$
            and then we can apply our inductive hypothesis on $\tau$ to prove
            properties (1) and (2).
          \item{\underline{(ty. case) $M_q \tau$.}}
            By our compatibility relation, we know that: $\sigma = (x \mapsto
            v_0);(x \mapsto v_1)$ for $d_{\tau}(\llbracket \Vdash v_0
            \rrbracket_{\tau}, \llbracket \Vdash v_1 \rrbracket_{\tau}) \leq q$.
            Unfolding our machine configuration interpretation definition, we
            obtain:
            \begin{equation*}
              \begin{aligned}
                \llbracket x \mapsto v_0; x \mapsto v_1 \Vdash x \rrbracket_{M_q
                \tau} &= (\llbracket x \mapsto v_0 \Vdash x \rrbracket_{\tau};
                \llbracket x \mapsto v_1 \Vdash x \rrbracket_{\tau}) \\
                &= (\llbracket \Vdash v_0 \rrbracket_{\tau}, \llbracket \Vdash
                v_1 \rrbracket_{\tau}) \in \llbracket M_q \tau \rrbracket
              \end{aligned}
            \end{equation*}
            which proves property 1. Since distance on the neighborhood monad is
            measured in terms of the first (ideal) component, property 2 also
            holds.
        \end{description}
      \item[\textit{(env. size) n + 1.}] 
    \end{description}
  \item[\textsc{(ty. rule) Ret.}] 
    From our inductive hypothesis, we have that $\Gamma \vDash e : \tau$ and
    wish to prove that $\Gamma \vDash \mathbf{ret} \ e : \tau$. It suffices to
    show that $(\sigma; \sigma)[\alpha \mapsto v:_1 \tau] \Vdash \alpha$.
  \item[\textsc{(ty. rule) Rnd.}]
\end{description}
\end{proof}

\begin{theorem}[Semantic type soundness]
$\Gamma \vdash e : \tau \implies \Gamma \vDash e : \tau$
\end{theorem}
\begin{proof}
  If $e$ is a value, we can apply our semantic type soundness theorem for values
  and we are done. If $e$ is a well-typed expression, we wish to show that (1)
  it will also step to a semantically well-typed term; and (2) it will
  eventually step to a value.
  TODO
\end{proof}

%%%% IGNORE BELOW %%%%

% % todo: Use whatever theorem name people like best here.
% Semantics is preserved under operational stepping. If $\sigma \Vdash e : \tau \rightsquigarrow \sigma'
% \Vdash e' : \tau$, then $\llbracket \sigma \Vdash e : \tau \rrbracket =
% \llbracket \sigma' \Vdash e' : \tau \rrbracket$.
%
% % todo: Use whatever theorem name people like best here.
% If the semantics of a program is equivalent to the semantics of a value, it must
% reduce to that value. If $\llbracket \sigma \Vdash e : \tau \rrbracket =
% \llbracket v : \tau \rrbracket$, then $\sigma \Vdash e : \tau
% \rightsquigarrow^{*} \sigma' \Vdash v : \tau$
%
% Syntactically well-typed programs are non-expansive. For $\Gamma \vdash e : \tau
% $ and $\llbracket \sigma \rrbracket, \llbracket \sigma' \rrbracket \in
% \llbracket \Gamma \rrbracket $ and
% $$\sigma \Vdash e : \tau \rightsquigarrow^* v : \tau$$ 
% and 
% $$\sigma' \Vdash e : \tau \rightsquigarrow^* v' : \tau$$
% then
% $$
% d_{\tau}(v, v') \leq d_{\Gamma}(\sigma, \sigma')
% $$
%
% Syntactically ok implies semantically ok. If $\Gamma \vdash e : \tau$, then
% $\forall \llbracket \sigma \rrbracket \in \llbracket \Gamma \rrbracket, \exists
% v, \llbracket \sigma \Vdash e : \tau \rrbracket = \llbracket v : \tau
% \rrbracket$.
%
% Syntactically ok implies operationally ok. If $\Gamma \vdash e : \tau$, then
% $\forall \llbracket \sigma \rrbracket \in \llbracket \Gamma \rrbracket, \exists
% v, \sigma \Vdash e : \tau \rightsquigarrow^{*} v : \tau$.

