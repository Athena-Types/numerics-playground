\section{Type Soundness}
\begin{lemma}[Termination]\label{thm:termination}
  If $\Gamma \vDash e : \tau$, then $\forall \sigma$ compatible with $\Gamma$, 
  $$\sigma \Vdash e \rightsquigarrow^* \sigma' \Vdash v$$
\end{lemma}
\begin{proof}
  If $\sigma \Vdash e$ is a value, then we are done. Otherwise, if $\sigma
  \Vdash e$ is an expression, we begin by unfolding and applying the definition
  of $\vDash$. By definition, know that $\llbracket \sigma \Vdash e \rrbracket_{\tau} \in
  \llbracket \tau \rrbracket$. We also know that $\bot \not\in \llbracket \tau
  \rrbracket$, a metric space, so $\llbracket \sigma \Vdash e \rrbracket_{\tau}
  \not= \bot$ and is well-defined.

  By inspection of the definition of cases of $\llbracket - \rrbracket_{\tau}$,
  we know that only last case (which deals with all expressions) must apply.
  Therefore, $\sigma \Vdash e \rightsquigarrow^* \sigma' \Vdash v$.
\end{proof}

% \begin{theorem}[Semantic type preservation]
% If a term $e$ is semantically well-typed in context $\Gamma$ with type $\tau$
%   then for all $\sigma \Vdash e \rightsquigarrow \sigma' \Vdash e'$, $\sigma$
%   compatible with $\Gamma$, then there exists a $\Gamma'$ such that $\Gamma'
%   \vDash e' : \tau$.
% \end{theorem}
% \begin{proof}
% If $e$ is a value, we're done. 
% If $e$ is not a value, we proceed by induction over the cases of the rewrite relation
% $\rightsquigarrow$.
% % Our inductive hypothesis is that $e \rightsquigarrow e'$ and $\Gamma \vDash e : \tau$, that is, 
% % $\forall \sigma \text{ compatible with } \Gamma, \llbracket \sigma \Vdash e
% % \rrbracket_{\tau} \in \llbracket \tau \rrbracket$.
% \begin{description}
%   \item[\textsc{Variable lookup.}] If $$
%   \item[\textsc{Ret.}]
%   \item[\textsc{Rnd.}]
% \end{description}
% \end{proof}

% \begin{definition}[Monadic type]\label{def:monadic}
%   A type $\tau$ is monadic if and only if $\exists \tau', q \text{ such that }
%   \tau = M_q \tau'$. It is non-monadic otherwise.
% \end{definition}
% %%% wrong theorem:
% \begin{lemma}[Non-monadic lookup]\label{thm:non-monadic-lookup}
%   For all $\tau$ non-monadic and all $\Gamma, x : \tau'$ compatible with
%   $\sigma$, $\exists \sigma', \sigma = \sigma'[x \mapsto v : \tau']$ is
%   well-defined. Stated equivalently, non-monadic $x$ uniformly maps to $v$ at
%   all enviroments in $\sigma$.
% \end{lemma}
% \begin{proof}
%   TODO
% \end{proof}

\begin{lemma}[Enviroment tree shape]
  The tree height for a 
\end{lemma}

\begin{definition}[Orderings]\label{def:orderings}
  We define the following well-founded partial order over configurations and
  types, which mirrors the definition of machine configuration interpretation
  and will be used for induction in the proceeding proofs.

  First, we define an ordering over types by size of the AST.
  \begin{equation}
    \begin{aligned}[c]
      s(\textbf{unit}) &= 1 \\
      s(\textbf{num}) &= 1 \\
      s(!_s \tau) &= s(\tau) + 1 \\
      s(M_q \tau) &= s(\tau) + 1 \\
      s(\tau_0 \times \tau_1) &= s(\tau_0) + s(\tau_1) + 1 \\
      s(\tau_0 \otimes \tau_1) &= s(\tau_0) + s(\tau_1) + 1 \\
      s(\tau_0 \multimap \tau_1) &= s(\tau_0) + s(\tau_1) + 1 \\
    \end{aligned}
  \end{equation}
  and
  \begin{equation}
    \tau_0 < \tau_0 \iff s(\tau_0) < s(\tau_1)
  \end{equation}

  Secondly, we define an ordering over environment leafs $\gamma$ using the number of bound
  variables:
  \begin{equation}
    \begin{aligned}[c]
      s(.) &= 0 \\
      s(\gamma, x \mapsto v :_s \tau) &= s(\gamma) + 1 \\
    \end{aligned}
  \end{equation}

  and

  \begin{equation}
    \gamma_0 < \gamma_0 \iff s(\gamma_0) < s(\gamma_1)
  \end{equation}

  We can now define an ordering over enviorment trees using the maximum size of
  each enviorment leaf in the tree:
  \begin{equation}
    \begin{aligned}[c]
      s(\gamma; \gamma') &= max(s(\gamma), s(\gamma')) \\
      s(\sigma; \sigma') &= max(s(\sigma), s(\sigma'))
    \end{aligned}
  \end{equation}

  We are now ready to define an ordering over $\textit{config} \times
  \textit{type}$:
  \footnote{This is our termination measure for our machine configuration
  interpretation function.}
  \footnote{Note that the type of the configuration (if the interpretation of
  the configuration belongs to the interpretation of the type) restricts the
  height and shape of environment tree. So we do not need to reason about the
  height of our enviroment trees in our ordering.}

  \begin{equation}
    \sigma \Vdash e : \tau < \sigma' \Vdash e' : \tau' \iff
    s(\tau) < s(\tau') \lor (s(\tau) = s(\tau') \land s(\sigma) < s(\sigma'))
  \end{equation}

\end{definition}

\begin{lemma}[Well-founded partial ordering]
  $\leq$ is a well-founded partial order over $\textit{config} \times
  \textit{type}$.
\end{lemma}
\begin{proof}
  Trivial.
\end{proof}

%%%%%%%%%%%%%%%%%%%%%%%%%%%%%%%%%%%%%%%%%%%%%%%%%%%%%%%%%%%%%%%%%%

The following property is used to prove that our denotation of configurations
(for function types) define \textit{total} functions. This is necessary because
our denotation for configurations is defined extensionally (by stepping
configs).

\begin{lemma}[Semantic totality]
  For all types $\tau$ and $x \in \llbracket \tau \rrbracket$, there exists
  $\sigma, e$ such that $\llbracket \sigma \Vdash e \rrbracket_{\tau} = x$.
\end{lemma}
\begin{proof}
  By induction over $\tau$. All cases except the $\tau_0 \multimap \tau_1$ cases
  hold trivially or by application of the inductive hypothesis. For the
  implication case, we observe that we can always pick an $\textbf{op}$
  witnessing the totality. Qed.
\end{proof}

\begin{theorem}[Semantic type soundness]
$\Gamma \vdash e : \tau \implies \Gamma \vDash e : \tau$
\end{theorem}
\begin{proof}
From our premise, we wish to show that:
  \begin{enumerate}
    \item $\forall \sigma$ compatible with $\Gamma$, $\llbracket \sigma \Vdash e
      \rrbracket_{\tau} \in \llbracket \tau \rrbracket$ and well-defined.
    \item $\llbracket - \Vdash e \rrbracket_{\tau}$ is a \text{1-sensitive} map
      for all inputs enviroments compatible with $\Gamma$.
  \end{enumerate}
We begin by induction over our typing derivation. 
  % In some of the cases, we will need to induct over the size of our enviroment. 
\begin{description}
  \item[\textsc{(ty. rule) Var.}] There is no premise containing a typing
    judgement to this rule. So we have no inductive hypothesis to rely on.
    (This is a base case.)
    % We proceed to induct over the type of our variable, $\tau$.
    % \begin{description}
    %   \item{\textbf{unit} and \textbf{num}.}
    %   \item{$\tau_0 \otimes \tau_1$, $\tau_0 + \tau_1$, and $\tau_0 \multimap
    %     \tau_1$.}
    %   \item{$!_s \tau$.}
    %   \item{$M_q \tau$}
    %   \item{$\tau_0 \times \tau_1$.}
    % \end{description}
    Instead, we induct over the size of our enviorment.
    \begin{description}
      \item[\textit{(env. size) 0.}] We observe that an enviorment size of zero
        implies that $\Gamma$ is empty. So we are trying to show that the
        following:
        $$x : \tau \vDash x : \tau$$
        implies:
        $$\llbracket \sigma \Vdash x \rrbracket_\tau \in \llbracket \tau
        \rrbracket$$
        for $\sigma$ compatible with typing context $x : \tau$ and is
        \text{1-sensitive}. We proceed to induct over the size of type $\tau$,
        namely $s(\tau)$. 
        \footnote{Note that when we perform inversion over our compatibility
        relation for several of the cases below, the case that $\sigma = x
        \mapsto y$, for $y$ a var, is impossible when the enviroment size is
        zero because $\llbracket \Vdash y \rrbracket = \bot$. This is why we are
        inducting over the enviroment size.}
        \begin{description}
          \item{\underline{(ty. case) \textbf{unit.}}} The only configuration compatible with this is
            $x \mapsto \langle \rangle \Vdash x$ and clearly 
            $$\llbracket x \mapsto \langle \rangle \Vdash x
            \rrbracket_{\textbf{unit}} = \llbracket \Vdash \langle \rangle
            \rrbracket_{\textbf{unit}} = * \in \{ * \} = \llbracket \textbf{unit}
            \rrbracket$$
            So, properties (1) and (2) hold by the reasoning above.
          \item{\underline{(ty. case) $\mathbf{num}$.}} 
            The only configurations compatible
            with this is $x \mapsto k \Vdash x$ for $k \in \mathbb{R}$ and
            clearly $$\llbracket x \mapsto k \Vdash x \rrbracket_{\textbf{num}}
            = \llbracket \Vdash k \rrbracket_{\textbf{num}} = k \in \mathbb{R} =
            \llbracket \textbf{num} \rrbracket$$ implies that $\llbracket -
            \Vdash x \rrbracket_{\mathbf{num}}$ for this case is the
            $\text{1-sensitive}$ identity function over the reals. Therefore,
            properties (1) and (2) both hold.
          \item{\underline{(ty. case) $\times$.}} By inversion on
            our compatibility relation, we know that we must have some $\sigma =
            (x \mapsto v_0; x \mapsto v_1)$ for $\llbracket \Vdash v_0
            \rrbracket_{\tau_0} \in \llbracket \tau_0 \rrbracket$ and
            $\llbracket \Vdash v_1 \rrbracket_{\tau_1} \in \llbracket \tau_1
            \rrbracket$. By definition, $\llbracket x \mapsto v_0; x \mapsto v_1
            \Vdash x \rrbracket_{\tau_0 \times \tau_1} = (\llbracket x \mapsto
            v_0 \Vdash x \rrbracket_{\tau_0}, \llbracket x \mapsto v_1 \Vdash x
            \rrbracket_{\tau_1}) = (\llbracket \Vdash v_0 \rrbracket_{\tau_0},
            \llbracket \Vdash v_1 \rrbracket_{\tau_1})$ which is clearly
            \text{1-sensitive} in the interpretation of our type. So properties
            (1) and (2) hold.
          \item{\underline{(ty. case) $\otimes$.}} By inversion on our
            compatibility relation, we know that we must have some $\sigma = (x
            \mapsto (v_0, v_1))$ for $\llbracket \Vdash v_0 \rrbracket_{\tau_0}
            \in \llbracket \tau_0 \rrbracket$ and $\llbracket \Vdash v_1
            \rrbracket_{\tau_1} \in \llbracket \tau_1 \rrbracket$. By
            definition, $\llbracket x \mapsto (v_0, v_1) \Vdash x
            \rrbracket_{\tau_0 \times \tau_1} = (\llbracket x \mapsto v_0 \Vdash
            x \rrbracket_{\tau_0}, \llbracket x \mapsto v_1 \Vdash x
            \rrbracket_{\tau_1}) = (\llbracket \Vdash v_0 \rrbracket_{\tau_0},
            \llbracket \Vdash v_1 \rrbracket_{\tau_1})$ which is clearly
            \text{1-sensitive} in the interpretation of our type. So properties
            (1) and (2) hold.
          \item{\underline{(ty. case) $+$.}} By inversion on our compatibility
            relation, we know that we have $\sigma = x \mapsto v$ for
            $\llbracket \Vdash v \rrbracket_{\tau_0 + \tau_1} \in \llbracket
            \tau_0 + \tau_1 \rrbracket = \llbracket \tau_0 \rrbracket \uplus
            \llbracket \tau_1 \rrbracket$. 
            There are two cases: 
            (1) $\llbracket \Vdash v \rrbracket_{\tau_0} \in \llbracket \tau_0
            \rrbracket$ and therefore $\llbracket x \mapsto v \Vdash x
            \rrbracket_{\tau_0 + \tau_1} = (0, \llbracket x \mapsto v \Vdash x
            \rrbracket_{\tau_0}) = (0, \llbracket \Vdash v
            \rrbracket_{\tau_0})$; 
            or, (2) $\llbracket \Vdash v \rrbracket_{\tau_1} \in \llbracket
            \tau_1 \rrbracket$ and therefore $\llbracket x \mapsto v \Vdash x
            \rrbracket_{\tau_0 + \tau_1} = (1, \llbracket x \mapsto v \Vdash x
            \rrbracket_{\tau_1}) = (1, \llbracket \Vdash v
            \rrbracket_{\tau_1})$. In both cases property 1 clearly holds.
            Property 2 holds by a similar case analysis.
          \item{\underline{(ty. case) $\tau_0 \multimap \tau_1$.}} 
            This case holds by the conditions on our compatibility relation and
            unfolding the interpretation of a machine configuration.
          \item{\underline{(ty. case) $!_s \tau$.}} 
            We begin by unfolding the definition of our compatibility relation
            and applying our machine configuration interpretation definition:
            $$ 
            \llbracket x \mapsto v :_s \tau \Vdash x \rrbracket_{!_s \tau} = 
            \llbracket x \mapsto v : \tau \Vdash x \rrbracket_{\tau}
            $$
            and then we can apply our inductive hypothesis on $\tau$ to prove
            properties (1) and (2).
          \item{\underline{(ty. case) $M_q \tau$.}}
            By our compatibility relation, we know that: $\sigma = (x \mapsto
            v_0);(x \mapsto v_1)$ for $d_{\tau}(\llbracket \Vdash v_0
            \rrbracket_{\tau}, \llbracket \Vdash v_1 \rrbracket_{\tau}) \leq q$.
            Unfolding our machine configuration interpretation definition, we
            obtain:
            \begin{equation*}
              \begin{aligned}
                \llbracket x \mapsto v_0; x \mapsto v_1 \Vdash x \rrbracket_{M_q
                \tau} &= (\llbracket x \mapsto v_0 \Vdash x \rrbracket_{\tau};
                \llbracket x \mapsto v_1 \Vdash x \rrbracket_{\tau}) \\
                &= (\llbracket \Vdash v_0 \rrbracket_{\tau}, \llbracket \Vdash
                v_1 \rrbracket_{\tau}) \in \llbracket M_q \tau \rrbracket
              \end{aligned}
            \end{equation*}
            which proves property 1. Since distance on the neighborhood monad is
            measured in terms of the first (ideal) component, property 2 also
            holds.
        \end{description}
      \item[\textit{(env. size) n + 1.}] We again induct over our type. 
        Note that in contrast to the base case, when we perform inversion on the
        compatibility relation for the cases below, $\sigma = \sigma'[x \mapsto
        y]$, for $y$ a var, is a possible case. 
        So when we induct over our type, for each type, we have that either:
        \begin{description}
          \item{\underline{\textit{$\sigma = \sigma'[x \mapsto y]$ for $y$ a
            variable.}}} By unfolding the definition of our machine config
            interpretation and inversion over our compatibility relation, we
            know that $\llbracket \sigma \Vdash x \rrbracket_{\tau} = \llbracket
            \sigma' \Vdash y \rrbracket_{\tau} \in \llbracket \tau \rrbracket$
            and is \text{1-sensitive} by our inductive hypothesis (over env.
            size).
          \item{\underline{\textit{Otherwise.}}} Proof for this case mirrors the
            proof for each of the base cases.
        \end{description}
    \end{description}
  \item[\textsc{(ty. rule) Ret.}] 
    From our inductive hypothesis, we have that $\Gamma \vDash e : \tau$ and
    wish to prove that $\Gamma \vDash \mathbf{ret} \ e : M_0 \tau$. Stepping our
    rewrite relation, we have that 
    \begin{equation}
      \begin{aligned}
        \llbracket (\sigma; \sigma)[\alpha \mapsto v:_1 \tau] \Vdash \alpha
        \rrbracket_{M_0 \tau} 
        &= \\
        (\llbracket \sigma[\alpha \mapsto v:_1 \tau] \Vdash \alpha
        \rrbracket_{\tau}, \llbracket \sigma[\alpha
        \mapsto v:_1 \tau] \Vdash \rrbracket_{\tau})
        &\in 
        \llbracket M_0 \tau \rrbracket
      \end{aligned}
    \end{equation}
    by our inductive hypothesis and is clearly \text{1-sensitive} in the metric
    space denoted by its type.
  \item[\textsc{(ty. rule) Rnd.}]
    From our inductive hypothesis, we have that $\Gamma \vDash e : \mathbf{num}$ and
    wish to prove that $\Gamma \vDash \mathbf{rnd} \ e : M_q \mathbf{num}$. Stepping our
    rewrite relation, we can show that:
    \begin{equation}
      \begin{aligned}
        \llbracket (\sigma[\alpha \mapsto v:_1 \mathbf{num}]; \sigma[\alpha \mapsto
        \rho(v) :_1 \mathbf{num}]) \Vdash \alpha \rrbracket_{M_q \mathbf{num}} 
        &= \\
        (\llbracket \sigma[\alpha \mapsto v:_1 \mathbf{num}] \Vdash \alpha
        \rrbracket_{\mathbf{num}}, \llbracket \sigma[\alpha \mapsto \rho(v) :_1
        \mathbf{num}]) \Vdash \alpha \rrbracket_{\mathbf{num}})
        &\in \llbracket M_q \mathbf{num} \rrbracket
      \end{aligned}
    \end{equation}
    by relying on the fact that $d_{\mathbf{num}}(v, \rho(v)) \leq q$. It is
    also clearly \text{1-sensitive} in the metric space denoted by its type as
    distances between monadic values are measured as distances between their
    first (ideal) components.
  \item[\textsc{(ty. rule) $\multimap I$.}] We are given in our inductive
    hypothesis that $\Gamma, x :_1 \vDash e : \tau$ for all types $\tau$. That
    means for all enviroments $\sigma'$ compatible with $\Gamma, x :_1 \tau_0$,
    we have that: 
    $$\llbracket \sigma' \Vdash e \rrbracket_{\tau} \in \tau$$ and
    $$\llbracket _ \Vdash e \rrbracket_{\tau} \in \tau$$
    \text{1-sensitive}.
    From our inductive hypothesis, we wish to show that: $\Gamma \Vdash \lambda
    x . e : \tau_0 \multimap \tau$ and \text{1-sensitive}. 
    In the following proof, let $\sigma$ be an arbitrary enviroment compatible
    with $\Gamma$. Unfolding what we wish to show, it suffices to show that for
    all $\sigma$ compatible with $\Gamma$, $\llbracket \sigma \Vdash \lambda x .
    e \rrbracket_{\tau_0 \multimap \tau_1} \in \llbracket \tau_0 \multimap
    \tau_1 \rrbracket$. By semantic totality, it suffices to demonstrate that
    for any $\llbracket \sigma' \Vdash v \rrbracket_{\tau_0} \in \llbracket
    \tau_0 \rrbracket$, $\llbracket \sigma \Vdash \lambda x . e
    \rrbracket_{\tau_0 \multimap \tau_1} \llbracket \sigma' \Vdash v
    \rrbracket_{\tau_0} \in \llbracket \tau \rrbracket$ and \text{1-sensitive}.
    We proceed by inducting over the size of type $\tau_0$, namely $s(\tau_0)$.
    \begin{description}
      \item{\underline{(ty. case) $\textbf{unit}$.}} TODO.
      \item{\underline{(ty. case) $\textbf{num}$.}} TODO.
      \item{\underline{(ty. case) $\tau_0 \times \tau_1$.}} TODO. 
      \item{\underline{(ty. case) $\tau_0 \otimes \tau_1$.}} TODO.
      \item{\underline{(ty. case) $\tau_0 + \tau_1$.}} TODO.
      \item{\underline{(ty. case) $\tau_0 \multimap \tau_1$.}} TODO. 
      \item{\underline{(ty. case) $!_s \tau$.}} TODO.
      \item{\underline{(ty. case) $M_u \tau$.}} TODO.
    \end{description}
    TODO.
  \item[\textsc{(ty. rule) $\multimap E$.}] TODO.
  \item[\textsc{(ty. rule) $! I$.}] TODO.
  \item[\textsc{(ty. rule) $! E$.}] TODO.
  \item[\textsc{(ty. rule) Let.}] TODO.
  \item[\textsc{(ty. rule) $M_u \ E$.}] TODO.
  \item[\textsc{(ty. rule) Factor.}] TODO.
  \item[\textsc{(ty. rule) $\times I$.}] TODO.
  \item[\textsc{(ty. rule) $\times E$.}] TODO.
  \item[\textsc{(ty. rule) $\otimes I$.}] TODO.
  \item[\textsc{(ty. rule) $\otimes E$.}] TODO.
  \item[\textsc{(ty. rule) $+ I_i$.}] TODO.
  \item[\textsc{(ty. rule) $+ E$.}] TODO.
  \item[\textsc{(ty. rule) $\textit{op}(v)$.}] TODO.
  \item[\textsc{(ty. rule) Unit.}] Holds trivially.
  \item[\textsc{(ty. rule) Const.}] Holds trivially.
  \item[\textsc{(ty. rule) Subsumption.}] Holds trivially.
\end{description}
\end{proof}

\subsection{Modeling fidelity of real-world numerical programs}
Our operational and denotational semantics model both our ideal and approximate
numerical computations simultaneously. This enables an explicit encoding of the
computational content of $\textbf{factor}$ (which can be viewed as rearranging
various mixed ideal/approximate intermediate computations in a type-sound
fashion) and multiplication in our monad (which can be viewed as "forgetting"
mixed intermediate computations through the type-sound application of triangle
inequality).

However, a semantics that unifies both ideal and approximate computations makes
it less obvious that our type soundness theorem is actually useful in bounding
round-off error: the difference between what the user wanted (a program over the
reals with no round-off error) and what the user got (a program over the
floats).

In this section, we allay such concerns by annotating the operational semantics,
now colored with a \textcolor{red}{red} rewrite relation
\textcolor{red}{$\rightsquigarrow$}. The annotated operational semantics allow
us to cleanly track and separate the ideal and approximate computations during
stepping:
\begin{enumerate}
  \item For portions of the configuration that correspond to our \textit{ideal}
    computation, without round-off error, we \underline{underline} the
    computation. For example, our rounding stepping rule would look like (with
    only our ideal annotations):
    \begin{equation*}
      \sigma \Vdash \underline{\mathbf{rnd} \ v} : M_q \ \mathbf{num} \ \textcolor{red}{\rightsquigarrow} \
      \underline{\sigma [\alpha \mapsto v :_1 \mathbf{num}]}; \sigma [\alpha \mapsto \rho(v) :_1
      \mathbf{num}] \Vdash \underline{\alpha} : M_q \ \mathbf{num}
    \end{equation*}
  \item For portions of the configuration that correspond to our
    \textit{approximate} computation, with round-off error, we
    $\overline{\text{overline}}$ the computation. For example, our rounding
    stepping rule would look like (with only our approximate annotations):
    \begin{equation*}
      \sigma \Vdash \overline{\mathbf{rnd} \ v} : M_q \ \mathbf{num} \ \textcolor{red}{\rightsquigarrow} \ \sigma
      [\alpha \mapsto v :_1 \mathbf{num}]; \overline{\sigma [\alpha \mapsto \rho(v) :_1
      \mathbf{num}]} \Vdash \overline{\alpha} : M_q \ \mathbf{num}
    \end{equation*}
\end{enumerate}
For the fully annotated rounding stepping rule, see
Equation~\ref{eq:rnd-annotated}.
A user may choose to selectively consider the ideal semantics by \textit{only}
caring about the underlined portions of the operational semantics rewrite rules,
or, a user may selectively implement the approximate floating-point semantics by
\textit{only} computing the overlined portions. Our unified operational
semantics allows for both perspectives to co-exist.
Importantly, each perspective is self-contained: to consider the ideal
semantics, the user can completely ignore the approximate semantics (rounding)
and vice-versa.
With annotations, we prove our error soundness theorem
(Theorem~\ref{thm:error-soundness}), which states that all semantically
well-typed $e : M_q \ \mathbf{num}$ reduce to a pair of values, with the first
component side fully ideal (underlined; what the user wanted) and the second component fully
approximate (overlined; what the user got and can compute in-hardware), with the distance
between computations --- our round-off error --- bounded by $q$.

\subsubsection{Annotated operational semantics}

Note that different nodes in the enviroment tree may have mixed annotations,
corresponding to when the ideal and approximate computations are mixed. For
example, the fully annotated rounding stepping rule looks like:
\begin{equation}\label{eq:rnd-annotated}
  \sigma \Vdash \underline{\overline{\mathbf{rnd} \ v}} : M_q \ \mathbf{num} \ \textcolor{red}{\rightsquigarrow} \
  \underline{\sigma [\alpha \mapsto v :_1 \mathbf{num}]}; \overline{\sigma [\alpha \mapsto \rho(v) :_1
  \mathbf{num}]} \Vdash \underline{\overline{\alpha}} : M_q \ \mathbf{num}
\end{equation}
where $\sigma$ itself contain under and overlined annotations that are
propagated through. In some cases, expressions will be simultaneously
$\underline{\overline{\text{under-over-lined}}}$. 

\subsubsection{Annotated operational semantics}

\subsubsection{Relations over configurations and environment trees}
% TODO: fix off-by-one issue with grammars
To help prove error soundness, we define a unary relation over well-annotated
configurations and prove that it is preserved under stepping. The following
relation enforces two properties that are useful to maintain under stepping:
\begin{definition}[Well-annotated configuration]
  A configuration is well-annotated if and only if the following two properties
  hold:
  \begin{enumerate}
    \item On the left-hand side of $\textcolor{red}{\rightsquigarrow}$, annotated
      enviroment trees belong to the following grammar:
      \begin{alignat*}{3}
            &\sigma, \sigma_0, \sigma_1 &::=~ & \gamma
            \ \mid \ \underline{\sigma_0}; \overline{\sigma_1}
            & \ \mid \ \overline{\underline{\sigma_0}}; \overline{\underline{\sigma_1}}
      \end{alignat*}
    \item On the right-hand side of $\textcolor{red}{\rightsquigarrow}$, 
      expressions are always accessible from both the ideal and approximate
      perspectives. That is, $\sigma \Vdash \underline{\overline{e}}$.
  \end{enumerate}
\end{definition}

\begin{definition}[Fully ideal configuration]
  A configuration is fully ideal if and only if it has the form
  $\sigma_0 \Vdash \underline{e}$ where $\sigma$ belongs to the following grammar:
  \footnote{Note that e may also be overlined.}
      \begin{alignat*}{3}
        &\sigma_0, \sigma_1 &::=~ & \underline{\gamma}
            \ \mid \ \underline{\overline{\gamma}} \ \mid \ \sigma_0; \sigma_1
      \end{alignat*}

\end{definition}

\begin{definition}[Fully approximate configuration]
  Similarly, a configuration is fully approximate if and only if it has the form
  $\sigma_0 \Vdash \overline{e}$ where $\sigma$ belongs to the following
  grammar:
  \footnote{Note that e may also be underlined.}
      \begin{alignat*}{3}
        &\sigma_0, \sigma_1 &::=~ & \overline{\gamma}
            \ \mid \ \underline{\overline{\gamma}} \ \mid \ \sigma_0; \sigma_1
      \end{alignat*}
\end{definition}

\subsubsection{Error soundness}
\begin{lemma}[Well-annotation relation preservation]\label{thm:expressions-preserve-annotations} 
  For all $\sigma, \sigma', e, e'$, if $\sigma \Vdash e \rightsquigarrow \sigma'
  \Vdash e'$, then if $\sigma \Vdash \underline{\overline{e}}$ is well-annotated
  and $\sigma \Vdash \underline{\overline{e}} \
  \textcolor{red}{\rightsquigarrow} \ \sigma' \Vdash \underline{\overline{e'}}$,
  then $\sigma' \Vdash \underline{\overline{e'}}$ is well-annotated.
\end{lemma}
\begin{proof}
  By inspection of the annotated rewrite ($\textcolor{red}{\rightsquigarrow}$) relation.
\end{proof}

With annotations, we can now state the following theorem:
\begin{theorem}[Error soundness]\label{thm:error-soundness}
  For all semantically well-typed expressions $e : M_q \ \mathbf{num}$ in the
  empty context, $. \Vdash \underline{\overline{e}} \
  \textcolor{red}{\rightsquigarrow^*} \ \underline{\gamma}; \overline{\gamma'}
  \Vdash \underline{\overline{v}}$ where:
  $$
  \llbracket . \Vdash e \rrbracket_{M_q \ \mathbf{num}} 
  = \llbracket \underline{\gamma}; \overline{\gamma'} \Vdash
  \underline{\overline{v}} \rrbracket_{M_q \ \mathbf{num}} 
  = (\llbracket \underline{\gamma \Vdash v} \rrbracket_{\mathbf{num}},
  \llbracket \overline{\gamma' \Vdash v} \rrbracket_{\mathbf{num}}) 
  = (k_0, k_1)
  $$
  for $d_{\mathbf{num}}(k_0, k_1) \leq q$.
\end{theorem}
\begin{proof}
  Since $. \vDash e : M_q \ \mathbf{num}$, we know that $\llbracket . \Vdash e
  \rrbracket_{M_q \ \mathbf{num}} \in \llbracket M_q \ \mathbf{num}
  \rrbracket$. 
  By Lemma~\ref{thm:termination} (termination) and preservation of
  well-annotated configurations, we have that 
  $. \Vdash \underline{\overline{e}} \rightsquigarrow^{*} \sigma \Vdash
  \underline{\overline{v}}$ for some $\sigma$.
  It remains to be shown that $\sigma = \gamma; \gamma'$ where $\llbracket \underline{\gamma \Vdash v}
  \rrbracket_{\mathbf{num}} = k_0$ and $\llbracket \overline{\gamma' \Vdash v}
  \rrbracket_{\mathbf{num}} = k_1$ and $d_{\mathbf{num}}(k_0, k_1) \leq q$

  \begin{equation}
    \begin{aligned}
      \llbracket . \Vdash \underline{\overline{e}} \rrbracket_{M_q \ \mathbf{num}} 
        &=
      \llbracket \underline{\gamma}; \overline{\gamma'} \Vdash \underline{\overline{v}} \rrbracket_{M_q \ \mathbf{num}} \\
      &=
      (\llbracket \underline{\gamma} \Vdash \underline{\overline{v}} \rrbracket_{\mathbf{num}}, \llbracket \overline{\gamma'} \Vdash \underline{\overline{v}} \rrbracket_{\mathbf{num}}) \\
    \end{aligned}
  \end{equation}
  TODO.
\end{proof}

%%%% IGNORE BELOW %%%%

% % todo: Use whatever theorem name people like best here.
% Semantics is preserved under operational stepping. If $\sigma \Vdash e : \tau \rightsquigarrow \sigma'
% \Vdash e' : \tau$, then $\llbracket \sigma \Vdash e : \tau \rrbracket =
% \llbracket \sigma' \Vdash e' : \tau \rrbracket$.
%
% % todo: Use whatever theorem name people like best here.
% If the semantics of a program is equivalent to the semantics of a value, it must
% reduce to that value. If $\llbracket \sigma \Vdash e : \tau \rrbracket =
% \llbracket v : \tau \rrbracket$, then $\sigma \Vdash e : \tau
% \rightsquigarrow^{*} \sigma' \Vdash v : \tau$
%
% Syntactically well-typed programs are non-expansive. For $\Gamma \vdash e : \tau
% $ and $\llbracket \sigma \rrbracket, \llbracket \sigma' \rrbracket \in
% \llbracket \Gamma \rrbracket $ and
% $$\sigma \Vdash e : \tau \rightsquigarrow^* v : \tau$$ 
% and 
% $$\sigma' \Vdash e : \tau \rightsquigarrow^* v' : \tau$$
% then
% $$
% d_{\tau}(v, v') \leq d_{\Gamma}(\sigma, \sigma')
% $$
%
% Syntactically ok implies semantically ok. If $\Gamma \vdash e : \tau$, then
% $\forall \llbracket \sigma \rrbracket \in \llbracket \Gamma \rrbracket, \exists
% v, \llbracket \sigma \Vdash e : \tau \rrbracket = \llbracket v : \tau
% \rrbracket$.
%
% Syntactically ok implies operationally ok. If $\Gamma \vdash e : \tau$, then
% $\forall \llbracket \sigma \rrbracket \in \llbracket \Gamma \rrbracket, \exists
% v, \sigma \Vdash e : \tau \rightsquigarrow^{*} v : \tau$.

