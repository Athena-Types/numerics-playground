\begin{theorem}[Paired error theorem; a posteriori relative] \label{thm:paired-a-posteriori-rel}
  For any numbers $((r, a, b), (r', a', b')) \in \mathbb{P}^2$ and where we have
  $a^{\downarrow}, a^{\uparrow}, b^{\downarrow}, b^{\uparrow} \in \mathbb{R}^{+}$:
  the following bounds for $r^{\downarrow}, r^{\uparrow} \in \mathbb{R}$, 
  \begin{enumerate}
    \item $d_{\mathbb{R}}(a, a'), d_{\mathbb{R}}(b, b') \leq q$
    \item $r \in [r^{\downarrow}, r^{\uparrow}]$
    \item $a \in [a^{\downarrow}, a^{\uparrow}]$
    \item $b \in [b^{\downarrow}, b^{\uparrow}]$
  \end{enumerate}
  and we know $r'$, we have that $d_{\mathbb{R}}(r, r')$ is less than or equal
  to the following bounds if $0 < r'$: 
  \begin{enumerate}
    \item[(a)] $max(|ln(e^{-q} + \frac{a^{\uparrow}}{r'}(1-e^{-2q}))|, |ln(e^{-q} + \frac{a^{\downarrow}}{r'}(1-e^{2q}))|)$
    \item[(b)] $max(|ln(e^{q} + \frac{b^{\uparrow}}{r'}(e^{2q}- 1))|,|ln(e^q + \frac{b^{\downarrow}}{r'}(e^{-2q}-1))|)$
  \end{enumerate}
  and to the following bounds if $r' < 0$:
  \begin{enumerate}
    \item[(c)] $max(|ln(e^{-q} + \frac{a^{\downarrow}}{r'}(1-e^{-2q}))|, |ln(e^{-q} + \frac{a^{\uparrow}}{r'}(1-e^{2q}))|)$
    \item[(d)] $max(|ln(e^{q} + \frac{b^{\downarrow}}{r'}(e^{2q}- 1))|,|ln(e^q + \frac{b^{\uparrow}}{r'}(e^{-2q}-1))|)$
  \end{enumerate}
\end{theorem} 
\begin{proof}
  We first examine the scenario where $0 < r'$.
  Similarly to the previous theorem, there are two subcases, $r' \leq r$ or $r' >
  r$. In the first case, we wish to maximize the numerator and minimize the
  denominator. So, 
  \begin{equation}
  \begin{aligned}[c]
    1 \leq
    \frac{r}{r'} \leq
    \frac{a - b'e^{-q}}{a' - b'} =
    \frac{a - b' e^{-q} + a'e^{-q} - a'e^{-q}}{a' - b'} \\ =
    e^{-q} + \frac{a - a'e^{-q}}{a' - b'} \leq
    e^{-q} + \frac{a - ae^{-2q}}{a' - b'} =
    e^{-q} + \frac{a}{r'}(1-e^{-2q}) \\ \leq
    e^{-q} + \frac{a^{\uparrow}}{r'}(1-e^{-2q})
  \end{aligned}
  \end{equation}
  Therefore,
  \begin{equation} \label{eq:a-posterori-bnd-a-pos}
  \begin{aligned}[c]
  0 \leq ln(\frac{r'}{r}) \leq 
  ln(e^{-q} + \frac{a^{\uparrow}}{r'}(1-e^{-2q}))
  \end{aligned}
  \end{equation}
  Similarly, for the other case, we have:
  \begin{equation}
  \begin{aligned}[c]
    1 \geq
    \frac{r}{r'} \geq
    \frac{a - b'e^{q}}{a' - b'} =
    \frac{a - b'e^{q} + a'e^{q} - a'e^{q}}{a' - b'} \\ =
    e^{q} + \frac{a - a'e^{q}}{a' - b'} \geq
    e^{q} + \frac{a - ae^{2q}}{a' - b'} =
    e^{q} + \frac{a}{r'}(1-e^{2q}) \\ \geq
    e^{q} + \frac{a^{\downarrow}}{r'}(1-e^{2q})
  \end{aligned}
  \end{equation}
  Therefore, 
  \begin{equation} \label{eq:a-posterori-bnd-a-neg}
  \begin{aligned}[c]
    0 \geq ln(\frac{r'}{r}) \geq
    ln(e^{-q} + \frac{a^{\downarrow}}{r'}(1-e^{2q}))
  \end{aligned}
  \end{equation}
  A sound bound for both cases is absolute value of the maximum of
  Equation~\ref{eq:a-posterori-bnd-a-pos} and Equation~\ref{eq:a-posterori-bnd-a-neg},
  which is exactly our first bound (a). The proof strategy for the second bound
  (b) mirrors that of the first bound.

  We now examine the scenario where $r' < 0$. Recall that we assume that $r$ and
  $r'$ have the same sign.
  This scenario largely mirrors the first one, except some directions are
  flipped. Similarly to the previous case, there are two subcases, $r' \leq r$
  or $r' > r$. In the first case, we wish to maximize our numerator and minimize
  our denominator. 
  So, 
  \begin{equation}
  \begin{aligned}[c]
    1 \geq
    \frac{r}{r'} \geq
    \frac{a - b'e^{-q}}{a' - b'} =
    \frac{a - b'e^{-q} + a'e^{-q} - a'e^{-q}}{a' - b'} \\ =
    e^{-q} + \frac{a - a'e^{-q}}{a' - b'} \geq
    e^{-q} + \frac{a - ae^{-2q}}{a' - b'} =
    e^{-q} + \frac{a}{r'}(1-e^{-2q}) \\ \geq
    e^{-q} + \frac{a^{\downarrow}}{r'}(1-e^{-2q})
  \end{aligned}
  \end{equation}
  Therefore, 
  \begin{equation} \label{eq:a-posterori-bnd-a-pos-neg}
  \begin{aligned}[c]
    0 \geq ln(\frac{r'}{r}) \geq
    ln(e^{-q} + \frac{a^{\downarrow}}{r'}(1-e^{-2q}))
  \end{aligned}
  \end{equation}
  Similarly, for the other case, we have:
  \begin{equation}
  \begin{aligned}[c]
    1 \leq
    \frac{r}{r'} \leq
    \frac{a - b'e^{q}}{a' - b'} =
    \frac{a - b' e^{q} + a'e^{q} - a'e^{q}}{a' - b'} \\ =
    e^{q} + \frac{a - a'e^{q}}{a' - b'} \leq
    e^{q} + \frac{a - ae^{2q}}{a' - b'} =
    e^{q} + \frac{a}{r'}(1-e^{2q}) \\ \leq
    e^{q} + \frac{a^{\uparrow}}{r'}(1-e^{2q})
  \end{aligned}
  \end{equation}
  Therefore,
  \begin{equation} \label{eq:a-posterori-bnd-a-neg-neg}
  \begin{aligned}[c]
  0 \leq ln(\frac{r'}{r}) \leq 
  ln(e^{q} + \frac{a^{\uparrow}}{r'}(1-e^{2q}))
  \end{aligned}
  \end{equation}
  A sound bound for both cases is absolute value of the the maximum of
  Equation~\ref{eq:a-posterori-bnd-a-pos-neg} and
  Equation~\ref{eq:a-posterori-bnd-a-neg-neg}, which is exactly our first bound
  (c). The proof strategy for the second bound (d) mirrors that of the first
  bound.
\end{proof}

\begin{corollary}
  For $. \vdash e : M_q~\mathbf{num}_{(k_0, k_1)}$, $e \mapsto^{*} ((r, a, b),
  (r', a', b'))$ where $d_{\mathbb{R}}(r, r')$ is less than or equal to the
  following bounds if $0 < r'$: 
  \begin{enumerate}
    \item[(a)] $max(|ln(e^{-q} + \frac{a^{\uparrow}}{r'}(1-e^{-2q}))|, |ln(e^{-q} + \frac{a^{\downarrow}}{r'}(1-e^{2q}))|)$
    \item[(b)] $max(|ln(e^{q} + \frac{b^{\uparrow}}{r'}(e^{2q}- 1))|,|ln(e^q + \frac{b^{\downarrow}}{r'}(e^{-2q}-1))|)$
  \end{enumerate}
  and to the following bounds if $r' < 0$:
  \begin{enumerate}
    \item[(c)] $max(|ln(e^{-q} + \frac{a^{\downarrow}}{r'}(1-e^{-2q}))|, |ln(e^{-q} + \frac{a^{\uparrow}}{r'}(1-e^{2q}))|)$
    \item[(d)] $max(|ln(e^{q} + \frac{b^{\downarrow}}{r'}(e^{2q}- 1))|,|ln(e^q + \frac{b^{\uparrow}}{r'}(e^{-2q}-1))|)$
  \end{enumerate}
\end{corollary} 
\begin{proof}
  By our logical relation and metric preservation theorem (Theorem
  \ref{thm:metric-preservation}), we know that $e \mapsto ((r, a, b), (r', a',
  b'))$ such that $d_{\mathbb{P}}((r, a, b), (r', a', b')) \leq q$. Therefore,
  $d_{\mathbb{R}}(a, a'), d_{\mathbb{R}}(b, b') \leq q$. Applying the previous
  theorem proves the corollary.
\end{proof}

% As an illustrative toy example, suppose we
% have two large numbers like 1,000,001 and 1,000,000 with little relative error.
% Suppose that the relative error bounds for both large numbers is ``small" at 1
% per cent. Then the relative error for $1,000,001 - 1,000,000 = 1$ is quite
% large, upper-bounded at roughly 2,000,000 per cent.

% The problem is essentially:
% compositional type-based error bounds seem to require relative notions of error;
% however, relative notions of error are not well-behaved in the presence of
% subtraction and negative numbers. 
% This fundamental phenomenon is known as \textit{catastrophic cancellation}: two
% large nearby numbers with small relative error can be subtracted to produce a
% small number with arbitrarily high relative error.

% When catestrophic cancellation occurs, it is impossible to obtain tight error
% bounds. Importantly, we often know when catestrophic cancellation cannot not
% occur (\textit{a priori}) and can detect when it has occured (\textit{a
% posteori}). \textit{A posterori} error bounds are particularly promising as
% catestrophic cancellation only occurs when two large numbers are subtracted to
% produce a \textit{small} number with arbitrarily high error. Knowing that the
% final value is large allows for the ruling out of catestrophic cancellation,
% sidestepping the problem entirely.
% In both circumstances, our approach is able to reason about the presence of
% catestrophic cancellation (or rule it out) in order to produce useful error
% bounds.

% Now, we can track error on the paired representation.
% We wish to turn our relative bounds on the paired components $a$ and $b$ into a
% useful bound on $r$ (ideal) and $\tilde{r}$ (approximate). To produce useful
% bounds, we take advantage of the invariant that $r = a - b$ and $\tilde{r} =
% \tilde{a} - \tilde{b}$. 
% If we can obtain bounds on the paired ideal components $a$ and $b$, we would be
% able to statically produce bounds on the distance beteen $r$ and $\tilde{r}$.

% To accomplish this, we incorporate a bounds analysis in the type system. 
% A challenge of doing a bounds analysis in the type system is how to assign a
% precise bound to a function that has multiple call sites. A na{\"i}ve approach
% might widen the bounds for each call site, drastically reducing the precision
% of the bounds analysis. We address this challenge by introducing \textit{bound
% polymorphism}, allowing 
%    At each function call site, we specialize the polymorphic function to the
%    appropriate bounds. 


% we would expect Numerical Fuzz to infer that the error on our paired summation
% example grows in $O(log_2(n))$
% It is well known that round-off error grows linearly in the height of a
% summation tree. 
% This would match our intuition 
%
% pairwise summation will recursively a list of numbers in half and compute the
% sum of the halves.
% This produces a computation tree of height $\left \lceil{log_2(n)}\right
% \rceil$ with error growing in .
% By contrast, naive iterative summation takes a list of numbers and sequentialy
% adds from left to right over the list. This produces a computation tree of
% height $O(n)$, with error also growing in $O(n)$.
% linearly in size of the number of \textit{nodes}, which is undesirable.
