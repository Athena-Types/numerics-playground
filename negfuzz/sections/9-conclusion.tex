\section{Conclusion}
Our work is the first type-based approach that is able to reason about
forwards round-off error in the presence of subtraction and negative numbers. 
It is faster than comparable approaches, often by orders of magnitude, and
offers competitive precision.
We see several directions for future work. Firstly, it would be interesting to
extend Negative Fuzz to support more operations, like division. Secondly, we
noticed that the choice of bounds domain greatly impacted the precision of our
analysis, and the analysis precision might be improved by using a more
sophisticated bounds domain. Finally, rewriting a program to use
\textbf{factor} is quite tedious; it would be useful to automatically infer where
to insert our \textbf{factor} operations.

Overall, we believe that a type-based approach has additional qualitative
benefits. For instance, programs that call a library
function would not need to type-check the underlying library code and could
instead rely on the function type as an interface specification.  Type checking
could also be performed in parallel or incrementally over the dependency graph
of the program. It would be interesting to explore these aspects in future work.
